\documentclass[12pt, a4paper, openany, oneside]{book}
\usepackage[utf8]{inputenc}
\usepackage[T1]{fontenc}
\usepackage[brazilian]{babel}
\usepackage{libertine}


\usepackage{amsthm}
\usepackage{amsmath}
\usepackage{amssymb}
\usepackage{amsfonts}
\usepackage{mathtools}

\usepackage{csquotes}
\usepackage{microtype}
\usepackage{enumitem}
\usepackage{esvect}
\usepackage{verbatim}
%\usepackage{cancel}
%\usepackage{xcolor}

%%%%%%%%%%%%%%%%%%%%%%%%%%%%%%

\setlength{\parskip}{8pt}
\setlength{\parindent}{0pt}
\setlist[itemize]{itemsep=0pt, topsep=0pt}
\setlist[enumerate]{itemsep=0pt, topsep=0pt}

\allowdisplaybreaks
\raggedbottom
%\setlength\emergencystretch{0.5em}
%\tolerance=300

%%%%%%%%%%%%%%%%%%%%%%%%%%%%%%

\usepackage[backend=biber]{biblatex}
\addbibresource{bibliografia.bib}

%%%%%%%%%%%%%%%%%%%%%%%%%%%%%%

\newtheoremstyle{meuestilo}
  {8pt}                     % Espaço acima
  {0pt}                     % Espaço abaixo
  {\normalfont}             % Fonte do corpo do teorema
  {}                        % Indentação (vazio = nenhuma)
  {\bfseries}               % Fonte do cabeçalho do teorema
  {.}                       % Pontuação após o cabeçalho
  { }                       % Espaço após o cabeçalho
  {}                        % Especificação do cabeçalho (vazio = `Teorema 1.1`)

\theoremstyle{meuestilo}

\newtheorem{defi}{Definição}[chapter]
\newtheorem{teo}[defi]{Teorema}
\newtheorem{prop}[defi]{Proposição}
\newtheorem{lem}[defi]{Lema}
\newtheorem{cor}[defi]{Corolário}
\newtheorem{obs}[defi]{Observação}
\newtheorem{ex}[defi]{Exemplo}
\newtheorem{ax}[defi]{Axioma}
\newtheorem{fato}[defi]{Fato}
\newtheorem{nota}[defi]{Notação}

\renewenvironment{proof}[1][Prova.]{\par\noindent\textbf{#1} }%{\hfill$\blacksquare$\par}

\newcommand{\itemproof}{\hfill$\blacksquare$\par}

%%%%%%%%%%%%%%%%%%%%%%%%%%%%%%

\newcommand{\F}{\mathbb{F}}
\newcommand{\C}{\mathbb{C}}
\newcommand{\R}{\mathbb{R}}
\newcommand{\Q}{\mathbb{Q}}
\newcommand{\Z}{\mathbb{Z}}
\newcommand{\N}{\mathbb{N}}
\newcommand{\Pb}{\mathbb{P}}
\newcommand{\ds}{\displaystyle}
\newcommand{\op}{\operatorname}
\newcommand{\ind}{\operatorname{Ind}}
\newcommand{\id}{\operatorname{Id}}
\newcommand{\Dom}{\operatorname{Dom}}
\renewcommand{\Im}{\operatorname{Im}}
\renewcommand{\sin}{\operatorname{sen}}


\makeatletter
\NewDocumentCommand{\lowerint}{}{\mathop{}\mathpalette\lowerint@\relax\!\int}
\NewDocumentCommand{\upperint}{}{\mathop{}\mathpalette\upperint@\relax\!\int}
\newcommand{\lowerint@}[2]{%
  \begingroup
  \sbox\z@{$\m@th#1\int$}%
  \lowup@{l}{\underline}{#1}%
  \endgroup
}
\newcommand{\upperint@}[2]{%
  \begingroup
  \sbox\z@{$\m@th#1\int$}%
  \lowup@{r}{\overline}{#1}%
  \endgroup
}
\newcommand{\lowup@}[3]{%
  % #1 = l (lower) or r (upper)
  % #2 = \underline (lower) or \overline (upper)
  % #3 = math style
  \rlap{%
    \hspace{0.05\wd\z@}%
    \makebox[0.9\wd\z@][#1]{%
      $\m@th#2{%
        \hspace{0.4\wd\z@}%
        \ifx#3\displaystyle\else\hspace{0.1\wd\z@}\fi
        \vphantom{\copy\z@}%
      }$%
    }%
    \hspace{0.05\wd\z@}%
  }%
}
\makeatother

%%%%%%%%%%%%%%%%%%%%%%%%%%%%%%

\usepackage[
    colorlinks=true,
    linkcolor=blue,
    citecolor=blue,
    urlcolor=blue,
    breaklinks=true
]{hyperref}

%%%%%%%%%%%%%%%%%%%%%%%%%%%%%%

\title{demonstrações erradas de teoremas falsos}
\author{Alexander Kahleul}
\date{\today}

%%%%%%%%%%%%%%%%%%%%%%%%%%%%%%

\begin{document}
\maketitle
\setcounter{page}{2}
\tableofcontents

%%%%%%%%%%%%%%%%%%%%%%%%%%%%%%

\part{Fundamentos}

%!TEX root = main.tex

\chapter{Lógica Proposicional}

Seguimos \cite{herculespaulovich}.

\begin{defi}
    \leavevmode
    \begin{enumerate}[leftmargin=*, align=left, label=\textbf{(\alph*)}]
            \item O \textit{alfabeto proposicional} $\mathrm{Alf}$ é uma coleção infinita de símbolos distintos, nenhum deles propriamente contido em outro, separados nas seguintes categorias:
                \begin{enumerate}[label=\roman*.]
                    \item Conectivos: $\neg$, $\rightarrow$.
                    \item Parênteses: $($, $)$.
                    \item Variáveis proposicionais: $p_1, p_2, p_3 \ldots, p_i, \ldots$.
            \end{enumerate}
            \item As \textit{fórmulas} sobre $\mathrm{Alf}$ são definidas indutivamente pelas seguintes regras:
                \begin{enumerate}[label=\roman*.]
                    \item se $\mathbf{p}$ é uma variável proposicional, então $\mathbf{p}$ é uma fórmula.
                    \item se $\mathbf{A}$ e $\mathbf{B}$ são fórmulas, então $(\neg \mathbf{A})$ e $(\mathbf{A} \rightarrow \mathbf{B})$ são fórmulas;
                    \item todas as fórmulas são obtidas por um número finito de aplicações das regras acima. 
            \end{enumerate}
            O conjunto de todas as fórmulas é denotado por $\mathrm{Form}$, enquanto o conjunto de todas as fórmulas atômicas é denotado por $\mathrm{Form}_{\mathrm{At}}$ 
        \item A \textit{linguagem proposicional} é o par $\mathcal{L} := (\mathrm{Alf}, \mathrm{Form})$
    \end{enumerate}
\end{defi}

\begin{defi}
    \leavevmode
        \begin{enumerate}[leftmargin=*, align=left, label=\textbf{(\alph*)}]
            \item Um \textit{sistema de dedução proposicional} é uma tripla $(\mathcal{L}, \mathrm{Ax}, \mathrm{R})$, onde $\mathcal{L}$ é a linguagem proposicional, $\mathrm{Ax}$ é um conjunto de esquemas de axiomas e $R$ é um conjunto de regras de inferência.
            \item A Lógica Proposicional é o sistema $\mathcal{L}_{P} := (\mathcal{L}, \Lambda, \mathrm{MP})$, onde $\Lambda$ é um conjunto formado pelos esquemas de axiomas 
                \begin{enumerate}[label=$\mathrm{Ax}_{\arabic*}$.]
                    \item $(\mathbf{A} \rightarrow (\mathbf{B} \rightarrow \mathbf{A}))$
                    \item $((\mathbf{A} \rightarrow (\mathbf{B} \rightarrow \mathbf{C})) \rightarrow ((\mathbf{A} \rightarrow \mathbf{B})\rightarrow(\mathbf{A} \rightarrow \mathbf{C})))$
                    \item $(((\neg \mathbf{B}) \rightarrow (\neg \mathbf{A})) \rightarrow ( ((\neg \mathbf{B}) \rightarrow \mathbf{A}) \rightarrow \mathbf{B}))$
                \end{enumerate}
            e MP é a regra de inferência \textit{Modus Ponens}, a saber,
                \[
                    \mathrm{MP} := \{(\{\mathbf{A}, (\mathbf{A} \rightarrow \mathbf{B})\}, \mathbf{B} ) : \mathbf{A}, \mathbf{B} \in \mathrm{Form}\}.
                \]
        \end{enumerate}
\end{defi}

\begin{defi}
    Sejam $\Delta \subseteq \mathrm{Form}$ e $\mathbf{A} \in \mathrm{Form}$. 
        \begin{enumerate}[leftmargin=*, align=left, label=\textbf{(\alph*)}]
            \item Uma \textit{dedução} de $\mathbf{A}$ a partir de $\Delta$ é uma sequência $(A_1, \ldots, A_n)$ tal que $A_n \equiv \mathbf{A}$ e, para cada $k \in [n]$, vale  pelo menos uma das seguintes afirmações.
                \begin{enumerate}
                    \item $A_k \in \Lambda$.
                    \item $A_k \in \Delta$.
                    \item Existem índices $i,j < k$ tais que $A_k$ é obtida de $A_i$ e $A_j$ via $\mathrm{MP}$.
                \end{enumerate}
            Isso é denotado por $\Delta \vdash \mathbf{A}$.
            \item Dizemos que $\mathbf{A}$ é uma \textit{consequência sintática} de $\Delta$ se $\Delta \vdash \mathbf{A}$.
            \item Dizemos que $\mathbf{A}$ é um \textit{teorema} se $\emptyset \vdash \mathbf{A}$. Isso é denotado por $\vdash \mathbf{A}$.
        \end{enumerate}
\end{defi}

\begin{prop}
    $\vdash (\mathbf{A} \rightarrow \mathbf{A})$.
\end{prop}

\begin{proof}
    Pois tome:
        \[
            \begin{array}{rll}
                1. & ((\mathbf{A} \rightarrow ((\mathbf{A} \rightarrow \mathbf{A}) \rightarrow \mathbf{A})) \rightarrow ((\mathbf{A} \rightarrow (\mathbf{A} \rightarrow \mathbf{A})) \rightarrow (\mathbf{A} \rightarrow \mathbf{A}) )) & \mathrm{Ax}_2 \\
                2. & (\mathbf{A} \rightarrow ((\mathbf{A} \rightarrow \mathbf{A}) \rightarrow \mathbf{A})) & \mathrm{Ax}_1 \\
                3. & ((\mathbf{A} \rightarrow (\mathbf{A} \rightarrow \mathbf{A})) \rightarrow (\mathbf{A} \rightarrow \mathbf{A}) ) & \text{MP}(1,2) \\
                4. & (\mathbf{A} \rightarrow (\mathbf{A} \rightarrow \mathbf{A})) & \mathrm{Ax}_1 \\
                5. & (\mathbf{A} \rightarrow \mathbf{A}) & \text{MP}(4,5)
            \end{array}
        \]
\end{proof}

\begin{teo}[da Dedução] \label{teo.fund:deduçãoproposicional}
    Sejam $\Delta \subseteq \mathrm{Form}$ e $\mathbf{A}, \mathbf{B} \in \mathrm{Form}$.
        \begin{enumerate}[leftmargin=*, align=left, label=\textbf{(\alph*)}]
            \item Se $\Delta \cup \{\mathbf{A}\} \vdash \mathbf{B}$, então $\Delta \vdash (\mathbf{A} \rightarrow \mathbf{B})$.
            \item Se $\Delta \vdash (\mathbf{A} \rightarrow \mathbf{B})$, então $\Delta \cup \{\mathbf{A}\} \vdash \mathbf{B}$.
        \end{enumerate}
\end{teo}

\begin{proof}
    \leavevmode
        \begin{enumerate}[leftmargin=*, align=left, label=\textbf{(\alph*)}]
            \item Façamos indução no número de fórmulas que ocorrem na dedução de $\mathbf{B}$ a partir de $\Delta \cup \{\mathbf{A}\}$. Se $(A_1)$ é uma dedução de $\mathbf{B}$, então $A_1 \equiv \mathbf{B}$.
                \begin{enumerate}[label=\roman*.]
                    \item Se $\mathbf{B} \in \Lambda$, então $\Delta \vdash \mathbf{B}$, e como $\Delta \vdash (\mathbf{B} \rightarrow (\mathbf{A} \rightarrow \mathbf{B}))$, temos, via MP, que $\Delta \vdash (\mathbf{A} \rightarrow \mathbf{B})$.
                    \item Se $\mathbf{B} \in \Delta$, então $\Delta \vdash \mathbf{B}$, e como $\Delta \vdash (\mathbf{B} \rightarrow (\mathbf{A} \rightarrow \mathbf{B}))$, temos, via MP, que $\Delta \vdash (\mathbf{A} \rightarrow \mathbf{B})$.
                    \item Se $\mathbf{B} \equiv \mathbf{A}$, então de $\Delta \vdash (\mathbf{A} \rightarrow \mathbf{A})$ vem $\Delta \vdash (\mathbf{A} \rightarrow \mathbf{B})$.
                \end{enumerate}
            Agora, seja $(A_1, \ldots, A_n)$ uma dedução de $\mathbf{B}$ a partir de $\Delta \cup \{\mathbf{A}\}$ e suponha, por hipótese de indução, que o resultado vale para toda fórmula que pode ser deduzida a partir de $\Delta \cup \{\mathbf{A}\}$ por uma dedução com menos de $n$ fórmulas. Se $\mathbf{B} \in \Lambda$, $\mathbf{B} \in \Delta$ ou $\mathbf{B} \equiv \mathbf{A}$, então podemos deduzir $(\mathbf{A} \rightarrow \mathbf{B})$ a partir de $\Delta$ exatamente do mesmo modo que fizemos na base da indução. Suponha, então, que $\mathbf{B}$ é obtida de duas fórmulas de índices $<n$ via MP. Essas duas fórmulas têm as formas $\mathbf{C}$ e $(\mathbf{C} \rightarrow \mathbf{B})$, e como elas foram deduzidas de $\Delta \cup \{\mathbf{A}\}$ por menos de $n$ fórmulas, temos que $\Delta \vdash (\mathbf{A} \rightarrow \mathbf{C})$ e $\Delta \vdash (\mathbf{A} \rightarrow (\mathbf{C} \rightarrow \mathbf{B}))$. Com isso, podemos deduzir $(\mathbf{A} \rightarrow \mathbf{B})$ a partir de $\Delta$ do seguinte modo.
                \[
                    \begin{array}{rll}
                        & \vdots & \vdots \\
                        i. & \Delta \vdash (\mathbf{A} \rightarrow \mathbf{C}) & \\
                        & \vdots & \vdots \\
                        j. & \Delta \vdash (\mathbf{A} \rightarrow (\mathbf{C} \rightarrow \mathbf{B})) & \\[1ex]
                        k. & \Delta \vdash ((\mathbf{A} \rightarrow (\mathbf{C} \rightarrow \mathbf{B})) \rightarrow ((\mathbf{A} \rightarrow \mathbf{C}) \rightarrow (\mathbf{A} \rightarrow \mathbf{B}))) & \mathrm{Ax}_2 \\[1ex]
                        l. & \Delta \vdash ((\mathbf{A} \rightarrow \mathbf{C}) \rightarrow (\mathbf{A} \rightarrow \mathbf{B})) & \text{MP}(j, k) \\[1ex]
                        m. & \Delta \vdash (\mathbf{A} \rightarrow \mathbf{B}) & \text{MP}(i, l)
                \end{array}
                \]
            Assim, se $\Delta \cup \{\mathbf{A}\} \vdash \mathbf{B}$, então $\Delta \vdash (\mathbf{A} \rightarrow \mathbf{B})$. \blackproof
            \item Como $\Delta \vdash (\mathbf{A} \rightarrow \mathbf{B})$ e $\Delta \subseteq \Delta \cup \{\mathbf{A}\}$, temos $\Delta \cup \{\mathbf{A}\} \vdash (\mathbf{A} \rightarrow \mathbf{B})$. Daí, como $\Delta \cup \{\mathbf{A}\} \vdash \mathbf{A}$, temos que $\Delta \cup \{\mathbf{A}\} \vdash \mathbf{B}$. \blackproof
        \end{enumerate}
\end{proof}

\begin{prop} \label{prop.fund:propriedadesvdash}
    \leavevmode
        \begin{enumerate}[leftmargin=*, align=left, label=\textbf{(\alph*)}]
            \item $\{\mathbf{A} \rightarrow \mathbf{B}, \mathbf{B} \rightarrow \mathbf{C} \} \vdash \mathbf{A} \rightarrow \mathbf{C}$.
            \item $\vdash \mathbf{A} \leftrightarrow \neg \neg \mathbf{A}$.
        \end{enumerate}
\end{prop}

\begin{defi}
    \leavevmode
        \begin{enumerate}[leftmargin=*, align=left, label=\textbf{(\alph*)}]
            \item Uma \textit{valoração proposicional} é uma função $\bar{v} : \mathrm{Form}_{\mathrm{At}} \to \{\mathfrak{t}, \mathfrak{f} \}$.
            \item Uma \textit{valoração} é uma função $v: \mathrm{Form} \to \{\mathfrak{t}, \mathfrak{f} \}$ tal que
                \begin{enumerate}[label=\roman*.]
                    \item $v |_{\mathrm{Form}_{\mathrm{At}}} = \bar{v}$;
                    \item $v[(\neg \mathbf{A})] = \mathfrak{f}$ se, e somente se, $v(\mathbf{A}) = \mathfrak{t}$;
                    \item $v[(\mathbf{A} \rightarrow \mathbf{B})] = \mathfrak{f}$ se, e somente se, $v(\mathbf{A}) = \mathfrak{t}$ e $v(\mathbf{B}) = \mathfrak{f}$.
                \end{enumerate}
        \end{enumerate}
\end{defi}

\begin{defi}
    Uma fórmula $\mathbf{A}$ é \textit{válida}, ou \textit{tautológica}, se $v(\mathbf{A}) = \mathfrak{t}$ para toda valoração $v$. Isso é denotado por $\vDash \mathbf{A}$.
\end{defi}

\begin{lem} \label{lem.fund:axregrasvalidos}
    \leavevmode
        \begin{enumerate}[leftmargin=*, align=left, label=\textbf{(\alph*)}]
            \item Os axiomas de $\mathcal{L}_P$ são válidos.
            \item Se $\vDash \mathbf{A}$ e $\vDash (\mathbf{A} \rightarrow \mathbf{B})$, então $\vDash \mathbf{B}$.
        \end{enumerate}
\end{lem}

\begin{proof}
    \leavevmode
        \begin{enumerate}[leftmargin=*, align=left, label=\textbf{(\alph*)}]
            \item Suponha, por absurdo, que $\not\vDash\mathbf{A} \rightarrow (\mathbf{B} \rightarrow \mathbf{A})$. Isso só é possível se $v[(\mathbf{B} \rightarrow \mathbf{A})] = \mathfrak{f}$, o que, por sua vez, só é possível se $v(\mathbf{A}) = \mathfrak{f}$, o que contraria a hipótese. Logo, $\vDash\mathbf{A} \rightarrow (\mathbf{B} \rightarrow \mathbf{A})$. A prova de que os outros (esquemas de) axiomas são válidos segue analogamente. \blackproof
            \item Suponha, por absurdo, que $\not\vDash \mathbf{B}$. Logo, existe uma valoração $v$ tal que $v(\mathbf{B}) = \mathfrak{f}$. Para essa valoração, como $\vDash \mathbf{A}$, temos $v(\mathbf{A}) = \mathfrak{t}$, de modo que $v[(\mathbf{A} \rightarrow \mathbf{B})] = \mathfrak{f}$, o que contraria a hipótese. Com isso, $\vDash \mathbf{B}$. \blackproof
        \end{enumerate}
\end{proof}


\begin{teo}[da Correção] \label{teo.fund:correçãoproposicional}
    Se $\vdash \mathbf{A}$, então $\vDash \mathbf{A}$.    
\end{teo}

\begin{proof}
    Façamos indução no número de fórmulas que ocorrem na dedução de $\mathbf{A}$. Se $(A_1)$ é a dedução de $\mathbf{A}$, então $A_1 \equiv \mathbf{A}$ e $\mathbf{A}$ é um axioma, e como todos os axiomas são válidos (\cref{lem.fund:axregrasvalidos}), temos $\vDash \mathbf{A}$. Agora, seja $(A_1, \ldots, A_n)$ a dedução de $\mathbf{A}$ e suponha, por hipótese de indução, que o resultado vale para toda fórmula que pode ser deduzida por uma dedução com menos de $n$ fórmulas. Se $\mathbf{A}$ é um axioma, então $\vDash \mathbf{A}$. Suponha, então, que $\mathbf{A}$ é obtida de duas fórmulas de índices $<n$ via MP. Essas duas fórmulas têm as formas $\mathbf{B}$ e $(\mathbf{B} \rightarrow \mathbf{A})$, e como elas foram deduzidas por menos de $n$ fórmulas, temos que $\vDash \mathbf{B}$ e $\vDash (\mathbf{B} \rightarrow \mathbf{A})$. Daí, como a regra MP conserva validade (\cref{lem.fund:axregrasvalidos}), temos que $\vDash \mathbf{A}$. \blackproof
\end{proof}

\begin{lem}[Kalmár] \label{lem.fund:troca}
    Sejam $\mathbf{A}(p_1, \ldots, p_n) \in \mathrm{Form}$ e $v : \mathrm{Form} \to \{\mathfrak{t}, \mathfrak{f} \}$. Se
        \[
            p'_i :\equiv
                \begin{cases}
                    p_i, & \text{se } v(p_i) = \mathfrak{t} \\
                    (\neg p_i), & \text{se } v(p_i) = \mathfrak{f}
                \end{cases}
            \quad \text{ e } \quad
            \mathbf{A}' :\equiv
                \begin{cases}
                    \mathbf{A}, & \text{se } v(\mathbf{A}) = \mathfrak{t} \\
                    (\neg \mathbf{A}), & \text{se } v(\mathbf{A}) = \mathfrak{f}
                \end{cases},
        \]
    então $\{p'_1, \ldots, p'_n \} \vdash \mathbf{A}'$.
\end{lem}

\begin{proof}
    Façamos indução no número de conectivos que ocorrem em $\mathbf{A}$.
\end{proof}

\begin{teo}[da Completude] \label{teo.fund:completudeproposicional}
    Se $\vDash \mathbf{A}$, então $\vdash \mathbf{A}$.
\end{teo}

\begin{proof}
    Sejam $p_1, \ldots, p_n$ as variáveis proposicionais que ocorrem em $\mathbf{A}$. Pelo \cref{lem.fund:troca}, temos $\{p'_1, \ldots, p'_n \} \vdash \mathbf{A}'$ para toda valoração $v$, e como $\vDash \mathbf{A}$, temos $\mathbf{A}' \equiv \mathbf{A}$, de modo que $\{p'_1, \ldots, p'_n \} \vdash \mathbf{A}$. Agora, definindo
        \[
            v_1(p_i) :=
                \begin{cases}
                    v(p_i), & \text{se } i < n \\
                    \mathfrak{t}, & \text{se } i = n
                \end{cases}
            \quad \text{ e } \quad
            v_2(p_i) :=
                \begin{cases}
                    v(p_i), & \text{se } i < n \\
                    \mathfrak{f}, & \text{se } i = n
                \end{cases},
        \]
    cada $p'_i$, para $i<n$, fica bem definido. Como $v_2(p_n) = \mathfrak{f}$, vem $\{p'_1, \ldots, p'_{n-1}, (\neg p_n) \} \vdash \mathbf{A}$, de modo que, pelo \Cref{teo.fund:deduçãoproposicional}, temos $\{p'_1, \ldots, p'_{n-1}\} \vdash ((\neg p_n) \rightarrow \mathbf{A})$. Analogamente, como $v_1(p_n) = \mathfrak{t}$, então $\{p'_1, \ldots, p'_{n-1}\} \vdash (p_n \rightarrow \mathbf{A})$. Assim:
        \[
            \begin{array}{rll}
                1. & \{p'_1, \ldots, p'_{n-1}\} \vdash ((\neg p_n) \rightarrow \mathbf{A}) & p. \\
                2. & \{p'_1, \ldots, p'_{n-1}\} \vdash (p_n \rightarrow \mathbf{A}) & p. \\
                3. & \{p'_1, \ldots, p'_{n-1}\} \vdash (p_n \rightarrow \mathbf{A}) \rightarrow (((\neg p_n) \rightarrow \mathbf{A}) \rightarrow \mathbf{A}) & \eqref{prop.fund:propriedadesvdash} \\
                4. & \{p'_1, \ldots, p'_{n-1}\} \vdash (((\neg p_n) \rightarrow \mathbf{A}) \rightarrow \mathbf{A}) & \text{MP}(2,3) \\
                5. & \{p'_1, \ldots, p'_{n-1}\} \vdash \mathbf{A} & \text{MP}(1,4)
            \end{array}
        \]
    Com isso, eliminamos $p_n$. Repetindo esse processo (um número finito de vezes), eliminamos $p_{n-1}, \ldots, p_1$, obtendo por fim $\vdash \mathbf{A}$.
    \blackproof
\end{proof}


\begin{cor}[Adequação]
    $\vDash \mathbf{A}$ se, e somente se, $\vdash \mathbf{A}$.
\end{cor}

\begin{proof}
    Segue dos \cref{teo.fund:correçãoproposicional,teo.fund:completudeproposicional}. \blackproof
\end{proof}



coisas

\begin{defi}
    Sejam $\mathbf{A} \in \mathrm{Form}$ e $\Gamma \subseteq \mathrm{Form}$.
        \begin{enumerate}[leftmargin=*, align=left, label=\textbf{(\alph*)}]
            \item Um \textit{modelo} de $\mathbf{A}$ é uma valoração $v$ tal que $v(\mathbf{A}) = \mathfrak{t}$. Dizemos que $v$ \textit{satisfaz} $\mathbf{A}$. Isso é denotado por $v \vDash \mathbf{A}$.
            \item Um \textit{modelo} de $\Gamma$ é uma valoração $v$ tal que $v \vDash \mathbf{B}$ para todo $\mathbf{B} \in \Gamma$.
        \end{enumerate}
\end{defi}

\begin{defi}
    Sejam $\mathbf{A}, \mathbf{B} \in \mathrm{Form}$ e $\Gamma \subseteq \mathrm{Form}$.
        \begin{enumerate}[leftmargin=*, align=left, label=\textbf{(\alph*)}]
            \item Dizemos que $\mathbf{B}$ é uma \textit{consequência semântica} de $\mathbf{A}$ se todo modelo de $\mathbf{A}$ é também um modelo de $\mathbf{B}$. Isso é denotado por $\{\mathbf{A} \} \vDash \mathbf{B}$.
            \item Dizemos que $\mathbf{B}$ é uma \textit{consequência semântica} de $\Gamma$ se todo modelo de $\Gamma$ é também um modelo de $\mathbf{B}$. Isso é denotado por $\Gamma \vDash \mathbf{B}$.
        \end{enumerate}
\end{defi}

\begin{teo}
    Se $\Gamma \vdash \mathbf{A}$, então $\Gamma \vDash \mathbf{A}$.
\end{teo}

\begin{proof}
    O caso $\Gamma = \emptyset$ é simplesmente o teorema da correção \eqref{teo.fund:correçãoproposicional}. Suponha, então, que $\Gamma \neq \emptyset$. Se $\mathbf{C}_1, \ldots, \mathbf{C}_n$ são as fórmulas de $\Gamma$ que aparecem na dedução de $\mathbf{A}$, então $\{\mathbf{C}_1, \ldots, \mathbf{C}_n\} \vdash \mathbf{A}$, de modo que, por sucessivas aplicações do teorema da dedução \eqref{teo.fund:deduçãoproposicional}, temos $\vdash \mathbf{C}_1 \rightarrow \cdots \rightarrow \mathbf{C}_n \rightarrow \mathbf{A}$. Com isso, para toda valoração $v$ tal que $v(\mathbf{C}_i) = \mathfrak{t}$ para todo $i \in [n]$, temos $v(\mathbf{A}) = \mathfrak{t}$. Como $\{\mathbf{C}_1, \ldots, \mathbf{C}_n\} \subseteq \Gamma$, temos $v \vDash \mathbf{A}$ para toda valoração $v$ tal que $v \vDash \Gamma$, isto é, $\Gamma \vDash \mathbf{A}$. \blackproof
\end{proof}

\chapter{Teorias de Primeira Ordem}

\section{Linguagens de Primeira Ordem}

\begin{defi}[Linguagens de Primeira Ordem] 
    \leavevmode
    \begin{enumerate}[leftmargin=*, align=left, label=\textbf{(\alph*)}]
            \item Um \textit{alfabeto} é uma coleção infinita de símbolos distintos, nenhum deles propriamente contido em outro, separados nas seguintes categorias:
                \begin{enumerate}[label=\roman*.]
                    \item Conectivos: $\lor$, $\neg$.
                    \item Quantificador universal: $\forall$.
                    \item Parênteses: $($, $)$.
                    \item Variáveis, uma para cada inteiro positivo $n$: $v_1, v_2, \ldots, v_n, \ldots$. % O conjunto de símbolos de variáveis será denotado por Vars.
                    \item Símbolos de função: para cada inteiro positivo $n$, uma coleção de símbolos de função $n$-ários.
                    \item Símbolos de predicado: para cada inteiro positivo $n$, uma coleção de símbolos de predicado $n$-ários.
                    \item Símbolo predicado binário de igualdade: $=$.
                    \item Símbolos de constantes: uma coleção de símbolos.
            \end{enumerate}
            \item Os \textit{termos} correspondentes a um alfabeto são definidos do seguinte modo:
                \begin{enumerate}[label=\roman*.]
                    \item as variáveis são termos;
                    \item as constantes são termos;
                    \item se $t_1, t_2, \ldots, t_n$ são termos e $f$ é um símbolo de função $n$-ário, então $f(t_1, t_2, \ldots, t_n)$ é um termo;
                    \item todos os termos têm uma das formas acima.
                \end{enumerate}
            \item As \textit{fórmulas} correspondentes a um alfabeto são definidas do seguinte modo:
                \begin{enumerate}[label=\roman*.]
                    \item se $t_1$ e $t_2$ são termos, então $= (t_1,  t_2)$ é uma fórmula;
                    \item se $t_1, t_2, \ldots, t_n$ são termos e $R$ é um símbolo de predicado $n$-ário, então $R(t_1, t_2, \ldots, t_n)$ é uma fórmula;
                    \item se $\alpha$ e $\beta$ são fórmulas, então $(\neg \alpha)$ e $(\alpha \lor \beta)$ são fórmulas;
                    \item se $x$ é uma variável e $\alpha$ é uma fórmula, então $(\forall x)(\alpha)$ é uma fórmula;
                    \item todas as fórmulas têm uma das formas acima.
            \end{enumerate}
        As fórmulas como definidas nos itens i. e ii. são ditas \textit{atômicas}. A fórmula $\alpha$ que aparece no item iv. é chamada de \textit{escopo} do quantificador $\forall$.
        \item Uma \textit{linguagem de primeira ordem} $\mathcal{L}$ consiste num alfabeto como descrito no item (a) e termos ($\mathcal{L}$-termos) e fórmulas ($\mathcal{L}$-fórmulas) como descritos nos itens (b) e (c).
        \item Para especificar uma linguagem de primeira ordem $\mathcal{L}$, basta especificar quais são suas constantes, seus símbolos de função e seus símbolos de predicado:
            \[
                \mathcal{L} \quad \text{é} \quad \{ c_1, c_2, \ldots, f^{a(f_1)}_1, f^{a(f_2)}_2, \ldots, R^{a(R_1)}_1, R^{a(R_2)}_2, \ldots \},
            \]
        onde cada $c_i$ é um símbolo de constante, cada $f^{a(f_i)}_i$ é um símbolo de função de aridade $a(f_i)$ e cada $R^{a(R_i)}_i$ é um símbolo de predicado de aridade $a(R_i)$.
    \end{enumerate}
\end{defi}

\begin{teo}[Legibilidade única] \label{teo.fund:termform}
    Seja $\mathcal{L}$ uma linguagem de primeira ordem.
        \begin{enumerate}[leftmargin=*, align=left, label=\textbf{(\alph*)}]
            \item Todo termo tem uma, e exatamente uma, das formas i.-iii. da definição de termo.
            \item Toda fórmula tem uma, e exatamente uma, das formas i.-iv. da definição de fórmula.
        \end{enumerate}
\end{teo}

\begin{proof}
    Ver \cite{shoenfield1967}, página 16, ou ainda, \cite{learykristiansen2017logica}, página 18.
\end{proof}


\begin{defi}[Subtermos e subfórmulas]
    Sejam $t$ um $\mathcal{L}$-termo e $\varphi$ uma $\mathcal{L}$-fórmula.
        \begin{enumerate}[leftmargin=*, align=left, label=\textbf{(\alph*)}]
            \item Um \textit{subtermo} de $t$ é um $\mathcal{L}$-termo definido recursivamente do seguinte modo:
                \begin{enumerate}[label=\roman*.]
                    \item se $t$ é uma variável ou uma constante, então $t$ é o único subtermo de si mesmo;
                    \item se $t$ é da forma $ft_1 t_2 \ldots t_n$, onde $f$ é um símbolo funcional $n$-ário e $t_1, t_2, \ldots, t_n$ são $\mathcal{L}$-termos, então os subtermos de $t$ são $t$ e os subtermos de $t_1, t_2, \ldots, t_n$.
                \end{enumerate}
            \item Uma \textit{subfórmula} de $\varphi$ é uma $\mathcal{L}$-fórmula definida recursivamente do seguinte modo:
                \begin{enumerate}[label=\roman*.]
                    \item se $\varphi$ é atômica, então  $\varphi$ é a única subfórmula de si mesma;
                    \item se \(\varphi\) é da forma \((\neg\alpha)\) ou da forma \((\forall x)(\alpha)\), então as subfórmulas de \(\varphi\) são \(\varphi\) e as subfórmulas de \(\alpha\);
                    \item se \(\varphi\) é da forma \((\alpha \vee \beta)\), então as subfórmulas de \(\varphi\) são \(\varphi\) e as subfórmulas de \(\alpha\) e de \(\beta\).
                \end{enumerate}
        \end{enumerate}
\end{defi}

\begin{defi} % leary, def1.5.2.
    Sejam $\mathcal{L}$ uma linguagem de primeira ordem, $x$ uma variável e $\varphi$ uma fórmula. 
        \begin{enumerate}[leftmargin=*, align=left, label=\textbf{(\alph*)}]
            \item (Variáveis livres) Dizemos que $x$ é \textit{livre} em $\varphi$ se
                \begin{enumerate}[label=\roman*.]
                    \item $\varphi$ é atômica e $x$ ocorre em (é um símbolo) $\varphi$; ou
                    \item $\varphi$ é da forma $(\neg \alpha)$ e $x$ é livre na fórmula $\alpha$; ou  
                    \item $\varphi$ é da forma $(\alpha \lor \beta)$ e $x$ é livre em pelo menos uma das fórmulas $\alpha$ ou $\beta$; ou  
                    \item $\varphi$ é da forma $(\forall y)(\alpha)$, com $x$ diferente de $y$ e livre na fórmula $\alpha$.
                \end{enumerate}
            Equivalentemente, podemos dizer que uma ocorrência de $x$ é livre em $\varphi$ se $x$ não ocorre no escopo de uma subfórmula $(\forall x)(\alpha)$ de $\varphi$.
            \item (Variáveis ligadas) Dizemos que $x$ é \textit{ligada} em $\varphi$ se não for livre em $\varphi$. %, isto é, se ocorre no escopo de uma subfórmula $(\forall x)(\alpha)$ de $\varphi$.
            \item (Sentenças) Uma \textit{sentença} de $\mathcal{L}$, ou uma $\mathcal{L}$-\textit{sentença}, é uma $\mathcal{L}$-fórmula que não possui variáveis livres.
        \end{enumerate}
\end{defi}

\begin{defi}[Substituição]
    Sejam $\mathcal{L}$ uma linguagem de primeira ordem, $t$ um termo e $x$ uma variável.
        \begin{enumerate}[leftmargin=*, align=left, label=\textbf{(\alph*)}]
            \item Seja $u$ um termo. O termo $u_t^x$, que resulta da substituição de todas as ocorrências de $x$ em $u$ por $t$, é definido recursivamente do seguinte modo:
                \begin{enumerate}[label=\roman*.]
                    \item se $u$ é uma variável diferente de $x$, então $u_t^x$ é $u$;
                    \item se $u$ é a variável $x$, então $u_t^x$ é $t$;
                    \item se $u$ é uma constante, então $u_t^x$ é $u$;
                    \item se $u$ é da forma $f t_1 \ldots t_n$, então $u_t^x$ é $f {t_1}_t^x \ldots {t_n}_t^x$.
                \end{enumerate}
            \item Seja $\varphi$ uma fórmula. A fórmula $\varphi_t^x$, que resulta da substituição de todas as ocorrências de $x$ em $\varphi$ por $t$, é definida recursivamente do seguinte modo:
                \begin{enumerate}[label=\roman*.]
                    \item se $\varphi$ é da forma $(t_1=t_2)$, então $\varphi_t^x$ é $({t_1}_t^x={t_2}_t^x)$; 
                    \item se $\varphi$ é da forma $R t_1 \ldots t_n$, então $\varphi_t^x$ é $R {t_1}_t^x \ldots {t_n}_t^x$;
                    \item se $\varphi$ é da forma $(\neg \alpha)$, então $\varphi_t^x$ é $(\neg \alpha_t^x)$;
                    \item se $\varphi$ é da forma $(\alpha \lor \beta)$, então $\varphi_t^x$ é $(\alpha_t^x \lor \beta_t^x)$;
                    \item se $\varphi$ é da forma $(\forall y) (\alpha)$, então $\varphi_t^x$ é
                        \[
                            \begin{cases}
                                \varphi, & \text{ se } y \text{ é } x; \text{ ou } \\
                                (\forall y)(\alpha_t^x), & \text{ caso contrário. }
                            \end{cases}
                        \]
                \end{enumerate}
        \end{enumerate}
\end{defi}

\begin{defi}[Substituibilidade]
    Sejam $\mathcal{L}$ uma linguagem de primeira ordem, $\varphi$ uma fórmula, $t$ um termo e $x$ uma variável. Dizemos que $x$ é \textit{substituível} por $t$ em $\varphi$ se
        \begin{enumerate}[label=\roman*.]
            \item $\varphi$ é atômica; ou
            \item $\varphi$ é da forma $(\neg \alpha)$ e $x$ é substituível por $t$ em $\alpha$;
            \item $\varphi$ é da forma $(\alpha \lor \beta)$ e $x$ é substituível por $t$ em $\alpha$ e em $\beta$;
            \item $\varphi$ é da forma $(\forall y)(\alpha)$ e, exclusivamente, ou $x$ é ligada em $\varphi$, ou $y$ não ocorre em $t$ e $x$ é substituível por $t$ em $\alpha$.
        \end{enumerate}
\end{defi}

\section{Estruturas}

\begin{defi}
    Seja $\mathcal{L}$ uma linguagem de primeira ordem. Uma $\mathcal{L}$-\textit{estrutura} $\mathfrak{A}$ consiste num conjunto $A$, chamado de \textit{universo} de $\mathfrak{A}$, tal que
        \begin{enumerate}[label=\roman*.]
            \item para cada símbolo de constante $c$ de $\mathcal{L}$, há um elemento $c^{\mathfrak{A}}$ em $A$;
            \item para cada símbolo de função $n$-ário $f$ de $\mathcal{L}$, há uma função $f^{\mathfrak{A}} : A^n \to A$;
            \item para cada símbolo de relação $n$-ário $R$ de $\mathcal{L}$, há uma relação $R^{\mathfrak{A}}$ em $A$ (isto é, $R^{\mathfrak{A}} \subseteq A^n$).
        \end{enumerate}
\end{defi}

\begin{defi}
    Seja $\mathfrak{A}$ uma $\mathcal{L}$-estrutura de universo $A$. 
        \begin{enumerate}[leftmargin=*, align=left, label=\textbf{(\alph*)}]
            \item Uma \textit{valoração} é qualquer função $s : \mathrm{Vars} \to A$.
            \item Sejam $s$ uma valoração, $x$ uma variável e $a$ um elemento de $A$. Uma \textit{$x$-modificação de $s$} é definida como
                \[
                    s[x|a](v) := \begin{cases}
                        s(v) & \text{ se } v \text{ é uma variável diferente de } x \\
                        a & \text{ se } v \text{ é a variável } x
                    \end{cases}.
                \]
            \item Seja $s : \mathrm{Vars} \to A$ uma valoração. Uma \textit{valoração de termos gerada por $s$} é uma função $\bar{s} : \mathrm{Term} \to A$ definida recursivamente do seguinte modo:
                \begin{enumerate}[label=\roman*.]
                    \item se $t$ é uma variável, então $\bar{s}(t) = s(t)$;
                    \item se $t$ é um símbolo de constante $c$, então $\bar{s}(t) = c^{\mathfrak{A}}$;
                    \item se $t$ é da forma $ft_1 \ldots t_n$, onde $f$ é um símbolo funcional $n$-ário e $t_1, \ldots, t_n$ são termos, então $\bar{s}(t) = f^{\mathfrak{A}}(\bar{s}(t_1), \ldots, \bar{s}(t_n))$.
                \end{enumerate}
            \item Sejam $\varphi$ uma $\mathcal{L}$-fórmula e $s : \mathrm{Vars} \to A$ uma valoração. Dizemos que $\mathfrak{A}$ \textit{satisfaz $\varphi$ com relação a $s$}, denotando isso por $\mathfrak{A} \models  \varphi[s]$, se
                \begin{enumerate}[label=\roman*.]
                    \item $\varphi$ é da forma $=t_1 t_2$ e $\bar{s}(t_1)$ coincide com $\bar{s}(t_2)$; ou
                    \item $\varphi$ é da forma $Rt_1 \ldots t_n$ e $(\bar{s}(t_1), \ldots, \bar{s}(t_n))$ é um elemento de $R^{\mathfrak{A}}$; ou
                    \item $\varphi$ é da forma $(\neg \alpha)$ e $\mathfrak{A} \not\models \alpha[s]$; ou
                    \item $\varphi$ é da forma $(\alpha \lor \beta)$ e $\mathfrak{A} \models \alpha[s]$ ou $\mathfrak{A} \models \beta[s]$; ou
                    \item $\varphi$ é da forma $(\forall x)(\alpha)$ e $\mathfrak{A} \models \alpha [s[x|a]]$ para cada elemento $a$ de $A$.
                \end{enumerate}
            Se $\Gamma$ é um conjunto de $\mathcal{L}$-fórmulas, dizemos que $\mathfrak{A}$ satisfaz $\Gamma$ com relação a $s$, escrevendo $\mathfrak{A} \models \Gamma[s]$, se $\mathfrak{A} \models \gamma[s]$ para cada fórmula $\gamma$ em $\Gamma$.
        \end{enumerate}
\end{defi}

\begin{teo}
    Seja $\mathfrak{A}$ uma $\mathcal{L}$-estrutura.
        \begin{enumerate}[leftmargin=*, align=left, label=\textbf{(\alph*)}]
            \item Se $s_1$ e $s_2$ são valorações tais que $s_1 (v) = s_2 (v)$ para toda variável $v$ que ocorre num termo $t$, então $\bar{s}_1(t) = \bar{s}_2(t)$.
            \item Se $s_1$ e $s_2$ são valorações tais que $s_1 (v) = s_2 (v)$ para toda variável livre $v$ que ocorre na fórmula $\varphi$, então $\mathfrak{A} \models \varphi[s_1]$ se, e somente se, $\mathfrak{A} \models \varphi[s_2]$.
            \item Se $\psi$ é uma sentença, então ou $\mathfrak{A} \models \psi[s]$ para todas as valorações $s$, ou $\mathfrak{A} \models \psi[s]$ para nenhuma valoração $s$. 
        \end{enumerate}
\end{teo}

\begin{proof}
    Ver \cite{learykristiansen2017logica}, seção 1.7.
\end{proof}

\begin{defi}
    Seja $\mathfrak{A}$ uma $\mathcal{L}$-estrutura.
        \begin{enumerate}[leftmargin=*, align=left, label=(\alph*)]
            \item Seja $\varphi$ uma fórmula. Diremos que $\mathfrak{A}$ é um \textit{modelo} de $\varphi$, denotando isso por $\mathfrak{A} \models \varphi$, se $\mathfrak{A} \models \varphi[s]$ para toda função de atribuição de variável $s$.
            \item Seja $\Phi$ um conjunto de fórmulas. Diremos que $\mathfrak{A}$ \textit{modela} $\Phi$, denotando isso por $\mathfrak{A} \models \Phi$, se  $\mathfrak{A} \models \varphi$ para cada fórmula $\varphi$ de $\Phi$.
        \end{enumerate}
\end{defi}

\chapter{Teoria dos Conjuntos}

A linguagem (de primeira ordem) da teoria dos conjuntos, denotada por $\mathcal{L}_{ST}$, consiste em somente um símbolo de predicado binário dito de \textit{pertencimento} $\in$. A seguir apresentamos os axiomas de ZFC que constituem a teoria de primeira ordem da teoria dos conjuntos. Na primeira seção apresentamos e discutimos os axiomas básicos, deixando os axiomas do infinito (que garante a existência do conjunto dos números naturais $\omega$), da escolha (que tem muitas equivalências) e da substituição (fundamental para a teoria dos ordinais), os mais importantes, e mais complicados, para serem tratados nas próximas seções.

\section{Primeiros Axiomas} \label{sec1}

\subsection{O Axioma da Extensão}

\begin{ax}[da Extensão] \label{ax:1}
    Dois conjuntos são iguais se, e somente se, eles têm os mesmos elementos.
        \[
            \boxed{
                \forall x \forall y ( (x=y) \leftrightarrow \forall z((z \in x) \leftrightarrow (z \in y)))
            }
        \]
\end{ax}

\begin{defi}[Inclusão]
    Um conjunto $x$ está \textit{contido} num conjunto $y$, ou é um \textit{subconjunto} de $y$, se todo elemento de $x$ é um elemento de $y$:
        \[ \ds
            (x \subseteq y) \overset{\text{def}}{\leftrightarrow} \forall z ((z \in x) \rightarrow (z \in y)).
        \]
\end{defi}

\begin{obs}
    De maneira análoga podemos definir $\subsetneq$, $\not\subseteq$, $\supseteq$, etc. Além disso, com a definição de $\subseteq$, o axioma da extensão pode ser enunciado assim:
        \[
            \forall x \forall y ( (x=y) \leftrightarrow ((x \subseteq y) \land (y \subseteq x))).
        \] 
\end{obs}

\begin{prop}\footnote{Conforme a definição \eqref{defi.fund:ordemparcial}, isso significa que a relação $\subseteq$ é uma relação de ordem parcial.} \label{prop.fund:inclusaoparcial}
    \leavevmode
        \begin{enumerate}[leftmargin=*, align=left, label=\textbf{(\alph*)}]
            \item $\forall x (x \subseteq x)$.
            \item $\forall x \forall y ((x \subseteq y) \land (y \subseteq x) \rightarrow (x=y))$.
            \item $\forall x \forall y \forall z (((x \subseteq y) \land (y \subseteq z)) \rightarrow (x \subseteq z))$.
        \end{enumerate}
\end{prop}

\begin{proof}
    \leavevmode
        \begin{enumerate}[leftmargin=*, align=left, label=\textbf{(\alph*)}]
            \item Trivialmente, a implicação $(z \in x) \rightarrow (z \in x)$ é verdadeira para todo $z$. Logo, pela definição de inclusão $\subseteq$, temos $x \subseteq x$. \blackproof
            \item Se $x \subseteq y$ e $y \subseteq x$, então todo elemento de $x$ pertence a $y$ e todo elemento de $y$ pertence a $x$. Pelo axioma da extensão \eqref{ax:1}, $x=y$. \blackproof
            \item Suponha que $x \subseteq y$ e $y \subseteq z$. Seja $w \in x$. Como $x \subseteq y$, temos $w \in y$. Daí, como $y \subseteq z$, de $w \in y$ segue que $w \in z$. Assim, todo elemento de $x$ é elemento de $z$, isto é, $x \subseteq z$. \blackproof
        \end{enumerate}
\end{proof}

\subsection{O Axioma do Vazio}

\begin{defi}
    Um conjunto $x$ é \textit{vazio} se $\forall y (y \notin x)$.
\end{defi}

\begin{ax}[do Vazio] \label{ax:2}
    Existe um conjunto vazio.
        \[
            \boxed{
                \exists x \forall y (y \notin x)
            }
        \]
    Onde $(y \notin x) \overset{\text{def}}{\leftrightarrow} (\neg (y \in x))$.
\end{ax}

\begin{prop} \label{prop.fund:extensaounico1}
    Quaisquer dois conjuntos vazios são iguais.
        \[
            \forall x_1 \forall x_2 ( (\forall y (y \notin x_1) \land \forall y (y \notin x_2)) \rightarrow (x_1 = x_2) ).
        \]
\end{prop}

\begin{proof}
    Se $x_1 \neq x_2$, então ou existe $z \in x_1$ tal que $z \notin x_2$, ou existe $z \in x_2$ tal que $z \notin x_1$. Em ambos os casos, $x_1$ e $x_2$ não são vazios, uma contradição. Logo $x_1 = x_2$. \blackproof
\end{proof}

\begin{obs}
    O axioma do vazio \eqref{ax:2}, junto com a proposição \eqref{prop.fund:extensaounico1}, nos permite estabelecer que existe um único conjunto vazio:
        \[
            \exists x (\forall y  (y \notin x) \land \forall z(\forall y ( y \notin z ) \rightarrow (z=x) )).
        \]
    Podemos então falar \textit{do} conjunto vazio (em vez de \textit{de um}). Ele é denotado por $\emptyset$.
\end{obs}

\begin{prop}
    O conjunto vazio está contido em qualquer conjunto.
        \[
          \forall x (\emptyset \subseteq x)
        \]
     %\forall E \left( \forall y (y \notin E) \rightarrow \forall x (E \subseteq x) \right)$
\end{prop}

\begin{proof}
    Suponha que existe $x$ tal que $\emptyset \not\subseteq x$. Então existe $y \in \emptyset$ tal que $y \notin x$, uma contradição pois $\forall y (y \notin \emptyset)$. Logo $\forall x (\emptyset \subseteq x)$. \blackproof
\end{proof}

\begin{proof}
    Pela definição de $\subseteq$, precisamos provar que $\forall x (\forall y ( (y \in \emptyset) \rightarrow (y \in x) ) )$. Como a fórmula $y \in \emptyset$ é sempre falsa, $(y \in \emptyset) \rightarrow (y \in x)$ é sempre verdadeira, donde $\forall y ( (y \in \emptyset) \rightarrow (y \in x) )$ é sempre verdadeira, donde $\forall x (\forall y ( (y \in \emptyset) \rightarrow (y \in x) ) )$ é sempre verdadeira. Isto prova que a fórmula $\forall x (\emptyset \subseteq x)$ é sempre verdadeira. \blackproof
\end{proof}

\subsection{O Axioma do Par}

\begin{ax}[do Par] \label{ax:3}
    Para quaisquer conjuntos $x$ e $y$, existe um conjunto cujos elementos são $x$ e $y$.
        \[
            \boxed{
                \forall x \forall y \exists z \forall w ((w \in z) \leftrightarrow ((w = x) \lor (w = y)))
            }
        \]
\end{ax}

\begin{prop} \label{prop.fund:extensaounico2}
    O conjunto $z$ do axioma do par é único. Notação: $z := \{x,y\}$. %As chaves podem ser entendidas como um símbolo funcional binário (futuramente, $n$-ário), apesar de terem uma sintaxe um pouco diferente.
\end{prop}

\begin{proof}
    Pelo axioma do par, $z$ é tal que
        \[
            \forall w ((w \in z) \leftrightarrow ((w = x) \lor (w = y))).
        \]
    Se $z'$ é tal que
        \[
            \forall w ( (w \in z') \leftrightarrow ((w=x) \lor (w=y)) ),
        \]
    então $\forall w ((w \in z') \leftrightarrow (w \in z) )$, de modo que $z' = z$ pelo axioma da extensão. \blackproof
\end{proof}

\begin{prop}
    $\forall x \forall y((x \in y) \leftrightarrow \{x\} \subseteq y)$.
\end{prop}

\begin{proof}
    Por um lado ($\Rightarrow$), suponha que $x \in y$. Pela definição de inclusão, precisamos provar que $\forall z ((z \in \{x\}) \rightarrow (z \in y))$. Se $z \in \{x\}$, então $z = x$. Como $x \in y$ por hipótese e $z=x$, temos $z \in y$. Logo $\{x\} \subseteq y$. Por outro lado ($\Leftarrow$), suponha que $\{x\} \subseteq y$. Temos que $x \in \{x\}$. Como $\{x\} \subseteq y$, pela definição de inclusão, todo elemento de $\{x\}$ pertence a $y$. Logo $x \in y$. \blackproof
\end{proof}

\subsection{O Axioma da União}

\begin{ax}[da União] \label{ax:4}
    Para todo conjunto $x$ existe o conjunto de todos os conjuntos que pertencem a algum elemento de $x$.
        \[
            \boxed{
                \forall x \exists y \forall z ((z \in y) \leftrightarrow \exists w ( (z \in w) \land (w \in x)))
            }
        \]
\end{ax}

\begin{prop} \label{prop.fund:extensaounico3}
    O conjunto $y$ do axioma da união é único. Notação: $y := \bigcup x $.
\end{prop}

\begin{proof}
    Pelo axioma da união, $y$ é tal que
        \[
            \forall z ((z \in y) \leftrightarrow \exists w ( (z \in w) \land (w \in x))).
        \]
    Se $y'$ é tal que 
        \[
            \forall z ((z \in y') \leftrightarrow \exists w ( (z \in w) \land (w \in x))),
        \]
    então $\forall z ((z \in y') \leftrightarrow (z \in y)) $, donde $y' = y$ pelo axioma da extensão. \blackproof
\end{proof}

\begin{teo}
    Para quaisquer conjuntos $x$ e $y$, existe o conjunto dos conjuntos que pertencem a $x$ ou a $y$.
        \[
            \forall x \forall y \exists z \forall w ((w \in z) \leftrightarrow ((w \in x) \lor (w \in y)))
        \]
    Ademais, esse conjunto é único, sendo denotado por $x \cup y$.
\end{teo}

\begin{proof}
    Provemos que $z := \bigcup \{x,y\}$, que existe pelos axiomas do par e da união, funciona. De fato, para todo $w$, temos $w \in z$ se, e somente se, existe $u \in \{x,y \}$ tal que $w \in u$. Mas $u \in \{ x,y \}$ se, e somente se, $u = x$ ou $u=y$, de modo que $w \in x$ ou $w \in y$, como queríamos. A unicidade de $z$ segue do axioma da extensão, de modo que podemos denotar $x \cup y := z$. \blackproof
\end{proof}

\begin{proof}
    Uma prova alternativa é a seguinte. Precisamos provar que
        \[
            \forall w \left( \left(w \in \bigcup \{ x,y \}  \right) \leftrightarrow ((w \in x) \lor (w \in y))\right). \tag{$\lozenge$}
        \]
    Pelos axiomas do par e da união, temos
        \begin{align*}
            w \in \bigcup \{ x,y \} &\leftrightarrow \exists u ((u \in \{x,y \}) \land (w \in u) ) \\
            &\leftrightarrow \exists u ( ( (u=x) \lor (u=y) ) \land (w \in u) ) \\
            &\leftrightarrow \exists u ( ( (u=x) \land (w \in u) )  \lor ( (u=y) \land ( w \in u ) ) ) \\
            &\leftrightarrow \exists u ( (w \in x) \lor (w \in y) ) \\
            &\leftrightarrow (w \in x) \lor (w \in y).
        \end{align*}
    Com isso, temos $\lozenge$, como queríamos. \blackproof
\end{proof}

\subsection{O Axioma das Partes}

\begin{ax}[das Partes] \label{ax:5}
    Para todo conjunto $x$, existe o conjunto dos subconjuntos de $x$.
        \[
            \boxed{
                \forall x \exists y \forall z ( (z \in y) \leftrightarrow (z \subseteq x) )
            }
        \]
\end{ax}

\begin{prop} \label{prop.fund:extensaounico4}
    O conjunto $y$ do axioma das partes é único. Notação: $y:=\mathcal{P}(x)$.
\end{prop}

\begin{proof}
    Pelo axioma das partes, o conjunto $y$ cumpre $\forall z ( (z \in y) \leftrightarrow (z \subseteq x) )$. Se $y'$ cumpre $\forall z ( (z \in y') \leftrightarrow (z \subseteq x) )$, então $\forall z ((z \in y' )\leftrightarrow(z \in y))$, de modo que $y' = y$ pelo axioma da extensão. \blackproof
\end{proof}

\subsection{O Esquema de Axiomas da Separação}

\begin{ax}[da Separação] \label{ax:6}
    Para cada fórmula $P$ em que $z$ não ocorre livre, a fórmula
        \[
            \boxed{
                \forall y \exists z \forall x ( (x \in z) \leftrightarrow ((x \in y) \land P) )
            }    
        \]
    é um axioma.
\end{ax}

\begin{obs}
    O conjunto $y$ é o ``universo'' da discussão.  O axioma da separação também é chamado de axioma da compreensão ou axioma da especificação.
\end{obs}

\begin{prop}
    O conjunto $z$ do axioma da separação \eqref{ax:6} é único. Notação: $z:=\{x \in y : P(x) \}$.
\end{prop}

\begin{proof}
    Segue do axioma da extensão. \blackproof
\end{proof}

\begin{teo}[Paradoxo de Russell] Não existe o conjunto de todos os conjuntos.
        \[
            \forall x \exists y (y \notin x)
        \]
\end{teo}

\begin{proof}
    Suponha que $\exists x \forall y (y \in x)$. Pelo axioma da separação com universo $x$ e a fórmula $y \notin y$, existe $z$ tal que $\forall y ( (y \in z) \leftrightarrow ((y \in x) \land (y \notin y)))$ (note que $z$ não ocorre livre em $y \notin y$). Como $\forall y (y \in x)$, temos $\forall y ((y \in z) \leftrightarrow (y \notin y))$. Particularmente para $y = z$, temos $((z \in z) \leftrightarrow (z \notin z))$, uma contradição. Logo $\forall x \exists y (y \notin x)$. \blackproof
\end{proof}

\begin{teo} \label{teo.fund:intfamilia}
    Para todo conjunto $x \neq \emptyset$ existe o conjunto de todos os conjuntos que pertencem simultaneamente a todos os elementos de $x$.
        \[
            \forall x ((x \neq \emptyset) \rightarrow \exists y \forall z ((z \in y) \leftrightarrow \forall w ( (w \in x) \rightarrow (z \in w))))
        \]
    Ademais, esse conjunto é único, sendo denotado por $\bigcap x$.
\end{teo}

\begin{proof}
    Precisamos provar que existe $y$ tal que
        \[
           \forall z ( (z \in y) \leftrightarrow \forall w ( (w \in x) \rightarrow (z \in w)) ). \tag{$\lozenge$}
        \]
    Observe inicialmente que o axioma da separação pode ser escrito como
        \[
            \forall x \exists y \forall z  ( (z \in y) \leftrightarrow ( (z \in x) \land P ) ),
        \]
    onde $P$ é uma fórmula em que $y$ não ocorre livre. Agora, como $x \neq \emptyset$, tome $v \in x$. Pelo axioma da separação com universo $v$ e a fórmula $\forall w ( (w \in x) \rightarrow (z \in w))$, onde $y$ não ocorre livre, existe $y$ tal que
        \[
            \forall z ( (z \in y) \leftrightarrow ( (z \in v) \land  (\forall w ( (w \in x) \rightarrow (z \in w))) ) ),
        \]
    isto é, existe 
        \[
            y := \{ z \in v : \forall w ( (w \in x) \rightarrow (z \in w)) \}.
        \]
    Afirmamos que vale  ($\lozenge$) nesse $y$. De fato,
        \begin{itemize}
            \item por um lado ($\Rightarrow$), se $z \in y$, então trivialmente $\forall w ((w \in x) \rightarrow (z \in w))$;
            \item por outro lado ($\Leftarrow$), se $z$ é tal que $\forall w ((w \in x) \rightarrow (z \in w))$, então, particularmente para $w = v$, temos $v \in x \rightarrow z \in v$, e como $v \in x$, temos $z \in v$. Como $z \in v$ e $\forall w ((w \in x) \rightarrow (z \in w))$, temos que $z \in y$.
        \end{itemize}
    Logo, existe $y$ tal que $\lozenge$. A unicidade de $y$ segue do axioma da extensão, de modo que podemos denotar $\bigcap x := y$.  \blackproof
\end{proof}

\begin{defi} \label{defi.fund:capminus}
    Sejam $x$ e $y$ conjuntos.
        \begin{enumerate}[leftmargin=*, align=left, label=\textbf{(\alph*)}]
            \item A \textit{interseção} entre $x$ e $y$ é definida como
                \[
                    x \cap y :=  \{ z \in x : z \in y\}.
                \]
            Dizemos que $x$ e $y$ são \textit{disjuntos} se $x \cap y = \emptyset$. 
            \item A \textit{diferença} entre $x$ e $y$ é definida como
                \[
                    x \setminus y := \{z \in x : z \notin y \}.
                \]
            Dizemos que $x \setminus y$ é o \textit{complementar} de $y$ relativo a $x$ se $y \subseteq x$. Isso é denotado por $y^{C} := x \setminus y$.
            \item A \textit{diferença simétrica} entre $x$ e $y$ é definida como
                \[
                    x \Delta y := \{ z \in x \cup y :z \notin x \cap y \}.
                \]
        \end{enumerate}
\end{defi}

\begin{obs}
    As definições \eqref{defi.fund:capminus} se dão pelo axioma da separação. Vejamos como isso é feito, por exemplo, na definição de $x \cap y$. Sendo $x$ o universo, o axioma da separação é a fórmula $\forall x \exists w \forall z ((z \in w) \leftrightarrow ((z \in x) \land P) )$, onde $P$ é uma fórmula em que $w$ não ocorre livre. Se $P$ é a fórmula $z \in y$, então existe $w$ que cumpre $\forall z ( (z \in w) \leftrightarrow (z \in x) \land (x \in y) )$, isto é, $w = \{z \in x : z \in y \}$. Denotamos esse $w$ por $x \cap y$.
\end{obs}

\subsection{Propriedades Algébricas}

\begin{prop}[Propriedades da União]
    \leavevmode
        \begin{enumerate}[leftmargin=*, align=left, label=\textbf{(\alph*)}]
            \item $\forall x (x \cup x = x)$.
            \item $\forall x (x \cup \emptyset = x)$.
            \item $\forall x \forall y (x \cup y = y \cup x)$.
            \item $\forall x \forall y \forall z (x \cup (y \cup z) = (x \cup y) \cup z)$.
            \item $\forall x \forall y (x \cup y = y \leftrightarrow x \subseteq y)$.
            \item $\forall x \forall y ( (x \subseteq x \cup y) \land (y \subseteq x \cup y))$.
            \item $\forall x \forall y \forall z (x \subseteq y \rightarrow x \cup z \subseteq y \cup z )$.
            \item $\forall x \forall y (x \subseteq y \rightarrow \bigcup x \subseteq \bigcup y)$.
        \end{enumerate}
\end{prop}

\begin{proof}
    Trivial. \blackproof
\end{proof}

\begin{prop}[Propriedades da Interseção]
    \leavevmode
        \begin{enumerate}[leftmargin=*, align=left, label=\textbf{(\alph*)}]
            \item $\forall x (x \cap x = x)$.
            \item $\forall x (x \cap \emptyset = \emptyset)$.
            \item $\forall x \forall y (x \cap y = y \cap x)$.
            \item $\forall x \forall y \forall z (x \cap (y \cap z) = (x \cap y) \cap z)$.
            \item $\forall x \forall y (x \cap y = x \leftrightarrow x \subseteq y)$.
            \item $\forall x \forall y ( (x \cap y \subseteq x) \land (x \cap y \subseteq y))$.
            \item $\forall x \forall y \forall z (x \subseteq y \rightarrow x \cap z \subseteq y \cap z)$.
            \item $\forall x \forall y (x \subseteq y \land x \neq \emptyset \rightarrow \bigcap y \subseteq \bigcap x)$.
        \end{enumerate}
\end{prop}

\begin{proof}
    Trivial. \blackproof
\end{proof}

\begin{prop}[Distributividade]
    \leavevmode
        \begin{enumerate}[leftmargin=*, align=left, label=\textbf{(\alph*)}]
            \item $\forall x \forall y \forall z (x \cap (y \cup z) = (x \cap y) \cup (x \cap z))$.
            \item $\forall x \forall y \forall z (x \cup (y \cap z) = (x \cup y) \cap (x \cup z))$.
        \end{enumerate}
\end{prop}

\begin{proof}
    Trivial. \blackproof
\end{proof}

\begin{prop}[Propriedades da Diferença]
    \leavevmode
        \begin{enumerate}[leftmargin=*, align=left, label=\textbf{(\alph*)}]
            \item (Imediatas).
                \begin{enumerate}[label=\roman*.]
                    \item $\forall x (x \setminus \emptyset = x)$;
                    \item $\forall x (x \setminus x = \emptyset)$;
                    \item $\forall x (\emptyset \setminus x = \emptyset)$.
                \end{enumerate}
            \item \leavevmode
                \begin{enumerate}[label=\roman*.]
                    \item $\forall x \forall y (x \setminus y = x \leftrightarrow x \cap y = \emptyset)$;
                    \item $\forall x \forall y (x \setminus y = \emptyset \leftrightarrow x \subseteq y)$.
                \end{enumerate}
            \item (Leis de De Morgan).
                \begin{enumerate}[label=\roman*.]
                    \item $\forall x \forall y \forall z ( x \setminus (y \cup z) = (x \setminus y) \cap (x \setminus z) )$;
                    \item $\forall x \forall y \forall z ( x \setminus (y \cap z) = (x \setminus y) \cup (x \setminus z) )$.
                \end{enumerate}
            \item $\forall x \forall y \forall z ( x \setminus (y \setminus z) = (x \setminus y) \cup (x \cap z) )$.
            \item (Diferenças entre interseções).
                \begin{enumerate}[label=\roman*.]
                    \item $\forall x \forall y \forall z ( x \cap (y \setminus z) = (x \cap y) \setminus (x \cap z) )$;
                    \item $\forall x \forall y \forall z ( (x \setminus y) \cap z = (x \cap z) \setminus y )$.
                \end{enumerate}
            \item (Monotonocidade da diferença).
                \begin{enumerate}[label=\roman*.]
                    \item $\forall x \forall y \forall z (x \subseteq y \rightarrow x \setminus z \subseteq y \setminus z)$.
                    \item $\forall x \forall y \forall z (y \subseteq z \rightarrow x \setminus z \subseteq x \setminus y)$.
                \end{enumerate}
        \end{enumerate}
\end{prop}

\begin{proof}
    Trivial. \blackproof
\end{proof}

\begin{prop}[Propriedades do Complemento Relativo]
    \leavevmode
        \begin{enumerate}[leftmargin=*, align=left, label=\textbf{(\alph*)}]
            \item $\forall x \forall y (y \subseteq x \rightarrow (y^C)^C = y)$.
            \item $\forall x \forall y (y \subseteq x \rightarrow (y \cup y^C = x) \land (y \cap y^C = \emptyset))$.
            \item $\forall x \forall y \forall z (z \subseteq y \subseteq x \rightarrow y^C \subseteq z^C)$.
            \item $\forall x \forall y \forall z (y \subseteq x \land z \subseteq x \rightarrow y \setminus z = y \cap z^C)$.
            \item (Leis de De Morgan para complementos).
                \begin{enumerate}[label=\roman*.]
                    \item $\forall x \forall y \forall z (y \subseteq x \land z \subseteq x \rightarrow (y \cup z)^C = y^C \cap z^C)$.
                    \item $\forall x \forall y \forall z (y \subseteq x \land z \subseteq x \rightarrow (y \cap z)^C = y^C \cup z^C)$.
                \end{enumerate}
        \end{enumerate}
\end{prop}

\begin{proof}
    Trivial. \blackproof
\end{proof}




\begin{prop}
    \leavevmode
        \begin{enumerate}[leftmargin=*, align=left, label=\textbf{(\alph*)}]
            \item $\forall x \forall y \forall z ( ((x \in y) \land (y \in z)) \rightarrow ( ( x \in \bigcup z) \land (y \subseteq \bigcup z)))$.
        \end{enumerate}
\end{prop}

\begin{prop}
    $\forall A (\bigcup \mathcal{P}{(A)} = A)$.
\end{prop}



\subsection{O Axioma da Regularidade}

\begin{ax}[da Regularidade]
    Para todo conjunto $x \neq \emptyset$ existe $y \in x$ tal que $x \cap y = \emptyset$.
        \[
            \boxed{
                \forall x ((x \neq \emptyset) \rightarrow \exists y ( (y \in x) \land (x \cap y = \emptyset)))
            }
        \]
\end{ax}

\begin{prop} \label{prop.fund:xinyeyinxabs}
    Não existem conjuntos $x$ e $y$ tais que $x \in y$ e $y \in x$.
        \[
            \neg (\exists x \exists y ((x \in y) \land (y \in x)))
        \]
\end{prop}

\begin{proof}
    Basta provar que $\forall x \forall y ((x \notin y) \lor (y \notin x))$. Pelo axioma do par, tome $z := \{ x,y \}$. Como $z \neq \emptyset$, pelo axioma da regularidade existe $w \in z$ tal que $w \cap z = \emptyset$. Se $w = x$, então $y \notin x$, porque se fosse $y \in x$ teríamos $x \cap z = \{ y\} \neq \emptyset$, uma contradição. Analogamente, se $w = y$, então $x \notin y$. \blackproof
\end{proof}

\begin{cor} \label{cor.fund:xnotinx}
    Não existe $x$ tal que $x \in x$.
        \[
            \forall x(x \notin x)
        \]
\end{cor}

\begin{proof}
    Segue do teorema anterior com $y = x$. \blackproof
\end{proof}

\section{Relações}

\subsection{Produto Cartesiano}

\begin{defi}[Par ordenado]
    Sejam $a$ e $b$ conjuntos. O \textit{par ordenado} $(a,b)$ é definido como o conjunto $\{ \{ a \}, \{a,b \} \}$, isto é,
        \[
            (a,b) := \{ \{ a \}, \{a,b \} \}.
        \]
\end{defi}

\begin{prop}
    Dois pares ordenados $(a,b)$ e $(c,d)$ são iguais se, e somente se, $a = c$ e $b=d$.
        \[
            \forall a \forall b \forall c \forall d ( ((a,b) = (c,d)) \leftrightarrow ((a=c) \land (b=d)))
        \]
\end{prop}

\begin{proof}
    Ver \cite{fajardo2024conjuntos}, teorema 4.2, página 95. Ver \cite{hercules}, teorema 4.2, página 50. \blackproof
\end{proof}

\begin{teo} \label{teo.fund:prodcart}
    Para quaisquer conjuntos $A$ e $B$ existe o conjunto de todos os pares ordenados $(a,b)$ tais que $a \in A$ e $b \in B$.
        \[
            \forall A \forall B \exists C \forall x (x \in C \leftrightarrow \exists a \exists b (a \in A \land b \in B \land x = (a,b))) 
        \]
    Ademais, esse conjunto é único, sendo denotado por $A \times B$.
\end{teo}

\begin{proof}
    Pelos axiomas da união e das partes, considere o conjunto $\mathcal{P}{(\mathcal{P}{(A \cup B)})}$; pelo axioma da separação, considere o conjunto
        \[
            C := \{x \in \mathcal{P}{(\mathcal{P}{(A \cup B)})} : \exists a \exists b (a \in A \land b \in B \land x = (a,b)) \}.
        \]
    Afirmamos que $C$ cumpre as condições do enunciado. Se $x \in C$, então pela definição de $C$ existem $a \in A$ e $b \in B$ tais que $x = (a,b)$. Provemos então que se existem $a \in A$ e $b \in B$ tais que $x = (a,b)$, então $x \in C$. Para isso, basta provar que $x \in \mathcal{P}(\mathcal{P}(A \cup B))$. Qualquer que seja o par ordenado $(a,b)$, onde $a \in A$ e $b \in B$,
        \begin{align*}
            (a,b) \in \mathcal{P}{(\mathcal{P}{(A \cup B)})} &\leftrightarrow
            \{ \{ a\}, \{ a,b\} \} \in \mathcal{P}{(\mathcal{P}{(A \cup B)})} \\ &\leftrightarrow \{ \{ a\}, \{ a,b\} \} \subseteq \mathcal{P}{(A \cup B)} \\
            &\leftrightarrow \{a\} \in \mathcal{P}{(A \cup B)} \land \{ a,b\} \in \mathcal{P}{(A \cup B)} \\
            &\leftrightarrow \{a\} \subseteq (A \cup B) \land \{ a,b\} \subseteq (A \cup B),
        \end{align*}
    o que sabemos ser verdade. A unicidade de $C$ segue do axioma da extensão, de modo que podemos denotar $A \times B := C$. \blackproof
\end{proof}

\begin{defi}
    O \textit{produto cartesiano} dos conjuntos $A$ e $B$ é definido como $A \times B$.
\end{defi}

Defina $\pi_1 : A \times B \to A$ e $\pi_2 : A \times B \to B$ por $\pi_1(a,b) = a$ e $\pi_2(a,b) = b$ para quaisquer $(a,b) \in A \times B$.

\subsection{Relações}

\begin{defi}
    \leavevmode
        \begin{enumerate}[leftmargin=*, align=left, label=\textbf{(\alph*)}]
            \item Uma \textit{relação binária}, ou simplesmente uma \textit{relação}, é um conjunto de pares ordenados.
            \item Um símbolo de predicado para ``$R$ é relação'', onde $R$ ocorre livre, é
                \[
                    \operatorname{Rel}(R) \overset{\text{def}}{\leftrightarrow} \forall x \left(x \in R \rightarrow \exists a \exists b \left(x = (a,b)\right)\right).
                \]
            Denotamos $(a,b) \in R$ por $aRb$.
            %\forall x \left(x \in R \rightarrow \exists a \exists b \forall y (y \in x \leftrightarrow (\forall z (z \in y \leftrightarrow z = a) \lor \forall z (z \in y \leftrightarrow (z = a \lor z = b) ) ) )\right).
            \item Dizemos que $R$ é uma relação de $A$ em $B$ se $R \subseteq A \times B$. Dizemos que $R$ é uma relação em $A$ se $R \subseteq A \times A$.
        \end{enumerate}
\end{defi}

\begin{teo}
    Um conjunto $R$ é uma relação se, e somente se, existem conjuntos $A$ e $B$ tais que $R \subseteq A \times B$.
        \[
            \forall R (\operatorname{Rel}{(R)} \leftrightarrow \exists A \exists B(R \subseteq A \times B))
        \]
\end{teo}

\begin{proof}
     ($\Rightarrow$) Sendo $R$ uma relação, usando os axiomas da união e da separação, defina
        \begin{align*}
             A &:= \left\{ a \in \bigcup \bigcup R : \exists b \left(b \in \bigcup \bigcup R \land aRb \right) \right\} \\
             B &:= \left\{ b \in \bigcup \bigcup R : \exists a \left(a \in \bigcup \bigcup R \land aRb \right) \right\}
        \end{align*}
    Seja $x \in R$. Então existem $a$ e $b$ tais que $x = (a,b)$. Se $\{ \{a \}, \{a,b \} \} \in R$, então $\{ \{a \}, \{a,b \} \} \subseteq \bigcup R$, donde $\{a,b \} \in \bigcup R$, donde $\{ a,b \} \subseteq \bigcup \bigcup R$, donde $a,b \in \bigcup \bigcup R$. Como $aRb$, pelas definições de $A$ e $B$, temos $a \in A$ e $b \in B$. Como $x = (a,b)$ e $a \in A$ e $b \in B$, pelo teorema \eqref{teo.fund:prodcart} temos $x \in A \times B$, donde, por fim, segue que $R \subseteq A \times B$.

    ($\Leftarrow$) Os elementos de $A \times B$ são pares ordenados; logo, qualquer subconjunto de $A \times B$ terá pares ordenados como elementos. \blackproof
\end{proof}

\begin{defi}
    Seja $R$ uma relação.
        \begin{enumerate}[leftmargin=*, align=left, label=\textbf{(\alph*)}]
            \item O \textit{domínio} de $R$ é definido como
                \[
                    \Dom{(R)} := \left\{ a \in \bigcup \bigcup R : \exists b ((a,b) \in R) \right\}.
                \]
            \item A \textit{imagem} de $R$ é definida como
                \[
                    \Im{(R)} := \left\{ b \in \bigcup \bigcup R : \exists a ((a,b) \in R) \right\}.
                \]
            \item A \textit{relação inversa} de $R$ é definida como
                \[
                    R^{-1} := \{ (b,a) \in \Im{(R)} \times \Dom{(R)} : (a,b) \in R \}.
                \]
            \item A \textit{imagem de um conjunto $X$ por $R$} é definida como
                \[
                    R[X] := \{ b \in \bigcup \bigcup R : \exists a (a \in X \land (a,b) \in R) \}.
                \]
            \item A \textit{imagem inversa de um conjunto $Y$ por $R$} é definida como
                \[
                    R^{-1}[Y] := \left\{ a \in \bigcup \bigcup R : \exists b (b \in Y \land (a,b) \in R) \right\}
                \]
            Equivalentemente, $R^{-1}[Y]$ é a imagem de $Y$ pela relação $R^{-1}$. 
            \item A \textit{restrição de $R$ a $X$} é definida como
                \[
                    R \restriction_X := \{ (a,b) \in R : a \in X \}.
                \]
            \item A \textit{composição de $R$ e $S$} é definida como 
                \[
                    S \circ R := \{ (a,c) \in \Dom{(R)} \times \Im{(S)} : \exists b (b \in \Im{(R)} \cap \Dom{(S)} \land (aRb \land bSc)) \}.
                \]
        \end{enumerate}
\end{defi}

\begin{prop} \label{prop.fund:domeimdeRAB}
    Sejam $R$ e $S$ relações e $A$, $B$, $C$ e $D$ conjuntos tais que $R \subseteq A \times B$ e $S \subseteq C \times D$.
        \begin{enumerate}[leftmargin=*, align=left, label=\textbf{(\alph*)}]
            \item $\Dom{(R)} = \{x \in A : \exists y (y \in B \land xRy)\}$.
            \item $\Im{(R)} = \{ y \in B : \exists x (x \in A \land xRy)\}$.
            \item $S \circ R \subseteq A \times D$.
        \end{enumerate}
\end{prop}

\begin{proof}
    \leavevmode
        \begin{enumerate}[leftmargin=*, align=left, label=\textbf{(\alph*)}]
            \item Se $x \in \Dom{(R)}$, então existe $y \in \Im{(R)}$ tal que $(x,y) \in R$. Como $R \subseteq A \times B$, temos $(x,y) \in A \times B$, donde $x \in A$ e $y \in B$. Assim, $x \in A$ e existe $y \in B$ tal que $xRy$, o que prova a inclusão $\subseteq$. Reciprocamente, se $x \in A$ e existe $y \in B$ tal que $(x,y) \in R$, então $x \in \Dom{(R)}$, o que prova a inclusão $\supseteq$. \blackproof
            \item Se $y \in \Im{(R)}$, então existe $x \in \Dom{(R)}$ tal que $(x,y) \in R$. Como $R \subseteq A \times B$, temos $(x,y) \in A \times B$, donde $x \in A$ e $y \in B$. Assim, $y \in B$ e existe $x \in A$ tal que $xRy$, o que prova a inclusão $\subseteq$. Reciprocamente, se $y \in B$ e existe $x \in A$ tal que $(x,y) \in R$, então $y \in \Im{(R)}$, o que prova a inclusão $\supseteq$. \blackproof
            \item Se $(x,z) \in S \circ R$, então $x \in \Dom{(R)}$ e $z \in \Im{(S)}$. Como $\Dom{(R)} \subseteq A$ (primeiro item) e $\Im{(S)} \subseteq D$ (segundo item), temos $x \in A$ e $z \in D$, de modo que $(x,z) \in A \times D$. Com isso, $S \circ R \subseteq A \times D$. \blackproof
        \end{enumerate}
\end{proof}

\begin{prop} \label{prop.fund:relacoes}
    Sejam $R$, $S$ e $T$ relações. Valem as seguintes afirmações.
        \begin{enumerate}[leftmargin=*, align=left, label=\textbf{(\alph*)}]
            \item 
                \begin{enumerate}[label=\roman*.]
                    \item $(R^{-1})^{-1} = R$.
                    \item $\Dom{R^{-1}} = \Im{(R)}$.
                    \item $\Im{(R^{-1})} = \Dom{(R)}$
                \end{enumerate}
            \item $T \circ (S \circ R) = (T \circ S) \circ R$.
            \item $(S \circ R)^{-1} = R^{-1} \circ S^{-1}$.
            \item Se $\Im{(R)} \subseteq \Dom{(S)}$, então $\Dom{(S \circ R)} = \Dom{(R)}$.
            \item Se $\Dom{(S)} \subseteq \Im{(R)}$, então $\Im{(S \circ R)} = \Im{(S)}$.
        \end{enumerate}
\end{prop}

\begin{proof}
    \leavevmode
        \begin{enumerate}[leftmargin=*, align=left, label=\textbf{(\alph*)}]
            \item 
                \begin{enumerate}[label=\roman*.]
                    \item Temos $(a,b) \in R$ se, e somente se, $(b,a) \in R^{-1}$, o que é equivalente a $(a,b) \in (R^{-1})^{-1}$. Logo $R = (R^{-1})^{-1}$.
                    \item 
                    \item 
                \end{enumerate}
            \item Se $(a,d) \in T \circ (S \circ R)$, então $a \in \Dom{(S \circ R)}$, $d \in \Im{(T)}$ e existe $c \in \Im{(S \circ R)} \cap \Dom{(T)}$ tal que $(a,c) \in S \circ R$ e $(c,d) \in T$. De $(a,c) \in S \circ R$ segue que $a \in \Dom{(R)}$, $c \in \Im{(S)}$ e existe $b \in \Im{(R)} \cap \Dom{(S)}$ tal que $(a,b) \in R$ e $(b,c) \in S$. Como $b \in \Dom{(S)}$, $d \in \Im{(T)}$ e existe $c \in \Dom{(T)} \cap \Im{(S)}$ tal que $(b,c) \in S$ e $(c,d) \in T$, temos que $(b,d) \in T \circ S$. Com isso, $b \in \Dom{(T \circ S)}$ e $d \in \Im{(T \circ S)}$. Como $a \in \Dom{(R)}$, $d \in \Im{(T \circ S)}$ e existe $b \in \Dom{(T \circ S)} \cap \Im{(R)}$ tal que $(a,b) \in R$ e $(b,d) \in T \circ S$, temos que $(a,d) \in (T \circ S) \circ R$. Com isso, $T \circ (S \circ R) \subseteq (T \circ S) \circ R$. A prova de que $(T \circ S) \circ R \subseteq T \circ (S \circ R)$ é completamente análoga, de modo que $T \circ (S \circ R) = (T \circ S) \circ R$. \blackproof
            \item Se $(c,a) \in (S \circ R)^{-1}$, então $(a,c) \in S \circ R$, $a \in \Dom{(R)}$, $c \in \Im{(S)}$ e existe $b \in \Im{(R)} \cap \Dom{(S)}$ tal que $(a,b) \in R$ e $(b,c) \in S$ isto é, $(c,b) \in S^{-1}$, $(b,a) \in R^{-1}$, com $c \in \Dom{(S^{-1})}$, $a \in \Im{(R^{-1})}$ e $b \in \Im{(S^{-1})} \cap \Dom{(R^{-1})}$. Com isso, $(c,a) \in R^{-1} \circ S^{-1}$, de modo que $(S \circ R)^{-1} \subseteq R^{-1} \circ S^{-1}$. A prova de que $R^{-1} \circ S^{-1} \subseteq (S \circ R)^{-1}$ é completamente análoga, de modo que $(S \circ R)^{-1} = R^{-1} \circ S^{-1}$. \blackproof
            \item Se $a \in \Dom{(R)}$, então existe $b \in \Im{(R)}$ tal que $(a,b) \in R$. Se $\Im{(R)} \subseteq \Dom{(S)}$, então $b \in \Dom{(S)}$, de modo que existe $c \in \Im{(S)}$ tal que $(b,c) \in S$. Assim, $(a,c) \in S \circ R$, de modo que $a \in \Dom{(S \circ R)}$. Com isso, $\Dom{(R)} \subseteq \Dom{(S \circ R)}$. Agora, se $a \in \Dom{(S \circ R)}$, então existe $c \in \Im{(S \circ R)}$ tal que $(a,c) \in S \circ R$, donde $a \in \Dom{(R)}$ e $\Dom{(S \circ R)} \subseteq \Dom{(R)}$. Com isso, $\Dom{(R)} = \Dom{(S \circ R)}$. \blackproof
            \item Se $c \in \Im{(S)}$, então existe $b \in \Dom{(S)}$ tal que $(b,c) \in S$. Se $\Dom{(S)} \subseteq \Im{(R)}$, então $b \in \Im{(R)}$, de modo que existe $a \in \Dom{(R)}$ tal que $(a,b) \in R$. Assim, $(a,c) \in S \circ R$, de modo que $c \in \Im{(S \circ R)}$. Com isso, $\Im{(S)} \subseteq \Im{(S \circ R)}$. Agora, se $c \in \Im{(S \circ R)}$, então existe $a \in \Dom{(S \circ R)}$ tal que $(a,c) \in S \circ R$, donde $c \in \Im{(S)}$ e $\Im{(S \circ R)} \subseteq \Im{(S)}$. Com isso, $\Im{(S)} = \Im{(S \circ R)}$. \blackproof
        \end{enumerate}
\end{proof}

\subsection{Relações de Ordem}

\begin{defi}
    Seja $R$ uma relação em $X$.
        \begin{enumerate}[leftmargin=*, align=left, label=\textbf{(\alph*)}]
            \item Dizemos que $R$ é \textit{reflexiva} se
                \[
                    \forall x (x \in X \rightarrow (x,x) \in R).
                \]
            \item Dizemos que $R$ é \textit{irreflexiva} se
                \[
                    \forall x (x \in X \rightarrow (x,x) \notin R).
                \]
            \item Dizemos que $R$ é \textit{simétrica} se
                \[
                    \forall x \forall y (x,y \in X \rightarrow (xRy \rightarrow yRx)).
                \]
            \item Dizemos que $R$ é \textit{antissimétrica} se
                \[
                    \forall x \forall y (x,y \in X \rightarrow (xRy \land yRx \rightarrow x=y)).
                \]
            \item Dizemos que $R$ é \textit{transitiva} se
                \[
                    \forall x \forall y \forall z (x,y,z \in X \rightarrow (xRy \land yRz \rightarrow xRz)).
                \]
        \end{enumerate}
\end{defi}

\begin{defi}[Ordem parcial] \label{defi.fund:ordemparcial}
    \leavevmode
        \begin{enumerate}[leftmargin=*, align=left, label=\textbf{(\alph*)}]
            \item Uma \textit{relação de ordem parcial} em $X$ é uma relação $\leq  \, \, \subseteq X \times X$ que tem as seguintes propriedades.
                \begin{enumerate}[label=\roman*.]
                    \item Reflexividade: $\forall x(x \in X \rightarrow x \leq x)$;
                    \item Antissimetria: $\forall x \forall y (x,y \in X \rightarrow (x \leq y \land y \leq x \rightarrow x=y))$;
                    \item Transitividade: $\forall x \forall y \forall z (x,y,z \in X \rightarrow (x \leq y \land y \leq z \rightarrow x \leq z))$. 
                \end{enumerate}
            Dizemos que $X$ é o \textit{domínio} de $\leq$.
            \item Um \textit{conjunto parcialmente ordenado} é um par $(X, \leq)$ onde $\leq  \, \, \subseteq X \times X$ é uma relação de ordem parcial.
        \end{enumerate}
\end{defi}

\begin{nota}
    Sendo $\leq$ uma ordem parcial, abreviaremos $y \leq x$ por $x \geq y$, $x \leq y$ e $x \neq y$ por $x < y$ e $x<y$ por $y>x$. Quando não houver perigo de confusão, podemos escrever somente \textit{ordem} em vez de ordem parcial.
\end{nota}

\begin{ex}
    A relação de inclusão $\subseteq$ é uma relação de ordem parcial \eqref{prop.fund:inclusaoparcial}.
\end{ex}

\begin{defi}
    Dois conjuntos parcialmente ordenados $(X_1, \leq_1)$ e $(X_2, \leq_2)$ são \textit{ordem-isomorfos} se existe uma bijeção $f:X_1 \to X_2$ tal que 
        \[
          \forall x \forall y (x,y \in X_1 \rightarrow (x \leq_1 y \leftrightarrow f(x) \leq_2 f(y))).  
        \]
    Dizemos que a função $f$ é um \textit{isomorfismo de ordens parciais}.
\end{defi}

\begin{teo}
    Se $(X, \leq)$ é um conjunto ordenado, então existe um conjunto ordenado $(Y, \preceq)$ ordem-isomorfo a $(X,\leq)$ tal que 
        \[
            \preceq \, \, = \{ (x,y) \in Y \times Y : x \subseteq y \}.
        \]
\end{teo}

\begin{proof}
    Definindo $f : X \to \mathcal{P}{(X)}$ por $f(x) := \{y \in X : y \leq x \}$, temos que $f$ é bijetiva em relação a $Y := \Im{(f)}$. De fato, se $f(x) = f(y)$, então $x \in f(y)$ e $y \in f(x)$ já que $x \in f(x)$ e $y \in f(y)$; daí, pela definição de $f$, vem $x \leq y$ e $y \leq x$, de modo que $x = y$ e $f$ é injetiva. Provemos então que $x \leq y$ se, e somente se, $f(x) \subseteq f(y)$, para quaisquer $x,y \in X$. Se $z \in f(x)$, então $z \leq x$, e se $x \leq y$, então $z \leq y$, de modo que $z \in f(y)$ e $f(x) \subseteq f(y)$. Agora, se $x \in f(x) \subseteq f(y)$, então $x \in f(y)$, donde $x \leq y$. Com isso, $(X, \leq)$ é ordem-isomorfo a $(Y, \subseteq)$, como havíamos afirmado. \blackproof
\end{proof}

\begin{defi}
    Sejam $(X, \leq)$ um conjunto parcialmente ordenado e $S$ um subconjunto não vazio de $X$.
        \begin{enumerate}[leftmargin=*, align=left, label=\textbf{(\alph*)}]
            \item Dizemos que $M \in X$ é
                \begin{enumerate}[label=\roman*.]
                    \item uma \textit{cota superior} de $S$ se
                        \[
                            \forall x (x \in S \rightarrow x \leq M).
                        \]
                    Nesse caso, dizemos que $S$ é \textit{limitado superiormente} em $X$.
                    \item o \textit{elemento máximo} de $S$ se $M$ é uma cota superior de $S$ e $M \in S$. Isso é denotado por $\max{S} := M$.
                    \item o \textit{supremo} de $S$ se
                        \[
                            \forall x (x \in S \rightarrow x \leq M) \land \forall y (y \in X \land \forall x (x \in S \rightarrow x \leq y) \rightarrow M \leq y),
                        \]
                    isto é, se $M$ é a menor cota superior de $S$. Isso é denotado por $\sup{S} := M$.
                \end{enumerate}
            \item Dizemos que $m \in X$ é
                \begin{enumerate}[label=\roman*.]
                    \item uma \textit{cota inferior} de $S$ se
                        \[
                            \forall x (x \in S \rightarrow m \leq x).
                        \]
                    Nesse caso, dizemos que $S$ é \textit{limitado inferiormente} em $X$.
                    \item o \textit{elemento mínimo} de $S$ se $m$ é uma cota inferior de $S$ e $m \in S$. Isso é denotado por $\min{S} := m$.
                    \item o \textit{ínfimo} de $S$ se
                        \[
                            \forall x (x \in S \rightarrow m \leq x) \land \forall y (y \in X \land \forall x (x \in S \rightarrow m \leq y) \rightarrow y \leq m),
                        \]
                    isto é, se $m$ é a maior cota inferior de $S$. Isso é denotado por $\inf{S} := m$.
                \end{enumerate}
        \end{enumerate}
\end{defi}

\begin{obs}
    É fácil ver que o máximo de $S$, quando existe, é único. De fato, se $x, y \in S$ são máximos de $S$, então $x \leq y$ e $y \leq x$, de modo que $x = y$. Essa unicidade também vale para o mínimo, o ínfimo e o supremo de $S$, quando existem. Isso justifica o uso do artigo ``o'' (em \textit{o} máximo, em vez de \textit{um} máximo, por exemplo) e nos permite denotar esses elementos por $\max{S}$, $\min{S}$, $\inf{S}$ e $\sup{S}$, respectivamente.
\end{obs}

\begin{defi}
    Seja $(X, \leq)$ um conjunto parcialmente ordenado.
        \begin{enumerate}[leftmargin=*, align=left, label=\textbf{(\alph*)}]
            \item Dizemos que $\leq$ é uma relação de ordem \textit{total}\footnote{O termo \textit{ordem linear} também costuma ser usado. Nesse caso, dizemos que o par $(X,\leq)$ é um \textit{conjunto linearmente ordenado}.}, e que o par $(X,\leq)$ é um \textit{conjunto totalmente ordenado}, se 
                \[
                    \forall x \forall y (x,y \in X \rightarrow (x \leq y \lor y \leq x)).
                \]
            \item Dizemos que $\leq$ é uma \textit{boa ordem} em $X$, e que o par $(X,\leq)$ é um \textit{conjunto bem-ordenado}, se todo subconjunto não vazio de $X$ possui um elemento mínimo.
            \item Dizemos que $\leq$ é um \textit{reticulado} se para quaisquer $x,y \in X$, o conjunto $\{ x,y \}$ possui supremo e ínfimo.
            \item Dizemos que $\leq$ é uma \textit{árvore} se, para todo $x \in X$, o conjunto $S = \{ y \in X : y \leq x\}$ é tal que $(S, \leq \cap \ S^2)$ é um conjunto bem-ordenado.
        \end{enumerate}
\end{defi}

\begin{prop}
    \leavevmode
        \begin{enumerate}[leftmargin=*, align=left, label=\textbf{(\alph*)}]
            \item Toda boa ordem é uma ordem total.
            \item Toda boa ordem é uma árvore.
            \item Toda ordem total é um reticulado.
        \end{enumerate}
\end{prop}

\begin{proof}
    \leavevmode
        \begin{enumerate}[leftmargin=*, align=left, label=\textbf{(\alph*)}]
            \item Sejam $x, y \in X$. Como $\leq$ é uma boa ordem, todo subconjunto não vazio de $X$ possui elemento mínimo. Assim, o conjunto $\{x,y\} \subseteq X$ possui um elemento mínimo. Se $\min\{x,y\} = x$, então $x \leq y$. Se $\min\{x,y\} = y$, então $y \leq x$. Em qualquer caso, vale $x \leq y$ ou $y \leq x$, de modo que a ordem $\leq$ é total. \blackproof
            \item Pela definição de árvore, precisamos mostrar que, para todo $x \in X$, o conjunto $S = \{ y \in X : y \leq x\}$ é bem-ordenado pela ordem induzida. Seja $A \subseteq S$ um subconjunto não vazio. Como $S \subseteq X$, temos que $A \subseteq X$. Como $X$ é bem-ordenado, $A$ possui um elemento mínimo. Com isso, todo subconjunto não vazio de $S$ possui um elemento mínimo, isto é, $S$ é bem-ordenado. \blackproof
            \item Sejam $x, y \in X$. Como a ordem $\leq$ é total, temos $x \leq y$ ou $y \leq x$. Suponha, sem perda de generalidade, que $x \leq y$. Como $x \leq x$ e $x \leq y$, temos que $x$ é uma cota inferior de $\{x,y\}$. Se $z$ uma cota inferior qualquer de $\{x,y\}$, então, em particular, $z \leq x$. Logo, $x$ é a maior das cotas inferiores, isto é, $x = \inf\{x,y\}$. Como $y \geq x$ e $y \geq y$, temos que $y$ é uma cota superior de $\{x,y\}$. Se $w$ uma cota superior qualquer de $\{x,y \}$, então, em particular, $w \geq y$. Logo, $y$ é a menor das cotas superiores, isto é, $y = \sup\{x,y\}$. Portanto, $\{x,y\}$ possui supremo e ínfimo. \blackproof
        \end{enumerate}
\end{proof}

\begin{prop}
    Se $(X,\leq)$ é parcialmente ordenado e $Y \subseteq X$, então $(Y, \leq \cap \ Y^2)$ é parcialmente ordenado.
\end{prop}

\begin{proof}
    Defina $\leq_Y \ := \ \leq \cap \ Y^2$.
        \begin{enumerate}[label=\roman*.]
            \item (Reflexividade) Dado $y \in Y$, como $Y \subseteq X$, temos $y \in X$, donde $y \leq y$. Como $y \in Y$, vem $(y,y) \in \leq_Y$.
            \item (Antissimetria) Sejam $x, y \in Y$ tais que $(x,y) \in \leq_Y$ e $(y,x) \in \leq_Y$. Então $x \leq y$ e $y \leq x$. Pela antissimetria de $\leq$, vem $x=y$.
            \item (Transitividade) Sejam $x, y, z \in Y$ tais que $(x,y) \in \leq_Y$ e $(y,z) \in \leq_Y$. Então $x \leq y$ e $y \leq z$. Pela transitividade de $\leq$, vem $x \leq z$. Como $x, z \in Y$, vem $(x,z) \in \leq_Y$.
        \end{enumerate}
    Com isso, $\leq_Y \ := \ \leq \cap \ Y^2$ é uma ordem. \blackproof
\end{proof}

\begin{defi}
    Sejam $(X,\leq)$ parcialmente ordenado e $Y \subseteq X$. 
        \begin{enumerate}[leftmargin=*, align=left, label=\textbf{(\alph*)}]
            \item Dizemos que $\leq_Y \ := \ \leq \cap \ Y^2$ é uma \textit{subordem} de $\leq$.
            \item Dizemos que $(Y, \leq_Y)$ é um \textit{subconjunto parcialmente ordenado} de $(X, \leq)$. Isso é denotado por $(Y, \leq)$.
        \end{enumerate}
\end{defi}

\begin{prop}
    Sejam $(X,\leq)$ parcialmente ordenado e $Y \subseteq X$.
        \begin{enumerate}[leftmargin=*, align=left, label=\textbf{(\alph*)}]
            \item Se $\leq$ é uma ordem total, então $\leq_Y$ é uma ordem total.
            \item Se $\leq$ é uma boa ordem, então $\leq_Y$ é uma boa ordem.
            \item Se $\leq$ é uma árvore, então $\leq_Y$ é uma árvore.
        \end{enumerate}
\end{prop}

\begin{proof}
    \leavevmode
        \begin{enumerate}[leftmargin=*, align=left, label=\textbf{(\alph*)}]
            \item Sejam $x, y \in Y$. Como $Y \subseteq X$, temos $x, y \in X$, e como $\leq$ é total em $X$, temos $x \leq y$ ou $y \leq x$, de modo que $x \leq_Y y$ ou $y \leq_Y x$. \blackproof
            \item Seja $A \in \mathcal{P}(Y)_{\neq \emptyset}$. Como $Y \subseteq X$, temos $A \subseteq X$, e como $\leq$ é uma boa ordem em $X$, existe $\min{A} \in A$. Como para todo $z \in A \subseteq Y$ temos $\min{A} \leq z$, segue que $\min{A}$ é o mínimo de $A$ com relação a $\leq_Y$. \blackproof
            \item Sendo $x \in Y$ e $S := \{y \in Y : y \leq_Y x \}$, provemos que $(S, \leq_Y \cap \ S^2)$ é bem ordenado. Pondo $T := \{y \in X : y \leq x \}$, temos $S = T \cap Y$, e como $x \in X$ e $\leq$ é uma árvore, temos que $(T, \leq \cap \ T^2)$ é bem ordenado. Como $S \subseteq T$, temos que $(S, (\leq \cap \ T^2) \cap S^2 )$ é também uma boa ordem. Observando que
                \begin{align*}
                    \leq_Y \cap \ S^2 &= (\leq \cap \ Y^2) \cap (T \cap Y)^2 \\
                    &= \leq \cap \ Y^2 \cap T^2 \\
                    &= (\leq \cap \ T^2) \cap (T^2 \cap Y^2) \\
                    &= (\leq \cap \ T^2) \cap S^2, 
                \end{align*}
            segue que $(S, \leq_Y \cap \ S^2)$ é bem ordenado. \blackproof
        \end{enumerate}
\end{proof}

\subsection{Relações de Equivalência}

\begin{defi} (Relações de equivalência)
    \begin{enumerate}[leftmargin=*, align=left, label=\textbf{(\alph*)}]
        \item Uma \textit{relação de equivalência} em $X$ é uma relação $\sim \, \, \subseteq X \times X$ que tem as seguintes propriedades.
        \begin{enumerate}[label=\roman*.]
            \item Reflexividade: $\forall x(x \in X \rightarrow x \sim x)$;
            \item Simetria: $\forall x \forall y (x,y \in X \rightarrow (x \sim y \rightarrow y \sim x))$;
            \item Transitividade: $\forall x \forall y \forall z (x,y,z \in X \rightarrow (x \sim y \land y \sim z \rightarrow x \sim z))$.
        \end{enumerate}
        \item A \textit{classe de equivalência} de $a \in X$ por $\sim$ é definida como
            \[
                [a]_{\sim} := \{ x \in X : x \sim a \}. 
            \]
        \item O conjunto das classes de equivalência de $\sim$ é definido como
            \[
                X / \sim \ := \{ Y \in \mathcal{P} (X) : \exists x \forall y (y \in Y \leftrightarrow x \sim y) \} = \{ [x]_{\sim} : x \in X \}.
            \]
    \end{enumerate}
\end{defi}

\begin{prop}
    Seja $\sim$ uma relação de equivalência num conjunto $X$. As seguintes afirmações são equivalentes.
        \begin{enumerate}[leftmargin=*, align=left, label=\textbf{(\alph*)}]
            \item $a \sim b$.
            \item $a \in [b]$.
            \item $b \in [a]$.
            \item $[a] = [b]$.
        \end{enumerate}
\end{prop}

\begin{proof}
    \leavevmode
        \begin{enumerate}[leftmargin=*, align=left]
            \item[\textbf{(a)} $\Rightarrow$ \textbf{(b)}:] Por definição, $[b] = \{ x \in X : x \sim b \}$, e como $a \sim b$, segue $a \in [b]$. \blackproof
            \item[\textbf{(b)} $\Rightarrow$ \textbf{(c)}:] Se $a \in [b]$, então $a \sim b$, isto é, $b \in [a]$. \blackproof
            \item[\textbf{(c)} $\Rightarrow$ \textbf{(d)}:] Se $b \in [a]$, então $b \sim a$. Se $x \in [a]$, então $x \sim a$, de modo que $x \sim b$, isto é, $x \in [b]$. Com isso, $[a] \subseteq [b]$. Analogamente temos $[b] \subseteq [a]$, de modo que $[a] = [b]$. \blackproof
            \item[\textbf{(d)} $\Rightarrow$ \textbf{(a)}:] Se $a \in [a] = [b]$, então $a \sim b$. \blackproof  
        \end{enumerate}
\end{proof}

\begin{defi}
    Uma \textit{partição} de um conjunto $X \neq \emptyset$ é um subconjunto $\mathcal{P} \subseteq \mathcal{P}{(X)}$ que tem as seguintes propriedades.
        \begin{enumerate}[label=\roman*.]
            \item $\emptyset \notin \mathcal{P}$;
            \item $\bigcup \mathcal{P} = X$;
            \item $A \cap B = \emptyset$ para quaisquer $A, B \in \mathcal{P}$ tais que $A \neq B$.
        \end{enumerate}
    % tal que $\emptyset \notin \mathcal{P}$, $A \cap B = \emptyset$ para quaisquer $A, B \in \mathcal{P}$ tais que $A \neq B$ e $\bigcup \mathcal{P} = X$.
\end{defi}

\begin{teo}
    Se $\sim$ é uma relação de equivalência num conjunto $X$, então $X / \sim$ é uma partição de $X$, isto é, valem as seguintes afirmações.
        \begin{enumerate}[leftmargin=*, align=left, label=\textbf{(\alph*)}]
            \item $\emptyset \notin X / \sim$.
            \item $\bigcup X / \sim = X$.
            \item $\forall Y\forall Z((Y, Z \in X / \sim)\rightarrow (Y=Z \lor Y \cap Z = \emptyset))$.
        \end{enumerate}
\end{teo}

\begin{proof}
    \leavevmode
        \begin{enumerate}[leftmargin=*, align=left, label=\textbf{(\alph*)}]
            \item Se $\emptyset \in X/ \sim$, então existiria $x \in X$ tal que $\emptyset = [x]$, mas $x \in [x]$, uma contradição. \blackproof
            \item Se $y \in \bigcup X / \sim$, então existe $Y \in X / \sim$ tal que $y \in Y$. Como $Y \in X / \sim$, existe $x \in X$ tal que $Y = [x]$. Como $[x] \subseteq X$, vem $y \in X$, de modo que $\bigcup X / \sim \subseteq X$. Agora, se $x \in X$, então $x \sim x$ e $x \in [x]$, e como $[x] \in X / \sim$, vem $x \in \bigcup X / \sim$, de modo que $X \subseteq \bigcup X / \sim$. Logo $\bigcup X / \sim = X$. \blackproof
            \item Se $Y \cap Z = \emptyset$, nada há de ser provado. Se $Y \cap Z \neq \emptyset$, então existe $x \in X$ tal que $x \in Y \cap Z$. Sendo $y_0, z_0 \in X$ tais que $Y = [y_0]$ e $Z = [z_0]$, temos $x \sim y_0$ e $x \sim z_0$, de modo que $[y_0] = [z_0]$, isto é, $Y = Z$. \blackproof
        \end{enumerate}
\end{proof}

\begin{teo}
    Se $\mathcal{P}$ é uma partição de um conjunto $X \neq \emptyset$, então existe uma relação de equivalência $R$ em $X$ tal que $X / R = \mathcal{P}$.
\end{teo}

\begin{proof}
    Pois tome $R := \{(x,y) \in X \times X : \exists A (A \in \mathcal{P} \land x,y \in A) \}$. \blackproof
\end{proof}

\section{Funções}

\begin{defi}[Função] \label{defi.fund:função}
    \leavevmode
        \begin{enumerate}[leftmargin=*, align=left, label=\textbf{(\alph*)}]
            \item Uma relação $f$ é uma \textit{função} se $(a,b) \in f$ e $(a,c) \in f$ implicam $b=c$. A \textit{imagem} de $a \in \Dom{(f)}$ por $f$ é denotada por $f(a)$.
            \item Uma \textit{função parcial de $A$ em $B$} é uma função $f$ tal que $\Dom{(f)} \subseteq A$ e $\Im{(f)} \subseteq B$.
            \item Uma \textit{função total de $A$ em $B$} é uma função $f$ tal que $\Dom{(f)} = A$ e $\Im{(f)} \subseteq B$. Isso é denotado por $f : A \to B$. O conjunto de todas as funções de $A$ em $B$ é denotado por $B^{A}$, isto é,
                \begin{align*}
                    B^{A} := \{f \in \mathcal{P}{(A \times B)} : & \forall a \forall b \forall c ((a,b) \in f \land (a,c) \in f \rightarrow b=c) \\ 
                    & \land \forall x(x \in A \rightarrow \exists y((x,y) \in f)) \}.
                \end{align*}
        \end{enumerate}
\end{defi}

\begin{obs}
    Escrevemos apenas ``função de $A$ em $B$'', omitindo o ``total''.
\end{obs}

\begin{defi}
    A \textit{função identidade} de um conjunto $A$ é a função $\id_A : A \to A$ definida por $\id_A{(x)} = x$ para todo $x \in A$.
\end{defi}

\begin{prop}
    Se $f: A \to B$ é uma função, então $\id_B \circ f = B$ e $f \circ \id_A = A$.
\end{prop}

\begin{proof}
    Teste só para ver se está funcionando. \blackproof
\end{proof}

\begin{prop}
    Sejam $f$ e $g$ funções e $X$ um conjunto.
        \begin{enumerate}[leftmargin=*, align=left, label=\textbf{(\alph*)}]
            \item $f \restriction_X$ é uma função e seu domínio é $\Dom{(f)} \cap X$.
            \item $g \circ f$ é uma função.
        \end{enumerate}
\end{prop}

\begin{proof}
    \leavevmode
        \begin{enumerate}[leftmargin=*, align=left, label=\textbf{(\alph*)}]
            \item Por definição, $f\restriction_X = \{ (a,b) \in f : a \in X \}$, isto é, $f$ é um conjunto de pares ordenados e, portanto, uma relação. Se $(a,b) \in f\restriction_X$ e $(a,c) \in f\restriction_X$, então, pela definição de $f\restriction_X$, $(a,b) \in f$, $(a,c) \in f$ e $a \in X$; como $f$ é função, $b=c$, de modo que $f\restriction_X$ é também uma função. Provemos, por fim, que $\Dom{(f\restriction_X)} =  \Dom{(f)} \cap X$. Se $a \in \Dom{(f\restriction_X)}$, então existe $b \in \Im{(f\restriction_X)}$ tal que $(a,b) \in f\restriction_X$; pela definição de $f\restriction_X$, vem $(a,b) \in f$ e $a \in X$, e de $(a,b) \in f$ vem $a \in \Dom{(f)}$. Com isso, $a \in \Dom{(f)} \cap X$, de modo que $\Dom{(f\restriction_X)} \subseteq \Dom{(f)} \cap X$. Por outro lado, se $a \in \Dom{(f)} \cap X$, então de $a \in \Dom{(f)}$ segue que existe $b \in \Im{(f)}$ tal que $(a,b) \in f$, e como $a \in X$, vem $(a,b) \in f\restriction_X$, de modo que $\Dom{(f)} \cap X \subseteq \Dom{(f\restriction_X)}$. Logo $\Dom{(f\restriction_X)} = \Dom{(f)} \cap X$. Em particular, temos $f\restriction_X = f \restriction_{\Dom{(f)} \cap X}$. \blackproof
            \item Se $(a,x) \in g \circ f$ e $(a,y) \in g \circ f$, então, por definição, existe $b \in \Im{(f)} \cap \Dom{(g)}$ tal que $(a,b) \in f$ e $(b,x) \in g$ e existe $c \in \Im{(f)} \cap \Dom{(g)}$ tal que $(a, c) \in f$ e $(c,y) \in g$. Como $f$ é função, vem $b=c$; daí, vem $(b,x) \in g$ e $(b,y) \in g$, e como $g$ é função, vem $x=y$. \blackproof
        \end{enumerate}
\end{proof}


\subsection{Funções Injetivas}

\begin{defi}
    Uma função $f$ é \textit{injetiva} se
        \[
            \forall x \forall y (x,y \in \Dom{(f)} \rightarrow (x \neq y \rightarrow f(x) \neq f(y))).
        \]
\end{defi}

\begin{prop} \label{prop.fund:feginj}
    Sejam $f$ e $g$ funções.
        \begin{enumerate}[leftmargin=*, align=left, label=\textbf{(\alph*)}]
            \item Se $f$ e $g$ são injetivas, então $g \circ f$ é injetiva.
            \item Se $g \circ f$ é injetiva e $\Im{(f)} \subseteq \Dom{(g)}$, então $f$ é injetiva.
        \end{enumerate}
\end{prop}

\begin{proof}
    \leavevmode
        \begin{enumerate}[leftmargin=*, align=left, label=\textbf{(\alph*)}]
            \item Sejam $x,y \in \Dom{(g \circ f)}$. Se $g(f(x)) = g(f(y))$, então $f(x) = f(y)$ pela injetividade de $g$. Se $f(x) = f(y)$, então $x = y$ pela injetividade de $f$. Logo $g \circ f$ é injetiva. \blackproof
            \item Sejam $x,y \in \Dom{(f)}$ tais que $f(x) = f(y)$. Se $\Im{(f)} \subseteq \Dom{(g)}$, então $f(x), f(y) \in \Dom{(g)}$, e como $f(x) = f(y)$, temos $g (f(x)) = g(f(y))$. Daí, como $\Dom{(f)} = \Dom{(g \circ f)}$ (proposição \eqref{prop.fund:relacoes}) e $g \circ f$ é injetiva, vem $x=y$, de modo que $f$ é injetiva. \blackproof
        \end{enumerate}
\end{proof}

\begin{teo} \label{teo.fund:inj}
    Seja $f$ uma função.
        \begin{enumerate}[leftmargin=*, align=left, label=\textbf{(\alph*)}]
            \item Se a relação $f^{-1}$ é uma função, então $f^{-1}$ é injetiva.
            \item A relação $f^{-1}$ é uma função se, e somente se,
                \begin{enumerate}[label=\roman*.]
                    \item $f$ é injetiva.
                    \item $f^{-1} \circ f = \id_{\Dom{(f)}}$.
                \end{enumerate}
        \end{enumerate}
\end{teo}

\begin{proof}
    \leavevmode
        \begin{enumerate}[leftmargin=*, align=left, label=\textbf{(\alph*)}]
            \item Se $(y,x) \in f^{-1}$ e $(z,x) \in f^{-1}$, então $(x,y) \in f$ e $(x,z) \in f$, e como $f$ é função vem $y=z$, de modo que $f^{-1}$ é uma função injetiva. \blackproof
            \item A equivalência mais importante é com $f$ ser injetiva.
                \begin{enumerate}[label=\roman*.]
                    \item Se $f^{-1}$ é uma função, então $(x,y) \in f^{-1}$ e $(x,z) \in f^{-1}$ implicam $y=z$. Daí, como $(y,x) \in f$ e $(z,x) \in f$, sendo $y=z$ segue que $f$ é injetiva. Agora, se $f$ é injetiva, então $(y,x) \in f$ e $(z,x) \in f$ implicam $y=z$, e como $(x,y) \in f^{-1}$ e $(x,z) \in f^{-1}$, segue que $f^{-1}$ é uma função.
                \end{enumerate}
            Provemos que $f$ é injetiva se, e somente se, $f^{-1} \circ f = \id_{\Dom{(f)}}$.
                \begin{enumerate}[label=\roman*., resume]
                    \item ($\Rightarrow$) Se $(x,z) \in f^{-1} \circ f$, então existe $y \in \Im{(f)} \cap \Dom{(f^{-1})}$ tal que $(x,y) \in f$ e $(y,z) \in f^{-1}$. Com isso, $(z,y) \in f$, e como $f$ é injetiva vem $z=x$, de modo que $(x,x) \in \id_{\Dom{(f)}}$, isto é, $f^{-1} \circ f \subseteq \id_{\Dom{(f)}}$. Agora, se $(x,x) \in \id_{\Dom{(f)}}$, então existe $y \in \Im{(f)}$ tal que $(x, y) \in f$, isto é, $(y,x) \in f^{-1}$. Com isso, $(x,x) \in f^{-1} \circ f$, de modo que $\id_{\Dom{(f)}} \subseteq f^{-1} \circ f$. Com isso, vem $f^{-1} \circ f = \id_{\Dom{(f)}}$.
                    
                    ($\Leftarrow$) Sendo $(x,y) \in f$ e $(z,y) \in f$, temos $(y,x) \in f^{-1}$, de modo que $(z,x) \in f^{-1} \circ f$, e como $f^{-1} \circ f = \id_{\Dom{(f)}}$, vem $x=z$, o que prova a injetividade de $f$.
                \end{enumerate}
            Com isso, todas as equivalências foram provadas. \blackproof
        \end{enumerate}
\end{proof}

\begin{defi}
    Uma função $f$ é \textit{invertível à esquerda} se existe uma função $g$ tal que $g \circ f = \id_{\Dom{(f)}}$. Dizemos que $g$ é uma \textit{inversa à esquerda} de $f$.
\end{defi}

\begin{teo} \label{teo.fund:invesquerda}
    Uma função $f$ é invertível à esquerda se, e somente se, $f$ é injetiva.
\end{teo}

\begin{proof}
    Se $f$ é injetiva, então pelo teorema \eqref{teo.fund:inj} $f^{-1}$ é uma função e $f^{-1} \circ f = \id_{\Dom{(f)}}$, de modo que $f$ é invertível à esquerda. Agora, sendo $(x,y) \in f$ e $(z,y) \in f$, provemos que $x=z$. Como $f$ é invertível à esquerda, existe uma função $g$ tal que $g \circ f = \id_{\Dom(f)}$. Como $(x,x) \in g \circ f$, existe $w \in \Im{(f)} \cap \Dom{(g)}$ tal que $(x,w) \in f$ e $(w,x) \in g$. Como $f$ é uma função, vem $w=y$, de modo que $(y,x) \in g$. Analogamente temos $(y,z) \in g$, e como $g$ é uma função vem $x=z$, de modo que $f$ é injetiva. \blackproof
\end{proof}

\subsection{Funções Sobrejetivas}

\begin{defi}
    Uma função $f : A \to B$ é \textit{sobrejetiva $B$} se $\Im{(f)} = B$.
\end{defi}

\begin{prop}
    Uma função $f: A \to B$ é sobrejetiva em $B$ se, e somente se, para todo $y \in B$ existe $x \in A$ tal que $(x,y) \in f$.
\end{prop}

\begin{proof}
    Segue da proposição \eqref{prop.fund:domeimdeRAB}. \blackproof    
\end{proof}

\begin{lem} \label{lem.fund:fegsob}
    Sejam $f: A \to B$ e $g: C \to D$ funções.
        \begin{enumerate}[leftmargin=*, align=left, label=\textbf{(\alph*)}]
            \item $\Dom{(g \circ f)} = \{ x \in A : f(x) \in C\}$.
            \item $\Dom{(g \circ f)} = A$ se, e somente se, $f(A) \subseteq C$.
        \end{enumerate}
\end{lem}

\begin{proof}
    \leavevmode
        \begin{enumerate}[leftmargin=*, align=left, label=\textbf{(\alph*)}]
            \item Pela proposição \eqref{prop.fund:domeimdeRAB}, $\Dom{(g \circ f)} = \{ x \in A : \exists y(y \in D \land (x,y \in g \circ f))\}$.
                \begin{enumerate}[label=\roman*.]
                    \item Se $x \in \Dom{(g \circ f)}$, então existe $y \in D$ tal que $(x,y) \in g \circ f$. Logo existe $z \in \Im{(f)} \cap \Dom{(g)} \subseteq B \cap C$ tal que $(x,z) \in f$ e $(z,y) \in g$, isto é, $z = f(x)$ e $y = g(z)$. Com isso, $x \in A$ e $f(x) \in C$, de modo que $x \in \{ x \in A : f(x) \in C\}$, isto é, $\Dom{(g \circ f)} \subseteq \{ x \in A : f(x) \in C\}$.
                    \item Se $x \in \{ x \in A : f(x) \in C\}$, então $x \in A$ e $f(x) \in C$, isto é, existe (um único) $z \in C$ tal que $(x,z) \in f$. Como $z \in C$, existe $y \in \Im{g} \subseteq D$ tal que $(z,y) \in g$. Com isso, $x \in A$ e existe $y \in D$ tal que $(x,y) \in g \circ f$, de modo que $x \in \Dom{(g \circ f)}$, isto é, $\{ x \in A : f(x) \in C\} \subseteq \Dom{(g \circ f)}$.
                \end{enumerate}
            Logo $\Dom{(g \circ f)} = \{ x \in A : f(x) \in C\}$. \blackproof
            \item A volta ($\Leftarrow$) já foi provada (proposição \eqref{prop.fund:relacoes}). Agora, se $y \in f(A)$, então existe $x \in A$ tal que $y = f(x)$, e como $A = \Dom{(g \circ f)}$, vem $f(x) \in C$, isto é, $y \in C$. Logo $f(A) \subseteq C$. \blackproof
        \end{enumerate}
\end{proof}

\begin{prop} \label{prop.fund:fegsob}
    Sejam $f : A \to B$ e $g : C \to D$ funções.
        \begin{enumerate}[leftmargin=*, align=left, label=\textbf{(\alph*)}]
            \item Se $f$ e $g$ são sobrejetivas e $B=C$, então $g \circ f$ é sobrejetiva.
            \item Se $g \circ f$ é sobrejetiva e $f(A) \subseteq C$, então $g$ é sobrejetiva.
        \end{enumerate} 
\end{prop}

\begin{proof}
    \leavevmode
        \begin{enumerate}[leftmargin=*, align=left, label=\textbf{(\alph*)}]
            \item Se $B=C$, então $\Dom{(g \circ f)} = A$ pelo lema \eqref{lem.fund:fegsob}. Se $g$ é sobrejetiva, então para todo $z \in D$ existe $y \in C = B$ tal que $(y,z) \in g$. Se $f$ é sobrejetiva, então para esse $y \in B$ existe $x \in A$ tal que $(x,y) \in f$. Com isso, para todo $z \in D$ existe $x \in A$ tal que $(x,z) \in g \circ f$, o que prova a sobrejetividade de $g \circ f$. \blackproof
            \item Se $f(A) \subseteq C$, então $\Dom{(g \circ f)} = A$ pelo lema \eqref{lem.fund:fegsob}. Se $g \circ f$ é sobrejetiva, então para todo $z \in D$ existe $x \in A$ tal que $(x,z) \in g \circ f$. Com isso, existe $y \in f(A) \cap C = C$ tal que $(x,y) \in f$ e $(y,z) \in g$. o que prova a sobrejetividade de $g$. \blackproof
        \end{enumerate}
\end{proof}

\begin{lem}
    Para qualquer função $f$, tem-se $f \circ f^{-1} = \id_{\Im{(f)}}$.
\end{lem}

\begin{proof}
    Provemos que $f \circ f^{-1} \subseteq \id_{\Im{(f)}}$ e $f \circ f^{-1} \supseteq \id_{\Im{(f)}}$. Se $(y,z) \in f \circ f^{-1}$, então existe $x \in \Dom{(f)} \cap \Im{(f^{-1})}$ tal que $(y,x) \in f^{-1}$ e $(x,z) \in f$. Daí, $(x,y) \in f$, e como $f$ é uma função, vem $y=z$. Logo $f \circ f^{-1} \subseteq \id_{\Im{(f)}}$. Por outro lado, se $(y,y) \in \id_{\Im{(f)}}$, então existe $x \in \Dom{(f)}$ tal que $(x,y) \in f$. Logo $(y,x) \in f^{-1}$, de modo que $(y, y) \in f \circ f^{-1}$, isto é, $f \circ f^{-1} \supseteq \id_{\Im{(f)}}$. Com isso, vem $f \circ f^{-1} = \id_{\Im{(f)}}$, como queríamos. \blackproof
\end{proof}

\begin{teo} \label{teo.fund:sob}
    Uma função $f : A \to B$ é sobrejetiva se, e somente se, $f \circ f^{-1} = \id_B$.
\end{teo}

\begin{proof}
    Se $f$ é sobrejetiva em $B$, então $\Im{(f)} = B$, de modo que $f \circ f^{-1} = \id_B$. Por outro lado, se $f \circ f^{-1} = \id_B$, então $f$ é sobrejetiva em $B$ porque, como também $f \circ f^{-1} = \id_{\Im{(f)}}$, vem $\Im{(f)} = B$. \blackproof
\end{proof}

\begin{defi}
    Uma função $f:A \to B$ é \textit{invertível à direita} se existe uma função $g : B \to A$ tal que $f \circ g = \id_{B}$. Dizemos que $g$ é uma \textit{inversa à direita} de $f$.
\end{defi}

\begin{teo} \label{teo.fund:invdireita}
    Uma função $f:A \to B$ é invertível à direita se, e somente se, $f$ é sobrejetiva em $B$.
\end{teo}

\begin{obs}
    A prova da volta ($\Leftarrow$) deste teorema depende do axioma da escolha. Mais precisamente, de um enunciado equivalente ao axioma da escolha: para toda relação $R$ existe uma função $f \subseteq R$ tal que $\Dom{(f)} = \Dom{(R)}$. Ainda assim, enunciamos este resultado aqui por uma questão de organização didática.  
\end{obs}

\begin{proof}
    \blackproof
\end{proof}

\subsection{Funções Bijetivas e Funções Inversas}

\begin{defi}
    Uma função $f: A \to B$ é \textit{bijetiva em relação a $B$} se é injetiva e sobrejetiva em $B$.
\end{defi}

\begin{prop} \label{prop.fund:defbij}
    Uma função $f: A \to B$ é bijetiva em $B$ se, e somente se, para todo $y \in B$ existe um único $x \in A$ tal que $(x,y) \in f$.
\end{prop}

\begin{proof}
    Segue imediatamente da definição. \blackproof
\end{proof}

\begin{prop}
    Se $f : A \to B$ e $g : B \to C$ são funções bijetivas, então a função $g \circ f : A \to C$ é uma função bijetiva.
\end{prop}

\begin{proof}
    Segue como corolário imediato das proposições \eqref{prop.fund:feginj} e \eqref{prop.fund:fegsob}. \blackproof
\end{proof}

\begin{teo} \label{teo.fund:bij1}
    Seja $f : A \to B$ uma função.
        \begin{enumerate}[leftmargin=*, align=left, label=\textbf{(\alph*)}]
            \item Se a relação $f^{-1}$ é uma função de $B$ em $A$, então $f^{-1}$ é bijetiva.
            \item A relação $f^{-1}$ é uma função de $B$ em $A$ se, e somente se,
                \begin{enumerate}[label=\roman*.]
                    \item $f$ é bijetiva em $B$.
                    \item $f^{-1} \circ f = \id_{A}$ e $f \circ f^{-1} = \id_{B}$.
                \end{enumerate}
        \end{enumerate}
\end{teo}

\begin{proof}
    \leavevmode
        \begin{enumerate}[leftmargin=*, align=left, label=\textbf{(\alph*)}]
            \item Como $f$ é função, para todo $x \in A$ existe um único $y \in B$ tal que $(x,y) \in f$, isto é, $(y,x) \in f^{-1}$. Daí, pela proposição \eqref{prop.fund:defbij}, $f^{-1}$ é bijetiva em $A$. \blackproof
            \item A equivalência que mais importa é com $f$ ser bijetiva em $B$.
                \begin{enumerate}[label=\roman*.]
                    \item Se $f^{-1} \subseteq B \times A$ é uma função tal que $\Dom{(f^{-1})} = B$, então para todo $y \in B$ existe um único $x \in A$ tal que $(y,x) \in f^{-1}$, isto é, $(x,y) \in f$. Daí, pela proposição \eqref{prop.fund:defbij}, temos que $f$ é bijetiva. Agora, pela mesma proposição, se $f$ é bijetiva, então para todo $y \in B$ existe um único $x \in A$ tal que $(x,y) \in f$, isto é, $(y,x) \in f^{-1}$. Daí, pela definição de função, $f^{-1}$ é uma função de $B$ em $A$. \blackproof
                \end{enumerate}
            Provemos que $f$ é bijetiva em $B$ se, e somente se, $f^{-1} \circ f = \id_A$ e $f \circ f^{-1} = \id_{B}$.
                \begin{enumerate}[label=\roman*., resume]
                    \item Se $f$ é bijetiva em $B$, então $f$ é injetiva e sobrejetiva, de modo que, pelos teoremas \eqref{teo.fund:inj} e \eqref{teo.fund:sob}, vem $f^{-1} \circ f = \id_A$ e $f \circ f^{-1} = \id_B$, respectivamente. Agora, se $f^{-1} \circ f = \id_A$ e $f \circ f^{-1} = \id_B$, então pelos mesmos teoremas $f$ é injetiva e sobrejetiva em $B$, isto é, $f$ é bijetiva em $B$. \blackproof
                \end{enumerate}
            Com isso, todas as equivalências foram provadas. \blackproof
        \end{enumerate}
\end{proof}

\begin{defi}
    Uma função $f: A \to B$ é \textit{invertível} se existe uma função $g : B \to A$ tal que $g \circ f = \id_A$ e $f \circ g = \id_B$. Dizemos que $g$ é a \textit{inversa} de $f$.
\end{defi}

\begin{prop}
    A função inversa de uma função invertível é única.\footnote{Note que é esta proposição que nos permite dizer ``a função inversa'' em vez de ``uma função inversa''. }
\end{prop}

\begin{proof}
    Seja $f : A \to B$ uma função invertível. Sejam $g_1, g_2 : B \to A$ funções inversas de $f$. Provemos que $g_1 = g_2$. De fato,
        \[
            g_1 = g_1 \circ \id_B = g_1 \circ (f \circ g_2) = (g_1 \circ f) \circ g_2 = \id_A \circ g_2 = g_2.    
        \]
    Logo, a função inversa da função $f : A \to B$, quando existe, é única. \blackproof
\end{proof}

\begin{teo} \label{teo.fund:bij2}
    Seja $f: A \to B$ uma função.
        \begin{enumerate}[leftmargin=*, align=left, label=\textbf{(\alph*)}]
            \item $f$ é invertível se, e somente se, $f$ é bijetiva.
            \item Se $f$ é invertível, então a função inversa de $f$ é a relação inversa $f^{-1}$.
        \end{enumerate}
\end{teo}

\begin{proof}
    \leavevmode
        \begin{enumerate}[leftmargin=*, align=left, label=\textbf{(\alph*)}]
            \item Se $f$ é invertível, então existe uma função $g: B \to A$ tal que $g \circ f = \id_A$ e $f \circ g = \id_B$, de modo que $f$ é invertível à esquerda e à direita, isto é, $f$ é injetiva (teorema \eqref{teo.fund:invesquerda}) e sobrejetiva em $B$ (teorema \eqref{teo.fund:invdireita}), isto é, $f$ é bijetiva em $B$. Por outro lado, se $f$ é bijetiva em $B$, então pelo teorema \eqref{teo.fund:bij1} sua relação inversa $f^{-1} \subseteq B \times A$ é uma função de $B$ em $A$ e, mais ainda, satisfaz $f^{-1} \circ f = \id_{A}$ e $f \circ f^{-1} = \id_{B}$, de modo que $f$ é invertível. \blackproof   
            \item Segue imediatamente do primeiro item. \blackproof
        \end{enumerate}
\end{proof}

\begin{cor}
    Uma função é invertível se, e somente se, é invertível à esquerda e à direita.
\end{cor}

\begin{proof}
    A ida ($\Rightarrow$) decorre imediatamente das definições. Provemos então a volta ($\Leftarrow$). Se $f$ é invertível à esquerda e à direita, então pelos teoremas \eqref{teo.fund:invesquerda} e \eqref{teo.fund:invdireita} $f$ é bijetiva em $B$, de modo que, pelo teorema \eqref{teo.fund:bij2}, $f$ é invertível. \blackproof
\end{proof}

\begin{obs}
    Vamos resumir o que está acontecendo. O resultado mais importante é a equivalência entre invertibilidade e bijetividade: por um lado, se $f : A \to B$ é bijetiva, então a relação inversa $f^{-1} \subseteq B \times A$ é uma função de $B$ em $A$ tal que $f^{-1} \circ f = \id_{A}$ e $f \circ f^{-1} = \id_{B}$, o que prova que $f$ é invertível. Por outro lado, se $f$ é invertível, então existe $f^{-1} : B \to A$ tal que $f^{-1} \circ f = \id_{A}$ e $f \circ f^{-1} = \id_{B}$, o que prova que $f$ é bijetiva.
\end{obs}

\section{O Axioma do Infinito e os Números Naturais}

\begin{defi}
    O \textit{sucessor} de um conjunto $x$ é definido como $x^+ := x \cup \{x\}$.
\end{defi}

\begin{prop}
    Valem as seguintes afirmações sobre o sucessor.
        \begin{enumerate}[leftmargin=*, align=left, label=\textbf{(\alph*)}]
            \item $\forall x \forall y ((y \in x^+) \leftrightarrow ((y \in x) \lor (y = x)))$
            \item $\forall x (x \in x^+)$
            \item $\forall x (x \subseteq x^+)$
        \end{enumerate}
\end{prop}

\begin{proof}
    \leavevmode
        \begin{enumerate}[leftmargin=*, align=left, label=\textbf{(\alph*)}]
            \item Pela definição de sucessor, $x^+ = x \cup \{x\}$. Pela definição de união, $y \in x \cup \{x\}$ se, e somente se, $y \in x$ ou $y \in \{x\}$. Como $y \in \{x\}$ equivale a $y = x$, temos que $y \in x^+$ se, e somente se, $y \in x$ ou $y = x$. \blackproof
            \item Tomando $y=x$ no item anterior, obtemos $(x \in x^+) \leftrightarrow ((x \in x) \lor (x = x))$. Como vale $x = x$, vale também a disjunção $(x \in x) \lor (x = x)$, de modo que $x \in x^+$. \blackproof
            \item Se $z \in x$, então $(z \in x) \lor (z = x)$, de modo que $z \in x^+$.
            Com isso, $\forall z (z \in x \rightarrow z \in x^+)$, isto é, $x \subseteq x^+$. \blackproof
        \end{enumerate}
\end{proof}

\begin{defi}
    Um conjunto $x$ é \textit{indutivo} se
        \[
            (\emptyset \in x) \land \forall y ( (y \in x) \rightarrow (y^+ \in x)).
        \]
    Isso é denotado por $\ind{(x)}$.
\end{defi}

\begin{ax}[Infinito]
    Existe um conjunto indutivo.
        \[
            \boxed{
                \exists x ( (\emptyset \in x) \land \forall y ((y \in x) \rightarrow (y^+ \in x)) )
            }
        \]
\end{ax}

\begin{teo} \label{teo.fund:omega}
    Seja $I$ um conjunto indutivo. Defina
        \[
            \omega(I) := \bigcap \{ x \in \mathcal{P}{(I)} : \ind{(x)}\}.
        \]
        \begin{enumerate}[leftmargin=*, align=left, label=\textbf{(\alph*)}]
            \item Para todo conjunto $I$, se $I$ é indutivo, então $\omega(I)$ é indutivo.
                \[
                    \forall I ( \ind{(I)} \rightarrow \ind{(\omega(I))})
                \]
            \item Para quaisquer conjuntos $I$ e $J$, se $I$ e $J$ são indutivos, então $\omega(I) = \omega(J)$.
                \[
                    \forall I \forall J ( \ind{(I)} \land \ind{(J)} \rightarrow \omega(I) = \omega(J) )
                \] 
        \end{enumerate}  
\end{teo}

\begin{proof} 
    Provemos que $\omega(I)$ está bem definido. Pelo axioma das partes, existe o conjunto $\mathcal{P}{(I)}$. Pelo axioma da separação com o conjunto $\mathcal{P}{(I)}$ e a fórmula $\ind{(x)}$, existe $z(I) :=  \{ x \in \mathcal{P}{(I)} : \ind{(x)} \}$. Pelo \Cref{teo.fund:intfamilia}, como $z(I) \neq \emptyset$ já que $I \in z(I)$, existe $\omega(I) := \bigcap z(I)$.
        \begin{enumerate}[leftmargin=*, align=left, label=\textbf{(\alph*)}]
            \item \blackproof
            \item \blackproof
        \end{enumerate}
\end{proof}

\begin{obs}
    O \Cref{teo.fund:omega} nos diz que $\omega(I)$ é a interseção da família de todos os conjuntos indutivos e que o parâmtro $I$ pode ser suprimido. A seguinte definição só é possível devido a esse teorema.
\end{obs}

\begin{defi}
    O conjunto dos números naturais é definido como a interseção de todos os conjuntos indutivos. Ele é denotado por $\omega$.
\end{defi}

\begin{teo}[Indução]
    Se $A \subseteq \omega$ é indutivo, então $A = \omega$.
\end{teo}

\begin{proof}
    Pelo teorema \eqref{teo.fund:omega}, se $A$ é indutivo, então $\omega \subseteq A$. Daí, como $A \subseteq \omega$, vem $A = \omega$. \blackproof
\end{proof}

\begin{teo}[Axiomas de Peano] \label{teo.fund:peano}
    O conjunto $\omega$ dos números naturais satisfaz os axiomas de Peano, isto é, valem as seguintes afirmações.
        \begin{enumerate}[leftmargin=*, align=left, label=\textbf{(\alph*)}]
            \item $\forall x \forall y (x,y \in \omega  \rightarrow  (\neg (x=y) \rightarrow \neg (x^+ = y^+ )))$.
            \item $\forall x (x \in \omega \rightarrow (\neg (x^+ = \emptyset) ))$.
            \item Para toda fórmula $P$,
                \[
                    P(\emptyset) \land  \forall x (P(x) \rightarrow P(x^+)) \rightarrow \forall x (x \in \omega \rightarrow P(x)).
                \]
        \end{enumerate}
\end{teo}

\begin{proof}
    Ver \cite{fajardo2024conjuntos}, teorema 3.20, página 89. \blackproof
\begin{comment}
    \leavevmode
        \begin{enumerate}[leftmargin=*, align=left, label=\textbf{(\alph*)}]
            \item Suponha, por absurdo, que $x \neq y$ e $x^+ = y^+$. Então $x^+ = y \cup \{y \}$, e como $x \in x^+$, vem $x \in y \cup \{y\}$. Como $x \neq y$, então $x \in y$. Analogamente, $y \in x$, o que contraria a proposição \eqref{prop.fund:xinyeyinxabs}. Logo, se $x \neq y$, então $x^+ \neq y^+$. \blackproof
        \end{enumerate}
    Note que não usamos a hipótese de ser $x,y \in \omega$, ou seja, vale a afirmação mais forte $\forall x \forall y (\neg(x=y) \rightarrow \neg (x^+ = y^+) )$.
        \begin{enumerate}[leftmargin=*, align=left, label=\textbf{(\alph*)}, resume]
            \item Sabemos que $\forall x (x \in x^+)$. Se fosse $x^+ = \emptyset$ para algum $x$, seria $x \in \emptyset$, o que não é. \blackproof
        \end{enumerate}
    Novamente, não usamos a hipótese de ser $x \in \omega$, ou seja, vale a afirmação mais forte $\forall x (\neg(x^+ = \emptyset))$.
        \begin{enumerate}[leftmargin=*, align=left, label=\textbf{(\alph*)}, resume]
            \item Pelo axioma da separação, defina $A := \{ x \in \omega : P(x) \}$. Então $A \subseteq \omega$. Afirmamos que $A$ é indutivo. Como $P(\emptyset)$, então $\emptyset \in A$. Se $x \in A$, então $x \in \omega$ e $P(x)$, e como $P(x^+)$ e $x^+ \in \omega$, vem $x^+ \in A$. Com isso, $A$ é indutivo, de modo que $\omega \subseteq A$, e como $A \subseteq \omega$, vem $A = \omega$. \blackproof
        \end{enumerate}.
\end{comment}
\end{proof}

\begin{defi}
    Um conjunto $A$ é \textit{transitivo} se
        \[
            \forall x \forall y (x \in y \land y \in A \rightarrow x \in A).
        \]
\end{defi}

\begin{prop}
    Um conjunto $A$ é transitivo se, e somente se,
        \begin{enumerate}[label=\roman*.]
            \item $\forall a (a \in A \rightarrow a \subsetneq A)$.
            \item $\bigcup A \subseteq A$.
            \item $\bigcup A^+ = A$.
            \item $A \subseteq \mathcal{P}(A)$.
        \end{enumerate}    
\end{prop}

\begin{proof}
    \leavevmode
        \begin{enumerate}[leftmargin=*, align=left, label=\textbf{(\alph*)}]
            \item Por um lado ($\Rightarrow$), se $A$ é transitivo e $a \in A$, então para todo $x \in a$ temos $x \in A$, de modo que $a \subseteq A$. Se fosse $a = A$, teríamos $a \in a$ ou $A \in A$, o que contraria o corolário \eqref{cor.fund:xnotinx}. Com isso, $a \neq A$, donde $a \subsetneq A$. Por outro lado ($\Leftarrow$), se $a \subsetneq A$ para todo $a \in A$, então $a \subseteq A$. Logo, se $x \in y$ e $y \in A$, então $y \subseteq A$, de modo que $x \in A$. \blackproof
        \end{enumerate}
\end{proof}

\begin{teo}
    \leavevmode
        \begin{enumerate}[leftmargin=*, align=left, label=\textbf{(\alph*)}]
            \item Todo número natural é um conjunto transitivo.
            \item O conjunto $\omega$ é transitivo.
        \end{enumerate}
\end{teo}

\begin{proof}
    \leavevmode
        \begin{enumerate}[leftmargin=*, align=left, label=\textbf{(\alph*)}]
            \item A prova se dará por indução em $n$. O conjunto vazio $\emptyset$ é trivialmente transitivo (por vacuidade). Agora, supondo que $n \in \omega$ é transitivo, provemos que $n^+ \in \omega$ também é transitivo. Se $x \in n^+$, então $x \in n$ ou $x = n$. 
                \begin{itemize}
                    \item Se $x \in n$, então, dado $y \in x$, como $n$ é transitivo, vem $y \in n$, e como $n \subsetneq n^+$, vem $y \in n^+$, de modo que $n^+$ é transitivo.
                    \item Se $x = n$, então, dado $y \in x$, temos $y \in n$, e como $n \subsetneq n^+$, vem $y \in n^+$, de modo que $n^+$ é transitivo. 
                \end{itemize}
            Com isso, todo número natural é transitivo. \blackproof
            \item Provemos que $\forall n (n \in \omega \rightarrow n \subsetneq \omega)$ por indução. Trivialmente $\emptyset \subsetneq \omega$. Agora, se $n \in \omega$, então $\{n\} \subsetneq \omega$, e como $n \subsetneq \omega$ (hipótese de indução), temos $n \cup \{n\} \subseteq \omega$, isto é, $n^+ \subsetneq \omega$, pois $n^+ \neq \omega$. \blackproof
        \end{enumerate}
\end{proof}

\begin{lem}
    Para quaisquer $m,n \in \omega$, valem as seguintes afirmações.
        \begin{enumerate}[leftmargin=*, align=left, label=\textbf{(\alph*)}]
            \item Se $m \in n$, então $m^+ \in n$ ou $m^+=n$.
            \item Se $m \in n$, então $m \subsetneq n$.
        \end{enumerate}
\end{lem}

\begin{proof}
    \leavevmode
        \begin{enumerate}[leftmargin=*, align=left, label=\textbf{(\alph*)}]
            \item A prova se dará por indução em $n$. Se $m \in \emptyset$, então a conclusão segue trivialmente por vacuidade. Agora, supondo que $m \in n$ implica $m^+ \in n$ ou $m^+=n$, provemos que $m \in n^+$ implica $m^+ \in n^+$ ou $m^+ = n^+$. Se $m \in n^+$, então $m \in n$ ou $m=n$. Se $m \in n$, então $m^+ \in n$ ou $m^+=n$ (hipótese de indução). Como $n \subsetneq n^+$ e $n \in n^+$, vem $m^+ \in n^+$. Se $m=n$, então $m^+=n^+$. \blackproof
            \item A prova se dará por indução em $n$. Se $m \in \emptyset$, então a conclusão segue trivialmente por vacuidade. Agora, supondo que $m \in n$ implica $m \subsetneq n$, provemos que $m \in n^+$ implica $m \subsetneq n^+$. Se $m \in n^+$, então $m \in n$ ou $m=n$. Se $m \in n$, então $m \subsetneq n$ (hipótese de indução), e como $n \subsetneq n^+$, vem $m \subsetneq n^+$. Se $m = n$, então $m \subsetneq n^+$ pois $n \subsetneq n^+$. \blackproof
        \end{enumerate}
\end{proof}

\begin{teo} \label{teo.fund:tricotomiadopertence}
    Para quaisquer $m,n \in \omega$, valem as seguintes afirmações.
        \begin{enumerate}[leftmargin=*, align=left, label=\textbf{(\alph*)}]
            \item $m \in n$ se, e somente se, $m^+ \in n^+$.
            \item Ou $m \in n$, ou $n \in m$, ou $m = n$.
        \end{enumerate}
\end{teo}

\begin{proof}
    \leavevmode
        \begin{enumerate}[leftmargin=*, align=left, label=\textbf{(\alph*)}]
            \item Por um lado ($\Rightarrow$), se $m \in n$, então $m^+ \in n$ ou $m^+ = n$. Como $n \subsetneq n^+$ e $n \in n^+$, em ambos os casos temos $m^+ \in n^+$. Por outro lado ($\Leftarrow$), se $m^+ \in n^+$, então $m^+ \in n$ ou $m^+ = n$. Se $m^+ \in n$, então $m^+ \subsetneq n$, e como $m \in m^+$, segue que $m \in n$. Se $m^+ = n$, como $m \in m^+$, temos imediatamente $m \in n$. \blackproof


            \item A prova se dará por indução em $n$. Se $n = \emptyset$, não pode ser $m \in \emptyset$. Provemos rapidamente que ou $\emptyset \in m$ ou $m = \emptyset$.
                \begin{itemize}
                    \item Se $m = \emptyset$, a afirmação é trivialmente verdadeira. Supondo que $m = \emptyset$ ou $\emptyset \in m$, provemos que $m^+ = \emptyset$ ou $\emptyset \in m^+$. Como $m \in m^+$, não pode ser $m^+ = \emptyset$. Se $m = \emptyset$, então $m^+ = \{\emptyset\}$, de modo que $\emptyset \in m^+$. Se $\emptyset \in m$, então $\emptyset \in m^+$ pois $m \subseteq m^+$.
                \end{itemize}
            Com isso, o resultado vale para $n = \emptyset$. Agora, supondo que o resultado vale para $n$, provemos que ele também vale para $n^+$. Se $m \in \omega$, então, pela hipótese de indução,
                \begin{itemize}
                    \item se $m \in n$, como $n \subsetneq n^+$, temos $m \in n^+$;
                    \item se $m = n$, como $n \in n^+$, temos $m \in n^+$;
                    \item se $n \in m$, então $n^+ \in m$ ou $n^+ = m$.
                \end{itemize}
            Com isso, em qualquer caso, pelo menos uma das afirmações $m \in n^+$, $n^+ \in m$ ou $m = n^+$ é verdadeira. A exclusividade é garantida pela \Cref{prop.fund:xinyeyinxabs} e pelo \Cref{cor.fund:xnotinx}. \blackproof
        \end{enumerate}
\end{proof}

\begin{cor} \label{cor.fund:bigcupomega}
    \leavevmode
        \begin{enumerate}[leftmargin=*, align=left, label=\textbf{(\alph*)}]
            \item Para quaisquer $m,n \in \omega$, se $m \subsetneq n$, então $m \in n$.
            \item Para quaisquer $m,n \in \omega$, se $m \subsetneq n^+$, então $m \subseteq n$.
            \item $\bigcup \omega = \omega$.
        \end{enumerate}
\end{cor}

\begin{proof}
    \leavevmode
        \begin{enumerate}[leftmargin=*, align=left, label=\textbf{(\alph*)}]
            \item Se $m \subsetneq n$, então $m \neq n$. Pelo \Cref{teo.fund:tricotomiadopertence}, ou $m \in n$, ou $n \in m$. Se fosse $n \in m$, teríamos $n \in n$, uma contradição. Logo, só pode ser $m \in n$. \blackproof
            \item Como $m \subsetneq n^+$, pelo item anterior temos $m \in n^+$. Pela definição de sucessor, $m \in n \cup \{n\}$, o que implica $m \in n$ ou $m = n$. Se $m \in n$, então $m \subseteq n$ (pois todo número natural é um conjunto transitivo). Se $m = n$, trivialmente $m \subseteq n$. Em ambos os casos, $m \subseteq n$. \blackproof
            \item Se $x \in \bigcup \omega$, então existe $y \in \omega$ tal que $x \in y$, e como $y \subseteq \omega$, vem $x \in \omega$. Com isso, $\bigcup \omega \subseteq \omega$. Agora, se $x \in \omega$, então existe $x^+ \in \omega$ tal que $x \in x^+$, de modo que $x \in \bigcup \omega$. Com isso, $\bigcup \omega \supseteq \omega$. Logo, $\bigcup \omega = \omega$. \blackproof
        \end{enumerate}
\end{proof}

\begin{teo}
    \leavevmode
        \begin{enumerate}[leftmargin=*, align=left, label=\textbf{(\alph*)}]
            \item Para todo $n \in \omega$, o conjunto $(n, \subseteq)$ é bem ordenado.
            \item $(\omega, \subseteq)$ é um conjunto bem ordenado.
        \end{enumerate}
\end{teo}

\begin{proof}
    \leavevmode
        \begin{enumerate}[leftmargin=*, align=left, label=\textbf{(\alph*)}]
            \item A prova se dará por indução em $n$. Trivialmente $\emptyset$ é bem ordenado por $\subseteq$ (por vacuidade: não existem subconjuntos não vazios de $\emptyset$). Agora, suponha que $n \in \omega$ seja bem ordenado por $\subseteq$. Tome $S \in \mathcal{P}(n^+)_{\neq \emptyset}$ e defina $S' := S \setminus \{n\}$. Se $S' = \emptyset$, então $S = \{n\}$ e $S$ tem um elemento mínimo (nesse caso, $n = \min{S}$). Se $S' \neq \emptyset$, então, como $S' \subseteq n$, existe $\min{S'} \in S'$. Provemos que $\min{S'} \subseteq x$ para todo $x \in S$. Se $x \in S$, então $x \in n^+$, de modo que $x \in n$ ou $x = n$. Se $x \in n$, então, como $x \in S$ e $x \neq n$, vem $x \in S'$, donde $\min{S'} \subseteq x$. Se $x=n$, então, como $\min{S'} \in n$ já que $S' \subseteq n$, da transitividade de $n$ vem $\min{S'} \subseteq n$, de modo que $\min{S'} \subseteq x$. Com isso, $\min{S'}$ é o elemento mínimo de $S$, de modo que $n^+$ é bem ordenado por $\subseteq$. \blackproof
            \item Sejam $S \in \mathcal{P}(\omega)_{\neq \emptyset}$ e $n_0 \in S$. Como $n_0 \in S \cap n^+_0$, temos $S \cap n^+_0 \neq \emptyset$, e como $S \cap n^+_0 \subseteq n^+_0$, existe $m := \min{S \cap n^+_0}$. Agora, sendo $n \in S$, provemos que $m \subseteq n$. Se $n \in S$, então ou $n \in n_0$, ou $n = n_0$, ou $n_0 \in n$, pela tricotomia de $\in$. Se $n \in n_0$, então, como $n_0 \subseteq n^+_0$, vem $n \in S \cap n^+_0$, de modo que $m \subseteq n$. Se $n = n_0$, então $n \in n^+_0$, e como $n_0 \subseteq n^+_0$, vem $n \in S \cap n^+_0$, de modo que $m \subseteq n$. Se $n_0 \in n$, então $n_0 \subseteq n$ (transitividade de $n$), e como $m \subseteq n_0$ já que $n_0 \in S \cap n^+_0$, vem $m \subseteq n$. Com isso, $m = \min{S \cap n^+_0}$ é o elemento mínimo de $S$, de modo que $\omega$ é bem-ordenado por $\subseteq$. \blackproof
        \end{enumerate}
\end{proof}

\subsection{O Teorema da Recursão}

Para definir funções de domínio $\omega$ recursivamente, precisamos
    \begin{enumerate}
        \item estabelecer o valor da função em 0;
        \item estabelecer uma ``regra'' para definir o valor da função em $n^+$ uma vez que se conheça o seu valor em $n$.
    \end{enumerate}

\begin{teo}[da recursão finita] \label{teo.fund:recfin}
    Sejam $X$ um conjunto, $x_0 \in X$ e $f:X \to X$. Existe uma única função $\varphi : \omega \to X$ tal que
        \begin{itemize}
            \item $\varphi(0) = x_0$;
            \item $\varphi(n^+) = f(\varphi(n))$, para todo $n \in \omega$.
        \end{itemize}
\end{teo}

\begin{proof}
    A ideia é considerar todas as relações de $\omega$ em $X$ que têm as propriedades desejadas e provar que a interseção de todas elas resulta em uma única função de domínio $\omega$. Defina
        \[
            \mathcal{C} := \left\{ R \in \mathcal{P}(\omega \times X) : (0,x_0) \in R \land \forall n \forall x ((n,x) \in R \to (n^{+},f(x)) \in R) \right\}.
        \]
    Como $\omega \times X \in \mathcal{C}$, temos $\mathcal{C} \neq \emptyset$, de modo que, pelo \Cref{teo.fund:intfamilia}, existe $\varphi := \bigcap \mathcal{C}$. Essa é a função procurada. Provemos isso.
        \begin{claim}
           $\varphi \in \mathcal{C}$.
        \end{claim}
        \begin{midproof}
            Temos $(0,x_0) \in \varphi$ porque $(0,x_0) \in R$ para toda $R \in \mathcal{C}$. Analogamente, se $(n,x) \in \varphi$, então $(n,x) \in R$ para toda $R \in \mathcal{C}$, de modo que $(n^{+},f(x)) \in R$ para toda $R \in \mathcal{C}$, o que prova que $(n^{+},f(x)) \in \varphi$ e que $\varphi \in \mathcal{C}$. \whiteproof
        \end{midproof}
        \begin{claim}
            $\Dom{(\varphi)} = \omega$.
        \end{claim}
        \begin{midproof}
            Claramente, $\Dom{(\varphi)} \subseteq \omega$. Como $(0,x_0) \in \varphi$, temos $0 \in \Dom{(\varphi)}$. Agora, se $n \in \Dom{(\varphi)}$, então existe $x \in X$ tal que $(n,x) \in \varphi$. Com isso, vem $(n^+,f(x)) \in \varphi$, donde $n^+ \in \Dom{(\varphi)}$. Assim, $\Dom{(\varphi)}$ é indutivo, e como $\Dom{(\varphi)} \subseteq \omega$, temos $\Dom{(\varphi)} = \omega$. \whiteproof
        \end{midproof}
        \begin{claim}
            A relação $\varphi \subseteq \omega \times X$ é uma função.
        \end{claim}
        \begin{midproof}
            Provemos que $(n,x) \in \varphi$ e $(n,y) \in \varphi$ implicam $x=y$ por indução em $n$.
                \begin{itemize}
                    \item Base de indução. Como $(0, x_0) \in \varphi$, se existisse $y \in X$ tal que $(0, y) \in \varphi$ e $y \neq x_0$, então $(0, x_0) \in \varphi \setminus \{(0,y)\}$, e se $(m,z) \in \varphi \setminus \{(0,y)\}$, então $(m^+, f(z)) \in \varphi$, e como $m^+ \neq 0$ para todo $m \in \omega$, teríamos $(m^+, f(z)) \in \varphi \setminus \{(0,y)\}$, de modo que $\varphi \setminus \{(0,y)\} \in \mathcal{C}$, uma contradição: teríamos $\varphi \subseteq \varphi \setminus \{(0,y)\}$, mas como $\varphi \setminus \{(0,y)\} \subseteq \varphi$, seria $\varphi \setminus \{(0,y)\} = \varphi$, isto é, $(0,y) \notin \varphi$, o que contraria a hipótese $(0,y) \in \varphi$. Com isso, se $(0,y) \in \varphi$, então $y=x_0$.
                    \item Passo indutivo. Se $n \in \Dom{(\varphi)}$, então existe $x \in X$ tal que $(n,x) \in \varphi$. Suponha, por hipótese de indução, que $(n,y) \in \varphi$ implica $y=x$. De $(n,x) \in \varphi$, vem $(n^+,f(x)) \in \varphi$. Suponha que existe $y \in X$ tal que $(n^+,y) \in \varphi$ e $y \neq f(x)$. Como $n^+ \neq 0$, teríamos $(0,x_0) \in \varphi \setminus \{(n^+,y)\}$, e se $(m,z) \in \varphi \setminus \{(n^+,y)\}$, então $(m^+,f(z)) \in \varphi$. Se $m^+ = n^+$, então $m=n$, e pela hipótese de indução, teríamos $z=x$, de modo que $f(z) = f(x) \neq y$, o que implicaria $(m^+, f(z)) \in \varphi \setminus \{(n^+,y)\}$. Por outro lado, se $m \neq n$, então $m^+ \neq n^+$, de modo que $(m^+, f(z)) \in \varphi \setminus \{(n^+,y)\}$. Em qualquer caso, teríamos $\varphi \setminus \{(n^+,y)\} \in \mathcal{C}$, uma contradição (pelo mesmo motivo visto na base de indução). Logo, se $(n^+,y) \in \varphi$, então $y=f(x)$.
                \end{itemize}
            Com isso, temos que $\varphi$ é uma função de $\omega$ em $X$. \whiteproof
        \end{midproof}
    Assim, fica provada a existência de uma função $\varphi: \omega \to X$ que tem as propriedades do enunciado. Provemos, por fim, sua unicidade.
        \begin{claim}
            A função $\varphi : \omega \to X$ é única.
        \end{claim}
        \begin{midproof}
            A prova se dará por indução. Se $\Phi : \omega \to X$ é uma função tal que $\Phi(0) = x_0$ e $\Phi(n^+) = f(\Phi(n))$ para todo $n \in \omega$, então $\Phi(0) = \varphi(0)$ e, se $\Phi(n) = \varphi(n)$, então
                \[
                    \Phi(n^+) = f(\Phi(n)) = f(\varphi(n)) = \varphi(n^+),
                \]  
            de modo que $\Phi = \varphi$. \whiteproof
        \end{midproof}
    Logo, existe uma única função $\varphi: \omega \to X$ tal que $\varphi(0) = x_0$ e $\varphi(n^+) = f(\varphi(n))$ para todo $n \in \omega$. \blackproof 
\end{proof}

\begin{teo}[da recursão com parâmetro] \label{teo.fund:recpar}
    Sejam $X$ um conjunto, $x_0 \in X$ e $f : \omega \times X \to X$. Existe uma única função $\varphi : \omega \to X$ tal que
        \begin{itemize}
            \item $\varphi(0) = x_0$;
            \item $\varphi(n^+) = f(n, \varphi(n))$, para todo $n \in \omega$.
        \end{itemize}
\end{teo}

\begin{proof}
    Defina $g : \omega \times X \to \omega \times X$ por $g(n, y) = (n^+, f(n, y))$. Pelo \Cref{teo.fund:recfin}, existe uma única função $\psi : \omega \rightarrow \omega \times X$ tal que $ \psi(0) = (0, x_0)$ e $\psi(n^+) = g(\psi(n))$ para todo $n \in \omega$.
        \begin{claim}
            $\pi_1 (\psi(n)) = n$ para todo $n \in \omega$.
        \end{claim}
        \begin{midproof}
            A prova se dará por indução em $n$. Para $n=0$, como $\psi(0) = (0, x_0)$, temos que $\pi_1(\psi(0)) = 0$. Suponha, por hipótese de indução, que $\pi_1(\psi(n)) = n$. Com isso, $\psi(n) = (n, \pi_2(\psi(n)))$. Pelas definições de $\psi$ e $g$, temos 
                \[
                    \psi(n^+) = g(\psi(n)) = g(n, \pi_2(\psi(n))) = (n^+, f(n, \pi_2(\psi(n)))),
                \]
            de modo que $\pi_1(\psi(n^+)) = n^+$. \whiteproof
        \end{midproof}
    Agora, defina $\varphi := \pi_2 \circ \psi : \omega \to X$, isto é, $\varphi(n) := \pi_2(\psi(n))$ para todo $n \in \omega$. Essa é a função procurada. Provemos isso.
        \begin{claim}
            $\varphi(0) = x_0$ e $\varphi(n^+) = f(n, \varphi(n))$, para todo $n \in \omega$.
        \end{claim}
        \begin{midproof}
            Como $\psi(0) = (0,x_0)$, temos $\varphi(0) = \pi_2(0,x_0) = x_0$. Pela definição de $\varphi$, temos $\psi(n) = (n, \varphi(n))$ para todo $n \in \omega$, de modo que
                \[
                    \psi(n^+) = (n^+, f(n, \pi_2(\psi(n)))) = (n^+, f(n, \varphi(n))).
                \]
            Com isso, vem
                \[
                    \varphi(n^+) = \pi_2(\psi(n^+)) = \pi_2(n^+, f(n, \varphi(n))) = f(n,\varphi(n))
                \]
            para todo $n \in \omega$. \whiteproof 
        \end{midproof}
    Assim, fica provada a existência de uma função $\varphi: \omega \to X$ que tem as propriedades do enunciado. Provemos, por fim, sua unicidade.
        \begin{claim}
            A função $\varphi: \omega \to X$ é única.
        \end{claim}
        \begin{midproof}
            A prova se dará por indução. Se $\Phi : \omega \to X$ é uma função tal que $\Phi(0) = x_0$ e $\Phi(n^+) = f(n, \Phi(n))$ para todo $n \in \omega$, então $\Phi(0) = \varphi(0)$ e, se $\Phi(n) = \varphi(n)$, então
                \[
                    \Phi(n^+) = f(n, \Phi(n)) = f(n, \varphi(n)) = \varphi(n^+),
                \]
            de modo que $\Phi = \varphi$. \whiteproof
        \end{midproof}
    Logo, existe uma única função $\varphi: \omega \to X$ tal que $\varphi(0) = x_0$ e $\varphi(n^+) = f(n,\varphi(n))$ para todo $n \in \omega$. \blackproof
\end{proof}

\begin{defi}
    O conjunto de todas as funções de um certo $n \in \omega$ em $X$ é denotado por $X^{< \omega}$, isto é,
        \[
            X^{< \omega} := \{f \in \mathcal{P}(\omega \times X) : \exists n (n \in \omega \land f \in X^n) \}.
        \]
\end{defi}

\begin{teo}[da recursão completa] \label{teo.fund:reccom}
    Sejam $X$ um conjunto e $f :  X^{<\omega} \to X$ uma função. Existe uma única função $\varphi : \omega \to X$ tal que $\varphi(n) = f(\varphi \restriction_n)$ para todo $n \in \omega$.
\end{teo}

\begin{proof}
    Defina $h : X^{< \omega} \to X^{< \omega}$ por $h(g) := g \cup \{(\Dom{(g)}, f(g)) \}$. Pelo \Cref{teo.fund:recfin}, existe uma única função $\psi : \omega \to X^{< \omega}$ tal que $\psi(0) = \emptyset$ e $\psi(n^+) = h(\psi(n))$.
        \begin{claim} \label{fund:reccom.claim.dompsin}
            $\psi(n) \in X^n$ para todo $n \in \omega$.
        \end{claim}
        \begin{midproof}
            Dado $n \in \omega$, como $\psi(n) \in X^{< \omega}$, existe $m \in \omega$ para o qual $\psi(n) : m \to X$. Com isso, $\Im{(\psi(n))} \subseteq X$ e a afirmação equivale a provar que $m = n$, isto é, basta provar que $\Dom{(\psi(n))} = n$ para todo $n \in \omega$. Façamos isso por indução em $n$. Para $n=0$, como $\psi(0) = \emptyset$, temos $\Dom{(\psi(0))} = \Dom{(\emptyset)} = \emptyset = 0$. Suponha, por hipótese de indução, que $\Dom{(\psi(n))} = n$. Como
                \[
                    \psi(n^+) = h(\psi(n)) = \psi(n) \cup \{ (\Dom{(\psi(n))}, f(\psi(n)))\},
                \]
            obtemos
                \[
                    \Dom{(\psi(n^+))} = \Dom{(\psi(n))} \cup \{ \Dom{(\psi(n))} \},
                \]
            e como $\Dom{(\psi(n))} = n$, vem $\Dom{(\psi(n^+))} = n \cup \{ n \} = n^+$. Isso completa o passo indutivo. Como $\Dom{(\psi(n))} = n$ e $\Im{\psi(n)} \subseteq X$, vem $\psi(n) : n \to X$, isto é, $\psi(n) \in X^n$, para todo $n \in \omega$. \whiteproof
        \end{midproof}
        \begin{claim} \label{fund:reccom.claim.restricao}
            Para quaisquer $m,n \in \omega$, se $m \subseteq n$, então $\psi(n) \restriction_m = \psi(m)$.
        \end{claim}
        \begin{midproof}
            A prova se dará por indução em $n$. Para $n=0$, obtemos $\psi(0) \restriction_m = \emptyset \restriction_m = \emptyset$, e como $m \subseteq \emptyset$, vem $m = \emptyset$, de modo que $\psi(0) \restriction_{\emptyset} = \emptyset = \psi(\emptyset) = \psi(0)$. Supondo, por hipótese de indução, que $\psi(n) \restriction_m = \psi(m)$ para todo $m \subseteq n$, provemos que $\psi(n^+) \restriction_m = \psi(m)$ para todo $m \subseteq n^+$. Se $m = n^+$, então $\psi(n^+) \restriction_{n^+} = \psi(n^+)$ trivialmente. Se $m \neq n^+$, então $m \subsetneq n^+$, de modo que $m \subseteq n$ (\Cref{cor.fund:bigcupomega}). 
                \begin{itemize}
                    \item Por um lado ($\subseteq$), seja $(k,x) \in \psi(n^+) \restriction_m$. Então $(k,x) \in \psi(n^+)$, com $k \in m$. Como $\psi(n^+) = \psi(n) \cup \{(n, f(\psi(n)))\}$, vem $(k,x) \in \psi(n)$ ou $(k,x) = (n, f(\psi(n)))$. Se fosse $(k,x) = (n, f(\psi(n)))$, então seria $k=n$, de modo que $n \in n$ pois $k \in m \subseteq n$, uma contradição. Logo, só pode ser $(k,x) \in \psi(n)$, e como $k \in m$, vem $(k,x) \in \psi(m)$ pois $\psi(n) \restriction_m = \psi(m)$ pela hipótese de indução. Assim, fica provado que $\psi(n^+) \restriction_m \subseteq \psi(m)$.
                    \item Por outro lado ($\supseteq$), seja $(k,x) \in \psi(m)$. Então, como $\psi(m) = \psi(n) \restriction_m$ pela hipótese de indução, temos $(k,x) \in \psi(n) \restriction_m$. Como $\psi(n) \subseteq \psi(n^+)$, temos $(k,x) \in \psi(n^+)$, como $k \in m$, temos $(k,x) \in \psi(n^+) \restriction_m$. Assim, fica provado que $\psi(n^+) \restriction_m \supseteq \psi(m)$.
                \end{itemize}
            Com isso, vem $\psi(n^+) \restriction_m = \psi(m)$ para todo $m \subseteq n^+$. Isso completa o passo indutivo. \whiteproof
        \end{midproof}
    Agora, defina $\varphi := \bigcup \Im{(\psi)}$. Essa é a função procurada. Provemos isso.
        \begin{claim} \label{fund:reccom.claim.varphifuncao}
            $\varphi$ é uma função de $\omega$ em $X$.
        \end{claim}
        \begin{midproof}
            Comecemos com $\varphi$ ser função. Como $\varphi \subseteq \omega \times X$, provemos que se $(k,x) \in \varphi$ e $(k,y) \in \varphi$, então $x=y$. Se $(k,x), (k,y) \in \varphi = \bigcup \Im{(\psi)}$, então existem $n,m \in \omega$ tais que $(k,x) \in \psi(n)$ e $(k,y) \in \psi(m)$. Como a relação $\subseteq$ é total em $\omega$, suponha, sem perda de generalidade, que $m \subseteq n$. Com isso, $\psi(n) \restriction_m = \psi(m)$, e como $k \in m$ pois $(k,y) \in \psi(m)$, vem $(k,x) \in \psi(m)$. Como $\psi(m)$ é uma função, vem $x=y$, de modo que $\varphi$ é uma função. Agora, vejamos que $\Dom{(\varphi)} = \omega$. Como
                \[
                    \varphi = \bigcup \Im{(\psi)} = \bigcup \{\psi(n) : n \in \omega \} = \bigcup_{n \in \omega} \psi(n),
                \]
            obtemos
                \[
                    \Dom{(\varphi)} = \Dom{\left( \bigcup_{n \in \omega} \psi(n) \right)} = \bigcup_{n \in \omega} \Dom{(\psi(n))} = \bigcup_{n \in \omega} n = \bigcup \omega = \omega,
                \]
            pois $\Dom{(\psi(n))} = n$ para todo $n \in \omega$ (\Cref{fund:reccom.claim.dompsin}) e $\bigcup \omega = \omega$ (\Cref{cor.fund:bigcupomega}). Por fim, verifiquemos que $\Im{(\varphi)} \subseteq X$. Para isso, basta ver que
                \[
                    \Im{(\varphi)} = \Im{\left(\bigcup_{n \in \omega} \psi(n) \right)} = \bigcup_{n \in \omega} \Im{(\psi(n))} \subseteq X,
                \]
            pois $\Im{(\psi(n))} \subseteq X$ para todo $n \in \omega$. Com isso, $\varphi : \omega \to X$. \whiteproof
        \end{midproof}
        \begin{claim} \label{fund:reccom.claim.propriedaderecursiva}
            $\varphi(n) = f(\varphi \restriction_n)$ para todo $n \in \omega$.
        \end{claim}
        \begin{midproof}
            Seja $n \in \omega$. Temos $(n, f(\psi(n))) \in \psi(n^+)$ pois 
                \[
                    \psi(n^+) = \psi(n) \cup \{ (n, f(\psi(n)))\},
                \]
            e como $\varphi = \bigcup \Im{(\psi)}$, temos $\psi(n^+) \subseteq \varphi$, de modo que $(n, f(\psi(n))) \in \varphi$; como $\varphi : \omega \to X$ é uma função (\Cref{fund:reccom.claim.varphifuncao}), vem  $\varphi(n) = f(\psi(n))$. Para finalizarmos, vejamos que $\psi(n) = \varphi \restriction_n$. Por um lado ($\subseteq$), se $(k,x) \in \psi(n)$, então, como $\psi(n) \in \Im{(\psi)}$, temos $(k,x) \in \bigcup \Im{(\psi)} = \varphi$. Com isso, como $k \in n$ pois $\Dom{(\psi(n))} = n$ (\Cref{fund:reccom.claim.dompsin}), vem $(k,x) \in \varphi \restriction_n$. Por outro lado ($\supseteq$), se $(k,x) \in \varphi \restriction_n$, então $(k,x) \in \varphi$ com $k \in n$ e existe $m \in \omega$ tal que $(k,x) \in \psi(m)$. Se $n \subseteq m$, então $\psi(m) \restriction_n = \psi(n)$ (\Cref{fund:reccom.claim.restricao}), e como $k \in n$, temos $(k,x) \in \psi(m) \restriction_n$, de modo que $(k,x) \in \psi(n)$. Se $m \subseteq n$, então $\psi(m) = \psi(n) \restriction_m$, e como $k \in m$, temos $(k,x) \in \psi(n) \restriction_m$, isto é, $(k,x) \in \psi(n)$. Com isso, temos $\psi(n) = \varphi \restriction_n$ para todo $n \in \omega$, de modo que, por fim, como $\varphi(n) = f(\psi(n))$, temos $\varphi(n) = f(\varphi \restriction_n)$ para todo $n \in \omega$. \whiteproof
        \end{midproof}
    Com isso, fica provada a existência de uma função $\varphi: \omega \to X$ que tem as propriedades do enunciado. Provemos, por fim, sua unicidade.
        \begin{claim}
            A função $\varphi : \omega \to X$ é única.
        \end{claim}
        \begin{midproof}
            Seja $\Phi : \omega \to X$ tal que $\Phi(n) = f(\Phi \restriction_n)$ para todo $n \in \omega$. Provemos que $\Phi \restriction_n = \varphi \restriction_n$ para todo $n \in \omega$ por indução em $n$. Para $n=0$, trivialmente $\Phi \restriction_{0} = \emptyset = \varphi \restriction_{0}$. Suponha, por hipótese de indução, que $\Phi \restriction_n = \varphi \restriction_n$. Com isso,
                \begin{align*}
                    \Phi \restriction_{n^+} &= \Phi \restriction_n \cup \{ (n, \Phi(n)) \} \\
                    &= \varphi \restriction_n \cup \{ (n, f(\Phi \restriction_n)) \} \\
                    &= \varphi \restriction_n \cup \{ (n, f(\varphi \restriction_n)) \} \\
                    &= h(\varphi \restriction_n) = h(\psi(n)) \\
                    &= \psi(n^+) = \varphi \restriction_{n^+},
                \end{align*}
            o que completa o passo indutivo. Logo
                \[
                    \Phi(n) = \Phi \restriction_{n^+}(n) = \varphi \restriction_{n^+}(n) = \varphi(n)
                \]
            para todo $n \in \omega$, de modo que $\Phi = \varphi$. \whiteproof
        \end{midproof}
    Logo, existe uma única função $\varphi : \omega \to X$ tal que $\varphi(n) = f(\varphi \restriction_n)$ para todo $n \in \omega$. \blackproof
\end{proof}

\subsection{Aritmética dos Números Naturais}

\section{O Axioma da Escolha}

\begin{ax}[da Escolha]
    Para todo conjunto $x$ de conjuntos não vazios existe uma função $\varphi : x \to \bigcup x$ tal que $\varphi(y) \in y$ para todo $y \in x$.
        \[
            \boxed{
                \forall x \left(\emptyset \notin x \rightarrow \exists \varphi \left( \varphi : x \to \bigcup x \land \forall y (y \in x \rightarrow \varphi(y) \in y) \right)\right)
            }        
        \]
\end{ax}

\begin{obs}
    Usamos a sigla \textsf{AC} (do inglês \textit{Axiom of Choice}) para nos referirmos ao axioma da escolha. Por seu caráter não construtivo, o axioma da escolha é o axioma mais controverso da matemática, evitado por uns e usado indiscriminadamente por outros. Desastres acontecem com e sem \textsf{AC}: por exemplo, sem \textsf{AC}, muitos resultados matemáticos fundamentais falham, sendo equivalentes em \textsf{ZF} a \textsf{AC} ou a alguma forma fraca de \textsf{AC}.
\end{obs}

\begin{prop}
    O axioma da escolha é equivalente à seguinte afirmação.
        \[
            \forall x \exists \varphi \left( \varphi : x \setminus \{\emptyset\} \to \bigcup x \land \forall y (y \in x \setminus \{\emptyset\} \rightarrow \varphi(y) \in y) \right).
        \]
\end{prop}

\begin{proof}
    \blackproof
\end{proof}

\begin{defi}
    Uma \textit{sequência} em $X$ \textit{indexada} $I$ é uma função $x : I \to X$. Isso é denotado por $\left( x_i \right)_{i \in I}$.
        \begin{enumerate}[leftmargin=*, align=left, label=\textbf{(\alph*)}]
            \item Denotamos por $x_i$ a imagem de $i \in I$ pela sequência $x : I \to X$, isto é, $x_i := x(i)$.
            \item Denotamos por $\left( x_i \right)_{i \in I}$ a sequência $x : I \to X$.
            \item Denotamos por $\{x_i : i \in I \}$ a imagem da sequência $\left( x_i \right)_{i \in I}$.
            \item Denotamos por $\bigcup_{i \in I} x_i$ a união da imagem da sequência $\left( x_i \right)_{i \in I}$, isto é, $\bigcup_{i \in I} x_i := \bigcup \left\{x_i : i \in I \right\}$ 
        \end{enumerate}
\end{defi}

\begin{defi}
    O \textit{produto cartesiano} de uma sequência $\left(x_i \right)_{i \in I}$ é definido como
        \[
           \prod_{i \in I} x_i := \left\{ f \in \left(\bigcup_{i \in I} x_i\right)^{I} : \forall i(i \in I \rightarrow f(i) \in x_i) \right\},
        \]
    isto é, $\prod_{i \in I} x_i$ é definido como o conjunto de todas as funções $f$ de domínio $I$ tais que $f(i) \in x_i$ para todo $i \in I$.
\end{defi}

\begin{prop}
    O axioma da escolha é equivalente à seguinte afirmação: se $\left(X_i \right)_{i \in I}$ é uma sequência com $X_i \neq \emptyset$ para todo $i \in I$, então $\prod_{i \in I} X_i \neq \emptyset$.
\end{prop}

\begin{proof}
    Usemos a notação usual de função escrevendo $g = (X_i)_{i \in I}$.

    ($\Rightarrow$) Como $g(i) \neq \emptyset$ para todo $i \in I$, temos $\emptyset \notin \Im{(g)}$, de modo que, pelo axioma da escolha, existe uma função $\varphi : \Im{(g)} \to \bigcup \Im{(g)}$ tal que $\varphi(y) \in y$ para todo $y \in \Im{(g)}$. Tomando $f := \varphi \circ g : I \to \bigcup \Im{g}$, temos, para todo $i \in I$,
        \[
            f(i) = (\varphi \circ g)(i) = \varphi[g(i)] \in g(i),
        \]
    isto é, $f(i) \in g(i)$. Como $g(i) := X_i$, vem $f(i) \in X_i$ para todo $i \in I$, de modo que $f \in \prod_{i \in I} X_i$, isto é, $\prod_{i \in I} X_i \neq \emptyset$.

    $(\Leftarrow)$ Agora, seja $x \neq \emptyset$ tal que $\emptyset \notin x$. Defina $g = (X_i)_{i \in I}$ assim: $I = x$ e $g := \id_I$. Como $X_i = g(i) = \id_I(i) = i \in I = x$ e $\emptyset \notin x$, temos $X_i \neq \emptyset$ para todo $i \in I$, de modo que $\prod_{i \in I} X_i \neq \emptyset$. Com isso, existe $\varphi \in \prod_{i \in I} X_i$ tal que $\varphi(i) \in X_i$ para todo $i \in I$, e como $X_i = i$ e $I = x$, vem $\varphi(i) \in i$ para todo $i \in x$, de modo que $\varphi : x \to \bigcup x$ é uma função de escolha em $x$. \blackproof
\end{proof}

\begin{defi}
    Sejam $(X, \leq)$ um conjunto parcialmente ordenado e $S$ um subconjunto não vazio de $X$.
        \begin{enumerate}[leftmargin=*, align=left, label=\textbf{(\alph*)}]
            \item Dizemos que $x \in S$ é \textit{maximal} em $S$ se não existe $y \in S$ tal que $x < y$.
            \item Dizemos que $x \in S$ é \textit{minimal} em $S$ se não existe $y \in S$ tal que $y < x$.
            \item Dizemos que $S$ é uma cadeia em $X$ se
                \[
                    \forall x \forall y (x,y \in S \rightarrow (x \leq y \lor y \leq x)).
                \]
        \end{enumerate}
\end{defi}

\begin{teo}[Lema de Zorn]
    Seja $(X, \leq)$ um conjunto parcialmente ordenado. Se toda cadeia em $X$ é limitada superiormente em $X$, então $X$ possui um elemento maximal. 
\end{teo}

\begin{proof}
    \blackproof
\end{proof}

\part{Números Reais}

%!TEX root = main.tex

\chapter{Números Reais como na Análise}

\section{Corpos}

\begin{defi} \label{reais:def.corpo}
    Uma tripla $(\F, +, \cdot)$ é um \textit{corpo} se no conjunto $\F \neq \emptyset$ existem duas operações, $+ :\F \times \F \to \F$ e $\cdot: \F \times \F \to \F$, para as quais
        \begin{itemize}
            \item A1: $x + (y + z) = (x + y) + z$ para quaisquer $x,y,z \in \F$;
            \item A2: $x + y = y + x$ para quaisquer $x,y \in \F$; 
            \item A3: existe $0 \in \F$ tal que $x + 0 = x$ para todo $x \in \F$;
            \item A4: para cada $x \in \F$ existe $y \in \F$ tal que $x+y=0$;
            \item M1: $x \cdot (y \cdot z) = (x \cdot y) \cdot z$ para quaisquer $x,y,z \in \F$; 
            \item M2: $x \cdot y = y \cdot x$ para quaisquer $x,y \in \F$;
            \item M3: existe $1 \in \F_{\neq 0}$ tal que $x\cdot 1 = x$ para todo $x \in \F$;
            \item M4: para cada $x \in \F_{\neq 0}$ existe $y \in \F$ tal que $x \cdot y = 1$;
            \item D: $x \cdot ( y + z) = x \cdot y + x \cdot z$ para quaisquer $x,y,z \in \F$.
        \end{itemize}
    Para simplificar a notação, e quando não houver perigo de confusão, vamos nos referir ao corpo $(\F, +, \cdot)$ simplesmente como o conjunto $\F$.
\end{defi}

\begin{obs}
    As operações $+$ e $\cdot$ são chamadas, respectivamente, de \textit{adição} e \textit{multiplicação}. As propriedades descritas em A1 e M1 são chamadas de \textit{associatividade}; em A2 e M2, de \textit{comutatividade}; em A3 e M3, de existência de \textit{elementos neutros}; em A4, de existência de um \textit{oposto aditivo}; em M4, de existência de um \textit{inverso multiplicativo}; e em D, de \textit{distributividade}.
\end{obs}

\begin{prop}
    Seja $(\F,+,\cdot)$ um corpo. Valem as seguintes afirmações.
        \begin{enumerate}[leftmargin=*, align=left, label=\textbf{(\alph*)}]
            \item (Unicidade)
                \begin{enumerate}[label=\roman*.]
                    \item O elemento neutro $0$ de $+$ é único.
                    \item O elemento neutro $1$ de $\cdot$ é único.
                    \item O inverso multiplicativo de cada elemento de $\F_{\neq 0}$ é único.
                \end{enumerate}
            \item (Leis do corte) Para quaisquer $x,y,z \in \F$, temos
                \begin{enumerate}[label=\roman*.]
                    \item $x+z=y+z \Rightarrow x=y$;
                    \item $x \cdot z = y \cdot z \ \text{e} \ z \in \F_{\neq 0} \Rightarrow x = y$.
                \end{enumerate}
            \item (Integridade) Para quaisquer $x,y \in \F$, temos
                \begin{enumerate}[label=\roman*.]
                    \item $x \cdot 0 = 0$;
                    \item $x \cdot y = 0 \Rightarrow x = 0 \ \text{ou} \ y = 0$;
                \end{enumerate}
            \item (Regras dos sinais) Para quaisquer $x,y \in \F$, temos
                \begin{enumerate}[label=\roman*.]
                    \item $(-1) \cdot x = -x$;
                    \item $-(-x) = x$;
                    \item $(-x) \cdot y = x \cdot (-y) = - (x \cdot y)$;
                    \item $(-x) \cdot (-y) = x \cdot y$.
                \end{enumerate}
            \item Para quaisquer $x, y \in \F$, temos
                \[
                    x^2 = y^2 \Leftrightarrow x = y \ \text{ou} \ x = -y.
                \]
        \end{enumerate}
\end{prop}

\begin{proof}
\end{proof}

\begin{defi}
    Um corpo $(\F,+,\cdot)$ é \textit{ordenado} se existe uma relação $\leq \ \subseteq \F \times \F$ tal que $(\F, \leq)$ é um conjunto totalmente ordenado e
        \begin{enumerate}[label=\roman*.]
            \item para quaisquer $x,y,z \in \F$, se $x \leq y$, então $x + z \leq y + z$;
            \item para quaisquer $x,y,z \in \F$, se $x \leq y$ e $0 \leq z$, então $x \cdot z \leq y \cdot z$.
        \end{enumerate}
    Isso é denotado por $(\F, +, \cdot, \leq)$.
\end{defi}

\begin{prop}
    Seja $(\F, +, \cdot, \leq)$ um corpo ordenado.
        \begin{enumerate}[leftmargin=*, align=left, label=\textbf{(\alph*)}]
            \item A relação $< \ \subseteq \F \times \F$ definida como
                \[
                    < \ := \{(x,y) \in \F \times \F : x \leq y \land x \neq y\}
                \]
            é de ordem estrita total.
            \item Existe um subconjunto $\F_{>0} \subseteq \F$ tal que
                \begin{enumerate}[label=\roman*.]
                    \item se $x,y \in \F_{>0}$, então $x+y \in \F_{>0}$ e $x \cdot y \in \F_{>0}$;
                    \item se $x \in \F$, então ou $x=0$, ou $x \in \F_{>0}$, ou $-x \in \F_{>0}$, exclusivamente.
                \end{enumerate}
        \end{enumerate}
\end{prop}

\begin{proof}
    \leavevmode
        \begin{enumerate}[leftmargin=*, align=left, label=\textbf{(\alph*)}]
            \item 
            \item Como a notação ``$\F_{>0}$'' sugere, basta tomar $\F_{>0} := \{x \in \F : x > 0\}$, onde $y > x$ significa $x < y$.
        \end{enumerate}
\end{proof}

\begin{obs}
    Sendo $\F$ um corpo ordenado, escrevemos $x<y$ quando $y>x$ e $x \leq y$ quando $y \geq x$.
\end{obs}

\begin{prop}
    Propriedades cringe de ordem
\end{prop}

\begin{defi}
    Seja $\F$ um corpo ordenado. A função $|\cdot| : \F \to \F_{\geq 0}$ definida por
        \[ |x|:=
            \begin{cases}
                x & \text{ se }  x \in \F_{\geq 0} \\
                -x & \text{ se } x \in \F_{<0}
            \end{cases}
        \]
    é chamada de \textit{função modular}. O \textit{módulo}, ou o \textit{valor absoluto}, de $x \in \F$, é a imagem de $x$ pela função modular, isto é, $|x| \in \F_{\geq 0}$.
\end{defi}

\begin{prop}
    Seja $\F$ um corpo ordenado. Valem as seguintes afirmações.
        \begin{enumerate}[leftmargin=*, align=left, label=\textbf{(\alph*)}]
            \item $x \leq |x|$ para todo $x \in \F$;
            \item $|x \cdot y| = |x| \cdot |y|$ para quaisquer $x,y \in \F$;
            \item $|x+y| \leq |x|+|y|$ para quaisquer $x,y \in \F$;
            \item $|x| - |y| \leq ||x|-|y|| \leq |x-y|$ para quaisquer $x,y \in \F$;
            \item $|x-z| \leq |x-y| + |y-z|$;
            \item $|x| \leq \epsilon \Leftrightarrow -\epsilon \leq x \leq \epsilon$ para quaisquer $x \in \F$ e $\epsilon \in \F_{>0}$.
        \end{enumerate}
\end{prop}

\begin{proof}
    Ver \cite{johnpfaffen}, teorema 4.5, página 14. \itemproof
\end{proof}



\section{Números Naturais}

Em toda esta seção, $(\F, +, \cdot, \leq)$ é um corpo ordenado qualquer.

\begin{defi}
    Um subconjunto $I \subseteq \F$ é \textit{indutivo} se $1 \in I$ e $n \in I \Rightarrow n+1 \in I$. Isso é denotado por $\ind{I}$.
\end{defi}

\begin{ex}
    $\F$ é um conjunto indutivo. Com isso, o conjunto de todos os subconjuntos indutivos de $\F$ é não vazio, isto é, $\{ I \in \mathcal{P}{(\F)} : \ind{(I)} \} \neq \emptyset$. Em particular, isso nos permite considerar $\bigcap \{ I \in \mathcal{P}{(\F)} : \ind{(I)} \}$.
\end{ex}

\begin{prop}
    Se $\mathcal{A}$ é uma coleção não vazia de subconjuntos indutivos de $\F$, isto é, se 
        \[
            \mathcal{A} \in \mathcal{P}{(\{ I \in \mathcal{P}{(\F)} : \ind{(I)} \})}_{\neq \emptyset},
        \]
    então $\bigcap \mathcal{A}$ é um conjunto indutivo.
\end{prop}

\begin{proof} 
    Como $1 \in A$ para todo $A \in \mathcal{A}$, temos $1 \in \bigcap \mathcal{A}$. Agora, se $n \in \bigcap \mathcal{A}$, então $n \in A$ para todo $A \in \mathcal{A}$; como cada $A \in \mathcal{A}$ é indutivo, temos $n + 1 \in A$, donde $n+1 \in \bigcap \mathcal{A}$. \itemproof
\end{proof}

\begin{defi}
    O conjunto dos números naturais é definido como o menor subconjunto indutivo de $\F$:
        \[
            \N_{\F} := \bigcap \{ I \in \mathcal{P}{(\F)} : \ind{(I)} \}.
        \]
    % Em particular, $\N_0 := \N \cup \{0\}$.
\end{defi}

\begin{obs}
    Explicação
\end{obs}

\begin{teo}[Indução] \label{teo.reais:indução}
    \leavevmode
        \begin{enumerate}[leftmargin=*, align=left, label=\textbf{(\alph*)}]
            \item Se um subconjunto $A \subseteq \N$ é indutivo, então $A = \N$.
            \item Seja $s(n)$ uma proposição bem definida para cada $n \in \N$. Se $s(1)$ é verdadeira e se $s(n+1)$ é verdadeira sempre que $s(n)$ é verdadeira, então $s(n)$ é verdadeira para todo $n \in \N$.
        \end{enumerate}
\end{teo}

\begin{proof}
    \leavevmode
        \begin{enumerate}[leftmargin=*, align=left, label=\textbf{(\alph*)}]
            \item Se $A$ é um conjunto indutivo, então, pela definição de $\N$, temos $\N \subseteq A$. Daí, se $A \subseteq \N$, então $A = \N$. \itemproof
            \item Definindo $A := \{ n \in \N : s(n)\}$, temos $A \subseteq \N$. Além disso, $1 \in A$ e $n+1 \in A$ sempre que $n \in A$, de modo que $A$ é indutivo. Com isso, $\N \subseteq A$, de modo que $A = \N$, isto é, vale $s(n)$ para todo $n \in \N$. \itemproof
        \end{enumerate}
\end{proof}

\begin{prop}
    \leavevmode
        \begin{enumerate}[leftmargin=*, align=left, label=\textbf{(\alph*)}]
            \item Para quaisquer $m,n \in \N$ tem-se $m+n \in \N$.
            \item Para quaisquer $m,n \in \N$ tem-se $m \cdot n \in \N$.
            \item Para qualquer $n \in \N$, tem-se $n \geq 1$. Isso significa, em particular, que $\N$ é limitado inferiormente.
        \end{enumerate}
\end{prop}

\begin{proof}
    \leavevmode
        \begin{enumerate}[leftmargin=*, align=left, label=\textbf{(\alph*)}]
            \item Fixe $m \in \N$ e defina $A := \{n \in \N : m + n \in \N \}$. Temos $1 \in A$ pois se $m \in \N$ então $m+1 \in \N$ já que $\N$ é indutivo. Agora, se $n \in A$, então $m + n \in \N$, e como $\N$ é indutivo vem $(m+n)+1 \in \N$, isto é, $m+(n+1) \in \N$, de modo que $n+1 \in A$. Com isso, $A$ é indutivo, isto é, $\N \subseteq A$. Como $A \subseteq \N$ pela definição de $A$, segue que $A = \N$. Como $m$ foi fixado arbitrariamente, segue que $m + n \in \N$ para quaisquer $m,n \in \N$. \itemproof
            \item Fixe $m \in \N$ e defina $A := \{n \in \N : m \cdot n \in \N \}$. Temos $1 \in A$ pois $m \cdot 1 = m \in \N$. Agora, se $n \in A$, então $m \cdot n \in \N$, e pelo item anterior $m \cdot n + m \in \N$, isto é, $m \cdot (n+1) \in \N$, de modo que $n+1 \in A$. Com isso, $A$ é indutivo, isto é, $\N \subseteq A$. Como $A \subseteq \N$ pela definição de $A$, segue que $A = \N$. Como $m$ foi fixado arbitrariamente, segue que $m \cdot n \in \N$ para quaisquer $m,n \in \N$. \itemproof
            \item Definindo $A := \{ n \in \N : n \geq 1\}$, temos $A \subseteq \N$. Claramente $1 \in A$ pois $1 \geq 1$. Agora, se $n \in A$, então $1>0 \Rightarrow n+1 > n \geq 1 \Rightarrow n+1 \geq 1$, de modo que $n+1 \in A$. Com isso $\N \subseteq A$, donde $A = \N$. \itemproof
        \end{enumerate}
\end{proof}

\begin{lem} \label{lema.reais:discretude}
    \leavevmode
        \begin{enumerate}[leftmargin=*, align=left, label=\textbf{(\alph*)}]
            \item Para qualquer $n \in \N$, se $n \neq 1$, então $n-1 \in \N$.
            \item Para quaisquer $m,n \in \N$, se $n<m$, então $m-n \in \N$.
            \item Para qualquer $n \in \N$ não existe $m \in \N$ tal que $n < m < n+1$.
        \end{enumerate}
\end{lem}

\begin{proof}
    \leavevmode
        \begin{enumerate}[leftmargin=*, align=left, label=\textbf{(\alph*)}]
            \item Suponha que existe $p \in \N$ com $p \neq 1$ tal que $p-1 \notin \N$, e seja $A = \N \setminus \{p\}$. Como $1 \in \N$ e $p \neq 1$, temos $1 \in A$. Agora, se $n \in A$, então $n \neq p$, e também $n+1 \neq p$ (se fosse $n+1 = p$, então $p-1 = n \in \N$, mas supomos $p-1 \notin \N$), de modo que $A$ é indutivo. Assim, $\N \setminus \{ p\} = \N$, uma clara contradição, de modo que não existe tal $p$. \itemproof
            \item Definindo $A := \{n \in \N : \forall m (m \in \N \land n < m \Rightarrow m - n \in \N) \}$, temos $A \subseteq \N$. Para todo $m \in \N$ com $m > 1$, temos $m \neq 1$, de modo que $m-1 \in \N$ pelo item anterior. Com isso, $1 \in A$. Agora, se $n \in A$, então para todo $m \in \N$ com $m > n$ tem-se $m-n \in \N$, e precisamos provar que $n+1 \in A$, isto é, que para todo $m \in \N$ com $m > n+1$ tem-se $m-(n+1) \in \N$. Se $m>n+1$, então $m>n+1>n$, de modo que, pela hipótese de indução, temos $m-n \in \N$. Agora, como $m>n+1$, temos $m-n \neq 1$, e como $m-n \in \N$, pelo item anterior temos $(m-n)-1 \in \N$, isto é, $m-(n+1) \in \N$, de modo que $n+1 \in A$. Com isso, $A$ é indutivo, isto é, $\N \subseteq A$, de modo que $A = \N$. \itemproof
            \item Sendo $n \in \N$, suponha que existe $m \in \N$ tal que $n < m < n+1$. Daí, $m-n<1$ e, pelo item anterior, $m - n \in \N$, absurdo! Logo, tal $m$ não pode existir. \itemproof
        \end{enumerate}
\end{proof}

\begin{teo}[Princípio da Boa Ordenação]
    Todo subconjunto não vazio de números naturais possui um elemento mínimo. Isto é, se $A \in \mathcal{P}{(\N)}_{\neq \emptyset}$, então existe $a \in A$ tal que $a \leq x$ para todo $x \in A$.
\end{teo}

\begin{proof}
    Suponha por contradição que existe um subconjunto $A \in \mathcal{P}{(\N)}_{\neq \emptyset}$ que não tem um elemento mínimo. Definindo $[n] := \{ x \in \N : x \leq n \}$ e $X := \{ n \in \N : [n] \cap A = \emptyset \}$, temos $1 \in X$ (de fato, se $1 \notin X$, então  $[1] \cap A \neq \emptyset$, de modo que $1 \in A$ seria o elemento mínimo de $A$, absurdo!). Agora, se $n \in X$, então $[n] \cap A = \emptyset$; daí, se fosse $n+1 \in A$, este seria o elemento mínimo de $A$, absurdo! Logo, só pode ser $n+1 \in X$, de modo que $X$ é indutivo e $\N \subseteq X$. Como $X \subseteq \N$ por definição, temos que $X = \N$, donde $A = \emptyset$, uma contradição. \itemproof
\end{proof}

\begin{proof}
    Suponha por contradição que existe um subconjunto $A \in \mathcal{P}{(\N)}_{\neq \emptyset}$ que não tem um elemento mínimo. Definindo
        \[
            X := \{ n \in \N : \forall r (r \in \N \land 1 \leq r \leq n \Rightarrow r \in \N \setminus A) \},
        \]
    temos $1 \in X$. De fato, se fosse $1 \notin X$, então existiria $r \in \N$ tal que $1 \leq r \leq 1$ e $r \notin \N \setminus A$, isto é, teríamos $1 \in A$, de modo que $A$ teria um elemento mínimo, uma contradição. Agora, provemos que se $n \in X$ então $n+1 \in X$. Se fosse $n+1 \notin X$, então teríamos $n+1 \in A$, e como $A$ não tem um elemento mínimo existiria $p \in A$ tal que $p < n+1$. Como $p \in \N$, pelo lema \eqref{lema.reais:discretude} teríamos $1 \leq p \leq n$ (não poderia ser $n < p < n+1$ justamente pelo lema), mas como $n \in X$ teríamos $p \in \N \setminus A$, uma contradição pois $p \in A$. Com isso, $n+1 \in X$ e $X$ é indutivo, de modo que, $X = \N$. Logo, para todo $n \in \N$ temos $n \in \N \setminus A$, isto é, $A = \emptyset$, de modo que não existe um subconjunto não vazio de $\N$ que não tenha um elemento mínimo. \itemproof
\end{proof}

\begin{proof}
    Defina $X := \{ n \in \N : [n] \subseteq \N \setminus A\}$. Se $1 \in A$, então $1 = \min A$. Se $1 \notin A$, então $[1] = \{ 1 \} \subset \N \setminus A$, de modo que $1 \in X$. Agora, como $A \neq \emptyset$, temos que $X \neq \N$. Se $1 \in X$ e $X \neq \N$, então existe $n_0 \in X$ tal que $n_0 + 1 \notin X$ (se um tal $n_0$ não existisse, $X$ seria indutivo e teríamos $X = \N$), isto é, $[n_0] \cap A = \emptyset$ e $[n_0+1] \cap A \neq \emptyset$. Com isso, temos $n_0 + 1 \in A$, sendo este o elemento mínimo de $A$ (pois, pelo lema \eqref{lema.reais:discretude}, não existe um natural entre $n_0$ e $n_0 + 1$). \itemproof
\end{proof}

\begin{cor}
    Todo subconjunto não vazio de números naturais limitado superiormente possui um elemento máximo. Isto é, se $A \in \mathcal{P}{(\N)}_{\neq \emptyset}$ é limitado superiormente, então existe $a \in A$ tal que $x \leq a$ para todo $x \in A$.
\end{cor}

\begin{proof}
    
\end{proof}

\begin{teo}[Indução forte]
    Se $A \subseteq \N$ é tal que $n \in A$ sempre que $m \in A$ para todo $m < n$, então $A = \N$.
\end{teo}

\begin{proof}
    Provemos que $X := \N - A$ é vazio. De fato, se $X \neq \emptyset$, então pela Boa Ordenação existiria $p \in X$ mínimo; daí, todo $m < p$ seria $m \in A$, de modo que, pela definição de $A$, $p \in A$, absurdo! Logo, $X = \emptyset$. \itemproof
\end{proof}

\begin{defi}
    Um corpo ordenado $\F$ é \textit{arquimediano} se para quaisquer $a \in \F_{>0}$ e $b \in \F$ existe $n \in \N_{\F}$ tal que $n \cdot a > b$.
\end{defi}

\begin{teo} \label{teo.reais:arquimedes}
    $\N_{\F}$ é ilimitado superiormente em $\F$ se, e somente se,
        \begin{enumerate}[label=\roman*.]
            \item $\F$ é arquimediano;
            \item para todo $\epsilon \in \F_{>0}$ existe $n \in \N$ tal que $0 < \frac{1}{n} < \epsilon$.
        \end{enumerate}
\end{teo}

\begin{proof}
    Ver \cite{cursodeanalise1}, teorema 3 do capítulo 3. \itemproof    
\end{proof}

\subsection{A Unicidade dos Números Naturais}

\begin{teo}[da Recursão]
    Sejam $(\F, +_{\F}, \cdot_{\F}, \leq_{\F})$ corpo ordenado, $X$ um conjunto, $a \in X$ e $f:X \to X$ uma função. Existe e é única a função $\varphi : \N_{\F} \to X$ tal que
        \begin{itemize}
            \item $\varphi(1_{\F}) = a$;
            \item $\varphi(n +_{\F} 1_{\F}) = f(\varphi(n))$, para todo $n \in \N_{\F}$.
        \end{itemize}
\end{teo}

\begin{proof}
    Para simplificar a notação vamos omitir o índice $_{\F}$. \itemproof
\end{proof}

\begin{teo}
    Sejam $(\F_1, +_1, \cdot_1, \leq_1)$ e $(\F_2, +_2, \cdot_2, \leq_2)$ corpos ordenados. Existe uma única bijeção $\varphi : \N_{\F_1} \to \N_{\F_2}$ tal que,
        \begin{itemize}
            \item $\varphi (m +_1 n) = \varphi(m) +_2 \varphi(n)$,
            \item $\varphi (m \cdot_1 n) = \varphi(m) \cdot_2 \varphi(n)$, e
            \item $m \leq_1 n \Leftrightarrow \varphi(m) \leq_2 \varphi(n)$
        \end{itemize}
    para quaisquer $m,n \in \N_{\F_1}$, 
    %$\varphi (m +_1 n) = \varphi(m) +_2 \varphi(n)$, $\varphi (m \cdot_1 n) = \varphi(m) \cdot_2 \varphi(n)$ e $m \leq_1 n \Leftrightarrow \varphi(m) \leq_2 \varphi(n)$.
    %Os conjuntos $\N_{\F_1}$ e $\N_{\F_2}$ são isomorfos.
\end{teo}

\begin{proof}
    \itemproof
\end{proof}

\begin{comment}
\subsection*{Algumas equivalências}

O objetivo desta subseção é provar que, em $\N$, indução, indução forte e boa ordenação são equivalentes.
\end{comment}

\section{Conjuntos Finitos}

Seguimos \cite{cursodeanalise1} e \cite{analisereal1} de perto.

\begin{defi}
    Um conjunto $X \neq \emptyset$ é \textit{finito} se existem um natural $n \in \N$ e uma bijeção $f: [n] \to X$. Isso é denotado por $|X|=n$. O natural $n$ é o \textit{número de elementos} de $X$, enquanto $f$ é uma \textit{contagem dos elementos} de $X$. Em particular, o conjunto vazio $\emptyset$ é finito e tem $0$ elementos.
\begin{comment}
    \leavevmode
        \begin{enumerate}[leftmargin=*, align=left, label=\textbf{(\alph*)}]
            \item Um conjunto $X \neq \emptyset$ é dito \textit{finito} quando existem um natural $n \in \N$ e uma bijeção $f: [n] \to X$. Denotamos isso por $|X|=n$. O natural $n$ é chamado de \textit{número de elementos} de $X$, enquanto $f$ é chamada de \textit{contagem dos elementos} de $X$. Se $X = \emptyset$, diremos que $X$ é também finito e tem $0$ elementos.

            \item Diremos que $X \subseteq \N$ é \textit{limitado} se existir $n \in \N$ tal que $x \leq n$ para todo $x \in X$.
        \end{enumerate}
\end{comment}
\end{defi}

\begin{teo} \label{teo:1}
    \leavevmode
        \begin{enumerate}[leftmargin=*, align=left, label=\textbf{(\alph*)}]
            \item Para todo $n \in \N$, não existe uma bijeção $f : A \subsetneq [n] \to [n]$.

            \item Para todo $n \in \N$, se existe uma bijeção $f: [n] \to A \subseteq [n]$, então $A = [n]$.
        \end{enumerate}
\end{teo}

\begin{proof}
    \leavevmode
        \begin{enumerate}[leftmargin=*, align=left, label=\textbf{(\alph*)}]
            \item Comecemos com um lema: se existe uma bijeção $f:X \to Y$, então, dados $a \in X$ e $b \in Y$, existe também uma bijeção $g: X \to Y$ tal que $g(a)=b$. De fato, como $f$ é sobrejetora, então existe $a' \in X$ tal que $f(a') = b$; sendo $b' = f(a)$, definamos $g: X \to Y$ pondo $g(a) = b$, $g(a')=b'$ e $g(x) = f(x)$ para todo $x \neq a, a'$ em $X$. É fácil ver que $g$ é uma bijeção. Agora, seja $n_0$ o menor natural para o qual existe uma bijeção $f: A \subsetneq [n_0] \to [n_0]$. Se $n_0 \in A$, então, pelo lema, existe uma bijeção $g: A \subsetneq [n_0] \to [n_0]$ com $g(n_0) = n_0$; daí, a restrição $\tilde{g} : A \setminus \{n_0\} \subsetneq [n_0 -1] \to [n_0 -1]$ é uma bijeção, o que contraria a minimalidade de $n_0$. Por outro lado, se $n_0 \notin A$, então $A \subsetneq [n_0 -1]$; tomando $a \in A$ com $f(a) = n_0$, a restrição $\tilde{f} : A \setminus \{a\} \subsetneq A \subseteq [n_0 -1] \to [n_0 -1]$ é uma bijeção, o que, novamente, contraria a minimalidade de $n_0$. \itemproof

            \item Basta ver que esse enunciado é a contrapositiva do item anterior. No entanto, ainda assim, daremos uma outra prova, que se dará por indução em $n$. Evidentemente, o resultado vale para $n=1$. Agora, supondo que o resultado vale para $n \in \N$, tomando uma bijeção $f : [n+1] \to A \subseteq [n+1]$ provaremos que $A = [n+1]$. Sendo $a := f(n+1)$, a restrição $\tilde{f} : [n] \to A \setminus \{a \}$ é uma bijeção.
                \begin{itemize}
                    \item Se for $A \setminus \{ a \} \subseteq [n]$, então, pela hipótese de indução, $A \setminus \{a \} = [n]$, donde $a=n+1$ e $A = [n+1]$.
                    \item Se for $A \setminus \{ a \} \not\subseteq [n]$, então $n+1 \in A \setminus \{ a \}$ e existe $p \in [n]$ tal que $f(p) = n+1$. Agora, definindo a bijeção $g: [n+1] \to A \subseteq [n+1]$ por
                        \[
                            g(x) = 
                                \begin{cases}
                                    f(x), & \text{se } x \neq p \text{ e } x \neq n+1 \\
                                    a, & \text{se } x = p \\
                                    n+1, & \text{se } x=n+1
                                \end{cases},
                        \]
                    a restrição $\tilde{g} : [n] \to A \setminus \{n+1\}$ é uma bijeção; daí, como $A \setminus \{ n+1\} \subseteq [n]$, pela hipótese de indução $A \setminus \{ n+1\}  = [n]$, donde $A = [n+1] $.
                \end{itemize}
            Com isso, temos $A = [n+1]$ em ambos os casos, como queríamos provar. \itemproof
        \end{enumerate}
\end{proof}

\begin{cor}
    \leavevmode
        \begin{enumerate}[leftmargin=*, align=left, label=\textbf{(\alph*)}]
            \item O número de elementos de um conjunto finito está bem definido. Isto é, se $f:[m] \to X$ e $g:[n] \to X$ são bijeções, então $m=n$.

            \item (Princípio bijetivo) Sejam $A,B \neq \emptyset$ conjuntos finitos. Temos $|A|=|B|$ se, e somente se, existe uma bijeção $f:A \to B$.
        \end{enumerate}
\end{cor}

\begin{proof}
    \leavevmode
        \begin{enumerate}[leftmargin=*, align=left, label=\textbf{(\alph*)}]
            \item Se fosse $m<n$, teríamos $[m] \subsetneq [n]$, donde existiria uma bijeção $g^{-1} \circ f : [m] \to [n]$, o que contraria o teorema \eqref{teo:1}. Analogamente, se fosse $n<m$, teríamos $[n] \subsetneq [m]$, donde existiria uma bijeção $f^{-1} \circ g : [n] \to [m]$, o que novamente contraria o teorema \eqref{teo:1}! Logo, só pode ser $m=n$.

            Uma outra prova é o que segue. Se $h = g^{-1} \circ f :[m] \to [n]$ é uma bijeção, então $m=n$. De fato, se $m \leq n$, então $[m] \subseteq [n]$, e como $h^{-1} : [n] \to [m] \subseteq [n]$, pelo teorema \eqref{teo:1} só pode ser $[m]= [n]$, donde $m=n$. \itemproof
            
            \item Como $A,B \neq \emptyset$ são finitos, existem $n,m \in \N$ e bijeções $g: [n] \to A$ e $h : [m] \to B $.

            ($\Rightarrow$) Sendo $|B| = n$, existe uma bijeção $\varphi : [n] \to B$, donde $ g^{-1} \circ \varphi : A \to B$ é uma bijeção.

            ($\Leftarrow$) Existindo uma bijeção $f : A \to B$, temos que $g^{-1} \circ f^{-1} \circ h : [m] \to [n]$ é também uma bijeção, donde $m=n$, isto é, $|A|=|B|$. \itemproof
        \end{enumerate}
\end{proof}

\subsection{Resultadinhos}

\begin{prop}
    Seja $X$ um conjunto finito.
        \begin{enumerate}[leftmargin=*, align=left, label=\textbf{(\alph*)}]
            \item Para todo subconjunto próprio $Y \subsetneq X$ não existe uma bijeção $f : X \to Y$.
            \item Todo subconjunto $Y \subseteq X$ também é finito. Se $Y \subsetneq X$, então $|Y| < |X|$, sendo $|Y| = |X|$ somente quando $Y = X$.
        \end{enumerate}
\end{prop}

\begin{proof}
    \leavevmode
        \begin{enumerate}[leftmargin=*, align=left, label=\textbf{(\alph*)}]         
            \item Suponha que existe uma tal bijeção. Se $X$ é finito, então existe uma bijeção $h : [n] \to X$ para algum $n \in \N$. Definindo $A := h^{-1}(Y)$, temos $A \subsetneq [n]$ e, além disso, a restrição de $h$ a $A$ é uma bijeção $h_A : A \to Y$. Com isso, a composta $h^{-1} \circ f^{-1} \circ h_A : A \to [n]$ é uma bijeção de $A \subsetneq [n]$ em $[n]$, o que contraria o teorema \eqref{teo:1}! Logo, não pode existir uma bijeção $f:X \to Y$ onde $X$ é finito e $Y \subsetneq X$. \itemproof

            Observe que esse item é uma mera reformulação do teorema \eqref{teo:1}.
    
            \item Ver \cite{cursodeanalise1}, página 31, teorema 4. Ver \cite{analisereal1}, página 5, teorema 2.
        \end{enumerate}
\end{proof}

\begin{cor} \label{cor.reais:funcfin}
    \leavevmode
        \begin{enumerate}[leftmargin=*, align=left, label=\textbf{(\alph*)}]
            \item Se $X$ é um conjunto finito, então uma função $f:X \to X$ será injetora se, e somente se, for sobrejetora. 
            
            \item Seja $f:X \to Y$ uma função injetora. Se $Y$ é finito, então $X$ é finito e $|X| \leq |Y|$.

            \item Seja $f:X \to Y$ uma função sobrejetora. Se $X$ é finito, então $Y$ é finito e $|Y| \leq |X|$.
        \end{enumerate}
\end{cor}

\begin{proof}
    \leavevmode
        \begin{enumerate}[leftmargin=*, align=left, label=\textbf{(\alph*)}]
            \item Ver \cite{analisereal1}, página 4, corolário 2.

            \item Ver \cite{analisereal1}, página 5, corolário 1. Ver \cite{cursodeanalise1}, página 31, corolário 1.

            \item Ver \cite{analisereal1}, página 5, corolário 1. Ver \cite{cursodeanalise1}, página 31, corolário 2.
        \end{enumerate}
\end{proof}

\begin{defi}
    \leavevmode
        \begin{enumerate}[leftmargin=*, align=left, label=\textbf{(\alph*)}]
            \item Um conjunto $X \subseteq \N$ é \textit{limitado} se existir $n \in \N$ tal que $x \leq n$ para todo $x \in X$.

            \item (Maior elemento)
        \end{enumerate}
\end{defi}

\begin{teo} \label{teo.reais:caracterizacao}
    Dado $\emptyset \neq X \subseteq \N$, as seguintes afirmações são equivalentes.
        \begin{enumerate}
            \item $X$ é finito;
            \item $X$ é limitado;
            \item $X$ possui um maior elemento.
        \end{enumerate}
\end{teo}

\begin{proof}
      Ver \cite{cursodeanalise1}, página 32, teorema 5. Para finito sse limitado, ver \cite{analisereal1}, página 5, corolário 2.
\end{proof}


\section{Conjuntos Infinitos}

\begin{defi}
    \leavevmode
        \begin{enumerate}[leftmargin=*, align=left, label=\textbf{(\alph*)}]
            \item Um conjunto $X$ é \textit{infinito} quando ele não é finito, isto é, quando $X \neq \emptyset$ e quando não existe uma bijeção $f : [n] \to X$ para todo $n \in \N$.
            \item Diremos que $X \subseteq \N$ é \textit{ilimitado} quando ele não é limitado, isto é, quando para cada $n \in \N$ existe $p \in X$ tal que $p > n$.
        \end{enumerate}
\end{defi}

\begin{cor}
    $\N$ é infinito.
\end{cor}

\begin{prop}
    Segue como contrapositiva do teorema \eqref{teo.reais:caracterizacao}: um conjunto $\emptyset \neq X \subseteq \N$ é infinito se, e somente se, não é limitado. Como $\N$ não é limitado, ele é infinito.
\end{prop}

\begin{teo}
    Se $X$ é um conjunto infinito, então existe uma função $f : \N \to X$ injetora.
\end{teo}

\begin{proof}
\end{proof}

\begin{cor}
    Um conjunto $X$ é infinito se, e somente se, existe uma bijeção $f : X \to Y \subsetneq X$.
\end{cor}

\begin{proof}
\end{proof}

\begin{cor}
    \leavevmode
        \begin{enumerate}[leftmargin=*, align=left, label=\textbf{(\alph*)}]
            \item Seja $f:X \to Y$ uma função injetora. Se $X$ é infinito, então $Y$ é infinito.
            \item Seja $f:X \to Y$ uma função sobrejetora. Se $Y$ é infinito, então $X$ é infinito.
        \end{enumerate}
\end{cor}

\begin{proof}
    Basta ver que essas afirmações são equivalentes às afirmações do resultado \eqref{cor.reais:funcfin} por contrapositiva.
\end{proof}

\begin{prop}
    Se $X$ é um conjunto finito e $Y$ é um conjunto infinito, então existem funções $f : X \to Y$ injetora e $g : Y \to X$ sobrejetora. 
\end{prop}

\begin{proof}
\end{proof}

\section{Conjuntos Enumeráveis e Não-Enumeráveis}

\begin{defi}
    Um conjunto $X$ é dito \textit{enumerável} quando é finito ou quando existe uma bijeção $f : \N \to X$. A função $f$ é chamada de uma \textit{enumeração} dos elementos de $X$.
\end{defi}

\begin{teo}
    Todo subconjunto de $\N$ é enumerável.
\end{teo}

\begin{proof}
    Seja $X \subseteq \N$. Se $X$ é finito, nada há de ser provado.
\end{proof}

\begin{cor}
    \textbf{(a)} Seja $f:X \to Y$ uma função injetora. Se $Y$ é enumerável, então $X$ é enumerável.

    \textbf{(b)} Seja $f:X \to Y$ uma função sobrejetora. Se $X$ é enumerável, então $Y$ é enumerável.
\end{cor}

\begin{proof}
\end{proof}

\begin{cor}
    \textbf{(a)} O produto cartesiano de um número finito de conjuntos enumeráveis é enumerável.
    
    \textbf{(b)} A união de uma família enumerável de conjuntos enumeráveis é enumerável.
\end{cor}

\section{Números Inteiros}

\begin{defi}
    O conjunto dos números inteiros é definido como
        \[
            \Z_{\F} := \N_{\F} \cup \{ 0\} \cup -\N_{\F}.
        \]
\end{defi}

\begin{prop}
    \leavevmode
        \begin{enumerate}[leftmargin=*, align=left, label=\textbf{(\alph*)}]
            \item Para quaisquer $m,n \in \Z$ tem-se $m+n \in \Z$.
            \item Para quaisquer $m,n \in \Z$ tem-se $m-n \in \Z$ e $n-m \in \Z$.
            \item Para quaisquer $m,n \in \Z$ tem-se $m \cdot n \in \Z$.
        \end{enumerate}
\end{prop}

\begin{proof}
\end{proof}

\begin{prop}
    Para todo $x \in \F$ existe um único $n \in \Z$ tal que $n \leq x < n+1$.
\end{prop}

\subsection{Teoria Elementar dos Números}

\section{Números Racionais}

(Racionais) Sendo $y,w \neq 0$, temos que $x \cdot w = y \cdot z \Leftrightarrow x \cdot y^{-1} = z \cdot w^{-1}$.

\begin{defi}
    O conjunto dos números racionais é definido como
        \[
            \Q_{\F} := \{x \in \F : \exists a \exists b (a \in \Z_{\F} \land b \in \Z_{\F} \land b \neq 0 \land x = a \cdot b^{-1} ) \}.
        \]
\end{defi}

\begin{prop}
    Para quaisquer $p,q \in \Q$, temos $p+q \in \Q$ e $p \cdot q \in \Q$.
\end{prop}

\begin{proof}
\end{proof}

\begin{defi}
    Dado $x \in \R$, definimos, para cada $n \in \N$, $x^1 := x$ e $x^{n+1} := x^n \cdot x$, e sendo $x \neq 0$, definimos, para cada $n \in \N_0$, $x^{0} := 1$ e $x^{-n} := \frac{1}{x^n}$.
\end{defi}

\begin{teo}
    \textbf{(a)} Seja $n \in \N$. Para todo $a \in \R_{\geq 0}$ existe um único $b \in \R_{\geq 0}$ tal que $b^n = a$. Notação: $b := \sqrt[n]{a}$

    \textbf{(b)} Seja $n \in \N$ ímpar. Para todo $a \in \R$ existe $b \in \R$ tal que $b^n=a$. Notação: $b := \sqrt[n]{a}$.
\end{teo}

\begin{proof}
\end{proof}

\begin{defi}
    \textbf{(a)} Seja $n \in \N$. Dado $x \in \R_{\geq 0}$, definimos $x^{\frac{1}{n}} := \sqrt[n]{x}$. Se $n$ for ímpar, dado $x \in \R$, definimos $x^{\frac{1}{n}} := \sqrt[n]{x}$.

    \textbf{(b)} Seja $n \in \N$. Dado $x \in \R$, definimos $x^{- \frac{1}{n}} := \dfrac{1}{x^{\frac{1}{n}}}$, desde que $x^{\frac{1}{n}} \neq 0$ esteja definido.

    \textbf{(c)} Seja $r := \frac{p}{q} \in \Q$, com $p \in \Z$ e $q \in \Z_{\neq 0}$. Dado $x \in \R$, definimos $x^r := \left( x^{\frac{1}{q}} \right)^p$, desde que $x^{\frac{1}{q}}$ esteja definido. 
\end{defi}

\begin{teo}
    \textbf{(a)} Se $a,b \in \R$ cumprem $a<b$, então existe $r \in \Q$ tal que $a<r<b$.

    \textbf{(b)} Se $a,b \in \R$ cumprem $a<b$, então existe $s \in \R \setminus \Q$ tal que $a<s<b$.
\end{teo}

\begin{proof}
\end{proof}

\section{Números Reais}

\begin{defi}
    Sejam $(\F, +, \cdot, \leq)$ um corpo ordenado e $S \in \mathcal{P}(\F)_{\neq \emptyset}$.
        \begin{enumerate}[leftmargin=*, align=left, label=\textbf{(\alph*)}]
            \item Dizemos que $S$ é
                \begin{enumerate}[label=\roman*.]
                    \item \textit{limitado superiormente} se existe $M \in \F$ tal que $x \leq M$ para todo $x \in S$. Nesse caso, dizemos que $M$ é uma \textit{cota superior} de $S$.
                    \item \textit{limitado inferiormente} se existe $m \in \F$ tal que $m \leq x$ para todo $x \in S$. Nesse caso, dizemos que $m$ é uma \textit{cota inferior} de $S$.
                    \item \textit{limitado} se $S$ é limitado superiormente e inferiormente.
                \end{enumerate}
            \item Dizemos que $\alpha \in \F$ é o 
                \begin{enumerate}[label=\roman*.]
                    \item \textit{supremo} de $S$ se $\alpha$ é uma cota superior de $S$ e $\alpha \leq x$ para toda cota superior $x \in \F$ de $S$. Denotamos $\alpha$ por $\sup{S}$.
                    \item \textit{ínfimo} de $S$ se $\alpha$ é uma cota inferior de $S$ e $x \leq \alpha$ para toda cota inferior $x \in \F$ de $S$. Denotamos $\alpha$ por $\inf{S}$.
                \end{enumerate}
        \end{enumerate}
\end{defi}

\begin{prop}
    Sejam $(\F, +, \cdot, \leq)$ um corpo ordenado e $S \in \mathcal{P}(\F)_{\neq \emptyset}$.
        \begin{enumerate}[leftmargin=*, align=left, label=\textbf{(\alph*)}]
            \item O supremo de $S$, quando existe, é único.
            \item O ínfimo de $S$, quando existe, é único.
        \end{enumerate}
\end{prop}

\begin{proof}
    \leavevmode
        \begin{enumerate}[leftmargin=*, align=left, label=\textbf{(\alph*)}]
            \item Sejam $\alpha$ e $\beta$ supremos de $S$. Como $\alpha$ é a menor das cotas superiores de $S$ e $\beta$ é uma cota superior de $S$, temos $\alpha \leq \beta$. Como $\beta$ é a menor das cotas superiores de $S$ e $\alpha$ é uma cota superior de $S$, temos $\beta \leq \alpha$. Logo $\alpha = \beta$. \itemproof
            \item Segue analogamente. \itemproof
        \end{enumerate}
\end{proof}

\begin{teo} \label{teo.reais:supinf}
    Sejam $(\F,+,\cdot, \leq)$ um corpo ordenado e $S \in \mathcal{P}(\F)_{\neq \emptyset}$. 
        \begin{enumerate}[leftmargin=*, align=left, label=\textbf{(\alph*)}]
            \item Suponha que existe $\sup{S}$.
                \begin{enumerate}[label=\roman*.]
                    \item Se $\beta \in \F$ e  $\beta < \sup{S}$, então existe $x \in S$ tal que $x > \beta$.
                    \item Para todo $\epsilon \in \F_{>0}$ existe $x \in S$ tal que $\sup{S} - \epsilon < x$.
                \end{enumerate}
            \item Suponha que existe $\inf{S}$.
                \begin{enumerate}[label=\roman*.]
                    \item Se $\beta \in \F$ e  $\inf{S} < \beta$, então existe $x \in S$ tal que $x < \beta$.
                    \item Para todo $\epsilon \in \F_{>0}$ existe $x \in S$ tal que $x < \inf{S} + \epsilon$.
                \end{enumerate}
        \end{enumerate}
\end{teo}

\begin{proof}
    \leavevmode
        \begin{enumerate}[leftmargin=*, align=left, label=\textbf{(\alph*)}]
            \item Segue facilmente por contradição.
                \begin{enumerate}[label=\roman*.]
                    \item Do contrário, seria $x \leq \beta < \sup{S}$ para todo $x \in S$, de modo que $\beta$ seria uma cota superior de $S$ menor que $\sup{S}$, um absurdo. \itemproof
                    \item Do contrário, existiria $\epsilon \in \F_{>0}$ tal que $x \leq \sup{S} - \epsilon < \sup{S}$ para todo $x \in S$, de modo que $\sup{S} - \epsilon$ seria uma cota superior de $S$ menor que $\sup{S}$, um absurdo. \itemproof
                \end{enumerate}
            \item Segue analogamente. \itemproof
        \end{enumerate}
\end{proof}

\begin{defi}
    Um corpo ordenado $\F$ é \textit{completo} se todo subconjunto não vazio de $\F$ limitado superiormente possui supremo em $\F$.
\end{defi}

\begin{cor}
    Um corpo ordenado $\F$ é \textit{completo} se, e somente se, todo subconjunto não vazio de $\F$ limitado inferiormente possui ínfimo em $\F$.
\end{cor}

\begin{proof}
    Ver \cite{ajwhite}, teorema 1-10, página 32. Ver \cite{stromberg}, corolário 1.12, página 13. \itemproof
\end{proof}

\begin{prop}
    Todo corpo ordenado completo é arquimediano.
\end{prop}

\begin{proof}
    Provemos que em todo corpo ordenado completo $\F$ o conjunto $\N_{\F}$ é ilimitado superiormente. Daí, pelo teorema \eqref{teo.reais:arquimedes}, seguirá que $\F$ é arquimediano.
    
    Suponha que $\N_{\F}$ seja limitado superiormente. Como $\N_{\F} \in \mathcal{P}(\F)_{\neq \emptyset}$, existe $a := \sup{\N_{\F}}$. Por \eqref{teo.reais:supinf}, para $\epsilon = 1$ existe $n \in \N_{\F}$ tal que $a-1 < n$, isto é, $a < n+1$, e como $n+1 \in \N_{\F}$, temos uma contradição. \itemproof
\end{proof}


\begin{teo}
    Existe um corpo ordenado completo.
\end{teo}

\begin{obs}
    Veremos mais a frente que, a menos de isomorfismos, existe um único corpo ordenado completo. Ele é denotado por $\R$ e seus elementos são chamados de \textit{números reais}.
\end{obs}

\subsection*{Propriedades do Supremo e do Ínfimo}

\begin{proof} 
\end{proof}

























\chapter{Números Reais como na Álgebra}

\begin{defi}[Grupo]
    \leavevmode
        \begin{enumerate}[leftmargin=*, align=left, label=\textbf{(\alph*)}]
            \item Um par $(G, *)$ é um \textit{grupo} se no conjunto $G \neq \emptyset$ existe uma operação $* : G \times G \to G$ para a qual
                \begin{itemize}
                    \item G1:
                        $x * (y * z) = (x * y) * z$
                    para quaisquer $x, y, z \in G$;
                    \item G3: existe $e \in G$ tal que
                        $x * e = x = e * x$
                    para todo $x \in G$;
                    \item G4: para cada $x \in G$ existe $y \in G$ tal que
                        $x * y = e = y * x$.
                \end{itemize}
            \item Umm grupo $(G,*)$ é \textit{comutativo}, ou \textit{abeliano}, se
                \begin{itemize}
                    \item G2: $x * y = y * x$ para quaisquer $x, y \in G$,
                \end{itemize}
        \end{enumerate}
\end{defi}

\begin{obs}
    As propriedades descritas em G1--G4 se chamam, respectivamente, \textit{associatividade}, \textit{comutatividade}, existência de um \textit{elemento neutro} e \textit{invertibilidade} (ou existência de \textit{inversos operativos}). 
\end{obs}

\begin{prop}
    Seja $(G,*)$ um grupo.
        \begin{enumerate}[leftmargin=*, align=left, label=\textbf{(\alph*)}]
            \item O elemento neutro de $*$ é único.
            \item O inverso de cada elemento de $G$ é único.
        \end{enumerate}
\end{prop}

\begin{proof}
    \leavevmode
        \begin{enumerate}[leftmargin=*, align=left, label=\textbf{(\alph*)}]
            \item Se $e' \in G$ é um elemento neutro de $*$, então
                \[
                    e = e * e' = e' * e = e',
                \]
            como havíamos afirmado. \itemproof
            \item Segue analogamente. \itemproof
        \end{enumerate}
\end{proof}

Anéis



\begin{defi}
    Uma tripla $(A, +, \cdot)$ é um \textit{anel} se no conjunto $A \neq \emptyset$ existem duas operações, $+:A \times A \to A$ e $\cdot: A \times A \to A$, para as quais
        \begin{itemize}
            \item A1: $a + (b + c) = (a + b) + c$ para quaisquer $a,b,c \in A$;
            \item A2: $a + b = b + a$ para quaisquer $a, b \in A$; 
            \item A3: existe $0 \in A$ tal que $a + 0 = a$ para todo $a \in A$;
            \item A4: para cada $a \in A$ existe $b \in A$ tal que $a + b = 0$;
            \item M1: $a \cdot (b \cdot c) = (a \cdot b) \cdot c$ para quaisquer $a,b,c \in A$;
            \item AM: $a \cdot ( b + c) = a \cdot b + a \cdot c$ e $(a+b) \cdot c = a \cdot c + b \cdot c$ para quaisquer $a,b,c \in A$.
        \end{itemize}
\end{defi}

\begin{prop}
    Seja $(A,+, \cdot)$ um anel.
        \begin{enumerate}[leftmargin=*, align=left, label=\textbf{(\alph*)}]
            \item O elemento neutro $0$ de $+$ é único.
            \item (Lei do corte) Para quaisquer $a, b, c \in A$, vale
                \begin{enumerate}[label=\roman*.]
                    \item $a+c = b+c \Rightarrow a=b$;
                    \item $a + b = a \Rightarrow b = 0$.
                \end{enumerate}
            \item $a \cdot 0 = 0$ para todo $a \in A$.
        \end{enumerate}
\end{prop}

\begin{proof}
    \leavevmode
        \begin{enumerate}[leftmargin=*, align=left, label=\textbf{(\alph*)}]
            \item Se $0' \in A$ é um elemento neutro de $+$, então
                \[
                    0' = 0' + 0 = 0 + 0' = 0,
                \]
            conforme afirmado. \itemproof
            \item
            \item Observando que $a \cdot 0 = a \cdot (0+0) = a \cdot 0 + a \cdot 0$, somando $-(a \cdot 0)$ aos dois lados de $ a \cdot 0 = a \cdot 0 + a \cdot 0$, segue que $0 \cdot a = 0$. \itemproof
        \end{enumerate}
\end{proof}

\begin{defi}
    Um anel $(A, +, \cdot)$ é um \textit{anel comutativo} se
        \begin{itemize}
            \item M2: $a \cdot b = b \cdot a$ para quaisquer $a,b \in A$.
        \end{itemize}
\end{defi}

\begin{defi}
    Um anel $(A, +, \cdot)$ é um \textit{anel com unidade} se
        \begin{itemize}
            \item M3: existe $1 \in A_{\neq 0}$ tal que $a \cdot 1 = 1 \cdot a = a$ para todo $a \in A$.
        \end{itemize}
\end{defi}

\begin{prop}
    Seja $(A, +, \cdot)$ um anel com unidade.
        \begin{enumerate}[leftmargin=*, align=left, label=\textbf{(\alph*)}]
            \item O elemento neutro $1$ de $\cdot$ é único.
            \item (Regras dos sinais) Para quaisquer $a,b \in A$, vale
                \begin{enumerate}[label=\roman*.]
                    \item $(-1) \cdot a = -a$;
                    \item $-(-a) = a$;
                    \item $(-a) \cdot b = a \cdot (-b) = - (a \cdot b)$;
                    \item $(-a) \cdot (-b) = a \cdot b$.
                \end{enumerate}
        \end{enumerate}
\end{prop}

\begin{proof}
    \itemproof
\end{proof}

\begin{defi}
    Um anel comutativo com unidade $(A, +, \cdot)$ é um \textit{domínio de integridade} se
        \begin{itemize}
            \item M4: $a \cdot b = 0 \Rightarrow a = 0 \lor b = 0$ para quaisquer $a, b \in A$.
        \end{itemize}
\end{defi}


\begin{prop}
    Seja $(A, +, \cdot)$ um domínio de integridade.
        \begin{enumerate}[leftmargin=*, align=left, label=\textbf{(\alph*)}]
            \item (Leis do corte) Para quaisquer $a, b, c \in A$, com $c \neq 0$,
                \begin{enumerate}[label=\roman*.]
                    \item $a \cdot c = b \cdot c \Rightarrow a = b$;
                    \item $a \cdot b = a \Rightarrow a = 0 \lor b = 1$;
                    \item $\ds a^2 = 
                        \begin{cases}
                            0 \Leftrightarrow a = 0 \\
                            1 \Leftrightarrow a = 1 \text{ ou } a = -1 \\
                            a \Leftrightarrow a = 0 \text{ ou } a = 1
                        \end{cases}$.
                \end{enumerate}
        \end{enumerate}
\end{prop}

\begin{proof}
    
\end{proof}

\begin{defi}
    Um anel comutativo com unidade $(A, +, \cdot)$ é um \textit{corpo} se
        \begin{itemize}
            \item M5: para cada $a \in A_{\neq 0}$ existe $b \in A$ tal que $a \cdot b = 1$.
        \end{itemize}
\end{defi}

\begin{prop}
    Todo corpo é um domínio de integridade.
\end{prop}

\begin{proof}
    \itemproof
\end{proof}

\begin{obs}
    Para simplificar a linguagem, um anel comutativo com unidade será chamado simplesmente de anel.
\end{obs}

\begin{defi}
    Um anel $(A, +, \cdot)$ é um \textit{anel ordenado} se existe uma relação de ordem total $\leq \, \subseteq A \times A$ tal que
        \begin{itemize}
            %\item O1: $a \leq a$ para todo $a \in A$;
            %\item O2: $a \leq b \land b \leq a \Rightarrow a = b$ para quaisquer $a,b \in A$;
            %\item O3: $a \leq b \land b \leq c \Rightarrow a \leq c$ para quaisquer $a,b,c \in A$;
            %\item O4: $a \leq b \lor b \leq a$ para quaisquer $a,b \in A$;
            \item OA: $a \leq b \Rightarrow a + c \leq b + c$ para quaisquer $a,b,c \in A$;
            \item OM: $a \leq b \Rightarrow a \cdot c \leq b \cdot c$ para quaisquer $a,b,c \in A$ com $0 \leq c$.
        \end{itemize}
\end{defi}

\begin{prop}
    Se $(A, +, \cdot, \leq)$ é um anel ordenado e $a, b, c, d \in A$, então

    \textbf{(a)} $a \geq 0 \Rightarrow -a \leq 0$ e $a \leq 0 \Rightarrow -a \geq 0$;
    
    \textbf{(b)} $a + c \leq b + c \Rightarrow a \leq b$;

    \textbf{(c)} $a \leq b, c \leq d \Rightarrow a + c \leq b + d$;

    \textbf{(d)} $a \leq b, c \leq 0 \Rightarrow a \cdot c \geq b \cdot c$;

    \textbf{(e)} $a \geq 0, b \leq 0 \Rightarrow a \cdot b \leq 0$ e $a \leq 0, b \leq 0 \Rightarrow a \cdot b \geq 0$;

    \textbf{(f)} $a^2 \geq 0$, $1 > 0$ e $-1 < 0$.
\end{prop}

\begin{proof}
    
\end{proof}

\begin{defi}
    Seja $(A, +, \cdot, \leq)$ um anel ordenado. O \textit{valor absoluto} de $a \in A$ é definido como
        \[
            |a| := 
                \begin{cases}
                    a, & \text{ se } a \geq 0 \\
                    -a, & \text{ se } a < 0
                \end{cases} .
        \]
\end{defi}

\begin{prop}
    Seja $(A, +, \cdot, \leq)$ um anel ordenado. Para quaisquer $a, b \in A$, vale
        \begin{enumerate}[leftmargin=*, align=left, label=\textbf{(\alph*)}]
            \item $|a \cdot b| = |a| \cdot |b|$.
            \item $- |a| \leq a \leq |a|$.
            \item $|a| \leq b \Leftrightarrow -b \leq a \leq b$.
            \item $||a| - |b|| \leq |a \pm b| \leq |a|+|b|$.
        \end{enumerate}
\end{prop}

\begin{proof}
    \itemproof
\end{proof}

\begin{defi}
    Seja $(A, +, \cdot, \leq)$ um anel ordenado.
        \begin{enumerate}[leftmargin=*, align=left, label=\textbf{(\alph*)}]
            \item Um subconjunto $X \subseteq A$ é
                \begin{enumerate}[label=\roman*.]
                    \item \textit{limitado inferiormente} se existe $a \in A$ tal que $a \leq x$ para todo $x \in X$;
                    \item \textit{limitado superiormente} se existe $a \in A$ tal que $a \geq x$ para todo $x \in X$.
                \end{enumerate}
            \item Um subconjunto $X \subseteq A$ tem um
                \begin{enumerate}[label=\roman*.]
                    \item \textit{menor elemento} se existe $a \in X$ tal que $a \leq x$ para todo $x \in X$;
                    \item \textit{maior elemento} se existe $a \in X$ tal que $a \geq x$ para todo $x \in X$.
                \end{enumerate}
        \end{enumerate}
\end{defi}

\begin{prop}
    Seja $(A, +, \cdot, \leq)$ um domínio ordenado. Todo subconjunto não vazio de $A$, limitado inferiormente, possui um menor elemento se, e somente se, todo subconjunto não vazio de $A$, limitado superiormente, possui um maior elemento.
\end{prop}

\begin{proof}
    \itemproof
\end{proof}

\begin{defi}
    Um domínio ordenado $(A, +, \cdot, \leq)$ é um \textit{domínio bem ordenado} se
        \begin{itemize}
            \item PBO: todo subconjunto não vazio de $A$, limitado inferiormente, possui um menor elemento.
        \end{itemize}
\end{defi}

\begin{teo}
    Existe um único domínio bem ordenado.
\end{teo}

\begin{teo}
    Seja $(A, +, \cdot, \leq)$ um domínio bem ordenado.
        \begin{enumerate}[leftmargin=*, align=left, label=\textbf{(\alph*)}]
            \item Para quaisquer $a, b \in A$, vale
                \begin{enumerate}[label=\roman*.]
                    \item $a > 0 \Rightarrow a \geq 1$;
                    \item $a > b \Rightarrow a \geq b+1$;
                    \item $b \neq 0 \Rightarrow |a \cdot b| \geq |a|$.
                \end{enumerate}
            \item Para quaisquer $a, b \in A$, com $b \neq 0$, existe $n \in A$ tal que $n \cdot b \geq a$.
        \end{enumerate}
\end{teo}

\begin{proof}
    \itemproof
\end{proof}

\section{Homomorfismos}

\begin{defi}
    \leavevmode
        \begin{enumerate}[leftmargin=*, align=left, label=\textbf{(\alph*)}]
            \item Um \textit{homomorfismo de anéis} $(A, +, \cdot)$ e $(B, +, \cdot)$ é uma função $f : A \to B$ tal que $f(a+b) = f(a) + f(b)$ e $f(a \cdot b) = f(a) \cdot f(b)$ para quaisquer $a, b \in A$.
            \item Um \textit{homomorfismo de anéis com unidade} $(A, +, \cdot)$ e $(B, +, \cdot)$ é um homomorfismo $f : A \to B$ tal que $f(1_A) = 1_B$.
            \item Um \textit{homomorfismo de anéis ordenados} $(A, +, \cdot, \leq)$ e $(B, +, \cdot, \leq)$ é um homomorfismo $f : A \to B$ tal que $a \leq b \Rightarrow f(a) \leq f(b)$ para quaisquer $a, b \in A$.
        \end{enumerate}
\end{defi}

Isso é denotado por $A \cong B$.





\part{Análise Real I}

%!TEX root = main.tex

\chapter{Sequências}

\begin{defi}
    Uma \textit{sequência numérica} é qualquer função $x : \N \to \R$, que associa a cada número natural $n$ um número real $x_n := x(n)$ (isto é, $n \mapsto x_n$), que será chamado de $n$-\textit{ésimo} termo da sequência. Escreveremos $(x_1, x_2, \ldots, x_n, \ldots)$, $\left(x_n\right)_{n \in \N}$ ou $(x_n)$ para indicar a sequência $x : \N \to \R$ cujo $n$-ésimo termo é $x_n \in \R$.
\end{defi}

\begin{defi}
    Uma sequência $(x_n)$ é
        \begin{enumerate}[label=\roman*.]
            \item \textit{crescente} se $n > m \Rightarrow x_n \geq x_m$;
            \item \textit{decrescente} se $n > m \Rightarrow x_n \leq x_m$;
            \item \textit{estritamente crescente} se $n > m \Rightarrow x_n > x_m$;
            \item \textit{estritamente decrescente} se $n > m \Rightarrow x_n < x_m$;
            \item \textit{monótona} se cumprir exatamente uma das condições acima.
        \end{enumerate}
\end{defi}

\begin{defi}
    Uma sequência $(x_n)$ é
    \begin{enumerate}[label=\roman*.]
        \item \textit{limitada superiormente} se existe $M \in \R$ tal que $x_n \leq M$ para todo $n \in \N$;
        \item \textit{limitada inferiormente} se existe $m \in \R$ tal que $m \leq x_n$ para todo $n \in \N$;
        \item \textit{limitada} se é limitada superiormente e limitada inferiormente.
        \item \textit{ilimitada} se não é limitada.
    \end{enumerate}
\end{defi}

\begin{prop}
    Uma sequência $(x_n)$ é limitada se, e somente se, existe $L \in \R_{>0}$ tal que $|x_n| \leq L$ para todo $n \in \N$.
\end{prop}

\begin{proof}
    Absolutamente trivial. \itemproof
\end{proof}

\begin{ex}
    A sequência $(x_n)$ é limitada se, e somente se, a sequência $\{|x_n|\}$ é limitada.
\end{ex}

\begin{proof}
    Elão, início da seção 4.1.
\end{proof}

\begin{defi}
    \leavevmode
        \begin{enumerate}[leftmargin=*, align=left, label=\textbf{(\alph*)}]
            \item Uma sequência $(x_n)$ é \textit{convergente} e \textit{converge} para $a \in \R$ se para todo $\epsilon \in \R_{>0}$ existe $n_0 \in \N$ tal que $n > n_0 \Rightarrow |x_n - a| < \epsilon$. Isso é denotado por
                \[ \ds
                    \lim_{n \to +\infty} x_n = a.
                \]
            \item Uma sequência $(x_n)$ é \textit{divergente} se não for convergente.
        \end{enumerate}
\end{defi}

\begin{obs}
    As notações ``$\ds \lim_{n \in \N} x_n = a$'', ``$\lim x_n =a$'', ``$x_n \to a$ quando $n \to +\infty$'' e ``$x_n \to a$'' também são frequentemente usadas para indicar que $\ds \lim_{n \to +\infty} x_n = a$.
\end{obs}

\begin{prop}[Unicidade]
    Uma sequência convergente converge para um único limite.
\end{prop}

\begin{proof}
    Provemos que se a sequência $(x_n)$ converge para $a \in \R$ e para $b \in \R$, então $a=b$.
\end{proof}

\begin{prop}
    Toda sequência convergente é limitada.
\end{prop}

\begin{proof}
\end{proof}

\begin{teo}[Convergência monótona] \label{teo.anal:convmon}
    \leavevmode
        \begin{enumerate}[leftmargin=*, align=left, label=\textbf{(\alph*)}]
            \item Toda sequência crescente e limitada superiormente é convergente.
            \item Toda sequência decrescente e limitada inferiormente é convergente.
            \item Toda sequência monótona e limitada é convergente.
        \end{enumerate}
\end{teo}

\begin{proof}
    \textbf{(a)} Seja $\left(x_n\right)_{n \in \N}$ uma sequência crescente e limitada superiormente. Como o conjunto $X := \{x_n \mid n \in \N\}$ é, por hipótese, não vazio e limitado superiormente, pela propriedade do supremo existe $\sup X$. Como, para todo $\epsilon \in \R_{>0}$, $\sup{X} - \epsilon$ não é uma cota superior de $X$, existe $n_0 \in \N$ tal que $\sup{X} - \epsilon < x_{n_0} \leq \sup{X}$. Como $\left(x_n\right)_{n \in \N}$ é crescente, para todo $n \in \N$, se $n > n_0$, então $\sup{X} - \epsilon < x_{n_0} \leq x_n \leq \sup{X} < \sup{X} + \epsilon$. Temos então que $x_n \to \sup{X}$, isto é, $\left(x_n\right)_{n \in \N}$ converge para $\sup{X}$. \itemproof

    \textbf{(b)} Segue analogamente: sendo $\left(x_n\right)_{n \in \N}$ uma sequência decrescente e limitada inferiormente, basta provar que $x_n \to \inf{\{x_n :n \in \N \}}$.
\end{proof}

\begin{defi} \label{defi:subseq}
    Uma \textit{subsequência} de uma sequência $x : \N \to \R$ dada por $n \mapsto x_n$ é qualquer composição $x \circ n : \N \to \R$, onde $n : \N \to \N$ dada por $k \mapsto n_k$ é uma sequência estritamente crescente de números naturais. A subsequência de $\left( x_n \right)_{n \in \N}$ definida por $\left( n_k \right)_{k \in \N}$ será denotada por $\left(x_{n_k} \right)_{k \in \N}$.
\end{defi}

\begin{prop} \label{prop.anal:subseqconv}
    Se uma sequência $\left(x_n\right)_{n \in \N}$ converge para $a \in \R$, então toda subsequência $\left(x_{n_k}\right)_{k \in \N}$ converge para $a$.
\end{prop}

\begin{proof}
\end{proof}

\begin{teo} \label{teo.anal:bolzanoweierstrass}
    (Bolzano-Weierstrass) Toda sequência limitada possui uma subsequência convergente.
\end{teo}

\begin{proof}
    Pelo teorema \eqref{teo.anal:convmon}, basta mostrar que toda sequência possui uma subsequência monótona. Seja $\left( x_n \right)_{n \in \N}$ uma sequência limitada. Um índice $k \in \N$ é dito \textit{básico} quando $x_p \geq x_k$ para todo $p > k$, isto é, $x_k$ é menor ou igual aos termos que o sucedem. 
        \begin{itemize}
            \item Se existem infinitos índices básicos $n_1 < n_2 < n_3 < \cdots$, então $x_{n_1} \leq x_{n_2} \leq x_{n_3} \leq \cdots$, de modo que a subsequência $\left(x_{n_k} \right)_{k \in \N}$ é crescente; como ela é limitada, pelo teorema \eqref{teo.anal:convmon}, ela é convergente.
            \item Por outro lado, se o número de índices básicos é finito, seja $n_1 \in \N$ maior que todos eles (se o número de índices básicos for 0, qualquer $n_1$ funciona). Como $n_1$ não é um índice básico, existe um índice $n_2 \in \N_{>n_1}$ tal que $x_{n_2} < x_{n_1}$. Como $n_2$ não é um índice básico, existe um índice $n_3 \in \N_{>n_2}$ tal que $x_{n_3} < x_{n_2}$. Prosseguindo deste modo, obtemos uma subsequência $\left(x_{n_k} \right)_{k \in \N}$ estritamente decrescente; como ela é limitada, pelo teorema \eqref{teo.anal:convmon}, ela é convergente.
        \end{itemize}
    Com isso, vemos que toda sequência possui uma subsequência monótona, e como a sequência original é limitada, a subsequência monótona também é (proposição \eqref{prop.anal:subseqconv}), sendo, portanto, convergente.
\end{proof}







\chapter{Limites e Continuidade}

\section{Topologia da Reta}

\begin{defi}
    Uma \textit{vizinhança} de $a \in \R$ de raio $r \in \R_{>0}$ é definida como
        \[
            V_r(a) := \{x \in \R : |x-a|<r \}.
        \]
\end{defi}

\begin{cor}
    Para quaisquer $a \in \R$ e $r \in \R_{>0}$, temos $V_r(a) = (a-r, a+r)$.
\end{cor}

\begin{defi}
    Um ponto $a \in \R$ é um \textit{ponto de acumulação} de $A \subseteq \R$ se 
        \[
            V_{\delta}(a) \cap A_{\neq a} \neq \emptyset
        \]
    para todo $\delta \in \R_{>0}$. O conjunto de todos os pontos de acumulação de $A$ é denotado por $A'$.
\end{defi}

\begin{prop}
    Seja $A \subseteq \R$.
        \begin{enumerate}[leftmargin=*, align=left, label=\textbf{(\alph*)}]
            \item $a \in A'$ se, e somente se, para todo $\delta \in \R_{>0}$ existe $x \in A$ tal que $0 < |x - a| < \delta$.
            \item $a \in A'$ se, e somente se, $0 \in B'$, onde $B := \{h \in \R_{\neq 0} : a+h \in A \}$.
        \end{enumerate}
\end{prop}

\begin{proof}
    \leavevmode
        \begin{enumerate}[leftmargin=*, align=left, label=\textbf{(\alph*)}]
            \item ($\Rightarrow$) blabla.
            
            ($\Leftarrow$) blabla. \itemproof
            \item ($\Rightarrow$) Para que $0$ seja um ponto de acumulação de $B$, para todo $\delta \in \R_{>0}$ deve existir $h \in B$ tal que $0 < |h-0| < \delta$. Bem, como $a$ é um ponto de acumulação de $A$, para todo $\delta \in \R_{>0}$ existe $x \in A$ tal que $0 < |x-a| < \delta$. Pois tomando $h := x-a$, temos que $x = a+h$, e como $x \in A$, temos $a+h \in A$, de modo que $h \in B$. Daí, segue a conclusão.

            ($\Leftarrow$) Para que $a$ seja um ponto de acumulação de $A$, para todo $\delta \in \R_{>0}$ deve existir $x \in A$ tal que $0 < |x-a| < \delta$. Bem, como $0$ é ponto de acumulação de $B$, para todo $\delta \in \R_{>0}$ existe $h \in B$ tal que $0<|h|<\delta$. Pois tome $x:=a+h$: como $h \in B$, temos que $a+h \in A$, de modo que $x \in A$. Daí, segue a conclusão. \itemproof
        \end{enumerate}
\end{proof}

\begin{defi}
    \leavevmode
        \begin{enumerate}[leftmargin=*, align=left, label=\textbf{(\alph*)}]
            \item Diremos que $a \in \R$ é um \textit{ponto de acumulação à direita} de $A \subseteq \R$ se $(a, a+ \delta) \cap A \neq \emptyset$ para todo $\delta \in \R_{>0}$.
            \item Diremos que $a \in \R$ é um \textit{ponto de acumulação à esquerda} de $A \subseteq \R$ se $(a - \delta, a) \cap A \neq \emptyset$ para todo $\delta \in \R_{>0}$.
        \end{enumerate}
\end{defi}

\section{Limites}

\begin{defi}[Limite]
    Uma função $f : A \subseteq \R \to \R$ tem \textit{limite} $L \in \R$ quando $x$ \textit{tende} a $a \in A'$ se para todo $\epsilon \in \R_{>0}$ existe $\delta = \delta(\epsilon, a) \in \R_{>0}$ tal que
        \[
            0 < | x-a | < \delta \Rightarrow | f(x) - L| < \epsilon
        \]
    para todo $x \in A$. Isso é denotado por 
        \[ \ds
            \lim_{x \to a} f(x) = L.
        \]
\end{defi}

\begin{prop}[Unicidade]
\end{prop}

\begin{proof}
    \itemproof
\end{proof}


\begin{defi}[Limites laterais]
    \leavevmode
        \begin{enumerate}[leftmargin=*, align=left, label=\textbf{(\alph*)}]
            \item Diremos que uma função $f : A \subseteq \R \to \R$ tem \textit{limite lateral à direita} $L \in \R$ quando $x$ \textit{tende} ao ponto de acumulação à direita $a \in \R$ de $A$, indicando isso por 
                \[ \ds
                    \lim_{x \to a^+} f(x) = L,
                \]
            se para todo $\epsilon \in \R_{>0}$ existir $\delta = \delta(\epsilon, a) \in \R_{>0}$ tal que
                \[
                    a < x < a+ \delta \Rightarrow | f(x) - L| < \epsilon
                \]
            para todo $x \in A$.
            \item Diremos que uma função $f : A \subseteq \R \to \R$ tem \textit{limite lateral à esquerda} $L \in \R$ quando $x$ \textit{tende} ao ponto de acumulação à esquerda $a \in \R$ de $A$, indicando isso por
                \[ \ds
                    \lim_{x \to a^-} f(x) = L,
                \]
            se para todo $\epsilon \in \R_{>0}$ existir $\delta = \delta(\epsilon, a) \in \R_{>0}$ tal que
                \[
                    a - \delta < x  < a \Rightarrow | f(x) - L| < \epsilon
                \]
            para todo $x \in A$.
        \end{enumerate}
\end{defi}

\begin{prop}[Unicidade]
\end{prop}

\begin{proof}
    \itemproof
\end{proof}

\begin{teo}[Bilateral $\Leftrightarrow$ Laterais]
\end{teo}

\begin{proof}
    \itemproof
\end{proof}

\begin{defi} (Limites no infinito)

    \textbf{(a)} Diremos que uma função $f : A \subseteq \R \to \R$, onde $A$ é ilimitado superiormente, tem limite $L \in \R$ quando $x$ \textit{cresce indefinidamente}, ou \textit{tende ao infinito positivo}, indicando isso por 
        \[
            \ds \lim_{x \to +\infty} f(x) = L,
        \]
    se para todo $\epsilon \in \R_{>0}$ existir $\delta = \delta(\epsilon) \in \R_{>0}$ tal que
        \[
            x > \delta \Rightarrow |f(x) - L| < \epsilon
        \]
    para todo $x \in A$.

    \textbf{(b)} Diremos que uma função $f : A \subseteq \R \to \R$, onde $A$ é ilimitado inferiormente, tem limite $L \in \R$ quando $x$ \textit{decresce indefinidamente}, ou \textit{tende ao infinito negativo}, indicando isso por 
        \[
            \ds \lim_{x \to -\infty} f(x) = L,
        \]
    se para todo $\epsilon \in \R_{>0}$ existir $\delta = \delta(\epsilon) \in \R_{>0}$ tal que
        \[
            x < - \delta \Rightarrow |f(x) - L| < \epsilon
        \]
    para todo $x \in A$.
\end{defi}

\begin{teo} \label{teo:caracterizacao}
    \textbf{(a)} (Unicidade do Limite) Seja $f$ uma função. O limite de $f$ quando $x \to p, p^{\pm}, \pm \infty$, quando existe, é único, isto é, se $\ds \lim f(x) = L_1$ e $\ds \lim f(x) = L_2$, então $L_1 = L_2$.

    \textbf{(b)} (Bilateral $\Leftrightarrow$ Laterais) Sejam $f$ uma função e $p$ um número real. Se existem números reais $a$ e $b$, com $a < p < b$, tais que $ \left]a, p\right[ , \ \left]p, b\right[ \subset D_f$, então
        \[ \ds
            \lim_{x \to p} f(x) = L \in \R \Leftrightarrow \lim_{x \to p^+} f(x) = L = \lim_{x \to p^-} f(x).
        \]
    \textbf{(c)} (Cálculo de Limites) Sejam $f$ e $g$ funções para as quais existe $r>0$ tal que $f(x) = g(x)$ sempre que $0 < |x-p| < r$ (caso $x \to p$), ou $p<x<p+r$ (caso $x \to p^+$), ou $p-r<x<p$ (caso $x \to p^-$), ou $x>r$ (caso $x \to + \infty$), ou $x < -r$ (caso $x \to - \infty$). Nestas condições, se $\ds \lim f(x) = L \in \R$, então $\ds \lim g(x) = L$. \begin{comment}
        Sejam $f$ e $g$ duas funções para as quais existe $r > 0$ tal que $f(x) = g(x)$ sempre que $0 < \left| x - p \right| < r$. Nestas condições, se $\ds \lim_{x \to p} f(x) \in \R$, então $\ds \lim_{x \to p} g(x) = \lim_{x \to p} f(x)$.
    \end{comment}

    \textbf{(d)} (do \textit{Confronto}) Sejam $f$, $g$ e $h$ funções para as quais existe $r>0$ tal que $f(x) \leq g(x) \leq h(x)$ sempre que $0 < |x-p| < r$ (caso $x \to p$), ou $p<x<p+r$ (caso $x \to p^+$), ou $p-r<x<p$ (caso $x \to p^-$), ou $x>r$ (caso $x \to + \infty$), ou $x < -r$ (caso $x \to - \infty$). Nestas condições, se $\ds \lim f(x) = \lim h(x) = L \in \R$, então $\ds \lim g(x) = L$. \begin{comment}
    Nestas condições, se $\ds \lim f(x) = \lim h(x) = L \in \R$, então $\ds \lim g(x) = L$.
    
    Sejam $f$, $g$ e $h$ funções tais que $\ds \lim f(x) = \lim h(x) = L$, com $x \to p, p^{\pm}, \pm \infty$. Se existe $r>0$ tal que $f(x) \leq g(x) \leq h(x)$ sempre que 
    
    \begin{itemize} 
        \item $0 < |x-p| < r$, se $x \to p$; 
        \item ou $p<x<p+r$, se $x \to p^+$;
        \item ou $p-r<x<p$, se $x \to p^-$; 
        \item ou $x>r$, se $x \to + \infty$;
        \item ou $x < -r$, se $x \to - \infty$;
    \end{itemize}

    então $\ds \lim g(x) = L$.

    Sejam $f$, $g$ e $h$ funções tais que $\ds \lim f(x) = \lim h(x) = L$, com $x \to p, p^{\pm}, \pm \infty$. Se existe $r>0$ tal que $f(x) \leq g(x) \leq h(x)$ sempre que $0 < |x-p| < r$, se $x \to p$, ou $p<x<p+r$, se $x \to p^+$, ou $p-r<x<p$, se $x \to p^-$, ou $x>r$, se $x \to + \infty$, ou $x < -r$, se $x \to - \infty$, então $\ds \lim g(x) = L$.
    
    Sejam $f$, $g$ e $h$ funções para as quais existe $r>0$ tal que $0 < |x-p| < r \Rightarrow f(x) \leq g(x) \leq h(x)$. Nestas condições, se $\ds \lim_{x \to p} f(x) = L = \lim_{x \to p} h(x)$, então $\ds \lim_{x \to p} g(x) = L$. \end{comment}

    \textbf{(e)} (Limites Básicos) Dados $a,p \in \R$, temos que \begin{comment} $\ds \lim_{x \to p} a = \lim_{x \to \pm \infty} a = a$, $\ds \lim_{x \to p} x = p$ e $\ds \lim_{x \to \pm \infty} \dfrac{1}{x} = 0$. \end{comment}
        \[
            \ds \lim_{x \to p} a = \lim_{x \to \pm \infty} a = a;
            \qquad \lim_{x \to p} x = p;
            \qquad \lim_{x \to \pm \infty} \dfrac{1}{x} = 0.
        \]  
\end{teo}

\begin{proof}
    \textbf{(a)} Consideremos o caso em que $x \to p$. Como $\ds \lim_{x \to p} f(x) = L_1$ e $\ds \lim_{x \to p} f(x) = L_2$, temos por definição que para todo $\epsilon > 0$ existem $\delta_1, \delta_2 > 0$ para os quais
    \begin{align*}  
            0 < | x - p | < \delta_1 \Rightarrow | f(x) - L_1 | < \dfrac{\epsilon}{2}; \\
            0 < | x - p | < \delta_2 \Rightarrow | f(x) - L_2 | < \dfrac{\epsilon}{2}.
        \end{align*}
    Tomando $\delta := \min \{\delta_1, \delta_2\}$, temos que para todo $\epsilon > 0$ existe $\delta > 0$ tal que
        \[
            0 < | x - p | < \delta \Rightarrow |f(x) - L_1| + |f(x) - L_2| < \epsilon.
        \]
    Com isso, temos que, para todo $\epsilon > 0$,
        \begin{align*}
            | L_1 - L_2 | &= | L_1 - f(x) + f(x) - L_2 | \\
            &\leq | L_1 - f(x) | + | f(x) - L_2 | \\
            &= |f(x) - L_1| + |f(x) - L_2| \\
            &< \epsilon,
        \end{align*}
    donde $L_1 = L_2$. \itemproof

    \textbf{(b)}

    \textbf{(c)}

    \textbf{(d)} Consideremos o caso em que $x \to p$. Como, por hipótese, $\ds \lim_{x \to p} f(x) = L = \lim_{x \to p} h(x)$, temos
        \begin{align*}
            \forall \epsilon > 0, \exists \delta_1 > 0: 0 < \left| x - p \right| < \delta_1 \Rightarrow L - \epsilon < f(x) < L + \epsilon; \\
            \forall \epsilon > 0, \exists \delta_2 > 0: 0 < \left| x - p \right| < \delta_2 \Rightarrow L - \epsilon < h(x) < L + \epsilon.
        \end{align*}
    Pois tome $\delta = \min \left\{ \delta_1, \delta_2, r \right\}$; daí, vem
        \[
            \forall \epsilon > 0, \exists \delta > 0 : 0 < \left| x-p \right| < \delta \Rightarrow L - \epsilon < f(x) \leq g(x) \leq h(x) < L + \epsilon,
        \]
    e então
        \[
            \forall \epsilon > 0, \exists \delta > 0 : 0 < \left| x-p \right| < \delta \Rightarrow L - \epsilon < g(x) < L + \epsilon, 
        \]
    donde $\ds \lim_{x \to p} g(x) = L$.
\end{proof}

\begin{teo} (Propriedades Operatórias) \label{teo:proplimites}
    Se $f_1$, $f_2$, $\ldots$, $f_n$ são funções tais que $\ds \lim f_1(x) = L_1$, $\ds \lim f_2(x) = L_2$,  $\ldots$, $\ds \lim f_n(x) = L_n$, em que $x \to p, p^{\pm}, \pm \infty$, então:
    
    \textbf{(a)} O limite da soma é igual à soma dos limites:
        \[ \ds
            \lim \left[ \sum_{i=1}^{n} f_i(x) \right] = \sum_{i=1}^{n} \left[ \lim f_i(x) \right] = \sum_{i=1}^{n} L_i = L_1 + L_2 + \ldots + L_n.
        \]  
    \textbf{(b)} O limite do produto é igual ao produto dos limites:
        \[ \ds
            \lim \left[ \prod_{i=1}^{n} f_i(x) \right] = \prod_{i=1}^{n} \left[ \lim f_i(x) \right] = \prod_{i=1}^{n} L_i =  L_1 \cdot L_2 \cdot \ldots \cdot L_n.
        \]
    \textbf{(c)} O limite do quociente é igual ao quociente dos limites, desde que o denominador seja diferente de $0$:
        \[ \ds
            \lim \dfrac{f_1(x)}{f_2(x)} = \dfrac{\lim f_1(x)}{\lim f_2(x)} = \dfrac{L_1}{L_2}. \quad (L_2 \neq 0)
        \]
\end{teo}

\begin{proof}
\end{proof}

\begin{teo} (Composição de Limites) \label{teo:complimites}
    Sejam $f$ e $g$ funções tais que $Im_f \subset D_g$ e $\ds \lim f(x) := a$, com $x \to p, \pm \infty$.  

    \textbf{(a)} Se $\ds \lim_{u \to a} g(u) = g(a)$, então
        \[ \ds
            \lim g[f(x)] = \lim_{u \to a} g(u),
        \]
    sendo $u := f(x)$.

    \textbf{(b)} Se $\ds \lim_{u \to a} g(u) := L$ e $a \notin D_g$, então
        \[ \ds
            \lim g[f(x)] = \lim_{u \to a} g(u),
        \]
    sendo $u := f(x)$.
\end{teo}

\begin{proof}
\end{proof}

\begin{obs}
    Para o item (a) do teorema acima, como, por hipótese, $\ds \lim_{u \to a} g(u) = g(a)$, podemos expressar o teorema como
        $ \ds
            \lim g \left[ f(x) \right] = g \left[ \lim f(x) \right].
        $
\end{obs}

\begin{cor} \label{consin1}
    \textbf{(a)} (Conservação do sinal) Se $\ds \lim_{x \to p} f(x) := L \neq 0$, então existe $\delta > 0$ tal que, para todo $x \in D_f$, temos $0 < |x-p| < \delta \Rightarrow f(x) \neq 0$.
    
    \textbf{(b)} Temos 
        \begin{align*}
             \lim_{x \to p} f(x) = L &\Leftrightarrow \lim_{h \to 0} f(p+h) = L \\
             &\Leftrightarrow \lim_{x \to p} [f(x) - L] = 0 \\ &\Leftrightarrow \lim_{x \to p} |f(x) - L| = 0.
        \end{align*}
\end{cor}

\begin{proof}
    \textbf{(a)} Basta tomar $\epsilon = L$. \itemproof

    \textbf{(b)}
\end{proof}

\section{Continuidade}

\begin{defi}[Continuidade] \label{ar1.defi:continuidade}
    Seja $f : A \subseteq \R \to \R$ uma função.
        \begin{enumerate}[leftmargin=*, align=left, label=\textbf{(\alph*)}]
            \item $f$ é \textit{contínua} em $a \in A$ se para todo $\epsilon \in \R_{>0}$ existe $\delta = \delta(\epsilon, a) \in \R_{>0}$ tal que
                \[
                    \forall x (x \in A \cap V_\delta(a) \Rightarrow f(x) \in V_\epsilon(f(a))),
                \]
            ou ainda, equivalentemente,
                \[
                    \forall x (x \in A \land |x-a| < \delta \Rightarrow |f(x) - f(a)| < \epsilon).
                \]
            \item $f$ é \textit{contínua} em $X \subseteq A$ se $f$ é contínua em todos os pontos de $X$, isto é, se para cada $a \in X$ e todo $\epsilon \in \R_{>0}$ existe $\delta = \delta(\epsilon, a) \in \R_{>0}$ tal que
                \[
                    |x-a| < \delta \Rightarrow |f(x) - f(a)| < \epsilon
                \]
            para todo $x \in A$.
            \item $f$ é \textit{uniformemente contínua} em $A$ se para todo $\epsilon \in \R_{>0}$ existe $\delta = \delta(\epsilon) \in \R_{>0}$ tal que
                \[
                    |x-y| < \delta \Rightarrow |f(x) - f(y)| < \epsilon
                \]
            para quaisquer $x,y \in A$.
        \end{enumerate}
\end{defi}

\begin{teo}
    Uma função $f : A \subseteq \R \to \R$ é contínua em $a \in A \cap A'$ se, e somente se, $\ds \lim_{x \to a} f(x) = f(a)$.
\end{teo}

\begin{proof}
\end{proof}

\begin{teo}
    Se $f$ e $g$ são funções contínuas em $p$, então as funções $f+g$ e $f \cdot g$ são contínuas em $p$; e se $g(p) \neq 0$, então a função $\dfrac{f}{g}$ é contínua em $p$.
\end{teo}

\begin{proof}
    Segue como corolário do Teorema \eqref{teo:proplimites}.
\end{proof}

\begin{prop}
    \textbf{(a)} Seja $a \in \R$. A função constante $f(x) := a$ é contínua.

    \textbf{(b)} A função identidade $f(x) := x$ é contínua.

    \textbf{(c)} Toda função polinomial é contínua. E ainda, toda função racional é contínua.
\end{prop}

\begin{proof}
\end{proof}

\begin{teo}
    Sejam $f : A \subseteq \R \to \R$ e $g : B \subseteq \R \to \R$ funções tais que $f(A) \subseteq B$. Se $f$ é contínua em $a \in A$ e se $g$ é contínua em $f(a) \in B$, então a função composta $g \circ f : X \subseteq \R \to \R$ é contínua em $a$.    
\end{teo}

\begin{teo}
    Sejam $f$ e $g$ funções tais que $Im_f \subset D_g$. Se $f$ é contínua em $p$ e $g$ é contínua em $f(p)$, então a função composta $(g \circ f)(x) = g[f(x)]$ é contínua em $p$.
\end{teo}

\begin{proof}
    Pois tome:
        $
            \ds \lim_{x \to p} g[f(x)] =  g \left[ \lim_{x \to p} f(x) \right] = g[f(p)].
        $
\end{proof}

\begin{teo} (Intervalos) \label{teo:intervalos}
     Sejam $f$ uma função e $p \in D_f$ um número real.
        
    \textbf{(a)} Se para todo $\epsilon > 0$ existir um intervalo aberto $\left]a,b \right[$, com $p \in \left] a,b \right[$, tal que
        $ \ds
           \forall x \in D_f : x \in \left]a,b \right[ \Rightarrow | f(x) - f(p) | < \epsilon,
        $        
    então $f$ é contínua em $p$.

    \textbf{(b)} Seja $r>0$. Se para todo $0 < \epsilon < r$ existir um intervalo aberto $I$ (como no item anterior, com $p \in I$) tal que
            $ \ds
               \forall x \in D_f : x \in I \Rightarrow | f(x) - f(p) | < \epsilon,
            $        
    então $f$ é contínua em $p$.
\end{teo}

\begin{proof}
    \textbf{(a)} Segue imediatamente do seguinte fato: para todo intervalo $]a, b[$, existe $\delta > 0$ tal que $\left]p - \delta, p + \delta \right[ \subset \left] a,b \right[$. Com efeito, basta tomar $\delta = \min \{ b-p, p-a\}$. Com isso, escolhendo esse $\delta$, temos que 
        \[
            x \in \left] p - \delta, p + \delta \right[ \Rightarrow x \in \left] a,b \right[.
        \]
    Como, por hipótese, $x \in \left] a,b \right[ \Rightarrow | f(x) - f(p) | < \epsilon$, temos então que
        \[
            x \in \left] p - \delta, p + \delta \right[ \Rightarrow |f(x) - f(p)| < \epsilon.
        \]
    Como $ x \in \left] p - \delta, p + \delta \right[ \Leftrightarrow |x-p| < \delta $, vemos que para todo $\epsilon>0$ existe um $\delta > 0$ tal que $|x-p| < \delta \Rightarrow |f(x) - f(p)| < \epsilon$, isto é, $f$ é contínua em $p$. \itemproof

    \textbf{(b)} Pelo item anterior, se $\epsilon < r$, então nada há de ser provado. Temos que provar o resultado para todos os $\epsilon$'s, isto é, falta provar o caso $\epsilon \geq r$.

    Pois tome $\epsilon_1 < r$. Para esse $\epsilon_1$, existe (por hipótese) um intervalo aberto $I$ tal que $x \in I \Rightarrow | f(x) - f(p)| < \epsilon_1$. Como $\epsilon_1 < \epsilon$, também vale 
        \[
            x \in I \Rightarrow | f(x) - f(p)| < \epsilon,
        \]
    o que completa a prova.
\end{proof}

\begin{prop}[Conservação do sinal] \label{consin2}
    Seja $f : A \subseteq \R \to \R$ uma função contínua em $a \in A$.
        \begin{enumerate}[leftmargin=*, align=left, label=\textbf{(\alph*)}]
            \item Se $f(a) > 0$, então existe $\delta \in \R_{>0}$ tal que $f(x) > 0$ para todo $x \in V_{\delta}(a) \cap A$.
            \item Se $f(a) < 0$, então existe $\delta \in \R_{>0}$ tal que $f(x) < 0$ para todo $x \in V_{\delta}(a) \cap A$.
        \end{enumerate}
\end{prop}

\begin{proof}
    \leavevmode
        \begin{enumerate}[leftmargin=*, align=left, label=\textbf{(\alph*)}]
            \item Para $\epsilon = f(a)$ na definição de continuidade, existe $\delta \in \R_{>0}$ tal que
                \[
                    \forall x (x \in V_\delta(a) \cap A \Rightarrow |f(x) - f(a)| < f(a)).
                \]
            Observando que
                \[
                    |f(x) - f(a)| < f(a) \Leftrightarrow 0 = f(a) - f(a) < f(x) < 2f(a),
                \]
            a conclusão segue. \itemproof
            \item Tomando $\epsilon = -f(a)$, segue analogamente. \itemproof
        \end{enumerate}
\end{proof}


\section{Limites Infinitos}

\begin{defi}
    (Limites infinitos quando $x \to \pm \infty$) Seja $f$ uma função.
    
    \textbf{(a)} Suponha que existe um número real $a$ tal que $\left] a, +\infty \right[ \subset D_f$.
    
    \begin{enumerate}[label=\roman*.]
        \item Diremos que $f$ \textit{cresce indefinidamente}, ou \textit{tende ao infinito positivo}, quando $x$ tende ao infinito positivo, indicando
            $ \ds
                \lim_{x \to + \infty} f(x) = + \infty,
            $
        se para todo $\epsilon > 0$ existir $\delta > 0$, com $\delta > a$, tal que $x > \delta \Rightarrow f(x) > \epsilon$.
        \item Diremos que $f$ \textit{decresce indefinidamente}, ou \textit{tende ao infinito negativo}, quando $x$ tende ao infinito positivo, indicando
            $ \ds
                \lim_{x \to + \infty} f(x) = - \infty,
            $
        se para todo $\epsilon > 0$ existir $\delta > 0$, com $\delta > a$, tal que $x > \delta \Rightarrow f(x) < - \epsilon$.
    \end{enumerate}
    
    \textbf{(b)} Suponha que existe um número real $a$ tal que $\left] -\infty, a \right[ \subset D_f$. 
    
    \begin{enumerate}[label=\roman*.]
        \item Diremos que $f$ tende ao infinito positivo, quando $x$ tende ao infinito negativo, indicando
            $ \ds
                \lim_{x \to - \infty} f(x) = + \infty,
            $
        se para todo $\epsilon > 0$ existir $\delta > 0$, com $- \delta < a$, tal que $x < - \delta \Rightarrow f(x) > \epsilon$.
        \item Diremos que $f$ tende ao infinito negativo, quando $x$ tende ao infinito negativo, indicando
            $ \ds
                \lim_{x \to - \infty} f(x) = - \infty,
            $
        se para todo $\epsilon > 0$ existir $\delta > 0$, com $- \delta < a$, tal que $x < - \delta \Rightarrow f(x) < - \epsilon$.
    \end{enumerate}
\end{defi}

\begin{defi}
    (Limites infinitos quando $x \to p^{\pm}$) Seja $f$ uma função e $p$ um número real.

    \textbf{(a)} Suponha que existe um número real $b$ tal que $\left]p,b\right[ \subset D_f$.
    
    \begin{enumerate}[label=\roman*.]
        \item Diremos que $f$ tende ao infinito positivo, quando $x$ tende a $p$, pela direita, indicando
            $ \ds
                \lim_{x \to p^+} f(x) = + \infty,
            $
        se para todo $\epsilon > 0$ existir $\delta > 0$, com $p+ \delta < b$, tal que $p < x < p + \delta \Rightarrow f(x) > \epsilon$.
        \item Diremos que $f$ tende ao infinito negativo, quando $x$ tende a $p$, pela direita, indicando
            $ \ds
                \lim_{x \to p^+} f(x) = - \infty,
            $
        se para todo $\epsilon > 0$ existir $\delta > 0$, com $p+ \delta < b$, tal que $p < x < p + \delta \Rightarrow f(x) < - \epsilon$.
    \end{enumerate}

    \textbf{(b)} Suponha que existe um número real $a$ tal que $\left]a,p\right[ \subset D_f$.

    \begin{enumerate}[label=\roman*.]
        \item Diremos que $f$ tende ao infinito positivo, quando $x$ tende a $p$, pela esquerda, indicando
            $ \ds
                \lim_{x \to p^-} f(x) = + \infty,
            $
        se para todo $\epsilon > 0$ existir $\delta > 0$, com $a< p - \delta$, tal que $p - \delta < x < p \Rightarrow f(x) > \epsilon$.
        \item Diremos que $f$ tende ao infinito negativo, quando $x$ tende a $p$, pela esquerda, indicando
            $ \ds
                \lim_{x \to p^-} f(x) = - \infty,
            $
        se para todo $\epsilon > 0$ existir $\delta > 0$, com $a< p - \delta$, tal que $p - \delta < x < p \Rightarrow f(x) < - \epsilon$.
    \end{enumerate}
\end{defi}

\begin{defi}
    (Limites infinitos quando $x \to p$) Seja $f$ uma função e $p$ um número real. Suponha que existem números reais $a$ e $b$, com $a < p < b$, tais que $ \left]a, p\right[ , \ \left]p, b\right[ \subset D_f$.

    \begin{enumerate}[label=\roman*.]
        \item Diremos que $f$ tende ao infinito positivo, quando $x$ tende a $p$, indicando
            $ \ds
                \lim_{x \to p} f(x) = + \infty,
            $
        se para todo $\epsilon > 0$ existir $\delta > 0$, com $a < p - \delta$ e $p + \delta < b$, tal que $0 < |x-p| < \delta \Rightarrow f(x) > \epsilon$.
        \item Diremos que $f$ tende ao infinito negativo, quando $x$ tende a $p$, indicando
            $ \ds
                \lim_{x \to p} f(x) = - \infty,
            $
        se para todo $\epsilon > 0$ existir $\delta > 0$, com $a < p - \delta$ e $p + \delta < b$, tal que $0 < |x-p| < \delta \Rightarrow f(x) < -\epsilon$.
    \end{enumerate}
\end{defi}

\begin{teo}
    Seja $f$ uma função e $p$ um número real. Se existem números reais $a$ e $b$, com $a < p < b$, tais que $ \left]a, p\right[ , \ \left]p, b\right[ \subset D_f$, então
        \[ \ds
            \lim_{x \to p} f(x) = \pm \infty \Leftrightarrow \lim_{x \to p^+} f(x) = \pm \infty = \lim_{x \to p^-} f(x).
        \]
\end{teo}

\begin{proof}
\end{proof}

\begin{teo}
    Os resultados a seguir valem para $x \to p$, $x \to p^{\pm}$ e $x \to \pm \infty$.

    \textbf{(a)} Se $\lim f(x) = \lim g(x) = \pm \infty$, então $\lim [f(x) + g(x)] = \pm \infty$ e $\lim [f(x)g(x)] = + \infty$.
    
    \textbf{(b)} Se $\lim f(x) = - \infty$ e $\lim g(x) = + \infty$, então $\lim [f(x)g(x)] = - \infty$.

    \textbf{(c)} Seja $\lim f(x) = L$. Se $\lim g(x) = \pm \infty$, então $\lim [f(x) + g(x)] = \pm \infty$.

    \textbf{(d)} Seja $\lim f(x) = L > 0$. Se $\lim g(x) = \pm \infty$, então $\lim [f(x)g(x)] = \pm \infty$.

    \textbf{(e)} Seja $\lim f(x) = L < 0$. Se $\lim g(x) = \pm \infty$, então $\lim [f(x)g(x)] = \mp \infty$.
\end{teo}

\begin{proof}
\end{proof}

\begin{prop}
    \textbf{(a)} Seja $\ds \lim f(x) = 0$, com $x \to p^{\pm}$. Se existe $r>0$ tal que $f(x) > 0$ sempre que $p<x<p+r$, se $x \to p^+$, ou $p-r<x<p$, se $x \to p^-$, então $\ds \lim \dfrac{1}{f(x)} = + \infty$.

    \textbf{(b)} Sejam $\ds \lim f(x) = L \neq 0$ e $\ds \lim g(x) = 0$, com $x \to p^{\pm}$. Se existe $r>0$ tal que $f(x) > 0$ sempre que $p<x<p+r$, se $x \to p^+$, ou $p-r<x<p$, se $x \to p^-$, então ou $\ds \lim \dfrac{f(x)}{g(x)} = + \infty$, ou $\ds \lim \dfrac{f(x)}{g(x)} = - \infty$, ou $\ds \lim \dfrac{f(x)}{g(x)}$ não existe.
\end{prop}

\begin{proof}
\end{proof}

\section{Limites e Sequências}

\begin{defi}
    
    \textbf{(c)} Diremos que $\left(x_n\right)_{n \in \N}$
        \begin{enumerate}
            \item \textit{diverge} para $+ \infty$, indicando
            \[ \ds
                \lim_{n \to +\infty} x_n = + \infty,
            \]
        se para todo $\epsilon \in \R_{>0}$ existir $n_0 \in \N$ tal que $n > n_0 \Rightarrow x_n > \epsilon$.
        \item \textit{diverge} para $- \infty$, indicando
            \[ \ds
                \lim_{n \to +\infty} x_n = - \infty,
            \]
        se para todo $\epsilon \in \R_{>0}$ existir $n_0 \in \N$ tal que $n > n_0 \Rightarrow x_n < - \epsilon$.
        \end{enumerate}
\end{defi}

\begin{obs}
    \textbf{(a)} Note que as definições acima são análogas àquelas que demos aos limites no infinito de funções. Assim, os resultados sobre os limites da forma $\ds \lim_{x \to + \infty} f(x)$ também são válidos para os limites da forma $\ds \lim_{x \to + \infty} x_n$.

    \textbf{(b)} A notação ``$x_n \to a$ quando $n \to +\infty$'' também é frequentemente usada para indicar $\ds \lim_{n \to +\infty} x_n = a$. Quando não houver confusão, podemos escrever simplesmente $x_n \to a$. Analogamente, também podemos escrever $x_n \to \pm \infty$ quando $n \to \infty$ ou, simplesmente, $x_n \to \pm \infty$.
\end{obs}

\begin{teo} \label{teo:seqmonlim}
    (Convergência Monótona)

    \textbf{(a)} Toda sequência crescente e limitada superiormente é convergente.

    \textbf{(b)} Toda sequência decrescente e limitada inferiormente é convergente.
\end{teo}

\begin{proof}
    \textbf{(a)} Seja $\left(x_n\right)_{n \in \N}$ uma sequência crescente e limitada superiormente. Como o conjunto $X := \{x_n \mid n \in \N\}$ é, por hipótese, não vazio e limitado superiormente, pela propriedade do supremo existe $\sup X$. Como, para todo $\epsilon \in \R_{>0}$, $\sup{X} - \epsilon$ não é uma cota superior de $X$, existe $n_0 \in \N$ tal que $\sup{X} - \epsilon < x_{n_0} \leq \sup{X}$. Como $\left(x_n\right)_{n \in \N}$ é crescente, para todo $n \in \N$, se $n > n_0$, então $\sup{X} - \epsilon < x_{n_0} \leq x_n \leq \sup{X} < \sup{X} + \epsilon$. Temos então que $x_n \to \sup{X}$, isto é, $\left(x_n\right)_{n \in \N}$ converge para $\sup{X}$. \itemproof

    \textbf{(b)} Segue analogamente: sendo $\left(x_n\right)_{n \in \N}$ uma sequência decrescente e limitada inferiormente, basta provar que $x_n \to \inf{\{x_n :n \in \N \}}$.
    \begin{comment}
    
    para todo $\epsilon \in \R_{>0}$ existe $n_0 \in \N$ tal que 
        \[
            n > n_0 \Rightarrow \sup{X} - \epsilon < x_n < \sup{X} + \epsilon
        \]
    Em particular, isso significa que para todo $\epsilon \in \R_{>0}$ 
    
    existe $n_0 \in \N$ tal que $n > n_0 \Rightarrow \sup{A} - \epsilon < a_n$. Como $a_n \leq \sup A < \sup A + \epsilon$, temos então que $\ds \lim_{n \to +\infty} a_n = \sup A$, como havíamos afirmado.

    Provaremos que se a sequência $\left(a_n\right)_{n \geq 0}$ for crescente e limitada superiormente, então
        $ \ds
            \lim_{n \to +\infty} a_n = \sup \{a_n \mid n \geq 0\}.
        $
    (É possível provar, de forma análoga, que se a sequência $\left(a_n\right)_{n \geq 0}$ for decrescente e limitada inferiormente, então 
        $ \ds
            \lim_{n \to +\infty} a_n = \inf \{a_n \mid n \geq 0\}.)
        $    \end{comment}
\end{proof}

\begin{teo} \label{teo.calc:bolzanoweierstrass}
    (Bolzano-Weierstrass) Toda sequência limitada possui uma subsequência convergente.
\end{teo}

\begin{proof}
    Pelo teorema \eqref{teo:seqmonlim}, basta mostrar que toda sequência possui uma subsequência monótona. Seja $\left( x_n \right)_{n \in \N}$ uma sequência limitada. Um índice $k \in \N$ é dito \textit{básico} quando $x_p \geq x_k$ para todo $p > k$, isto é, $x_k$ é menor ou igual aos termos que o sucedem. 
        \begin{itemize}
            \item Se existem infinitos índices básicos $n_1 < n_2 < n_3 < \cdots$, então $x_{n_1} \leq x_{n_2} \leq x_{n_3} \leq \cdots$, de modo que a subsequência $\left(x_{n_i} \right)_{i \in \N}$ é crescente; como ela é limitada, pelo teorema \eqref{teo:seqmonlim}, ela é convergente.
            \item Por outro lado, se o número de índices básicos é finito, seja $n_1 \in \N$ maior que todos eles (se o número de índices básicos for 0, qualquer $n_1$ funciona). Como $n_1$ não é um índice básico, existe um índice $n_2 \in \N_{>n_1}$ tal que $x_{n_2} < x_{n_1}$. Como $n_2$ não é um índice básico, existe um índice $n_3 \in \N_{>n_2}$ tal que $x_{n_3} < x_{n_2}$. Prosseguindo deste modo, obtemos uma subsequência $\left(x_{n_i} \right)_{i \in \N}$ estritamente decrescente; como ela é limitada, pelo teorema \eqref{teo:seqmonlim}, ela é convergente.
        \end{itemize}
    Com isso, toda sequência possui uma subsequência monótona, e como a sequência original é limitada, a subsequência monótona também é, sendo, portanto, convergente.
\end{proof}

\begin{teo} \label{teo.calc:contseq}
    Seja $f : A \subseteq \R \to \R$ uma função e $a \in A$. As seguintes afirmações são equivalentes:
        \begin{itemize}
            \item $f$ é contínua em $a$;
            \item toda sequência $\left( x_n \right)_{n \in \N}$, com $x_n \in A$ para todo $n \in \N$, satisfaz 
                \[
                    x_n \to a \Rightarrow f(x_n) \to f(a).
                \]
        \end{itemize}
\end{teo}

\begin{proof}
\end{proof}

\begin{teo} \label{teo.calc:contunifocont}
    Toda função $f :[a,b] \to \R$ contínua em $[a,b]$ é uniformemente contínua em $[a,b]$.
\end{teo}

\begin{proof}
    Suponha que exista uma função $f$ contínua em $[a,b]$ que não seja uniformemente contínua em $[a,b]$. Negando a definição de continuidade uniforme \eqref{ar1.defi:continuidade}, isso significa que existe $\epsilon_0 \in \R_{>0}$ tal que, para todo $\delta \in \R_{>0}$, existem $x,y \in [a,b]$ tais que $|x-y| < \delta$ e $|f(x) - f(y)| \geq \epsilon_0$. Em particular, escolhendo $\delta_n = \frac{1}{n}$ para cada $n \in \N$, existem $x_n, y_n \in [a,b]$ tais que $|x_n-y_n| < \frac{1}{n}$ e $|f(x_n) - f(y_n)| \geq \epsilon_0$; definimos, assim, duas sequências $\left(x_n \right)_{n \in \N}$ e $\left(y_n \right)_{n \in \N}$. Como $x_n \in [a,b]$ para todo $n \in \N$, a sequência $\left(x_n \right)_{n \in \N}$ é limitada, de modo que, pelo teorema de Bolzano-Weierstrass \eqref{teo.calc:bolzanoweierstrass}, existe uma subsequência $\left(x_{n_k} \right)_{k \in \N}$ que converge para algum $L \in [a,b]$, isto é, $x_{n_k} \to L$. Considerando a subsequência correspondente $\left(y_{n_k} \right)_{k \in \N}$, temos, para todo $k \in \N$,
        \[
            | x_{n_k} - y_{n_k} | < \dfrac{1}{n_k}.
        \]
    Como $n_k \to +\infty$ (pois $\left(n_k\right)_{k \in \N}$ é uma sequência de índices), temos que $\frac{1}{n_k} \to 0$, de modo que
        \[
            \lim_{k \to + \infty} |x_{n_k} - y_{n_k} | = 0.
        \]
    Com isso, sendo $x_{n_k} \to L$, só pode ser $y_{n_k} \to L$. Como $f$ é contínua em $L \in [a,b]$, pelo teorema \eqref{teo.calc:contseq} temos que
        \[
            \lim_{k \to \infty} f(x_{n_k}) = f(L) \quad \text{e} \quad \lim_{k \to \infty} f(y_{n_k}) = f(L),
        \]
    de modo que
        $ \ds
            \lim_{k \to + \infty} |f(x_{n_k}) - f(y_{n_k})| = 0,
        $
    o que contraria a hipótese de ser $|f(x_{n_k}) - f(y_{n_k})| \geq \epsilon_0 > 0$ para todo $k \in \N$. Assim, uma tal função $f$ não pode existir.
\end{proof}

\begin{teo} \label{teo:intenc}
    Se $\left(a_n\right)_{n \geq 0}$ e $\left(b_n\right)_{n \geq 0}$ são sequências tais que $\ds \lim_{n \to + \infty} (b_n - a_n) = 0$ e, para todo $n \in \N$, $a_n \leq a_{n+1} \leq b_{n+1} \leq b_{n}$, então existe um único $\alpha \in \R$ tal que, para todo $n \in \N$, $a_n \leq \alpha \leq b_n$.
\end{teo}

\begin{proof}
    (Existência) A segunda condição nos diz que $\left(a_n\right)_{n \geq 0}$ é crescente (pois $a_n \leq a_{n+1}$) e limitada superiormente (pois todo $b_n$ é uma cota superior dessa sequência). Analogamente, $\left(b_n\right)_{n \geq 0}$ é decrescente e limitada inferiormente. Assim, pelo Teorema \eqref{teo:seqmonlim}, existem $\ds \alpha := \lim_{n \to + \infty} a_n$ e $\ds \beta := \lim_{n \to + \infty} b_n$, e então $\ds 0 = \lim_{n \to + \infty} (b_n - a_n) = \alpha - \beta$, donde $\alpha = \beta$. Ainda pelo Teorema \eqref{teo:seqmonlim}, $\alpha = \sup \{a_n \mid n \in \N \}$, e então, para todo $n \in \N$, $a_n \leq \alpha$. Analogamente, $\alpha = \beta = \inf{\{b_n \mid n \in \N \}}$, e então, para todo $n \in \N$, temos que $\alpha \leq b_n$. Logo, existe um $\alpha \in \R$ tal que, para todo $n \in \N$, $a_n \leq \alpha \leq b_n$.

    (Unicidade) Suponha que existe $\alpha_1 \in \R$ para o qual também vale $a_n \leq \alpha_1 \leq b_n$. Daí, $0 \leq \alpha_1 - a_n \leq b_n - a_n$; observando que $\ds \lim_{n \to + \infty} 0 = 0$ e $\ds \lim_{n \to + \infty} (b_n - a_n) = 0$, temos, pelo Teorema do Confronto, que $\ds 0 = \lim_{n \to + \infty} (\alpha_1  - a_n) = \alpha_1 - \alpha$, isto é, $\alpha_1 = \alpha$. Logo, é único o $\alpha \in \R$ tal que, para todo $n \in \N$, $a_n \leq \alpha \leq b_n$.
\end{proof}

\begin{cor} \label{teo:intenc2}
    (Intervalos Encaixantes) Se $\left([a_n, b_n]\right)_{n \geq 0}$ for uma sequência de intervalos fechados em que, para todo $n \in \N$, $[a_n, b_n] \supset [a_{n+1}, b_{n+1}]$, e $\ds \lim_{n \to + \infty} (b_n - a_n) = 0$, então o conjunto
        $
            \bigcap_{n=0}^{\infty} [a_n, b_n]
        $
    é unitário.
\end{cor}

\begin{proof}
    Este enunciado é equivalente ao enunciado do Teorema \eqref{teo:intenc}.
\end{proof}

\begin{cor} \label{teo:intenc3}
    Se $\left([a_n, b_n]\right)_{n \geq 0}$ for uma sequência de intervalos encaixantes, com $a_n, b_n \geq 0$, então $\left([a^m_n, b^m_n]\right)_{n \geq 0}$, com $m \geq 2$ natural, também será uma sequência de intervalos encaixantes.
\end{cor}

\begin{proof}
    Basta ver que, para todo $n \in \N$,
        \begin{align*}
            [a_n, b_n] \supset [a_{n+1}, b_{n+1}] &\Leftrightarrow a_n \leq a_{n+1} \leq b_{n+1} \leq b_n \\ 
            &\Leftrightarrow a^m_n \leq a^m_{n+1} \leq b^m_{n+1} \leq b^m_n \\
            &\Leftrightarrow [a^m_n, b^m_n] \supset [a^m_{n+1}, b^m_{n+1}],
        \end{align*}
    e ainda,
        \begin{align*}
            \lim_{n \to + \infty} (b^m_n - a^m_n) &= \left( \lim_{n \to + \infty} b_n \right)^m - \left( \lim_{n \to + \infty} a_n \right)^m \\ 
            &= \left( \lim_{n \to + \infty} a_n \right)^m - \left( \lim_{n \to + \infty} a_n \right)^m = 0.
        \end{align*}
    Logo, $\left([a^m_n, b^m_n]\right)_{n \geq 0}$ é de intervalos encaixantes.

    Ademais, se $\alpha$ é o real que satisfaz, para todo $k \in \N$, $a_k \leq \alpha \leq b_k$, então $\alpha^m$ é o real que satisfaz, para todo $k \in \N$, $a^m_k \leq \alpha^m \leq b^m_k$.\footnote{Prove! O argumento é semelhante ao argumento da unicidade no Teorema \eqref{teo:intenc}.}
\end{proof}

\section{Teoremas do Valor Intermediário e de Weierstrass}

\begin{teo}[Bolzano]
    Seja $f : [a,b] \to \R$ uma função contínua em $[a,b]$. Se $f(a) \cdot f(b) < 0$, então existe $c \in (a,b)$ tal que $f(c) = 0$.
\end{teo}

\begin{proof}
    Suponha, sem perda de generalidade, que $f(a) < 0$ e $f(b) > 0$. Pois tome
        \[
            S := \{x \in [a,b] : f(x) \leq 0\}.
        \]
    Temos $S \neq \emptyset$ pois $a \in S$ já que $f(a) < 0$. $S$ é limitado pois $S \subsetneq [a,b]$. Assim, existe $p := \sup{S}$. Provemos que $f(p) = 0$. De fato, pela tricotomia de $<$ em $\R$, ou $f(p) > 0$, ou $f(p) < 0$, ou $f(p) = 0$. 
        \begin{itemize}
            \item Se fosse $f(p) > 0$, pela conservação do sinal existiria $\delta \in \R_{>0}$ tal que $f(x) > 0$ para todo $x \in (p-\delta,p+\delta) \cap [a,b]$. Com isso, se $x \in S$, então $x \leq p - \delta$, de modo que $p - \delta$ é uma cota superior de $S$, uma contradição pois $p - \delta < p = \sup{S}$.
            \item Se fosse $f(p) < 0$, pela conservação do sinal existiria $\delta \in \R_{>0}$ tal que $f(x) < 0$ para todo $x \in (p-\delta,p+\delta) \cap [a,b]$. Em particular, se $x \in (p,p+\delta) \cap [a,b]$, então $f(x) > 0$ e $x \in S$, uma contradição pois $x > p = \sup{S}$.
        \end{itemize}
    Logo, só pode ser $f(p) = 0$. Além disso, como $p \in [a,b]$ e $f(a) < 0$ e $f(b) > 0$, temos que $p \in (a,b)$. \itemproof
\end{proof}

\begin{proof}
    Suponha, sem perda de generalidade, que $f(a) < 0$ e $f(b) > 0$. Construamos uma sequência de intervalos $\left([a_n, b_n]\right)_{n \geq 0}$ recursivamente do seguinte modo: $a_0 := a$, $b_0 := b$ e
        \[
            \begin{cases}
                a_{n+1} := \dfrac{a_n + b_n}{2} \text{ e } b_{n+1} := b_n, & \text{ se } f \left( \dfrac{a_n + b_n}{2} \right) < 0 \\
                a_{n+1} := a_n \text{ e } b_{n+1} := \dfrac{a_n + b_n}{2}, & \text{ se } f \left( \dfrac{a_n + b_n}{2} \right) \geq 0.
            \end{cases}
        \]
    É fácil ver que, para todo $n \in \N$, temos $a_n \leq a_{n+1} \leq b_{n+1} \leq b_n$ e $\ds \lim_{n \to +\infty} (b_n - a_n) = 0$. Com isso,  $\left([a_n, b_n]\right)_{n \geq 0}$ é uma sequência de intervalos encaixantes, de modo que existe um único $c \in [a,b]$ tal que, para todo $n \in \N$, $a_n \leq c \leq b_n$. Em particular, temos que $f(a_n) < 0 \leq f(b_n)$ para todo $n \in \N$.

    Pela continuidade de $f$, $\ds \lim_{n \to + \infty} f(a_n) = f(c)$ e $\ds \lim_{n \to + \infty} f(b_n) = f(c)$, e como $f(a_n) < 0 \leq f(b_n)$ para todo $n \in \N$, temos, pelo Teorema do Confronto, que $f(c) =0$.
\end{proof}

\begin{teo}[Valor Intermediário]
    Seja $f : [a,b] \to \R$ uma função contínua em $[a,b]$.
        \begin{enumerate}[leftmargin=*, align=left, label=\textbf{(\alph*)}]
            \item Se $f(a) \leq f(b)$, então para todo $\gamma \in [f(a),f(b)]$ existe $c \in [a,b]$ tal que $f(c) = \gamma$.
            \item Se $f(b) \leq f(a)$, então para todo $\gamma \in [f(b),f(a)]$ existe $c \in [a,b]$ tal que $f(c) = \gamma$.
            \item Para todo $\gamma \in [\min{\{f(a),f(b)\}}, \max{\{f(a),f(b)\}}]$ existe $c \in [a,b]$ tal que $f(c) = \gamma$.
        \end{enumerate}

\end{teo}

\begin{proof}
    Pois tome $g(x) := f(x) - \alpha$, com $x \in [a,b]$. Como $f$ é contínua em $[a,b]$, $g$ também o é. Em particular, $g(a) = f(a) - \alpha < 0$ e $g(b) = f(b) - \alpha > 0$, de modo que, pelo Teorema do Anulamento, existe $c \in [a,b]$ tal que $g(c) = 0$, isto é, $f(c) = \alpha$.
\end{proof}

\begin{teo} \label{teo.calc:limitacao}
    (Limitação) Se uma função $f : [a,b] \to \R$ é contínua em $[a,b]$, então $f$ é limitada em $[a,b]$.
\end{teo}

\begin{proof}
    Suponhamos, por absurdo, que $f$ não seja limitada em $[a,b]$. Colocando $a_0 := a$ e $b_0 := b$, existe $x_0 \in [a_0, b_0]$ tal que $|f(x_0)| > 0$. Suponha, indutivamente, que $[a_n,b_n] \subset [a_0, b_0]$ esteja bem definido, sendo $f$ não limitada em $[a_n,b_n]$. Em particular, existe $x_n \in [a_n, b_n]$ tal que $|f(x_n)| > n$. Agora, defina $\ds a_{n+1} := a_n$ e $b_{n+1} := \dfrac{a_n + b_n}{2}$, se $f$ não for limitada em $\ds \left[a_n, \dfrac{a_n + b_n}{2} \right]$, ou $a_{n+1} := \dfrac{a_n + b_n}{2}$ e $b_{n+1} := b_n$ se $f$ não for limitada em $\ds \left[\dfrac{a_n + b_n}{2}, b_n \right]$. No intervalo em que $f$ não for limitada, existirá $x_{n+1}$ nesse intervalo tal que $|f(x_{n+1})|>n+1$. 
    
    Assim, fica construída uma sequência $\left([a_n, b_n]\right)_{n \geq 0}$ de intervalos encaixantes tal que, para todo $n \in \N$, existe $x_n \in [a_n,b_n]$ com $|f(x_n)| > n$. Em particular, isso significa que $\ds \lim_{n \to + \infty} |f(x_n)| = + \infty$. Agora, sendo $c$ o único real tal que $a_n \leq c \leq b_n$ para todo $n \in \N$, pelo Teorema do Confronto temos que $x_n \to c$, e sendo $f$ contínua em $c$, temos que $\ds \lim_{n \to + \infty} |f(x_n)| = |f(c)|$, absurdo! Logo, $f$ não ser limitada em $[a,b]$ nos leva a uma contradição, de modo que $f$ é, então, limitada em $[a,b]$.
\end{proof}

\begin{teo}[Valor Extremo, ou Weierstrass] \label{teo.calc:weierstrass}
    Se uma função $f: [a,b] \to \R$ é contínua em $[a,b]$, então existem $x_1, x_2 \in [a,b]$ tais que $f(x_1) \leq f(x) \leq f(x_2)$ para todo $x \in [a,b]$.
\end{teo}

\begin{proof}
    Pelo teorema da limitação \eqref{teo.calc:limitacao}, $f$ é limitada em $[a,b]$, de modo que o conjunto $A:= \{ f(x) : x \in [a,b]\}$ admite $M := \sup A$ e $m := \inf A$. Isto significa que $m \leq f(x) \leq M$ para todo $x \in [a,b]$. Afirmamos que existe $x_2 \in [a,b]$ para o qual $M = f(x_2)$. De fato, se um tal $x_2$ não existisse, seria $f(x) < M$ para todo $x \in [a,b]$, de modo que a função $\ds g(x) := \dfrac{1}{M - f(x)}$, com $x \in [a,b]$, seria contínua, mas não limitada, em $[a,b]$, o que é uma contradição (se $g$ fosse limitada, então existiria $\gamma > 0$ tal que $\ds 0< \dfrac{1}{M-f(x)}<\gamma$, donde $f(x) < M - \dfrac{1}{\gamma}$, de modo que $M$ não seria supremo de $A$). Assim, não pode ser $f(x) < M$, e como $f(x) \leq M$, existirá $x_2 \in [a,b]$ para o qual $f(x_2) =M$. Analogamente, prova-se que existe $x_1 \in [a,b]$ para o qual $f(x_1) = m$.
\end{proof}
\section{Algumas Funções Transcendentais}

\subsection{Trigonometria, parte I}

\begin{teo} \label{teo:sencos}
    Existe um único par de funções, $s,c : \R \to \R$, para as quais
        \begin{itemize}
            \item $s(0)=0$ e $c(0)=1$;
            \item $\forall x \forall y : s(x-y)=s(x)c(y) - s(y)c(x) \text{ e } c(x-y) = c(x)c(y) + s(x)s(y)$;
            \item $\exists r>0 : 0<x<r \Rightarrow 0 < s(x)<x<\dfrac{s(x)}{c(x)}.$
        \end{itemize}
    A função $s$ é chamada de \textit{seno} e será indicada por $\sin{x}$, enquanto $c$ é chamada de \textit{cosseno} e será indicada por $\cos{x}$.
\end{teo}

\begin{proof}
\end{proof}

\begin{prop}
    \textbf{(a)} (Identidade Fundamental) Temos $\sin^2{x} + \cos^2{x} = 1$ para todo $x \in \R$.

    \textbf{(b)} $\sin$ é uma função ímpar, isto é, $\sin{-x} = - \sin{x}$ para todo $x \in \R$, enquanto $\cos$ é uma função par, isto é, $\cos{-x} =  \cos{x}$ para todo $x \in \R$.

    \textbf{(c)} Temos, para todos $x,y \in \R$,
        \begin{align*}
            \sin{(x+y)} &= \sin{x}\cos{y} + \sin{y}\cos{x} \\ \cos{(x+y)} &= \cos{x}\cos{y} - \sin{x}\sin{y}
        \end{align*}
    \textbf{(d)} Temos $\sin{2x} = 2 \sin{x} \cos{x}$ e $\cos{2x} = \cos^2{x} - \sin^2{x}$ para todo $x \in \R$.

    \textbf{(e)} Temos $\sin^2{x} = \dfrac{1}{2} - \dfrac{1}{2} \cos{2x}$ e $\cos^2{x}  = \dfrac{1}{2} + \dfrac{1}{2} \cos {2x}$ para todo $x \in \R$.
\end{prop}

\begin{proof}
\end{proof}

\begin{teo}
    As funções $\sin$ e $\cos$ são contínuas em $\R$.
\end{teo}

\begin{proof}
    Pelo terceiro item no resultado \eqref{teo:sencos}, existe $r>0$ tal que $|x| < r \Rightarrow |\sin{x}| \leq |x|$. Usaremos isso para provar que $|x-p|<2r \Rightarrow |\sin{x} - \sin{p}| \leq |x-p|$. Pois tome:
        \begin{align*}
            |\sin{x} - \sin{p}| &= \left|2 \sin{\left(\dfrac{x-p}{2}\right)} \cos{\left(\dfrac{x+p}{2}\right)} \right| \\ &= 2 \left| \sin{\left(\dfrac{x-p}{2}\right)}  \right| \left| \cos{\left(\dfrac{x+p}{2}\right)}  \right|;
        \end{align*}
    como $\ds \left| \cos{\left( \dfrac{x+p}{2} \right)} \right| \leq 1$, temos que 
        \[
            |\sin{x} - \sin{p}| \leq 2 \left| \sin{\left(\dfrac{x-p}{2}\right)}  \right|,
        \]
    e então, pelo fato mencionado acima, vem 
        \[
            |x-p| < 2r \Rightarrow \left| \sin{\left(\dfrac{x-p}{2}\right)}  \right| \leq \left| \dfrac{x-p}{2} \right|,
        \]
    donde $|x-p|<2r \Rightarrow |\sin{x} - \sin{p}| \leq |x-p|$. De maneira completamente análoga, prova-se que $|x-p|<2r \Rightarrow |\cos{x} - \cos{p}| \leq \cos{x} - \cos{p}$.

    Com isso, $|x-p| < 2r \Rightarrow 0 \leq |\sin{x} - \sin{p}| \leq |x-p|$, e como $\ds \lim_{x \to p} |x-p|=0$, pelo Teorema do Confronto vem $\ds \lim_{x \to p} |\sin{x} - \sin{p} | = 0$, donde $\ds \lim_{x \to p} \sin{x} = \sin{p}$. De modo completamente análogo, prova-se que $\ds \lim_{x \to p} \cos{x} = \cos{p}$. Logo, $\sin$ e $\cos$ são contínuas em todo $p \in \R$.  
\end{proof}

\begin{teo}
     Temos $\ds \lim_{x \to 0} \dfrac{\sin{x}}{x}=1$ e $\ds \lim_{x \to 0} \dfrac{1 - \cos x}{x} = 0$.
\end{teo}

\begin{proof}
\end{proof}

\subsection{Exponencial e Logaritmo}

\subsubsection{Expoentes Racionais}

O intuito aqui é definir $a^x$ quando $x \in \Q$. A referência é \cite{guidorizzi1}.

\begin{teo} \label{teo:raizes}
    \leavevmode
        \begin{enumerate}[leftmargin=*, align=left, label=\textbf{(\alph*)}]
            \item Para quaisquer $a \in \R_{>0}$ e $n \in \N_{\geq 2}$ existe um único $x \in \R_{>0}$ tal que $x^n=a$.
            \item Para quaisquer $a \in \R$ e $n \in \N$ ímpar existe um único $x \in \R$ tal que $x^n=a$.
        \end{enumerate}
\end{teo}

\begin{proof}
    \textbf{(a)} Iremos construir duas sequências, $\left(a_k\right)_{k \geq 0}$ e $\left(b_k\right)_{k \geq 0}$, no sentido dos Teoremas \eqref{teo:intenc} e \eqref{teo:intenc2}.

    Seja $A_0$ o maior natural tal que $A^n_0 \leq a < (A_0 + 1)^n$. Em particular, isso nos diz que, se o real $x>0$ existe, então ele satisfaz $A_0 \leq x < A_0 + 1$. Agora, para cada $k \geq 1$, seja $A_k$ um elemento do conjunto $\{ 0, 1, \ldots, 9\}$ (isto é, $A_k$ é um \textit{dígito}, ou \textit{algarismo}), e defina $\left(a_k\right)_{k \geq 0}$ e $\left(b_k\right)_{k \geq 0}$ por
        \begin{gather*}
            a_k := \max \left\{ \sum_{i=0}^{k} \dfrac{A_k}{10^k} \mid \left( \sum_{i=0}^{k} \dfrac{A_k}{10^k} \right)^n \leq a \right\},  \forall k \geq 0 \\
            b_k := 
                \begin{cases}
                    A_0 + 1, & \text{se } k = 0 \\
                    a_{k-1} + \dfrac{A_k + 1}{10^k}, & \text{se } k \geq 1
                \end{cases}
        \end{gather*}
    É imediato que, para todo $k \in \N$, $a^n_k \leq a < b^n_k$. Em particular, isso nos diz que, se o real $x>0$ existe, então ele satisfaz $a_k \leq x < b_k$. Provemos agora que $a_k \leq a_{k+1} \leq b_{k+1} \leq b_k$:
    
    \begin{enumerate}[label=\roman*.]
        \item $a_k \leq a_{k+1}$: para ver isso, basta ver que
            \[
                a_{k+1} = \sum_{i=0}^{k+1} \dfrac{A_k}{10^k} = \sum_{i=0}^{k} \dfrac{A_k}{10^k} + \dfrac{A_{k+1}}{10^k} = a_k + \dfrac{A_{k+1}}{10^k}.
            \]
        Daí, $a_{k+1} - a_k = \dfrac{A_{k+1}}{10^k} \geq 0 \Rightarrow a_k \leq a_{k+1}$.
        \item $a_{k+1} \leq b_{k+1}$: releia uma das afirmações ditas acima.
        \item $b_{k+1} \leq b_k$: basta ver que
            \begin{align*}
                b_k - b_{k+1} &= \left( a_{k-1} + \dfrac{A_k + 1}{10^k} \right) - \left( a_k + \dfrac{A_{k+1} + 1}{10^{k+1}} \right) \\ 
                &= \left( a_{k-1} + \frac{A_k + 1}{10^k} \right) - \left( a_{k-1} + \frac{A_k}{10^k} + \frac{A_{k+1} + 1}{10^{k+1}} \right) \\ &= \frac{A_k + 1}{10^k} - \frac{A_k}{10^k} - \frac{A_{k+1} + 1}{10^{k+1}} \\ &= \frac{1}{10^k} - \frac{A_{k+1} + 1}{10^{k+1}} = \frac{9 - A_{k+1}}{10^{k+1}}.
            \end{align*}
        Como $A_{k+1} \in \{1, 2, \ldots, 9\}$, temos que $\dfrac{9 - A_{k+1}}{10^{k+1}} \geq 0$, donde $b_{k+1} \leq b_k$.
    \end{enumerate}

    Por fim, provemos que $\ds \lim_{k \to + \infty} (b_k - a_k) = 0$. De fato, veja que
        \begin{align*}
            b_k - a_k &= a_{k-1} + \dfrac{A_k + 1}{10^k} - \sum_{i=0}^{k} \dfrac{A_k}{10^k} \\ &= \sum_{i=0}^{k-1} \dfrac{A_k}{10^k} + \dfrac{A_k}{10^k} + \dfrac{1}{10^k} - \sum_{i=0}^{k} \dfrac{A_k}{10^k} \\
            &= \sum_{i=0}^{k} \dfrac{A_k}{10^k} + \dfrac{1}{10^k} - \sum_{i=0}^{k} \dfrac{A_k}{10^k} = \dfrac{1}{10^k},
        \end{align*}
    e então $\ds \lim_{k \to + \infty} (b_k - a_k) = \lim_{k \to + \infty} \dfrac{1}{10^k} = 0$.

    Com isso, a sequência $\left([a_k, b_k]\right)_{k \geq 0}$ é de intervalos encaixantes, e então existe um único $x \in \R$ tal que, para todo $k \in \N$, $a_k \leq x \leq b_k$. Pelo resultado \eqref{teo:intenc3}, a sequência $\left([a^n_k, b^n_k]\right)_{k \geq 0}$ também é de intervalos encaixantes; temos então que, para todo $k \in \N$, $a^n_k \leq x^n \leq b^n_k$. No entanto, vimos isso também vale para o real $a>0$: para todo $k \in \N$, $a^n_k \leq a \leq b^n_k$. Daí, $x^n = a$.

    \textbf{(b)} Se $a>0$, então, pelo item (a), existe $x>0$ tal que $x^n=a$. Por outro lado, se $a<0$, então $-a>0$, e pelo item (a) existe $x>0$ tal que $x^n=-a$; daí, $(-x)^n=a$.
\end{proof}

\begin{defi}
    Seja $n \geq 1$ um natural.
    
    \textbf{(a)} Para cada real $a$, o único real $x$ tal que $x^n=a$ será chamado de \textit{raíz $n$-ésima} de $a$ e será denotado por $\sqrt[n]{a}$. Assim, temos que $\ds (\sqrt[n]{a})^n=a$.

    \textbf{(b)} Como para todo $x \in \R_+$ existe um único $\sqrt[n]{x} \in \R_+$, a relação $\{ (x,y) \in \R_+ \times \R_+ : y = \sqrt[n]{x} \}$ é uma função, que será chamada de \textit{função raiz}.
\end{defi}

\begin{cor}
    Se $a,b >0$ são reais, $m,n \geq 1$ são naturais e $p$ é um inteiro, então

    \textbf{(a)} $\ds \sqrt[n]{a^p} = \sqrt[nm]{a^{pm}}$;

    \textbf{(b)} $\ds \sqrt[n]{\sqrt[m]{a}} = \sqrt[nm]{a}$;

    \textbf{(c)} $\ds \sqrt[n]{a} \cdot \sqrt[n]{b} = \sqrt[n]{ab}$;

    \textbf{(d)} $\ds a<b \Leftrightarrow \sqrt[n]{a} < \sqrt[n]{b}$.
        \begin{comment} \[ \ds
            \textbf{(a)} \ \sqrt[n]{a^p} = \sqrt[nm]{a^{pm}};
            \quad \textbf{(b)} \ \sqrt[n]{\sqrt[m]{a}} = \sqrt[nm]{a};
            \quad \textbf{(c)} \ \sqrt[n]{a} \cdot \sqrt[n]{b} = \sqrt[n]{ab};
            \quad \textbf{(d)} \ a<b \Leftrightarrow \sqrt[n]{a} < \sqrt[n]{b}.
        \] \end{comment}
\end{cor}

\begin{proof}
\end{proof}

\begin{prop}
    A função $f(x) := \sqrt[n]{x}$ é contínua em todo seu domínio.
\end{prop}

\begin{proof}
\end{proof}

\begin{defi}
    (Expoente Racional) Seja $a>0$ um real. Para cada racional $r$ (isto é, $r := m/n$, com $m \in \Z$ e $n \in \N \backslash \{0 \}$), definimos $\ds a^r = a^{\frac{m}{n}} := \sqrt[n]{a^m}$.\footnote{Como $\sqrt[n]{a^p} = \sqrt[nm]{a^{pm}}$, a definição acima não depende da escolha da fração $m/n$. Em particular, $f: \Q \to \R_+^*$ tal que $f(r) = a^r$ fica bem definida como função.}
\end{defi}

\begin{prop}
    Para quaisquer $a,b >0$ reais e $r,s$ racionais, temos que
    
    \textbf{(a)} $a^r \cdot a^s = a^{r+s}$;
    
    \textbf{(b)} $(a^r)^s = a^{rs}$;
    
    \textbf{(c)} $(ab)^r = a^r b^r$;
    
    \textbf{(d)} $\dfrac{a^r}{a^s} = a^{r-s}$;
    
    \textbf{(e)} $\left( \dfrac{a}{b} \right)^r =  \dfrac{a^r}{b^r}$;
    
    \textbf{(f)} Se $1<a$ e $r<s$, então $a^r < a^s$;

    \textbf{(g)} Se $0 < a < 1$ e $r<s$, então $a^s < a^r$.

    \begin{comment} \textbf{(a)} $\ds a^r \cdot a^s = a^{r+s}$, $\ds (a^r)^s = a^{rs}$, $\ds (ab)^r = a^r b^r$, $\ds \dfrac{a^r}{a^s} = a^{r-s}$, $\ds \left( \dfrac{a}{b} \right)^r =  \dfrac{a^r}{b^r}$.\end{comment}
\end{prop}

\begin{proof}
\end{proof}

\subsubsection{Expoentes Reais}

O intuito aqui é definir $a^x$ quando $x \in \R$.

\begin{teo}
    \textbf{(a)} Se $f:\R \to \R$ é contínua em $\R$ e $f(x) = 0$ para todo $x \in \Q$, então $f(x) = 0$ para todo $x \in \R$.

    \textbf{(b)} Se $f,g:\R \to \R$ são contínuas em $\R$ e $f(x) = g(x)$ para todo $x \in \Q$, então $f(x) = g(x)$ para todo $x \in \R$.\footnote{Isto nos diz que se duas funções contínuas em $\R$ coincidem em $\Q$, então elas são iguais.}

    \textbf{(c)} Se $f,g:\R \to \R$ são contínuas em $\R$ e existe $0 < a \neq 1$ tal que $f(x) = a^x$ e $g(x)=a^x$ para todo $x \in \Q$, então $f(x) = g(x)$ para todo $x \in \R$.\footnote{Isto significa que poderá existir no máximo uma função definida e contínua em $\R$ que coincide com $a^x$ para todo $x \in \Q$.}
\end{teo}

\begin{proof}
    \textbf{(a)} Segue como corolário da conservação do sinal \eqref{consin2}. \itemproof

    \textbf{(b)} Basta aplicar o resultado do item (a) na função $h(x) := f(x) - g(x)$. \itemproof

    \textbf{(c)} Segue como corolário do item (b) acima.
\end{proof}

\begin{teo} \label{teo:expirra}
    \textbf{(a)} Se $a>1$ é um real, então para todo $\epsilon >0$ existe um natural $n$ tal que $a^{\frac{1}{n}} -1 < \epsilon$. \begin{comment} Simbolicamente,
        \[
            \forall a (a \in \R \land a>1 \to \forall \epsilon (\epsilon \in \R \land \epsilon > 0 \to \exists n (n \in \N \land a^{\frac{1}{n}} -1 < \epsilon)) ).
        \] \end{comment}
    
    \textbf{(b)} Se $a>1$ e $x$ são reais. então para todo $\epsilon >0$ existem racionais $r$ e $s$ tais que $r<x<s$ e $a^s-a^r<\epsilon$. \begin{comment} Simbolicamente,
        \[
            \forall a \forall x (a,x \in \R \land a>1 \to \forall \epsilon (\epsilon \in \R \land \epsilon > 0 \to \exists r \exists s (r,s \in \Q \land r<x<s \land a^s - a^r < \epsilon) ) ).
        \] \end{comment}
    
    \textbf{(c)} Se $a>1$ é um real, então para todo $x$ real existe um único real $\gamma$ tal que $a^r < \gamma < a^s$ para todos os racionais $r$ e $s$ com $r<x<s$. \begin{comment} Simbolicamente,
        \[
            \forall a \forall x (a,x \in \R \land a>1 \to \exists ! \gamma (\gamma \in \R \land \forall r \forall s (r,s \in \Q \land r < x < s \to a^r < \gamma < a^s ) ) ).
        \] \end{comment}
    
    \textbf{(d)} Se $0 < a \neq 1$ é um real, então existe uma única função definida e contínua em $\R$ tal que $f(r) = a^r$ para todo $x \in \Q$.
\end{teo}

\begin{proof} 
    \textbf{(a)} Sabemos que $(1+\epsilon)^n \geq 1 + n\epsilon$ para todo natural $n \geq 1$. Pois tome $n$ tal que $1 + n\epsilon > a$ (basta que $ n > \frac{a-1}{\epsilon}$); daí, $(1+\epsilon)^n>a$, donde $a^{\frac{1}{n}} - 1 < \epsilon$. \itemproof

    \textbf{(b)} Para racionais $t>x$ temos $a^r<a^t$ para todo racional $r<x$. Pelo item anterior, existe um natural $n$ para o qual $a^{\frac{1}{n}} - 1 < \epsilon \cdot a^{-t}$, donde $a^t (a^{\frac{1}{n}}-1) < \epsilon$. Tomando racionais $r$ e $s$, com $r<x<s$, para os quais $s-r < 1/n$, temos $a^s - a^r = a^r (a^{s-r} -1) < a^t (a^{\frac{1}{n}} - 1) < \epsilon$. \itemproof

    \textbf{(c)} O conjunto $A := \{a^r : r \in \Q \land r<x \}$ é não vazio e limitado superiormente (por todo $a^s$, com $s>x$). Assim, existe $\gamma := \sup A$. Claramente, $a^r \leq \gamma \leq a^s$, mas, mais geralmente, $a^r < \lambda < a^s$ (prove!). Provemos, agora, a unicidade de $\gamma$. Se $\gamma'$ for tal que $a^r < \gamma' < a^s$ para quaisquer racionais $r$ e $s$, com $r<x<s$, então $|\gamma - \gamma'| < a^s - a^r$. Pelo item anterior, para todo $\epsilon > 0$ existem $r_0$ e $s_0$, com $r_0 < x < s_0$, para os quais $a^{s_0} - a^{r_0} < \epsilon$; logo, temos $|\gamma - \gamma'| < \epsilon$ para todo $\epsilon > 0$, donde $\gamma = \gamma'$. \itemproof

    \textbf{(d)} Pelo item anterior, para quaisquer $a>1$ e $x$ reais existe um único $\gamma$; assim, basta tomar $f(x) := \gamma$. Antes de provar a continuidade de $f$, provemos que $f$ é estritamente crescente. De fato, tomando reais $x_1 < x_2$ temos que $a^{r_1} < f(x_1) < a^{s_1}$ e $a^{r_2} < f(x_2) < a^{s_2}$ para todos os racionais $r_1$, $r_2$, $s_1$ e $s_2$ tais que $r_1 < x_1 < s_1$ e $r_2 < x_2 < s_2$. Como existe $s$ racional tal que $x_1 < s < x_2$, temos $f(x_1)<a^s<f(x_2$, donde $f$ é estritamente crescente.

    Agora, sendo $p \in \R$, pelo item (b) deste teorema, para todo $\epsilon > 0$ existem racionais $r$ e $s$, com $r<x<s$, para os quais $a^s-a^r < \epsilon$. Em particular, para todo $x \in \left]r,s\right[$, temos $a^r<f(x)<a^s$, e como também $a^r<f(p)<a^s$, temos $|f(x) - f(p)| < a^s - a^r < \epsilon$. Assim, pelo Teorema \eqref{teo:intervalos}, $f$ é contínua em $p$. Como $p$ foi tomado de modo arbitrário, segue que $f$ é contínua em $\R$. 

    Por outro lado, se $0<a<1$, então $f(x) := \left( \frac{1}{a} \right)^{-x}$ está bem definida em $\R$, é contínua em $\R$ e coincide com $a^r$ nos racionais.
\end{proof}

\begin{defi}
    Seja $0<a\neq1$ um real. Para todo $x \in \R$, definimos $a^x := f(x)$, em que $f$ é a função a que se refere o item (d) do Teorema \eqref{teo:expirra}.
\end{defi}

\begin{prop}
    Para quaisquer $0 < a,b \neq 1$ reais e $x,y$ reais, temos que
    
    \textbf{(a)} $a^x \cdot a^y = a^{x+y}$;
    
    \textbf{(b)} $(a^x)^y = a^{xy}$;
    
    \textbf{(c)} $(ab)^x = a^x b^x$;
    
    \textbf{(d)} $\dfrac{a^x}{a^y} = a^{x-y}$;
    
    \textbf{(e)} $\left( \dfrac{a}{b} \right)^x =  \dfrac{a^x}{b^x}$;
    
    \textbf{(f)} Se $1<a$ e $x<y$, então $a^x < a^y$;

    \textbf{(g)} Se $0 < a < 1$ e $x<y$, então $a^y < a^x$.
\end{prop}

\begin{proof}
\end{proof}

\subsubsection{Logaritmos}

\begin{teo}
    Para quaisquer reais $0<a\neq 1$ e $b>0$, existe um único real $\gamma := \log_{a}{b}$ tal que $a^{\gamma} = a^{\log_{a}{b}} = b$. Em particular, $f:\R_+ \to \R$ tal que $f(x) := \log_{a}{x}$ fica bem definida. 
\end{teo}

\begin{proof}    
\end{proof}

\begin{teo}
    Para quaisquer $0<a,b \neq 1$ e $x,y>0$, temos que
        \begin{align*}
            \log_{a}{xy} &= \log_{a}{x} + \log_{a}{y} \\
            \log_{a}{x^y} &= y \log_{a}{x} \\
            \log_{a}{\dfrac{x}{y}} &= \log_{a}{x} - \log_{a}{y} \\
            \log_{a}{x} &= \dfrac{\log_{b}{x}}{\log_{b}{a}}
        \end{align*}
    E ainda, se $a>1$ e $x<y$, então $\ds \log_{a}{x} < \log_{a}{y}$ (isto é, se $a>1$ então $f(x) := \log_{a}{x}$ é crescente), e se $0<a<1$ e $x<y$, então $\ds \log_{a}{y} < \log_{a}{x}$ (isto é, se $0<a<1$, então $f(x) := \log_{a}{x}$ é decrescente).
\end{teo}

\begin{proof}    
\end{proof}

\begin{teo}
    Se $a>1$, então $\ds \lim_{x \to + \infty} \log_{a}{x} = \infty$ e $\lim_{x \to 0^+} \log_{a}{x} = - \infty$; se $0<a<1$, então $\lim_{x \to + \infty} \log_{a}{x} = -\infty$ e $\lim_{x \to 0^+} \log_{a}{x} = +\infty$.
\end{teo}

\begin{proof}
\end{proof}

\begin{teo}
    A função logarítmica $f(x) := \log_{a}{x}$ é contínua em todo seu domínio.
\end{teo}

\begin{proof}
\end{proof}

\chapter{Derivadas}

\section{Definições e Resultados Iniciais}

\begin{defi} \label{defi.calc:derivada}
    Uma função $f : A \subseteq \R \to \R$ é \textit{derivável} em $a \in A \cap A'$, se existe o limite
        \[
            f'(a) := \lim_{x \to a} \dfrac{f(x) - f(a)}{x-a}.
        \]
    Noutros termos, $f$ é derivável em $a$ se existe o limite $\ds \lim_{x \to a} \dfrac{\Delta{f}}{\Delta{x}}(a)$, onde $\dfrac{\Delta{f}}{\Delta{x}}(a) : A \setminus \{ a\} \to \R$ é a \textit{função quociente de diferenças}, definida por
        \[
            \dfrac{\Delta{f}}{\Delta{x}}(a) := \dfrac{f(x) - f(a)}{x-a}.
        \]
    Sendo $f$ derivável em $a$, o limite $f'(a)$ é a \textit{derivada} de $f$ em $a$.
\end{defi}

\begin{obs}
    O objetivo das proposições seguintes é estabelecer precisamente a equivalência
        \[ \ds
            \lim_{x \to a} \dfrac{f(x) - f(a)}{x-a} = L \Leftrightarrow \lim_{h \to 0} \dfrac{f(a+h) - f(a)}{h} = L,
        \]
    por vezes assumida sem mais explicações.
\end{obs}

\begin{prop} \label{prop.ar1:xmenosa}
    Sejam $f : A \subseteq \R \to \R$ e $a \in A'$. Tem-se $\ds \lim_{x \to a} f(x) = L$ para algum $L \in \R$ se, e somente se, $\ds \lim_{h \to 0} g(h) = L$, onde $g: \{h \in \R_{\neq 0} : a+h \in A \} \to \R$ é definida por $g(h) := f(a+h)$.
\end{prop}

\begin{proof}
    A proposição (???) estabelece que $a \in A' \Leftrightarrow 0 \in \{h \in \R_{\neq 0} : a+h \in A \}'$. 
    
    Daí, $\ds \lim_{x \to a} f(x) = L$ se, e somente se, para todo $\epsilon \in \R_{>0}$ existe $\delta \in \R_{>0}$ tal que $0<|x-a|<\delta \Rightarrow |f(x) - L| < \epsilon$ para todo $x \in A$. Tomando $h := x-a$, temos $x = a+h \in A$, de modo que $h \in B$. Assim, para todo $\epsilon \in \R_{>0}$ existe $\delta \in \R_{>0}$ tal que $0<|h|<\delta \Rightarrow |f(a+h) - L| < \epsilon$ para todo $h \in B$, de modo que $\ds \lim_{h \to 0} g(h) = L$.
\end{proof}

\begin{cor}
    Uma função $f: A \subseteq \R \to \R$ é derivável em $a \in A \cap A'$, com derivada $L \in \R$, se, e somente se, $\ds \lim_{h \to 0} g(h) = L$, onde $g : \{ h \in \R_{\neq 0} : a+h \in A \setminus \{ a\} \} \to \R$ é definida por $g(h) := \dfrac{\Delta{f}}{\Delta{x}}(a+h)$.
\end{cor}

\begin{proof}
    Por definição, $f$ ser derivável em $a$ com derivada $L$ significa que $\ds \lim_{x \to a} g(x) = L$. Definindo $\varphi : \{ h \in \R_{\neq 0} : a+h \in A \setminus \{a \} \} \to \R$ por
        \[
            \varphi(h) = g(a+h) = \dfrac{f(a+h) - f(a)}{(a+h) - a} = \dfrac{f(a+h) - f(a)}{h},
        \]
    pela proposição \eqref{prop.ar1:xmenosa} temos $\ds \lim_{x \to a} g(x) = L \Leftrightarrow \lim_{h \to 0} \varphi(h) = L$, como havíamos afirmado.
\end{proof}

\begin{comment}
\begin{defi}
    Seja $f$ uma função e $p$ um ponto no domínio de $f$.

    \textbf{(a)} O limite
        $ \ds
            \lim_{x \to p} \dfrac{f(x) - f(p)}{x-p},
        $
    ou, equivalentemente,
        $ \ds
            \lim_{h \to 0} \dfrac{f(x+h) - f(x)}{h},            
        $
    quando existe, denomina-se \textit{derivada} de $f$ em $p$ e indica-se por $f'(p)$. Diremos que $f$ é \textit{derivável}, ou \textit{diferenciável}, em $p$.

    \textbf{(b)} Definimos a \textit{reta tangente} ao gráfico de $f$ no ponto $(p, f(p))$ como sendo a reta de equação $y - f(p) = f'(p)(x-p)$. Assim, a derivada de $f$ em $p$ é a \textit{inclinação} da reta tangente ao gráfico de $f$ no ponto $(p, f(p))$.
\end{defi}
\end{comment}

\begin{defi}
    Seja $f$ uma função e $A \subseteq D_f$ o conjunto dos $x \in D_f$ para os quais existe $f'(x)$. A função $f': A \to \R$ dada por $x \to f'(x)$ denomina-se \textit{função derivada} ou, simplesmente, \textit{derivada} de $f$. Diremos, ainda, que $f'$ é a \textit{derivada de 1ª ordem} de $f$, que também pode ser denotada por $f^{(1)}$. Por fim, definimos, indutivamente, $f^{(n+1)} := \left[f^{(n)}\right]'$.
\end{defi}

\begin{defi}
    Seja $f$ uma função, sendo $y := f(x)$. O símbolo $\ds \dfrac{dy}{dx}$, que se lê ``derivada de $y$ em relação a $x$'', denota a derivada de $f$ em $x$, isto é, $\ds \dfrac{dy}{dx} := f'(x)$. Já $\ds \dfrac{d^ny}{dx^n}$ denota a $n$-ésima derivada de $f$ em $x$: $\ds \dfrac{d^ny}{dx^n} := f^{(n)}(x)$. E ainda, $\ds \dfrac{df}{dx}$ denota a função derivada de $y=f(x)$: $\ds \dfrac{df}{dx} := f'$. Naturalmente, então, $\ds \dfrac{df}{dx} (x) := f'(x)$. A derivada de $y=f(x)$ no ponto $p$ é denotada por $\ds \dfrac{dy}{dx} \bigg|_{x = p}$.
    
    \begin{comment} \textbf{(a)} $\ds \dfrac{dy}{dx}$, que se lê ``derivada de $y$ em relação a $x$'', denota a derivada de $f$ em $x$: $\ds \dfrac{dy}{dx} := f'(x)$.
    
    \textbf{(b)} $\ds \dfrac{d^ny}{dx^n}$ denota a $n$-ésima derivada de $f$ em $x$: $\ds \dfrac{d^ny}{dx^n} = f^{(n)}(x)$. 

    \textbf{(c)} $\ds \dfrac{df}{dx}$ denota a função derivada de $y=f(x)$: $\ds \dfrac{df}{dx} := f'$. Naturalmente, então, $\ds \dfrac{df}{dx} (x)$ denota a derivada de $f$ em $x$: $\ds \dfrac{df}{dx} (x) := f'(x)$.
    
    \textbf{(c)} A derivada de $y=f(x)$ no ponto $p$ é denotada por $\ds \dfrac{dy}{dx} \bigg|_{x = p} := f'(p)$. \end{comment}
\end{defi}

\begin{teo}
    Se $f$ for derivável em $p$, então $f$ será contínua em $p$.
\end{teo}

\begin{proof}
\end{proof}

\begin{teo}
    Se $f$ e $g$ são funções deriváveis em $p$, então

    \textbf{(a)} a função $f+g$ é derivável em $p$ e
        \[
            (f+g)'(p) = f(p) + g(p);
        \]
    \textbf{(b)} a função $f \cdot g$ é derivável em $p$ e 
        \[
            (f \cdot g)'(p) = f'(p) \cdot g(p) + g'(p) \cdot f(p);
        \]
    \textbf{(c)} a função $\dfrac{f}{g}$ é derivável em $p$, desde que $g(p) \neq 0$, sendo
        \[
            \left( \dfrac{f}{g} \right)'(p) = \dfrac{f'(p) \cdot g(p) - g'(p) \cdot f(p)}{[g(p)]^2}.
        \]
\end{teo}

\begin{proof}
\end{proof}

\begin{cor}
    derivada de n funções, derivada de kf.
\end{cor}

\begin{lem}
    Seja $f : D_f \subseteq \R \to \R$ uma função derivável em $p \in D_f$. Definindo $\rho : D_f \backslash \{p\} \to \R$ de modo que
        \[
            f(x) = f(p) + f'(p) (x-p) + \rho (x) (x-p),
        \]
    temos que $\ds \lim_{x \to p} \rho (x) = 0$.
\end{lem}

\begin{proof}
\end{proof}

\begin{teo}
    (Regra da Cadeia) Se $f : D_f \subseteq \R \to \R$ e $g : D_g \subseteq \R \to \R$ são funções deriváveis, com $Im_g \subseteq D_f$, então a função composta $h:D_g \to \R$ dada por $h(x) = f(g(x))$ é derivável e
        \[
            h'(x) = f'(g(x)) \cdot g'(x).
        \]
    Sendo $y = f(u)$ e $u = g(x)$, a regra da cadeia nos diz que
        \[
            \dfrac{dy}{dx} = \dfrac{dy}{du} \cdot \dfrac{du}{dx},
        \]
    em que $\dfrac{dy}{du}$ deve ser calculada em $u = g(x)$.
\end{teo}

\begin{proof}
\end{proof}

\begin{teo}
    (Derivada de Função Inversa) Seja $f$ uma função inversível e $g$ a função inversa de $f$. Se $f$ for derivável em $q = g(p)$, com $f'(q) \neq 0$, e se $g$ for contínua em $p$, então $g$ será derivável em $p$ e
        \[
            g'(p) = \dfrac{1}{f'(g(p))}.
        \]
\end{teo}

\begin{proof}
\end{proof}

\section{Teoremas de Rolle, do Valor Médio e de Cauchy}

\begin{defi}
    Sejam $f:D_f \subseteq \R \to \R$ uma função e $p \in D_f$ um ponto no domínio de $f$.
    
    \textbf{(a)} Diremos que $f(p)$ é o \textit{valor máximo global} de $f$, ou que $p$ é um \textit{ponto de máximo global} de $f$, se $f(x) \leq f(p)$ para todo $x \in D_f$. Diremos que $f(p)$ é o \textit{valor mínimo global} de $f$, ou que $p$ é um \textit{ponto de mínimo global} de $f$, se $f(x) \geq f(p)$ para todo $x \in D_f$.

    \textbf{(b)} Suponha ainda que $p \in A \subseteq D_f$. Diremos que $f(p)$ é o \textit{valor máximo} de $f$ em $A$, ou que $p$ é um \textit{ponto de máximo} de $f$ em $A$, se $f(x) \leq f(p)$ para todo $x \in A$. Diremos que $f(p)$ é o \textit{valor mínimo} de $f$ em $A$, ou que $p$ é um \textit{ponto de mínimo} de $f$ em $A$, se $f(x) \geq f(p)$ para todo $x \in A$.

    \textbf{(c)} Diremos que $f(p)$ é o \textit{valor máximo local} de $f$, ou que $p$ é um \textit{ponto de máximo local} de $f$, se existir $r>0$ tal que $f(x) \leq f(p)$ para todo $x \in \left]p-r, p+r\right[ \cap D_f$. Diremos que $f(p)$ é o \textit{valor mínimo local} de $f$, ou que $p$ é um \textit{ponto de mínimo local} de $f$, se existir $r>0$ tal que $f(x) \geq f(p)$ para todo $x \in \left]p-r, p+r\right[ \cap D_f$.
\end{defi}

\begin{defi}
    Dada uma função $f$, diremos que o ponto $p$ é \textit{interior} a $D_f$ se existir um intervalo aberto $I \subset D_f$ tal que $p \in I$.
\end{defi}

\begin{teo} \label{teo:maxlocal}
    Seja $f:D_f \subseteq \R \to \R$ uma função derivável no ponto interior $p \in D_f$. Se $p$ é ponto de máximo (mínimo) local de $f$, então $f'(p)=0$.
\end{teo}

\begin{proof}
    Como $p$ é ponto de máximo local de $f$, existe $r_1 >0$ tal que $f(x) \leq f(p)$ para todo $x \in \left]p-r_1, p+r_1\right[ \cap D_f$. Como $p$ é um interior a $D_f$, existe $r_2 > 0$ tal que $\left] p-r_2, p+r_2 \right[ \subseteq D_f$. Sendo $r := \min{\{ r_1, r_2\}}$, temos $f(x) \leq f(p)$ para todo $x \in \left]p-r, p+r\right[$. Como $f$ é derivável em $p$, temos que
        $ \ds
            f'(p) = \lim_{x \to p^+} \dfrac{f(x) - f(p)}{x-p} = \lim_{x \to p^-} \dfrac{f(x) - f(p)}{x-p}.
        $
    Sendo $p < x < p+r$, temos $\ds \dfrac{f(x) - f(p)}{x-p} \leq 0$; daí, pela conservação do sinal, $\ds \lim_{x \to p^+} \dfrac{f(x) - f(p)}{x-p} \leq 0$.  Analogamente, sendo $p-r<x<p$, temos $\ds \dfrac{f(x) - f(p)}{x-p} \geq 0$; daí, pela conservação do sinal, $\ds \lim_{x \to p^+} \dfrac{f(x) - f(p)}{x-p} \geq 0$. Com isso, temos $f'(p) \leq 0$ e $f'(p) \geq 0$, donde, $f'(p) = 0$. O caso em que $p$ é ponto de mínimo local de $f$ segue de forma completamente análoga.
\end{proof}

$\subseteq D_f$ e derivável em $]a,b[$.

\begin{teo} \label{teo:rolletvm}
    Seja $f: [a,b] \to \R$ contínua em $[a,b]$ e derivável em $]a,b[$.
        \begin{enumerate}[leftmargin=*, align=left, label=\textbf{(\alph*)}]
            \item (Rolle) Se $f(a) = f(b)$, então existe $c \in \left]a,b\right[$ tal que $f'(c) = 0$.
            \item (Valor Médio) Existe $c \in \left]a,b\right[$ tal que
                \[
                    f(b) - f(a) = f'(c) \cdot (b-a).
                \]
        \end{enumerate}
\end{teo}

\begin{proof}
    \textbf{(a)} Se $f$ for constante em $[a,b]$, então $f'(x) = 0$ para todo $x \in \left]a,b \right[$. Suponha, então, que $f$ não seja constante em $[a,b]$. Sendo $f$ contínua em $[a,b]$, pelo Teorema de Weierstrass \eqref{teo.calc:weierstrass}, existem $x_1, x_2 \in [a,b]$ tais que $f(x_1) \leq f(x) \leq f(x_2)$ para todo $x \in [a,b]$. Se fosse $f(x_1) = f(x_2)$, então $f$ seria constante; logo, $f(x_1) \neq f(x_2)$. E como, por hipótese, $f(a) = f(b)$, temos que $x_1$ ou $x_2$ estão em $]a,b[$. O $x_i \in \left] a,b \right[$ é um ponto de máximo local de $f$; daí, pelo Teorema \eqref{teo:maxlocal}, $f'(x_i)=0$. \itemproof

    \textbf{(b)} Defina $S: [a,b] \to \R$ por
        \[
            S(x) := f(a) + \dfrac{f(b) - f(a)}{b-a} (x-a).
        \]
    Observe que o gráfico de $S$ é a reta que passa pelos pontos $(a, f(a))$ e $(b, f(b))$. Agora, defina $g : [a,b] \to \R$ por $g(x) := f(x) - S(x)$. Note que $g$ é contínua em $[a,b]$ e derivável em $]a,b[$; daí, pelo Teorema de Rolle, existe $c \in \left] a,b \right[$ tal que $g'(c) = 0$. Como
        \[
            g'(x) = f'(x) - \dfrac{f(b) - f(a)}{b-a},
        \]
    temos que
        \[ \ds
            g'(c) = f'(c) - \dfrac{f(b) - f(a)}{b-a} = 0,
        \]
    donde $f(b) - f(a) = f'(c) (b-a)$.
\end{proof}

\begin{teo}
    (Cauchy) Se $f : D_f \subseteq \R \to \R$ e $g : D_g \subseteq \R \to \R$ são funções contínuas em $\left[ a,b \right] \subseteq D_f \cap D_g$ e deriváveis em $]a,b[$, então existe pelo menos um ponto $c \in \left] a,b \right[$ tal que
        \[
            [f(b) - f(a)] g'(c) = [g(b) - g(a)] f'(c).
        \]
    Em particular, se $g'(x) \neq 0$ para todo $x \in \left] a,b \right[$, então
        \[
            \dfrac{f(b) - f(a)}{g(b) - g(a)} = \dfrac{f'(c)}{g'(c)}.
        \]
\end{teo}

\begin{proof}
    Defina $h:[a,b] \to \R$ por
        \[
            h(x) := [f(b) - f(a)] g(x) -[g(b) - g(a)]f(x).
        \]
    É fácil ver que $h$ é contínua em $[a,b]$, derivável em $]a,b[$ e $h(a) = h(b)$. Daí, pelo Teorema de Rolle $\eqref{teo:rolletvm}$, existe $c \in \left] a,b \right[$ tal que
        \[
            [f(b) - f(a)] g(c) -[g(b) - g(a)]f(c) = 0,
        \]
    donde
        \[
            [f(b) - f(a)] g'(c) = [g(b) - g(a)] f'(c).
        \]
    Em particular, se $g'(x) \neq 0$ para todo $x \in \left] a,b \right[$, então
        \[
            \dfrac{f(b) - f(a)}{g(b) - g(a)} = \dfrac{f'(c)}{g'(c)},
        \]
    pois o Teorema do Valor Médio \eqref{teo:rolletvm} aplicado à função $g$ nos diz que existe $\tilde{c} \in \left] a,b \right[$ tal que $ g(b) - g(a) = g'(\tilde{c})(b-a)$; como $g'(\tilde{c}) \neq 0$ e $b \neq a$, temos $g(b) - g(a) \neq 0$.
    \end{proof}

\section{Gráficos de Funções}

\begin{teo}
    Seja $f: D_f \subseteq \R \to \R$ uma função derivável no intervalo aberto $I \subset D_f$.

    \textbf{(a)} Se $f'(x)>0$ para todo $x \in I$ interior, então $f$ será estritamente crescente em $I$.

    \textbf{(b)} Se $f'(x)<0$ para todo $x \in I$ interior, então $f$ será estritamente decrescente em $I$.
\end{teo}

\begin{proof}
    \textbf{(a)} Provemos que para quaisquer $a,b \in I$ temos $a<b \Rightarrow f(a) < f(b)$. Sejam, então, $a,b \in I$ com $a<b$. Evidentemente, temos que $f$ e contínua em $[a,b]$ e derivável em $]a,b[$; daí, pelo Teorema do Valor Médio \eqref{teo:rolletvm}, existe $c \in \left] a,b \right[$ tal que 
        $
            f(b) - f(a) = f'(c) (b-a).
        $
    Como $f'(c) > 0$ e $b>a$, temos que $f(b) - f(a) > 0$, donde $f(a) < f(b)$. \itemproof
    
    \textbf{(b)} Segue analogamente.
\end{proof}

\begin{cor}
    Seja $f : D_f \subseteq \R \to \R$ uma função derivável até a 2ª ordem em $\left] a,b \right[ \subseteq D_f$. Se $f''(x) > 0$ para todo $x \in \left] a,b \right[$ e se existe $c \in \left] a,b \right[$ tal que $f'(c) = 0$, então $f$ é estritamente decrescente em $]a,c[$ e estritamente crescente em $]c,b[$.
\end{cor}

\begin{proof}
    Se $f''(x) > 0$ para todo $x \in \left] a,b \right[$, então $f'$ é estritamente crescente em $]a,b[$. Com isso, $f'(x) < f'(c) = 0$ para todo $x \in \left] a,c \right[$ e $f'(x) > f'(c) = 0$ para todo $x \in \left] c,b \right[$. Com isso, $f$ é estritamente decrescente em $]a,c[$ e estritamente crescente em $]c,b[$.
\end{proof}

\begin{defi}
    Seja $f : D_f \subseteq \R \to \R$ uma função derivável no intervalo aberto $I \subseteq D_f$.

    \textbf{(a)} Se 
        $
            f(x) > f(p) + f'(p) (x-p)
        $
    para todos $x,p \in I$, com $x \neq p$, diremos que $f$ tem a \textit{concavidade para cima} em $I$, ou que $f$ é \textit{convexa} em $I$.

    \textbf{(b)} Se
        $
            f(x) < f(p) + f'(p) (x-p)
        $
    para todos $x,p \in I$, com $x \neq p$, diremos que $f$ tem a \textit{concavidade para baixo} em $I$, ou que $f$ é \textit{côncava} em $I$.
\end{defi}

\begin{teo}
    Seja $f : D_f \subseteq \R \to \R$ uma função derivável até a 2ª ordem no intervalo aberto $I \subseteq D_f$.

    \textbf{(a)} Se $f''(x) > 0$ em $I$, então $f$ terá a concavidade para cima em $I$.

    \textbf{(b)} Se $f''(x)<0$ em $I$, então $f$ terá a concavidade para baixo em $I$.
\end{teo}

\begin{proof}
    \textbf{(a)} Sendo $p \in I$, provemos que para todo $x \in I$, com $x \neq p$, temos
        $
            f(x) > f(p) + f'(p)(x-p).
        $
    Definindo $g: I \to \R$ por $g(x) := f(x) - f(p) - f'(p)(x-p)$, basta provar que $g(x)>0$ para todo $x \in I$, com $x \neq p$. É fácil ver que $g'(x) = f'(x) - f'(p)$. Como $f''(x)>0$ em $I$, temos que $f'$ é estritamente crescente em $I$. Com isso, $g'(x) > 0$ para $x>p$ e $g'(x) < 0$ para $x < p$. Com isso, $g$ é estritamente decrescente em $\{ x \in I : x<p \}$ e estritamente crescente em $\{x \in I : x>p \}$. Com isso, sendo $g(p) = 0$, temos $g(x) > 0$ para todo $x \in I$, com $x \neq p$. \itemproof
    
    \textbf{(b)} Segue analogamente.
\end{proof}

\begin{defi}
    Seja $f : D_f \subseteq \R \to \R$ uma função contínua em $p \in D_f$. Diremos que $p$ é um \textit{ponto de inflexão} de $f$ se existirem $a,b \in \R$, com $p \in \left] a,b \right[ \subseteq D_f$, para os quais a concavidade de $f$ em $]a,p[$ é diferente da concavidade de $f$ em $]p,b[$.
\end{defi}


\begin{prop}
    \textbf{(a)} Seja $f: D_f \subseteq \R \to \R$ uma função derivável até a 3ª ordem no intervalo aberto $\left] a,b \right[ \subseteq D_f$. Se $f'''$ é contínua em $p \in \left] a,b \right[$, $f'''(p) \neq 0$ e $f''(p) = 0$, então $p$ é um ponto de inflexão de $f$.

    \textbf{(b)} Seja $f: D_f \subseteq \R \to \R$ uma função derivável até a 2ª ordem no intervalo aberto $I \subseteq D_f$. Se $f''$ é contínua em $p \in I$ e $p$ é um ponto de inflexão de $f$, então $f''(p)=0$.
\end{prop}

\begin{proof}
    \textbf{(a)} Suponha, sem perda de generalidade, que $f'''(p) > 0$. Como $f'''$ é contínua em $p$, pela conservação do sinal \eqref{consin2} existe $r_1 > 0$ tal que $f'''(x) > 0$ para todo $x \in \left] p-r_1, p+r_1 \right[$. Por outro lado, existe $r_2 > 0$ tal que $\left] p-r_2, p+r_2 \right[ \subseteq \left] a,b \right[$ (de fato, basta tomar $r_2 = \min{\{ b-p, p-a \}}$). Sendo, então, $r := \min{\{r_1,r_2\}}$, temos que $f'''(x) > 0$ para todo $x \in \left] p-r,p+r \right[ \subseteq \left] a,b \right[$. Com isso, $f''$ é estritamente crescente em $]p-r,p+r[$, e como $f''(p) = 0$, só pode ser $f''(x) < 0$ para todo $x \in \left] p-r,p+r \right[$ e $f''(x)>0$ para todo $x \in ]p,p+r[$. Logo, $p$ é um ponto de inflexão de $f$.

    \textbf{(b)} Se fosse $f''(p) \neq 0$, como $f''$ é contínua em $p$, pela conservação do sinal existiria $r>0$ tal que $f''(p)$ e $f''(x)$ teriam o mesmo sinal em $]p-r,p+r[$, donde $p$ não seria ponto de inflexão de $f$, absurdo. Logo, $f''(p) = 0$.
\end{proof}

\section{Regras de L'Hospital}

\begin{teo}
    (Regra de L'Hospital para indeterminações do tipo $0 / 0$)
    
    \textbf{(a)} Sejam $f:D_f \subseteq \R \to \R$ e $g: D_g \subseteq \R \to \R$ funções para as quais existe $r \in \R_{>0}$ tal que $f$ e $g$ são deriváveis e $g'(x) \neq 0$ em 
        \begin{itemize}
            \item $I := \left] p,p+r \right[ \subseteq D_f \cap D_g$ (caso $x \to p^+$); ou
            \item $I := \left]p-r, p \right[ \subseteq D_f \cap D_g$ (caso $x \to p^-$); ou
            \item $I := \left]p-r, p+r \right[ \setminus \{p\} \subseteq D_f \cap D_g$ (caso $x \to p$).
        \end{itemize}
    Se $\ds \lim{f(x)} = \lim{g(x)} = 0$ e $\ds \lim{\dfrac{f'(x)}{g'(x)}} \in \R \cup \{ \pm \infty \}$, então
        \[ \ds
            \lim{\dfrac{f(x)}{g(x)}} = \lim{\dfrac{f'(x)}{g'(x)}}. 
        \]
    \textbf{(b)} Sejam $f:D_f \subseteq \R \to \R$ e $g: D_g \subseteq \R \to \R$ funções para as quais existe $r \in \R$ tal que $f$ e $g$ são deriváveis e $g'(x) \neq 0$ em 
        \begin{itemize}
            \item $I := \left] r, + \infty \right[ \subseteq D_f \cap D_g$ (caso $x \to + \infty$); ou
            \item $I := \left]- \infty, r \right[ \subseteq D_f \cap D_g$ (caso $x \to - \infty$).
        \end{itemize}
    Se $\ds \lim{f(x)} = \lim{g(x)} = 0$ e $\ds \lim{\dfrac{f'(x)}{g'(x)}} \in \R \cup \{ \pm \infty \}$, então
        \[ \ds
            \lim{\dfrac{f(x)}{g(x)}} = \lim{\dfrac{f'(x)}{g'(x)}}. 
        \]
\end{teo}

\begin{proof}
    \textbf{(a)} Suponha $\ds \lim{\dfrac{f'(x)}{g'(x)}} = L \in \R$. Façamos o caso $x \to p^+$, isto é, 
        \[  \ds
            \lim_{x \to p^+}{f(x)} = \lim_{x \to p^+}{g(x)} = 0 \quad \text{e} \quad \lim_{x \to p^+}{\dfrac{f'(x)}{g'(x)}} = L.
        \]
    Defina $F,G : I \to \R$ por
        \[
            F(x) :=
                \begin{cases}
                    f(x) & x \in I \\
                    0 & x = p
                \end{cases}
            \qquad
            \text{e}
            \qquad 
            G(x) :=
                \begin{cases}
                    g(x) & x \in I \\
                    0 & x = p
                \end{cases}
        \]
    Afirmamos que $G'(x) \neq 0$ para todo $x \in I$ e $G(p) = 0$ resultam em $G(x) \neq 0$ para todo $x \in I$. De fato, se não fosse $G(x) \neq 0$ para todo $x \in I$, então existiria $a \in I$ com $G(a) = 0$; pelo Teorema do Valor Médio, existiria $b \in \left]p,a \right[$ tal que
        \[  
            \underbrace{G(a) - G(p)}_{=0} = G'(b) \underbrace{(a - p)}_{\neq 0},
        \]
    donde $G'(b) = 0$, contrariando a hipótese de ser $G'(x) \neq 0$ para todo $x \in I$. Agora, sendo $\ds \lim_{x \to p^+}{\dfrac{f'(x)}{g'(x)}} = L$, para todo $\epsilon \in \R_{>0}$ existe $\delta \in \R_{>0}$, com $\delta < r$, tal que
        \[ \ds
            p < x < p + \delta \Rightarrow \left| \dfrac{f'(x)}{g'(x)} - L \right| < \epsilon,
        \]
    isto é, $\ds \left| \dfrac{F'(x)}{G'(x)} - L \right| < \epsilon$. Por outro lado, o Teorema de Cauchy aplicado às funções $F$ e $G$ no intervalo $[p,x]$ nos diz que existe $q \in \left] p,x \right[$ tal que
        \[
            \dfrac{F(x) - F(p)}{G(x) - G(p)} = \dfrac{F'(q)}{G'(q)};
        \]
    daí,
        \[ \ds
            \left| \dfrac{F(x)}{G(x)} - L \right| =  \left| \dfrac{F(x) - F(p)}{G(x) - G(p)} - L \right| =   \left| \dfrac{F'(q)}{G'(q)} - L \right| < \epsilon,
        \]
    pois $p < q < x < p + \delta$. Com isso, $\ds \lim_{x \to p^+}{\dfrac{f(x)}{g(x)}} = L$, como queríamos provar.

    \textbf{(b)}
\end{proof}

\begin{teo}
    (Regra de L'Hospital para indeterminações do tipo $\infty /  \infty $)

    \textbf{(a)} Sejam $f:D_f \subseteq \R \to \R$ e $g: D_g \subseteq \R \to \R$ funções para as quais existe $r \in \R_{>0}$ tal que $f$ e $g$ são deriváveis e $g'(x) \neq 0$ em 
        \begin{itemize}
            \item $I := \left] p,p+r \right[ \subseteq D_f \cap D_g$ (caso $x \to p^+$); ou
            \item $I := \left]p-r, p \right[ \subseteq D_f \cap D_g$ (caso $x \to p^-$); ou
            \item $I := \left]p-r, p+r \right[ \setminus \{p\} \subseteq D_f \cap D_g$ (caso $x \to p$).
        \end{itemize}
    Se $\ds \lim{f(x)} = \lim{g(x)} = \pm \infty$ e $\ds \lim{\dfrac{f'(x)}{g'(x)}} \in \R \cup \{ \pm \infty \}$, então
        \[ \ds
            \lim{\dfrac{f(x)}{g(x)}} = \lim{\dfrac{f'(x)}{g'(x)}}. 
        \]
    \textbf{(b)} Sejam $f:D_f \subseteq \R \to \R$ e $g: D_g \subseteq \R \to \R$ funções para as quais existe $r \in \R$ tal que $f$ e $g$ são deriváveis e $g'(x) \neq 0$ em 
        \begin{itemize}
            \item $I := \left] r, + \infty \right[ \subseteq D_f \cap D_g$ (caso $x \to + \infty$); ou
            \item $I := \left]- \infty, r \right[ \subseteq D_f \cap D_g$ (caso $x \to - \infty$).
        \end{itemize}
    Se $\ds \lim{f(x)} = \lim{g(x)} = \pm \infty$ e $\ds \lim{\dfrac{f'(x)}{g'(x)}} \in \R \cup \{ \pm \infty \}$, então
        \[ \ds
            \lim{\dfrac{f(x)}{g(x)}} = \lim{\dfrac{f'(x)}{g'(x)}}. 
        \]
\end{teo}

\begin{proof}
\end{proof}

\section{Trigonometria, parte II}

\begin{teo} \label{teo:pisobre2}
    Existe um menor real $a>0$ tal que $\cos{a} = 0$ e $\sin{a} = 1$.
\end{teo}

\begin{proof}
\end{proof}

\begin{defi}
    Definimos $\pi := 2a$, em que $a$ é o menor real a que se refere o Teorema \eqref{teo:pisobre2}. Assim, $\ds \cos{\dfrac{\pi}{2}} = 0$ e $\ds \sin{\dfrac{\pi}{2}}=1$.
\end{defi}

\begin{teo}
    As funções $\sin$ e $\cos$ são periódicas com período $2 \pi$, isto é, $\sin{(x + 2\pi)} = \sin{x}$ e $\cos{(x+2\pi)} = \cos{x}$ para todo $x \in \R$.
\end{teo}

\section{Polinômio de Taylor}

\begin{defi} \label{defi:taylor}
    Seja $I \subset \R$ um intervalo aberto, $f: I \to \R$ uma função $n$ vezes diferenciável e $x_0 \in I$ um ponto de $I$. O polinômio
        \[
            P_n(x) := \sum_{k=0}^{n}{\dfrac{f^{(k)}(x_0)}{k!}(x-x_0)^{k}} 
        \]
    chama-se \textit{polinômio de Taylor} de \textit{ordem} $n$ de $f$ \textit{centrado}, ou \textit{em volta}, de $x_0$.
\end{defi}

\begin{prop}
    Nas condições da definição \eqref{defi:taylor}, $P^{(k)}_n (x_0) = f^{(k)}(x_0)$ para todo $k \leq n$.
\end{prop}

\begin{proof}
    
\end{proof}

\chapter{Integrais}

\section{A integral de Darboux}

\begin{defi}
    \leavevmode
        \begin{enumerate}[leftmargin=*, align=left, label=\textbf{(\alph*)}]
            \item Uma \textit{partição} de um intervalo $[a,b]$ é um conjunto finito $P := \{x_0, \ldots, x_n \}$ tal que $a = x_0 < x_1 < \cdots < x_n = b$. Isso é denotado por
                \[
                    P : a = x_0 < \cdots < x_n = b.
                \]
            O conjunto de todas as partições de $[a,b]$ é denotado por $\mathcal{P}{[a,b]}$.
            \item A \textit{amplitude} do $i$-ésimo intervalo $[x_{i-1}, x_i]$, para cada $i \in [n]$, é definida como $\Delta x_i := x_i - x_{i-1}$.
            \item Um \textit{refinamento} de uma partição $P \in \mathcal{P}{[a,b]}$ é uma partição $Q \in \mathcal{P}{[a,b]}$ tal que $P \subseteq Q$.
        \end{enumerate}
\end{defi}

\begin{defi} \label{defi:somassupinf}
    Sejam $f:[a,b] \to \R$ uma função limitada e $P : a = x_0 < \cdots < x_n = b$ uma partição de $[a,b]$.
        \begin{enumerate}[leftmargin=*, align=left, label=\textbf{(\alph*)}]
            \item A \textit{soma superior} de $f$ com relação à $P$ é definida como
                \[ \ds
                    U(f,P) := \sum_{i=1}^{n}{M_i \Delta{x_i}},
                \]
            onde
                $ \ds
                    M_i := \sup_{[x_{i-1}, x_i]}{f}
                $
            para cada $i \in [n]$.
            \item A \textit{soma inferior} de $f$ com relação à $P$ é definida como
                \[ \ds
                    L(f,P) := \sum_{i=1}^{n}{m_i \Delta{x_i}},
                \]
            onde
                $ \ds
                    m_i := \inf_{[x_{i-1}, x_i]}{f}
                $
            para cada $i \in [n]$.
        \end{enumerate}
\end{defi}

\begin{prop} \label{prop:somassupinf}
    Sejam $f:[a,b] \to \R$ uma função limitada e $P : a = x_0 < \cdots < x_n = b$ uma partição de $[a,b]$.
        \begin{enumerate}[leftmargin=*, align=left, label=\textbf{(\alph*)}]
            \item Tem-se $L(f,P) \leq U(f,P)$.
            \item Se $Q$ é um refinamento de $P$, então
                $
                    U(f,Q) \leq U(f,P)
                $
            e
                $
                    L(f,Q) \geq L(f,P)
                $.
            \item Se $Q$ é uma qualquer outra partição de $[a,b]$, então $L(f,P) \leq U(f,Q)$.
        \end{enumerate}
\end{prop}

\begin{proof}
    \leavevmode
        \begin{enumerate}[leftmargin=*, align=left, label=\textbf{(\alph*)}]
            \item Para todo $i \in [n]$ temos $m_i \leq M_i$ e $\Delta x_i > 0$. Logo $m_i \Delta x_i \leq M_i \Delta x_i$ para todo $i \in [n]$. Tomando a soma, temos que
                \[
                    \sum_{i=1}^{n} m_i \Delta x_i \leq \sum_{i=1}^{n} M_i \Delta x_i,
                \]
            de modo que $L(f,P) \leq U(f,P)$, como havíamos afirmado. \itemproof
            
            \item Façamos indução em $|Q \setminus P|$. Se $|Q \setminus P| = 1$, então $Q$ só tem um ponto a mais que $P$, isto é, existe $\bar{x} \in Q$ tal que $\bar{x} \notin P$. Em particular, existe um único índice $j \in [n]$ tal que $x_{j-1} < \bar{x} < x_j$. Com isso, tomando
                \[
                    M'_j := \sup_{[x_{j-1}, \bar{x}]}{f} \quad \text{e} \quad M''_j := \sup_{[\bar{x}, x_j]}{f},
                \]
            temos $M'_j, M''_j \leq M_j$, donde
                \begin{align*}
                    U(f,Q) &= \sum_{\substack{i = 1 \\ i \neq j  }}^{n} M_i \Delta x_i + M'_j(\bar{x} - x_{j-1}) + M''_j(x_j - \bar{x}) \\
                    &\leq \sum_{\substack{i = 1 \\ i \neq j  }}^{n} M_i \Delta x_i + M_j(\bar{x} - x_{j-1}) + M_j(x_j - \bar{x}) \\
                    &= \sum_{\substack{i = 1 \\ i \neq j  }}^{n} M_i \Delta x_i + M_j(x_j - x_{j-1}) \\
                    &= \sum_{i = 1}^{n} M_i \Delta x_i = U(f,P).
                \end{align*}
            Isso completa a base da indução. Agora, suponha que se $|Q \setminus P| = k > 1$, então $U(f,Q) \leq U(f,P)$. Tomando $Q'$ com só um ponto a mais que $Q$, temos que $U(f,Q') \leq U(f,Q)$, de modo que se $|Q' \setminus P| = k+1$ então $U(f,Q') \leq U(f,P)$. Isso completa o passo indutivo e, portanto, completa a prova. A prova de que $L(f,P') \geq L(f,P)$ segue de modo completamente análogo. \itemproof
    
            \item Para quaisquer partições $P$ e $Q$ de $[a,b]$, sempre existe um refinamento comum a ambas. De fato, basta tomar a união $P \cup Q$ e reindexar os índices conforme a definição. Assim, pelos itens anteriores,
                \[
                    L(f,P) \leq L(f, P \cup Q) \leq U( f, P \cup Q) \leq U(f,Q),
                \]
            como havíamos afirmado. \itemproof
        \end{enumerate}
\end{proof}

\begin{defi}
    Seja $f:[a,b] \to \R$ uma função limitada.
        \begin{enumerate}[leftmargin=*, align=left, label=\textbf{(\alph*)}]
            \item A \textit{integral inferior} de $f$ é definida como
                \[
                    \lowerint_{a}^{b} f(x) \, dx := \sup_{P \in \mathcal{P}{[a,b]}} L(f,P).
                \]
            \item A \textit{integral superior} de $f$ é definida como
                \[
                    \upperint_{a}^{b} f(x) \, dx := \inf_{P \in \mathcal{P}{[a,b]}} U(f,P).
                \]
        \end{enumerate}
\end{defi}

\begin{obs}
    A proposição \eqref{prop:somassupinf} garante que $\left\{ L(f,P) \in \R : P \in \mathcal{P}{[a,b]} \right\}$ é limitado superiormente; logo, pela propriedade do supremo, existe $\ds \sup_{P \in \mathcal{P}{[a,b]}} L(f,P)$. Analogamente, existe $\ds \inf_{P \in \mathcal{P}{[a,b]}} U(f,P)$. Isso garante que as definições de integral superior e inferior são consistentes (estão bem definidas).
\end{obs}

\begin{prop}
    Se uma função $f: [a,b] \to \R$ é limitada, então
        \[
             \lowerint_{a}^{b} f(x) \, dx \leq \upperint_{a}^{b} f(x) \, dx.
        \]
\end{prop}

\begin{proof}
    Vimos que $L(f, P) \leq U(f, Q)$ para quaisquer partições $P, Q \in \mathcal{P}{[a,b]}$. Fixando $Q$, temos que $U(f,Q)$ é uma cota superior de $L(f,P)$, para qualquer $P \in \mathcal{P}{[a,b]}$, de modo que
        \[ \ds
            \lowerint_{a}^{b} f(x) \, dx \leq U(f,Q).
        \]
    Com isso, $\ds \lowerint_{a}^{b} f(x) \, dx$ é uma cota inferior de $U(f,Q)$, para qualquer $Q \in \mathcal{P}{[a,b]}$, de modo que
        \[ \ds
            \lowerint_{a}^{b} f(x) \, dx \leq \upperint_{a}^{b} f(x) \, dx,
        \]
    como havíamos afirmado. \itemproof
\end{proof}

\begin{defi}
    Seja $f : [a,b] \to \R$ uma função limitada.
        \begin{enumerate}[leftmargin=*, align=left, label=\textbf{(\alph*)}]
            \item $f$ é \textit{integrável em $[a,b]$ segundo Darboux} se
                \[
                    \lowerint_{a}^{b} f(x) \, dx = \upperint_{a}^{b} f(x) \, dx.
                \]
            \item Seja $f$ \textit{Darboux-integrável} em $[a,b]$. A \textit{integral de Darboux de $f$ em $[a,b]$} é definida como
                \[
                    \int_{a}^{b} f(x) \, dx := \lowerint_{a}^{b} f(x) \, dx = \upperint_{a}^{b} f(x) \, dx.
                \]
            \item Se $f$ está definida em $c \in \R$, estendemos a definição dizendo que $f$ é integrável em $c$ colocando
                \[
                    \int_{c}^{c} f(x) \, dx = 0.
                \]
                
            \item Se $f$ é integrável, estendemos a definição colocando
                \[
                    \int_{b}^{a} f(x) \, dx := - \int_{a}^{b} f(x) \, dx.
                \]
        \end{enumerate}
\end{defi}

\begin{obs}
    Decorre das definições que, sendo $f: [a,b] \to \R$ integrável, vale
        \[
            L(f,P) \leq \lowerint_{a}^{b} f(x) \, dx = \int_{a}^{b} f(x) \, dx = \upperint_{a}^{b} f(x) \, dx \leq U(f,P),
        \]
    para qualquer $P \in \mathcal{P}[a,b]$.
\end{obs}

\begin{comment}
\begin{ex}
    \textbf{(a)} Se $f: [a,b] \to \R$ é dada por $f(x) = c \in \R$, então $f$ é integrável e
        $ \ds
            \int_{a}^{b} c \, dx = c(b-a).
        $
        
    \textbf{(b)} Se $f: [0,1] \to \R$ é dada por $f(x) = x$, então $f$ é integrável e
        $ \ds
            \int_{0}^{1} x \, dx = \frac{1}{2}.
        $
\end{ex}
\end{comment}

\begin{teo}[Critério de integrabilidade]
    Uma função limitada $f:[a,b] \to \R$ é integrável em $[a,b]$ se, e somente se, para todo $\epsilon \in \R_{>0}$ existe $P \in \mathcal{P}[a,b]$ tal que $U(f,P) - L(f,P) < \epsilon$.
\end{teo}

\begin{proof}
    \textbf{(a)} ($\Rightarrow$)\footnote{Esta prova depende de um resultado sobre supremos e ínfimos.} Sendo $f$ integrável, para todo $\epsilon \in \R_{>0}$ existem partições $P_1, P_2 \in \mathcal{P}$ tais que
        \[
            S(P_2,f) - \int_{a}^{b} f(x) \, dx < \dfrac{\epsilon}{2} \quad \text{e} \quad \int_{a}^{b} f(x) \, dx - s(P_1,f) < \dfrac{\epsilon}{2}.
        \]
    Com isso, sendo $P$ uma partição comum à $P_1$ e $P_2$, temos que
        \[
            S^+(f,P) \leq S(P_2,f) < \int_{a}^{b} f(x) \, dx + \dfrac{\epsilon}{2} < s(P_1,f) + \epsilon \leq S_-(f,P) + \epsilon,
        \]
    de modo que $S^+(f,P) - S_-(f,P) < \epsilon$.

    ($\Leftarrow$) Agora, suponha que para todo $\epsilon \in \R_{>0}$ existe uma partição $P$ de $[a,b]$ tal que $S^+(f,P) - S_-(f,P) < \epsilon$. Como
        \[
            S_-(f,P) \leq \lowerint_{a}^{b} f(x) \, dx \leq \upperint_{a}^{b} f(x) \, dx \leq S^+(f,P),
        \]
    temos que
        $ \ds
            0 \leq \upperint_{a}^{b} f(x) \, dx - \lowerint_{a}^{b} f(x) \, dx \leq S^+(f,P) - S_-(f,P) < \epsilon,
        $
    de modo que
        $ \ds
            \upperint_{a}^{b} f(x) \, dx - \lowerint_{a}^{b} f(x) \, dx < \epsilon
        $
    para todo $\epsilon \in \R_{>0}$, donde
        $ \ds
            \upperint_{a}^{b} f(x) \, dx = \lowerint_{a}^{b} f(x) \, dx.
        $
    Assim, $f$ é integrável.
\end{proof}

\subsection{Estendendo a definição}

\begin{prop}
    \textbf{(a)} Se $f : [a,b] \to \R$ é uma função tal que $f(x) = 0$ para todo $x \in [a,b] \setminus \{c\}$, onde $c \in [a,b]$ e $f(c) \neq 0$, então $f$ é integrável em $[a,b]$ e
        \[
            \int_{a}^{b} f(x) \, dx = 0.
        \]
% é nula, exceto em um ponto $c \in [a,b]$, então $f$ é integrável em $[a,b]$ e
    \textbf{(b)} Se $f : [a,b] \to \R$ é uma função tal que $f(x) = 0$ para todo $x \in [a,b] \setminus \{c_1, c_2, \ldots, c_n\}$, onde $c_i \in [a,b]$ e $f(c_i) \neq 0$ para todo $i \in [n]$, então $f$ é integrável em $[a,b]$ e
        \[
            \int_{a}^{b} f(x) \, dx = 0.
        \]
%Se a função $f : [a,b] \to \R$ é nula, exceto em um número finito de pontos $c_1, c_2, \ldots, c_n \in [a,b]$, então $f$ é integrável em $[a,b]$ e
    \textbf{(c)} Sejam $f,g : [a,b] \to \R$ funções tais que $f(x) = g(x)$ para todo $x \in [a,b] \setminus \{c_1, c_2, \ldots, c_n \}$, onde $c_i \in [a,b]$ e $f(c_i) \neq g(c_i)$ para todo $i \in [n]$. Se $f$ é integrável, então $g$ é integrável e
        \[
            \int_{a}^{b} g(x) \, dx  = \int_{a}^{b} f(x) \, dx.
        \]
\end{prop}

\begin{proof}
    Táboas, observação 4.1.16, página 171.
\end{proof}

\begin{defi}
    Seja $f:[a,b] \setminus \{c_1, c_2, \ldots, c_n \} \to \R$ uma função, $c_i \in [a,b]$ para todo $i \in [n]$. Diremos que $f$ é integrável em $[a,b]$ se qualquer extensão $g$ de $f$ a $[a,b]$ o for, pondo
        \[
            \int_{a}^{b} f(x) \, dx  := \int_{a}^{b} g(x) \, dx.
        \]
\end{defi}

\section{Resultados}

\begin{teo}
    \textbf{(a)} Toda função $f: [a,b] \to \R$ contínua é integrável.

    \textbf{(b)} Se a função $f:[a,b] \to \R$ é limitada e tem apenas um número finito de pontos de descontinuidade, então $f$ é integrável.
\end{teo}

\begin{proof}
    \textbf{(a)} Pelo teorema da limitação \eqref{teo.calc:limitacao}, $f$ é limitada. Pelo teorema \eqref{teo.calc:contunifocont}, $f$ é uniformemente contínua, de modo que para todo $\epsilon \in \R_{>0}$ existe $\delta \in \R_{>0}$ tal que
        \[
            |x-y| < \delta \Rightarrow |f(x) - f(y)| < \dfrac{\epsilon}{b-a}.
        \]
    Escolha uma partição $P : a = x_0 < \cdots < x_n = b$ tal que $\Delta x_i < \delta$ para todo $i \in [n]$. Uma tal partição existe: tomando $n \in \N$ tal que $\ds n > \frac{b-a}{\delta}$, é fácil ver que definindo $\ds x_i = a + i \cdot \frac{b-a}{n}$ para cada $i \in [n]_0$ temos $\Delta x_i < \delta$ para cada $i \in [n]$. Agora, $f$ é contínua em cada intervalo $[x_{i-1}, x_i]$, de modo que, pelo teorema de Weierstrass \eqref{teo.calc:weierstrass}, existem $a_i,b_i \in [x_{i-1}, x_i]$ tais que $f(a_i) = m_i$ e $f(b_i) = M_i$, onde
        \begin{align*}
            M_i &:= \sup_{x \in [x_{i-1}, x_i]}{f(x)}; \\
            m_i &:= \inf_{x \in [x_{i-1}, x_i]}{f(x)}.
        \end{align*}
    Como $|b_i - a_i| \leq \Delta x_i < \delta$ para todo $i \in [n]$, pela continuidade uniforme de $f$ temos $\ds M_i - m_i = |f(b_i) - f(a_i)| < \dfrac{\epsilon}{b-a}$ para todo $i \in [n]$, de modo que
        \[
            S^+(f,P) - S_-(f,P) = \sum_{i=1}^{n} (M_i - m_i) \Delta x_i < \dfrac{\epsilon}{b-a} \sum_{i=1}^{n} \Delta x_i = \epsilon.
        \]
    Assim, pelo critério de integrabilidade, $f$ é integrável. \itemproof

    \textbf{(b)} Como $f$ é limitada, existe $M \in \R_{>0}$ tal que $|f(x)| \leq M$ para todo $x \in [a,b]$. Se $f$ tem, digamos $p \in \N$ pontos de descontinuidade, sejam eles $x_j \in [a,b]$, para cada $j \in [p]$. Agora, seja $\epsilon \in \R_{>0}$ arbitrário. Para cada $j \in [p]$, tome $[c_j, d_j]$ centrado em $x_j$ tal que $[c_i, d_i] \cap [c_j, d_j] = \emptyset$ se $i \neq j$ e $\sum_{j=1}^{p} (d_j - c_j) < \epsilon$. Tomando $[a_j, b_j] = [c_j, d_j] \cap [a,b]$ para cada $j \in [p]$, sendo $A := [a,b] \setminus \bigcup_{j=1}^{p} \left]a_j,b_j \right[$ temos que $f$ é uniformemente contínua em $A$.
\end{proof}

\begin{teo}
    \textbf{(a)} Toda função $f:[a,b] \to \R$ monótona é integrável.

    \textbf{(b)} Se a função $f:[a,b] \to [m, M]$ é integrável e a função $g : [m, M] \to \R$ é contínua, então a função $g \circ f : [a,b] \to \R$ é integrável.
\end{teo}

\begin{proof}
    \textbf{(a)} Suponha, num primeiro caso, que $f$ seja crescente. Com isso, $f(a) \leq f(x) \leq f(b)$ para todo $x \in [a,b]$ e $f$ é limitada em $[a,b]$. Para todo $\epsilon \in \R_{>0}$ existe $n = n(\epsilon) \in \N$ suficientemente grande de modo que
        \[
            n > \dfrac{(b-a)[f(b) - f(a)]}{\epsilon}.
        \]
    Agora, a partição $P : a = x_0 < \cdots < x_n = b$ definida por $x_i = a +  i \cdot \frac{b-a}{n}$ para todo $i \in [n] \cup \{ 0\}$ é tal que $\Delta x_i = \frac{b-a}{n}$,  $M_i = f(x_i)$ e $m_i = f(x_{i-1})$ (pois $f$ é crescente), donde
        \begin{align*}
            S^+(f,P) - S_-(f,P) &= \sum_{i=1}^{n} \left[ M_i \dfrac{b-a}{n} \right] - \sum_{i=1}^{n} \left[ m_i \dfrac{b-a}{n} \right] \\
            &= \dfrac{b-a}{n} \cdot \sum_{i=1}^{n} [f(x_i) - f(x_{i-1})] \\
            &= \dfrac{b-a}{n} [f(b) - f(a)] \\
            &<\epsilon.
        \end{align*}
    Assim, pelo critério de integrabilidade, $f$ é integrável. No caso em que $f$ é decrescente, a demonstração é análoga. \itemproof
    
    \textbf{(b)}
\end{proof}


\begin{teo}
    Se $f,g : [a,b] \to \R$ são funções integráveis em $[a,b]$, então
        
        \textbf{(a)} $f + g : [a,b] \to \R$ é integrável em $[a,b]$ e
            \[
                \int_{a}^{b} [f(x) + g(x)] \, dx = \int_{a}^{b} f(x) \, dx + \int_{a}^{b} g(x) \, dx.
            \]
        \textbf{(b)} $c \cdot f : [a,b] \to \R$ (onde $c \in \R$ é uma constante) é integrável em $[a,b]$ e
            \[
                \int_{a}^{b} [c \cdot f(x)] \, dx = c \cdot \int_{a}^{b} f(x) \, dx.
            \]
        \textbf{(c)} 
            $ \ds
                \int_{a}^{b} f(x) \, dx  \leq \int_{a}^{b} g(x) \, dx
            $
        sempre que $f(x) \leq g(x)$ para todo $x \in [a,b]$.
        
        \textbf{(d)} $f \cdot g : [a,b] \to \R$ é integrável em $[a,b]$.

        \textbf{(e)} $|f| : [a,b] \to \R$ é integrável em $[a,b]$ e
            \[
                \left| \int_{a}^{b} f(x) \, dx  \right| \leq \int_{a}^{b} |f(x)| \, dx.
            \]
\end{teo}

\begin{proof}
    Táboas, página 173.
\end{proof}

\begin{prop}
    Se $I \subseteq \R$ é um intervalo fechado e $f : I \to \R$ é uma função integrável em $I$, então 
        \[
            \int_{a}^{b} f(x) \, dx = \int_{a}^{c} f(x) \, dx + \int_{c}^{b} f(x) \, dx
        \]
    para quaisquer $a,b,c \in I$.
\end{prop}

\begin{proof}
\end{proof}

\section{O Teorema Fundamental do Cálculo}

\begin{teo} \label{teo:TFC0}
    Sejam $I \subseteq \R$ um intervalo e $f , g: I \to \R$ funções contínuas.
        \begin{enumerate}[leftmargin=*, align=left, label=\textbf{(\alph*)}]
            \item Se $f'(x) = 0$ para todo $x \in I$ interior, então existe uma constante $C \in \R$ tal que $f(x) = C$ para todo $x \in I$.

            \item Se $f'(x) = g'(x)$ para todo $x \in I$ interior, então existe uma constante $C \in \R$ tal que $f(x) = g(x) + C$ para todo $x \in I$.
        \end{enumerate}
\end{teo}

\begin{proof}
\end{proof}

\begin{defi}
    Seja $I \subseteq \R$ um intervalo. Diremos que a função $F : I \to \R$ é uma \textit{primitiva} da função $f:I \to \R$ se $F'(x) = f(x)$ para todo $x \in I$.
\end{defi}

\begin{teo}(Fundamental do Cálculo, parte I) \label{ar1.teo:TFC1}
     Seja $f: [a,b] \to \R$ uma função integrável.
        \begin{enumerate}[leftmargin=*, align=left, label=\textbf{(\alph*)}]
            \item  A função $F: [a,b] \to \R$ definida por
                \[
                    F(x) := \int_{a}^{x} f(t) \, dt
                \]
            é uniformemente contínua em $[a,b]$.
            
            \item Se $f$ é contínua em $x_0 \in [a,b]$, então $F$ é derivável em $x_0$ e $F'(x_0) = f(x_0)$.
        \end{enumerate} 
\end{teo}

\begin{proof}
    \textbf{(a)} Precisamos provar que para todo $\epsilon \in \R_{>0}$ existe $\delta \in \R_{>0}$ tal que
        \[
            |x-y| < \delta \Rightarrow |F(x) - F(y)| < \epsilon
        \]
    para quaisquer $x,y \in [a,b]$. Como $f$ é integrável, $f$ é limitada, de modo que existe $M \in \R_{>0}$ tal que $|f(x)| \leq M$ para todo $x \in [a,b]$. Como, para quaisquer $x,y \in [a,b]$, temos
        \[
            |F(x) - F(y)| = \left| \int_{x}^{y} f(t) \, dt \right| \leq |M(y-x)| = M |x-y|,
        \]
    basta tomar $\delta \leq \dfrac{\epsilon}{M}$. De fato, se $\delta = \dfrac{\epsilon}{M}$ e $x,y \in [a,b]$ são tais que $|x-y| < \dfrac{\epsilon}{M}$, então $|F(x) - F(y)| \leq M |x-y| < M \cdot \dfrac{\epsilon}{M} = \epsilon$, como queríamos.

    \textbf{(b)} Para provar que $F$ é derivável em $x_0$ e $F'(x_0) = f(x_0)$, basta provar que
        \[
            \lim_{h \to 0} \left[ \dfrac{F(x_0+h) - F(x_0)}{h} - f(x_0) \right] = 0.
        \]
    Mais precisamente, basta provar que para todo $\epsilon \in \R_{>0}$ existe $\delta \in \R_{>0}$ tal que
        \[
            0 < |h| < \delta \Rightarrow \left| \dfrac{F(x_0+h) - F(x_0)}{h} - f(x_0) \right| < \epsilon
        \]
    para todo $h \in \R_{\neq 0}$ tal que $x_0 + h \in [a,b]$. Da continuidade de $f$ em $x_0$, para todo $\epsilon \in \R_{>0}$ existe $\delta \in \R_{>0}$ tal que
        \[
            |t-x_0| < \delta \Rightarrow |f(t) - f(x_0)| < \epsilon
        \]
    para todo $t \in [a,b]$. Veja que todo $h \in \R_{\neq 0}$ com $0 < |h| < \delta$ e $x_0 + h \in [a,b]$ é tal que $|t - x_0| \leq |h| < \delta$ para todo $t$ no intervalo definido por $x_0$ e $x_0 + h$ (especificamente, $t \in [x_0, x_0+h]$ se $h>0$ ou $t \in [x_0+h,x_0]$ se $h<0$), de modo que $|f(t) - f(x_0)| < \epsilon$ para todo $t$ no intervalo definido por $x_0$ e $x_0 + h$. Com isso,
        \begin{align*}
            \left |\dfrac{F(x_0+h) - F(x_0)}{h} - f(x_0) \right| &= \left| \dfrac{1}{h} \int_{x_0}^{x_0+h} [f(t) - f(x_0))] \, dt \right| \\
            &\leq \dfrac{1}{|h|}  \left| \int_{x_0}^{x_0+h} |f(t) - f(x_0)| \, dt  \right| \\
            &<  \dfrac{1}{|h|} \left| \int_{x_0}^{x_0+h} \epsilon \, dt \right| = \dfrac{1}{|h|} \epsilon |h| = \epsilon,
        \end{align*}
    o que prova que
        \[
            \lim_{h \to 0} \left[ \dfrac{F(x_0+h) - F(x_0)}{h} - f(x_0) \right] = 0,
        \]
    como havíamos afirmado. \itemproof
\end{proof}

\begin{cor}
    Seja $f : [a,b] \to \R$ uma função contínua em $[a,b]$.
        \begin{enumerate}[leftmargin=*, align=left, label=\textbf{(\alph*)}]
            \item A função $F: [a,b] \to \R$ definida por
                \[
                    F(x) := \int_{a}^{x} f(t) \, dt
                \]
            é uma primitiva de $f$ em $[a,b]$.
            
            \item Se $G: [a,b] \to \R$ é qualquer outra primitiva de $f$, então
                \[
                    G(x) = G(a) + \int_{a}^{x} f(t) \, dt
                \]
            para todo $x \in [a,b]$. Particularmente para $x = b$, temos
                \[
                    \int_{a}^{b} f(t) \, dt = G(b) - G(a).
                \]
        \end{enumerate}
\end{cor}

\begin{proof}
    
\end{proof}

\begin{teo}[Fundamental do Cálculo, parte II]
    Se $f : [a,b] \to \R $ é uma função integrável e $F : [a,b] \to \R$ é uma primitiva qualquer de $f$, então
        \[
            F(x) = F(a) + \int_{a}^{x} f(t) \, dt
        \]
    para todo $x \in [a,b] $. Particularmente para $x=b$, temos
        \[
            \int_{a}^{b} f(t) \, dt = F(b) - F(a).
        \]
\end{teo}

\begin{proof}
\end{proof}


\section{A Integral de Riemann}

\begin{defi}
    Seja $P : a < x_0 \cdots < x_n = b$ uma partição do intervalo $[a,b]$.
        \begin{enumerate}[leftmargin=*, align=left, label=\textbf{(\alph*)}]
            \item Definimos $\max \Delta x_i$ como a \textit{norma} de $P$, a qual denotaremos por $\| P \|$, isto é, $\| P \| := \max \Delta x_i$.

            \item (Partição marcada)

            \item (Soma de Riemann)
        \end{enumerate}
\end{defi}

\begin{defi}
    Seja $f : [a,b] \to \R$ uma função. Diremos que a soma de Riemann $S(f, P, \xi)$ tem limite $L \in \R$ quando $\| P \|$ tende a $0$, denotando isso por
        \[
            \lim_{\| P \| \to 0} S(f, P, \xi) = L,
        \]
    se para todo $\epsilon \in \R_{>0}$ existir $\delta = \delta (\epsilon) \in \R_{>0}$ tal que
        \[
            |S(f, P, \xi) - L| < \epsilon
        \]
    para toda partição marcada $(P, \xi)$ de $[a,b]$ com $\| P \| < \delta$. 
\end{defi}

\begin{prop}
    Seja $f : [a,b] \to \R$ uma função. O limite das somas de Riemann, quando existe, é único, isto é, se
        \[ \ds
            \lim_{\| P \| \to 0} S(f, P, \xi) = L_1 \quad \text{e} \quad \lim_{\| P \| \to 0} S(f, P, \xi) = L_2,
        \]
    então $L_1 = L_2$.
\end{prop}

\begin{proof}
\end{proof}

\begin{defi}
    Diremos que uma função $f : [a,b] \to \R$ é \textit{integrável em $[a,b]$ segundo Riemann} se
        $ \ds
            \lim_{\| P \| \to 0} S(f, P, \xi)
        $
    existir. Nesse caso, esse número real será chamado de \textit{integral de $f$ em $[a,b]$ segundo Riemann}, o qual será denotado por
        \[
            \int_{a}^{b} f(x) \, dx,
        \]
    isto é,
        \[
            \int_{a}^{b} f(x) \, dx := \lim_{\| P \| \to 0} S(f, P, \xi).
        \]
\end{defi}

\begin{prop}
    Se $f : [a,b] \to \R$ é uma função integrável segundo Riemann, então $f$ é limitada em $[a,b]$.
\end{prop}

\begin{proof}
\end{proof}

\begin{teo}
    Seja $f : [a,b] \to \R$ uma função.
        \begin{enumerate}[leftmargin=*, align=left, label=\textbf{(\alph*)}]
            \item $f$ é integrável segundo Riemann se, e somente se, é integrável segundo Darboux.

            \item Sendo $f$ integrável, as integrais de Riemann e Darboux coincidem.
        \end{enumerate}
\end{teo}

\begin{proof}
\end{proof}



\section{Integrais Impróprias}

\begin{defi}
    Seja 
\end{defi}






























\chapter{Demonstrações}

\begin{proof}
    \textbf{(a)} Consideremos o caso em que $x \to p$. Como $\ds \lim_{x \to p} f(x) = L_1$ e $\ds \lim_{x \to p} f(x) = L_2$, temos por definição que para todo $\epsilon > 0$ existem $\delta_1, \delta_2 > 0$ para os quais
    \begin{align*}  
            0 < | x - p | < \delta_1 \Rightarrow | f(x) - L_1 | < \dfrac{\epsilon}{2}; \\
            0 < | x - p | < \delta_2 \Rightarrow | f(x) - L_2 | < \dfrac{\epsilon}{2}.
        \end{align*}
    Tomando $\delta := \min \{\delta_1, \delta_2\}$, temos que para todo $\epsilon > 0$ existe $\delta > 0$ tal que
        \[
            0 < | x - p | < \delta \Rightarrow |f(x) - L_1| + |f(x) - L_2| < \epsilon.
        \]
    Com isso, temos que, para todo $\epsilon > 0$,
        \begin{align*}
            | L_1 - L_2 | &= | L_1 - f(x) + f(x) - L_2 | \\
            &\leq | L_1 - f(x) | + | f(x) - L_2 | \\
            &= |f(x) - L_1| + |f(x) - L_2| \\
            &< \epsilon.
        \end{align*}
    Daí, $L_1 = L_2$. \itemproof

    \textbf{(b)} \itemproof

    \textbf{(c)} \itemproof

    \textbf{(d)} (Verificar) Como, por hipótese, $\ds \lim_{x \to p} f(x) = L = \lim_{x \to p} h(x)$, temos
        \begin{align*}
            \forall \epsilon > 0, \exists \delta_1 > 0: 0 < \left| x - p \right| < \delta_1 \Rightarrow L - \epsilon < f(x) < L + \epsilon; \\
            \forall \epsilon > 0, \exists \delta_2 > 0: 0 < \left| x - p \right| < \delta_1 \Rightarrow L - \epsilon < h(x) < L + \epsilon.
        \end{align*}
    Pois tome $\delta = \min \left\{ \delta_1, \delta_2, r \right\}$; daí, vem
        \[
            \forall \epsilon > 0, \exists \delta > 0 : 0 < \left| x-p \right| < \delta \Rightarrow L - \epsilon < f(x) \leq g(x) \leq h(x) < L + \epsilon,
        \]
    e então
        \[
            \forall \epsilon > 0, \exists \delta > 0 : 0 < \left| x-p \right| < \delta \Rightarrow L - \epsilon < g(x) < L + \epsilon, 
        \]
    isto é, $\ds \lim_{x \to p} g(x) = L$.
\end{proof}

\begin{proof}
    \textbf{(a)} (Verificar) Consideremos o caso em que $x \to p$. Precisamos provar que
        \[
            \forall \epsilon > 0, \exists \delta > 0 : 0 < \left| x-p \right| < \delta \Rightarrow \lim_{u \to a} g(u) - \epsilon < g[f(x)] < \lim_{u \to a} g(u) + \epsilon.
        \]
     Como $\ds \lim_{u \to a} g(u) = g(a)$, temos que provar que
        \[
            \forall \epsilon > 0, \exists \delta > 0 : 0 < \left| x-p \right| < \delta \Rightarrow g(a) - \epsilon < g[f(x)] <g(a) + \epsilon. \tag{1} \label{eq:1}
        \]
    Por definição,
        \begin{multline*}
            \lim_{u \to a} g(u) = g(a) \Leftrightarrow \forall \epsilon > 0, \exists \delta_1 > 0: \\
            a - \delta_1 < u < a + \delta_1 \Rightarrow g(a) - \epsilon < g(u) < g(a) + \epsilon,
        \end{multline*}
    sendo esta última parte equivalente a 
        \[
            a - \delta_1 < f(x) < a + \delta_1 \Rightarrow g(a) - \epsilon < g[f(x)] < g(a) + \epsilon. \tag{2} \label{eq:2}
        \]
    Como, por hipótese, $\ds \lim_{x \to p} f(x) = a$, temos que
        \[
            \forall \epsilon > 0, \exists \delta > 0 : 0 < \left| x-p \right| < \delta \Rightarrow a - \epsilon < f(x) < a + \epsilon. \tag{3} \label{eq:3}
        \]
    Para $\epsilon = \delta_1$ em $\eqref{eq:3}$, existe um $\delta > 0$ tal que 
        \[
            0 < \left| x-p \right| < \delta \Rightarrow a - \delta_1 < f(x) < a + \delta_1. \tag{4} \label{eq:4}
        \]
    Daí, $\eqref{eq:4}$, com $\eqref{eq:2}$, resulta que
        \[
            \forall \epsilon > 0, \exists \delta > 0 : 0 < \left| x-p \right| < \delta \Rightarrow g(a) - \epsilon < g[f(x)] < g(a) + \epsilon,
        \]
    como queríamos provar. \itemproof
    
    \textbf{(b)} (Verificar) Precisamos provar que
        \[
            \forall \epsilon > 0, \exists \delta > 0 : 0 < \left| x-p \right| < \delta \Rightarrow \left| g[f(x)] - \lim_{u \to a} g(u) \right| < \epsilon,
        \]
    isto é,
        \[  
            0 < \left| x-p \right| < \delta \Rightarrow \left| g[f(x)] - L \right| < \epsilon.
        \]
    Bem, por definição,
        \[
            \lim_{u \to a} g(u) = g(a) \Leftrightarrow \forall \epsilon > 0, \exists \delta_1 > 0: 0 < \left| u-a \right| < \delta_1 \Rightarrow \left| g(u) - L \right| < \epsilon.
        \]
    Lembrando que $u := f(x)$, temos então
        \[
            0 < \left| f(x) - a \right| < \delta_1 \Rightarrow \left| g[f(x)] - L \right| < \epsilon.
        \]
    Por outro lado,
        \[
            \lim_{x \to p} f(x) = a \Leftrightarrow \forall \epsilon > 0, \exists \delta_2 > 0 : 0 < \left| x-p \right| < \delta_2 \Rightarrow \left| f(x) - a \right| < \epsilon,
        \]
    e então, tomando $\epsilon = \delta_1$, $0 < \left| x-p \right| < \delta_2 \Rightarrow \left| f(x) - a \right| < \delta_1$.

    Pois tome $\delta = \left\{ \delta_2, r \right\}$; teremos $0 < \left| x-p \right| < \delta \Rightarrow 0 < \left| f(x) - a \right| < \delta_1$.
\end{proof}


\begin{proof}
\end{proof}

\newpage

Um \textit{designador} é uma expressão que é um termo ou uma fórmula. Como se vê na definição de termo e fórmula, todo designador tem a forma \( \mathbf{u} \mathbf{v}_1 \ldots \mathbf{v}_n \), onde \( \mathbf{u} \) é um símbolo, \( \mathbf{v}_1, \ldots, \mathbf{v}_n \) são designadores, e \( n \) é um número natural determinado por \( \mathbf{u} \). Por exemplo, se \( \mathbf{u} \) é uma variável, então \( n = 0 \); se \( \mathbf{u} \) é um símbolo funcional \( k \)-ário, então \( n = k \); se \( \mathbf{u} \) é \( \exists \), então \( n = 2 \). Chamamos \( n \) de \textit{índice} de \( \mathbf{u} \). %ok

Dizemos que duas expressões são \textit{compatíveis} se uma delas pode ser obtida adicionando alguma expressão (possivelmente a expressão vazia) ao final da outra. 
    \begin{itemize}
        \item Se \( \mathbf{u} \mathbf{v} \) e \( \mathbf{u}'\mathbf{v}' \) são compatíveis, então \( \mathbf{u} \) e \( \mathbf{u}' \) são compatíveis; 
        \item se \( \mathbf{u} \mathbf{v} \) e \( \mathbf{u} \mathbf{v}' \) são compatíveis, então \( \mathbf{v} \) e \( \mathbf{v}' \) são compatíveis.
    \end{itemize} %ok

\textbf{Lema.} Seja $n$ um natural fixo. Se $\mathbf{u}_1, \ldots, \mathbf{u}_n$ e $\mathbf{u}'_1, \ldots, \mathbf{u}'_n$ são designadores, e $\mathbf{u}_1 \ldots \mathbf{u}_n$ e $\mathbf{u}'_1 \ldots \mathbf{u}'_n$ são compatíveis, então $\mathbf{u}'_i$ é $\mathbf{u}_i$, para $i = 1, \ldots, n$.

\begin{proof}
    Façamos indução no comprimento de $\mathbf{u}_1 \ldots \mathbf{u}_n$, isto é, no número de símbolos totais dessa expressão. Sendo $|\mathbf{u}_1 \ldots \mathbf{u}_n| = L$, provemos que se o resultado vale para toda sequência de designadores de comprimento menor que $L$, então vale para as sequências de designadores de comprimento $L$. Escrevendo $\mathbf{u}_1$ como $\mathbf{v} \mathbf{v}_1 \ldots \mathbf{v}_k$, onde $\mathbf{v}$ é um símbolo e $\mathbf{v}_1, \ldots, \mathbf{v}_k$ são designadores, vemos que $\mathbf{u}'_1$ é da forma $\mathbf{v} \mathbf{v}'_1 \ldots \mathbf{v}'_k$, onde $\mathbf{v}'_1, \ldots, \mathbf{v}'_k$ são designadores. Agora, como $\mathbf{u}_1$ é compatível com $\mathbf{u}'_1$ (isso segue do primeiro bullet point, fazendo $\mathbf{u}$ ser $\mathbf{u}_1$, $\mathbf{v}$ ser $\mathbf{u}_2 \ldots \mathbf{u}_n$, $\mathbf{u}'$ ser $\mathbf{u}'_1$ e $\mathbf{v}'$ ser $\mathbf{u}'_2 \ldots \mathbf{u}'_n$), temos que $\mathbf{v} \mathbf{v}_1 \ldots \mathbf{v}_k$ é compatível com $\mathbf{v} \mathbf{v}'_1 \ldots \mathbf{v}'_k$, de modo que  $\mathbf{v}_1 \ldots \mathbf{v}_k$ é compatível com $\mathbf{v}'_1 \ldots \mathbf{v}'_k$ (isso segue do segundo bullet point). Como, evidentemente, $|\mathbf{v}_1 \ldots \mathbf{v}_k| < L$, pela hipótese de indução concluímos que $\mathbf{v}_i$ é $\mathbf{v}'_i$ para cada $i = 1, \ldots, k$. Com isso, temos que $\mathbf{u}_1$ é $\mathbf{u}'_1$, de modo que $\mathbf{u}_2 \ldots \mathbf{u}_n$ é compatível com $\mathbf{u}'_2 \ldots \mathbf{u}'_n$ (pela segunda observação), e como $|\mathbf{u}_2 \ldots \mathbf{u}_n| < L$, pela hipótese de indução concluímos que $\mathbf{u}_i$ é $\mathbf{u}'_i$ para $i = 2, \ldots, n$. Isso nos dá a tese de indução, já que já sabemos que $\mathbf{u}_1$ é $\mathbf{u}'_1$.
\end{proof}

\begin{proof}
    Façamos indução forte no comprimento de $\mathbf{u}_1 \ldots \mathbf{u}_n$, isto é, no número de símbolos totais dessa expressão. Sendo $|\mathbf{u}_1 \ldots \mathbf{u}_n| = L$, provemos que se o resultado vale para toda sequência de designadores de comprimento menor que $L$, então vale para as sequências de designadores de comprimento $L$. Escrevendo $\mathbf{u}_1$ como $\mathbf{v} \mathbf{v}_1 \ldots \mathbf{v}_k$, onde $\mathbf{v}$ é um símbolo e $\mathbf{v}_1, \ldots, \mathbf{v}_k$ são designadores, vemos que $\mathbf{u}'_1$ é da forma $\mathbf{v} \mathbf{v}'_1 \ldots \mathbf{v}'_k$, onde $\mathbf{v}'_1, \ldots, \mathbf{v}'_k$ são designadores. Agora, como $\mathbf{u}_1$ é compatível com $\mathbf{u}'_1$ (isso segue da primeira observação ``se \( \mathbf{u} \mathbf{v} \) e \( \mathbf{u}'\mathbf{v}' \) são compatíveis, então \( \mathbf{u} \) e \( \mathbf{u}' \) são compatíveis'', fazendo $\mathbf{u}$ ser $\mathbf{u}_1$, $\mathbf{v}$ ser $\mathbf{u}_2 \ldots \mathbf{u}_n$, $\mathbf{u}'$ ser $\mathbf{u}'_1$ e $\mathbf{v}'$ ser $\mathbf{u}'_2 \ldots \mathbf{u}'_n$), temos que $\mathbf{v} \mathbf{v}_1 \ldots \mathbf{v}_k$ é compatível com $\mathbf{v} \mathbf{v}'_1 \ldots \mathbf{v}'_k$, de modo que  $\mathbf{v}_1 \ldots \mathbf{v}_k$ é compatível com $\mathbf{v}'_1 \ldots \mathbf{v}'_k$ (isso segue da segunda observação ``se \( \mathbf{u} \mathbf{v} \) e \( \mathbf{u} \mathbf{v}' \) são compatíveis, então \( \mathbf{v} \) e \( \mathbf{v}' \) são compatíveis''). Como, evidentemente, $|\mathbf{v}_1 \ldots \mathbf{v}_k| < L$, pela hipótese de indução concluímos que $\mathbf{v}_i$ é $\mathbf{v}'_i$ para cada $i = 1, \ldots, k$. Com isso, temos que $\mathbf{u}_1$ é $\mathbf{u}'_1$, de modo que $\mathbf{u}_2 \ldots \mathbf{u}_n$ é compatível com $\mathbf{u}'_2 \ldots \mathbf{u}'_n$ (pela segunda observação), e como $|\mathbf{u}_2 \ldots \mathbf{u}_n| < L$, pela hipótese de indução concluímos que $\mathbf{u}_i$ é $\mathbf{u}'_i$ para $i = 2, \ldots, n$. Isso nos dá a tese de indução, já que já sabemos que $\mathbf{u}_1$ é $\mathbf{u}'_1$.
   \end{proof}
 \newpage
Consideremos um conjunto enumerável correspondente aos símbolos do alfabeto e $X$ o conjunto de todas as sequências finitas de símbolos. Seja $Z$ o conjunto de todos os subconjuntos $Y$ de $X$ tais que
\begin{itemize}
    \item as variáveis proposicionais pertencem a $Y$;
    \item se $A$ pertence a $Y$, então $(\neg A)$ pertence a $Y$;
    \item se $A$ e $B$ pertencem a $Y$, então $(A \land B)$, $(A \lor B)$, $(A \rightarrow B)$ e $(A \leftrightarrow B)$ pertencem a $Y$.
\end{itemize}

\begin{teo}
Suponha que uma propriedade vale para toda fórmula atômica e que, se vale para as fórmulas $A$ e $B$, também vale para $(\neg A)$, $(A \land B)$, $(A \lor B)$, $(A \rightarrow B)$ e $(A \leftrightarrow B)$. Então essa propriedade vale para todas as fórmulas da lógica proposicional.
\end{teo}

\begin{proof}
    Tomando $F$ a interseção de $Z$, temos que $F$ satisfaz as condições acima (isto é, pertence à família $Z$) e é o menor conjunto (na ordem da inclusão) que pertence a $Z$. Isto é, se $Y \in Z$, então $F \subset Y$. Segue facilmente, daí, o teorema. Deixamos os detalhes ao leitor.
\end{proof}

\part{Álgebra Linear}

%!TEX root = main.tex

\chapter{Matrizes e Sistemas Lineares}

\section{Definições Iniciais e Operações Matriciais}

% $abcdemz$ abcde \textit{abcdemz} $12345$ \textit{12345} 12345

\begin{defi}
    \textbf{(a)} Sejam $m,n \in \N$. Uma \textit{matriz} $A = (a_{ij})_{m \times n}$ é uma função $A : \left\{1,2, \ldots, m\right\} \times \left\{1, 2, \ldots, n\right\} \to \mathbb{R}$ que associa a cada par $(i,j)$, com $i \in [m]$ e $j \in [n]$, um elemento $a_{ij} \in \mathbb{R}$. A representação canônica de uma matriz é uma tabela com $m$ linhas e $n$ colunas
        \[
            A :=
                \begin{bmatrix}
                    a_{11} & a_{12} & \cdots & a_{1n} \\
                    a_{21} & a_{22} & \cdots & a_{2n} \\
                    \vdots & \vdots & \ddots &  \vdots \\
                    a_{m1} & a_{m2} & \cdots & a_{mn}
                \end{bmatrix}.
        \]
    Dizemos que a matriz $A$ definida acima tem \textit{tamanho}, ou \textit{tipo}, $m \times n$ (lê-se $m$ por $n$). Dizemos que $a_{ij}$, ou $[A]_{ij}$, é o \textit{elemento}, ou a \textit{entrada}, de posição $i, j$. O conjunto de todas as matrizes de tamanho $m \times n$ com entradas reais será denotado por $\mathcal{M}_{m \times n} (\R)$.
    
    \textbf{(b)} Duas matrizes são ditas \textit{iguais} se elas são do mesmo tipo e se os elementos correspondentes forem iguais. Mais especificamente, as matrizes $A = (a_{ij})_{m \times n}$ e $B = (b_{ij})_{p \times q}$ são iguais se $m=p$, $n=q$ e $a_{ij} = b_{ij}$ para todos $i \in [m]$ e $j \in [n]$.
    \begin{comment} Uma matriz $A = (a_{ij})_{m \times n}$ é um elemento de
        $
            \mathbb{K}^{m \times n} := \mathbb{K}^m \times \mathbb{K}^m \times \cdots \times \mathbb{K}^m,
        $
    podendo ser representada como
        \[ A =
            \begin{bmatrix}
                A_1 & A_2 & \cdots & A_n
            \end{bmatrix},
            \quad \text{em que} \quad
            A_i = \begin{bmatrix}
                a_{1i} \\
                a_{2i} \\
                \vdots \\
                a_{mi} \\
            \end{bmatrix}
        \] \end{comment}
\end{defi}

\begin{defi}
    Dada uma matriz $A := (a_{ij})_{m \times n}$, definimos
        \[
            L_i (A):=
                \begin{bmatrix}
                    a_{i1} & a_{i2} & \cdots & a_{in}
                \end{bmatrix}
        \]
    como a $i$-ésima linha da matriz $A$, com $i \in [m]$. Definimos, ainda,
        \[
            C_j(A) :=
                \begin{bmatrix}
                    a_{1j} \\
                    a_{2j} \\
                    \vdots \\
                    a_{mj} \\
                \end{bmatrix}
        \]
    como a $j$-ésima coluna da matriz $A$, com $j \in [n]$.
\end{defi}

\subsection*{Operações Matriciais}

\begin{defi}
    A \textit{soma}, ou \textit{adição}, de duas matrizes do mesmo tipo $A = (a_{ij})_{m \times n}$ e $B = (b_{ij})_{m \times n}$ é definida como a matriz $C = (c_{ij})_{m \times n}$ em que $c_{ij} = a_{ij} + b_{ij}$ para todos $i \in [m]$ e $j \in [n]$. Denotamos isso com $C = A + B$.
\end{defi}

\begin{prop}
    Para quaisquer matrizes $A = (a_{ij})_{m \times n}$, $B = (b_{ij})_{m \times n}$ e $C = (c_{ij})_{m \times n}$, valem

        \textbf{(a)} a associatividade da adição, isto é, $A + (B + C) = (A + B) + C$;

        \textbf{(b)} a comutatividade da adição, isto é, $A+B=B+A$;

        \textbf{(c)} a existência de um elemento neutro, isto é, $A + 0 = A$;

        \textbf{(d)} a existência de um oposto aditivo, isto é, $A + (-A) = 0$.
\end{prop}

\begin{proof}
\end{proof}

\begin{defi}
    A multiplicação de uma matriz $A = (a_{ij})_{m \times n}$ por um \textit{escalar} $\alpha$ é definida como a matriz $B = (b_{ij})_{m \times n}$ em que $b_{ij} = \alpha \cdot a_{ij}$ para todos $i \in [m]$ e $j \in [n]$. Denotamos isso com $B = \alpha \cdot A = \alpha A$.
\end{defi}

\begin{prop}
    Para quaisquer matrizes $A = (a_{ij})_{m \times n}$ e $B = (b_{ij})_{m \times n}$ e escalares $\alpha, \beta \in \R$, temos que

    \textbf{(a)} $\alpha \cdot (\beta \cdot A) = (\alpha \cdot \beta) \cdot A$;

    \textbf{(b)} $\alpha \cdot (A + B) = \alpha \cdot A + \alpha \cdot B$:

    \textbf{(c)} $(\alpha + \beta) \cdot A = \alpha \cdot A + \beta \cdot A$;

    \textbf{(d)} $1 \cdot A = A$.
\end{prop}

\begin{proof}
\end{proof}

\begin{defi}
    \textbf{(a)} O \textit{produto}, ou \textit{multiplicação}, de duas matrizes, em que o número de colunas da primeira matriz é igual ao número de linhas da segunda, $A = (a_{ij})_{m \times n}$ e $B = (b_{ij})_{n \times p}$, é definido como a matriz $C = (c_{ij})_{m \times p}$ em que
        \[
            c_{ij} = \sum_{k=1}^{n} a_{ik} b_{kj},
        \]
    para todos $i \in [m]$ e $j \in [p]$. Denotamos isso por $C = A \cdot B = AB$.

    \textbf{(b)} O produto de duas matrizes, em que o número de colunas da primeira matriz é igual ao número de linhas da segunda, $A = (a_{ij})_{m \times n}$ e $B = (b_{jk})_{n \times p}$ também pode ser definido como a matriz $C = (c_{ik})_{m \times p}$ em que
        \[
            c_{ik} = \sum_{j=1}^{n} a_{ij} b_{jk},
        \]
    para todos $i \in [m]$, $k \in [p]$ e $j \in [n]$. Essa definição pode ser útil para evitar confusões com os índices.
\end{defi}

\begin{prop}
    Para quaisquer matrizes $A$, $B$ e $C$, de tamanhos compatíveis, e escalares $\alpha, \beta \in \R$, valem

    \textbf{(a)} a associatividade do produto, isto é, $A \cdot (B \cdot C) = (A \cdot B) \cdot C$;

    \textbf{(b)} a existência de um elemento neutro, isto é $A \cdot \mathrm{I} = \mathrm{I} \cdot A = A$;

    \textbf{(c)} a distributividade da multiplicação com relação à adição, isto é, $A \cdot (B+C) = A \cdot B + A \cdot C$ e $(A+B) \cdot C = A \cdot C + B \cdot C$;

    \textbf{(d)} a associatividade do produto de matrizes com relação ao produto por escalar, isto é, $(\alpha \cdot A) \cdot B = \alpha \cdot (A \cdot B) = A \cdot (\alpha \cdot B)$;
\end{prop}

\begin{proof}
\end{proof}

\subsection*{Matrizes Especiais}

\begin{defi}
    Uma matriz $A \in \mathcal{M}_{m \times n}(\R)$ é dita
        \begin{enumerate}
            \item \textit{invertível à esquerda} se existir uma matriz $B \in \mathcal{M}_{n \times m}(\R)$ tal que $BA=\mathrm{I}_n$;
            \item \textit{invertível à direita} se existir uma matriz $C \in \mathcal{M}_{n \times m}(\R)$ tal que $AC = \mathrm{I}_m$;
            \item \textit{invertível}, ou ainda, \textit{não singular}, se for invertível à esquerda e à direita;
            \item \textit{singular} se não for invertível. 
        \end{enumerate}
\end{defi}

\begin{prop}
    Se uma matriz possui uma matriz inversa, então ela é única.
\end{prop}

\begin{proof} 
    Se $A \in \mathcal{M}_{m \times n}(\R)$ é invertível, então, por definição, existem $B \in \mathcal{M}_{n \times m}(\R)$ e $C \in \mathcal{M}_{n \times m}(\R)$ tais que $BA = \mathrm{I}_n$ e $AC = \mathrm{I}_{m}$. Com isso, basta ver que
        \[
            B = B \mathrm{I}_{m} = B(AC) = (BA)C = \mathrm{I}_{n} C = C,
        \]
    isto é, $B = C$.
\end{proof}

\section{Operações e Matrizes Elementares}

Dada uma matriz, podemos 

\begin{itemize}
    \item trocar a posição de duas de suas linhas;
    \item multiplicar uma de suas linhas por um escalar;\footnote{Esperamos ser evidente que nos referimos a uma multiplicação que  ocorre em \textit{cada entrada} da referida linha.}
    \item e somar a uma de suas linhas uma outra linha que foi multiplicada por um escalar.
\end{itemize}
Estas são as chamadas \textit{operações elementares}. Elas estão formalizadas na próxima definição e serão úteis no nosso estudo dos sistemas de equações lineares.

\begin{defi} Sejam $A \in \mathcal{M}_{m \times n}(\R)$ e $p, q \in [m]$, com $p < q$.
    
    \textbf{(a)} Definimos $A_{L_p \leftrightarrow L_q}$ como a matriz, também $m \times n$, tal que
        \[
            L_i \left( A_{L_p \leftrightarrow L_q} \right) :=
                \begin{cases}
                    L_q (A) & \text{se } i = p \\
                    L_p (A) & \text{se } i = q \\
                    L_i (A) & \text{se } i \neq p, q
                \end{cases}.
        \]
    \begin{comment} isto é,
        \[
            A :=
                \begin{bmatrix}
                    a_{11} & a_{12} & \cdots & a_{1n} \\
                    \vdots & \vdots & \ddots &  \vdots \\
                    a_{p1} & a_{p2} & \cdots & a_{pn} \\
                    \vdots & \vdots & \ddots &  \vdots \\
                    a_{q1} & a_{q2} & \cdots & a_{qn} \\
                    \vdots & \vdots & \ddots &  \vdots \\
                    a_{m1} & a_{m2} & \cdots & a_{mn}
                \end{bmatrix}
            \Rightarrow
            A_{L_p \leftrightarrow L_q} :=
                \begin{bmatrix}
                    a_{11} & a_{12} & \cdots & a_{1n} \\
                    \vdots & \vdots & \ddots &  \vdots \\
                    a_{q1} & a_{q2} & \cdots & a_{qn} \\
                    \vdots & \vdots & \ddots &  \vdots \\
                    a_{p1} & a_{p2} & \cdots & a_{pn} \\
                    \vdots & \vdots & \ddots &  \vdots \\
                    a_{m1} & a_{m2} & \cdots & a_{mn}
                \end{bmatrix}.
        \] \end{comment}
    \textbf{(b)} Seja $\lambda \in \R^*$ um escalar. Definimos $A_{L_p \gets \lambda L_p}$ como a matriz, também $m \times n$, tal que
        \[
            L_i \left( A_{L_p \gets \lambda L_p} \right) := 
            \begin{cases}
                \lambda L_p (A), & \text{se } i = p \\
                L_i (A), & \text{se } i \neq p
            \end{cases}.
        \]
    \begin{comment} isto é,
        \[
            A :=
                \begin{bmatrix}
                    a_{11} & a_{12} & \cdots & a_{1n} \\
                    \vdots & \vdots & \ddots &  \vdots \\
                    a_{p1} & a_{p2} & \cdots & a_{pn} \\
                    \vdots & \vdots & \ddots &  \vdots \\
                    a_{m1} & a_{m2} & \cdots & a_{mn}
                \end{bmatrix}
            \Rightarrow
            A_{L_p \gets \lambda L_p} :=
                \begin{bmatrix}
                    a_{11} & a_{12} & \cdots & a_{1n} \\
                    \vdots & \vdots & \ddots &  \vdots \\
                    \lambda a_{p1} & \lambda a_{p2} & \cdots & \lambda a_{pn} \\
                    \vdots & \vdots & \ddots &  \vdots \\
                    a_{m1} & a_{m2} & \cdots & a_{mn}
                \end{bmatrix}.
        \] \end{comment}
    \textbf{(c)} Seja $\lambda \in \R^*$ um escalar. Definimos $A_{L_p \gets L_p + \lambda L_q}$ como a matriz, também $m \times n$, tal que
        \[
            L_i \left( A_{L_p \gets L_p + \lambda L_q} \right) := 
                \begin{cases}
                    L_p (A) + \lambda L_q (A), & \text{se } i = p \\
                    L_i (A), & \text{se } i \neq p
                \end{cases},
        \]
    \begin{comment} isto é,
        \[
             A :=
                \begin{bmatrix}
                    a_{11} & \cdots & a_{1n} \\
                    \vdots & \ddots &  \vdots \\
                    a_{p1} & \cdots & a_{pn} \\
                    \vdots & \ddots &  \vdots \\
                    a_{q1} & \cdots & a_{qn} \\
                    \vdots & \ddots &  \vdots \\
                    a_{m1} & \cdots & a_{mn}
                \end{bmatrix}
            \Rightarrow
            A_{L_p \gets L_p + \lambda L_q} =
                \begin{bmatrix}
                    a_{11} & \cdots & a_{1n} \\
                    \vdots & \ddots &  \vdots \\
                    a_{p1} + \lambda a_{q1} & \cdots & a_{pn} + \lambda a_{qn} \\
                    \vdots & \ddots &  \vdots \\
                    a_{q1} & \cdots & a_{qn} \\
                    \vdots & \ddots &  \vdots \\
                    a_{m1} & \cdots & a_{mn}
                \end{bmatrix},
        \] \end{comment}
    e definimos $A_{L_q \gets L_q + \lambda L_p}$ como a matriz,  também $m \times n$, tal que
        \[
            L_i \left( A_{L_q \gets L_q + \lambda L_p} \right) :=
                \begin{cases}
                    L_q (A) + \lambda L_p (A), & \text{se } i = q \\
                    L_i (A), & \text{se } i \neq q
                \end{cases}.
        \]
    \begin{comment} isto é,
        \[
            A :=
                \begin{bmatrix}
                    a_{11} & \cdots & a_{1n} \\
                    \vdots & \ddots &  \vdots \\
                    a_{p1} & \cdots & a_{pn} \\
                    \vdots & \ddots &  \vdots \\
                    a_{q1} & \cdots & a_{qn} \\
                    \vdots & \ddots &  \vdots \\
                    a_{m1} & \cdots & a_{mn}
                \end{bmatrix}
            \Rightarrow
            A_{L_q \gets L_q + \lambda L_p} =
                \begin{bmatrix}
                    a_{11} & \cdots & a_{1n} \\
                    \vdots & \ddots &  \vdots \\
                    a_{p1} & \cdots & a_{pn} \\
                    \vdots & \ddots &  \vdots \\
                    a_{q1} + \lambda a_{p1} & \cdots & a_{qn} + \lambda a_{pn} \\
                    \vdots & \ddots &  \vdots \\
                    a_{m1} & \cdots & a_{mn}
                \end{bmatrix}.
        \] \end{comment}
\end{defi}

Na proposição a seguir, mostramos que cada operação elementar equivale à multiplicar a matriz $A$ por uma matriz dita \textit{elementar}, obtida pela aplicação de operações elementares na matriz identidade $\mathrm{I}_m$.

\begin{prop} 
    Sejam $A \in \mathcal{M}_{m \times n}(\R)$ e $p, q \in [m]$, com $p < q$.
    
    \textbf{(a)} Temos $\ds \mathrm{I}_{L_p \leftrightarrow L_q} \cdot A = A_{L_p \leftrightarrow L_q}$. \begin{comment} , isto é,
        \[
            \begin{bmatrix}
                    1 &         &          &         &        &      &      \\
                      & \ddots  &          &         &        &      &     \\
                      &         & 0        & \cdots  & 1      &      &     \\
                      &         & \vdots   & \ddots  & \vdots &      &      \\
                      &         & 1        & \cdots  & 0      &      &       \\
                      &         &          &         &        &      \ddots &  \\
                      &         &          &         &        &      &  1
            \end{bmatrix} \cdot
            \begin{bmatrix}
                a_{11} & a_{12} & \cdots & a_{1n} \\
                \vdots & \vdots & \ddots &  \vdots \\
                a_{p1} & a_{p2} & \cdots & a_{pn} \\
                \vdots & \vdots & \ddots &  \vdots \\
                a_{q1} & a_{q2} & \cdots & a_{qn} \\
                \vdots & \vdots & \ddots &  \vdots \\
                a_{m1} & a_{m2} & \cdots & a_{mn}
            \end{bmatrix} =
            \begin{bmatrix}
                a_{11} & a_{12} & \cdots & a_{1n} \\
                \vdots & \vdots & \ddots &  \vdots \\
                a_{q1} & a_{q2} & \cdots & a_{qn} \\
                \vdots & \vdots & \ddots &  \vdots \\
                a_{p1} & a_{p2} & \cdots & a_{pn} \\
                \vdots & \vdots & \ddots &  \vdots \\
                a_{m1} & a_{m2} & \cdots & a_{mn}
            \end{bmatrix}.
        \] \end{comment}
    
    \textbf{(b)} Temos $\ds \mathrm{I}_{L_p \gets \lambda L_p} \cdot A = A_{L_p \gets \lambda L_p}$. \begin{comment} , isto é,
        \[
            \begin{bmatrix}
                1   &        &         &        &          \\
                    & \ddots &         &        &          \\
                    &        & \lambda &        &          \\
                    &        &         & \ddots &          \\
                    &        &         &        & 1        \\
            \end{bmatrix} \cdot
            \begin{bmatrix}
                    a_{11} & a_{12} & \cdots & a_{1n} \\
                    \vdots & \vdots & \ddots &  \vdots \\
                    a_{p1} & a_{p2} & \cdots & a_{pn} \\
                    \vdots & \vdots & \ddots &  \vdots \\
                    a_{m1} & a_{m2} & \cdots & a_{mn}
            \end{bmatrix} =
            \begin{bmatrix}
                    a_{11} & a_{12} & \cdots & a_{1n} \\
                    \vdots & \vdots & \ddots &  \vdots \\
                    \lambda a_{p1} & \lambda a_{p2} & \cdots & \lambda a_{pn} \\
                    \vdots & \vdots & \ddots &  \vdots \\
                    a_{m1} & a_{m2} & \cdots & a_{mn}
            \end{bmatrix}.
        \] \end{comment}
    
    \textbf{(c)} Temos $\ds \mathrm{I}_{L_p \gets L_p + \lambda L_q} \cdot A = A_{L_p \gets L_p + \lambda L_q}$ e $\ds \mathrm{I}_{L_q \gets L_q + \lambda L_p} \cdot A = A_{L_q \gets L_q + \lambda L_p}$. \begin{comment}, isto é,
        \[
            \begin{bmatrix}
                    1 &         &         &         &         &        &     \\
                      & \ddots  &         &         &         &        &     \\
                      &         & 1       & \cdots  & \lambda &        &     \\
                      &         & \vdots  & \ddots  & \vdots  &        &     \\
                      &         & 0       & \cdots  & 1       &        &     \\
                      &         &         &         &         & \ddots &     \\
                      &         &         &         &         &        &  1
            \end{bmatrix} \cdot
            \begin{bmatrix}
                    a_{11} & \cdots & a_{1n} \\
                    \vdots & \ddots &  \vdots \\
                    a_{p1} & \cdots & a_{pn} \\
                    \vdots & \ddots &  \vdots \\
                    a_{q1} & \cdots & a_{qn} \\
                    \vdots & \ddots &  \vdots \\
                    a_{m1} & \cdots & a_{mn}
            \end{bmatrix} =
            \begin{bmatrix}
                    a_{11} & \cdots & a_{1n} \\
                    \vdots & \ddots &  \vdots \\
                    a_{p1} + \lambda a_{q1} & \cdots & a_{pn} + \lambda a_{qn} \\
                    \vdots & \ddots &  \vdots \\
                    a_{q1} & \cdots & a_{qn} \\
                    \vdots & \ddots &  \vdots \\
                    a_{m1} & \cdots & a_{mn}
                \end{bmatrix},
        \]
    e ainda, temos que $\ds \mathrm{I}_{L_q \gets L_q + \lambda L_p} \cdot A = A_{L_q \gets L_q + \lambda L_p}$, isto é,
        \[
            \begin{bmatrix}
                    1 &         &          &         &        &      &      \\
                      & \ddots  &          &         &        &      &     \\
                      &         & 1        & \cdots  & 0      &      &     \\
                      &         & \vdots   & \ddots  & \vdots &      &      \\
                      &         & \lambda  & \cdots  & 1      &      &       \\
                      &         &          &         &        &      \ddots &  \\
                      &         &          &         &        &      &  1
            \end{bmatrix} \cdot
            \begin{bmatrix}
                    a_{11} & \cdots & a_{1n} \\
                    \vdots & \ddots &  \vdots \\
                    a_{p1} & \cdots & a_{pn} \\
                    \vdots & \ddots &  \vdots \\
                    a_{q1} & \cdots & a_{qn} \\
                    \vdots & \ddots &  \vdots \\
                    a_{m1} & \cdots & a_{mn}
            \end{bmatrix} =
            \begin{bmatrix}
                    a_{11} & \cdots & a_{1n} \\
                    \vdots & \ddots &  \vdots \\
                    a_{p1} & \cdots & a_{pn} \\
                    \vdots & \ddots &  \vdots \\
                    a_{q1} + \lambda a_{p1} & \cdots & a_{qn} + \lambda a_{pn} \\
                    \vdots & \ddots &  \vdots \\
                    a_{m1} & \cdots & a_{mn}
            \end{bmatrix}.
        \] \end{comment}
\end{prop}

\begin{proof}
\end{proof}

\begin{prop}
    As operações (matrizes) elementares são invertíveis.
\end{prop}

\begin{proof}
A demonstração é feita exibindo-se, explicitamente, as inversas.
    \begin{itemize}
        \item A inversa de $\mathrm{I}_{L_p \leftrightarrow L_q}$ é ela mesma, isto é, $\mathrm{I}_{L_p \leftrightarrow L_q} \cdot \mathrm{I}_{L_p \leftrightarrow L_q} = \mathrm{I}$.
        \item A inversa de $\mathrm{I}_{L_p \gets \lambda L_p}$ é $\mathrm{I}_{L_p \gets \frac{1}{\lambda} L_p}$, isto é, \[\mathrm{I}_{L_p \gets \lambda L_p} \cdot \mathrm{I}_{L_p \gets \frac{1}{\lambda} L_p} = \mathrm{I}_{L_p \gets \frac{1}{\lambda} L_p} \cdot \mathrm{I}_{L_p \gets \lambda L_p} = \mathrm{I}.\]
        \item A inversa de $\mathrm{I}_{L_q \gets L_q + \lambda L_p}$ é $\mathrm{I}_{L_q \gets L_q -\lambda L_p}$, isto é, \[\mathrm{I}_{L_q \gets L_q + \lambda L_p} \cdot \mathrm{I}_{L_q \gets L_q - \lambda L_p} = \mathrm{I}_{L_q \gets L_q - \lambda L_p} \cdot \mathrm{I}_{L_q \gets L_q + \lambda L_p} = \mathrm{I}.\]
    \end{itemize}
\end{proof}

\begin{defi} \label{defi.III:eqporlinha}
    \textbf{(a)} (Informal) Uma matriz $A$ é dita \textit{equivalente por linhas} a uma matriz $B$, de mesmo tamanho, se existir uma sequência finita de operações elementares que, quando aplicadas em $A$, tem $B$ como resultado. Denotamos isso por 
        \[
            E \cdot A = B, \text{ em que } E = E_k \cdot E_{k-1} \cdot \ldots \cdot E_2 \cdot E_1,
        \]
    em que $E_i$ é uma matriz elementar para cada $i \in [k]$.

    \textbf{(b)} Diremos que uma matriz $A \in \mathcal{M}_{m \times n}(\R)$ é dita \textit{equivalente por linhas} a uma matriz $B \in \mathcal{M}_{m \times n}(\R)$, denotando isso por $A \sim B$, se existir uma sequência finita de matrizes elementares $E_1, \ldots, E_k \in \mathcal{M}_{m \times m}(\R)$ tais que 
        \[
            E_k \cdot E_{k-1} \cdot \ldots \cdot E_2 \cdot E_1 \cdot A = B.
        \]
\end{defi}

\begin{prop}
    A equivalência por linhas definida em \eqref{defi.III:eqporlinha} é uma relação de equivalência.
\end{prop}

\section{Eliminação Gaussiana e Decomposição LU}

\begin{defi}
    \textbf{(a)} (Intuitiva) Uma matriz será dita \textit{escalonada} se (i) o primeiro elemento não nulo de cada linha está à esquerda do primeiro elemento não nulo de cada uma das linhas seguintes e (ii) as linhas nulas (se houver) estão abaixo das demais.
    
    \textbf{(b)} Uma matriz $A = (a_{ij})_{m \times n}$ será dita \textit{escalonada} se existir uma sequência de índices $1 \leq b_1 < b_2 < \ldots < b_r \leq n$ tal que $a_{ib_i} \neq 0$ para todo $i = 1, 2, \ldots, r$ e $a_{ij} = 0$ para todo $1 \leq j < b_i$. Os termos $a_{ib_i}$ são chamados de \textit{pivôs}, enquanto o número de pivôs, $r$, é chamado de \textit{posto}.
\end{defi}

\begin{prop}
    Toda matriz é equivalente por linhas a uma matriz escalonada.
\end{prop}

\begin{proof}
    Hefez, 32 e 44.
\end{proof}

\begin{defi}
    \textbf{(a)} (Intuitiva) Uma matriz escalonada será dita \textit{reduzida} se todo pivô for unitário e se todos os outros elementos da coluna de um pivô forem iguais a 0.

    \textbf{(b)} Uma matriz $A = (a_{ij})_{m \times n}$ será dita \textit{escalonada reduzida} se existir uma sequência de índices $1 \leq b_1 < b_2 < \ldots < b_r \leq n$ tal que $a_{ib_i} = 1$ para todo $1 \leq i \leq r$, $a_{k b_i}$ = 0 para todo $k \neq i$ e $a_{ij} = 0$ para todo $1 \leq j < b_i$. 
\end{defi}

\begin{teo}
    Toda matriz é equivalente por linhas a uma única matriz escalonada reduzida.
\end{teo}

\begin{proof}
    Hefez, 32 e 44. Reginaldo, 68.
\end{proof}

\begin{prop}
    Se $A \in \mathcal{M}_{n \times n}(\R)$ é uma matriz escalonada reduzida e $A \neq \mathrm{I}_n$, então $A$ possui pelo menos uma linha nula.
\end{prop}

\begin{proof}
    Reginaldo, 47.
\end{proof}


\section{Sistemas Lineares}

\begin{defi}
    \textbf{(a)} Uma \textit{equação linear} nas incógnitas $x_1, x_2, \ldots, x_n$ é qualquer equação do tipo
        \[
            a_1 x_1 + a_2 x_2 + \cdots + a_n x_n = b,
        \]
    onde $a_1, a_2, \ldots , a_n, b \in \R$. Cada $a_{i}$ é chamado de \textit{coeficiente}, enquanto $b$ é chamado de \textit{termo independente}. 
    
    \textbf{(b)} O conjunto das soluções de uma equação linear é
        \[
            \{ (x_1, \ldots, x_n) \in \R^n : a_1 x_1 + \cdots + a_n x_n = b \}.
        \]
    
    %Definimos \textit{equação linear} como toda equação do tipo \[ a_{11} x_1 + a_{12} x_2 + \cdots + a_{1n} x_n = b \] nas incógnitas $x_1, x_2, \ldots, x_n$. Os números $a_{11}, a_{12}, \ldots , a_{1n} \in \R$ são denominados \textit{coeficientes}, enquanto o número $b \in \R$ é denominado \textit{termo independente}.
\end{defi}

\begin{defi}
    \textbf{(a)} Definimos \textit{sistema de equações lineares} como todo conjunto finito de equações lineares nas incógnitas $x_1, x_2, \ldots, x_n$. Com $m$ equações, a representação canônica é
        \[
            \left\{
                \begin{matrix}
                    a_{11} x_1 + a_{12} x_2 + \cdots + a_{1n} x_n = b_1 \\
                    a_{21} x_1 + a_{22} x_2 + \cdots + a_{2n} x_n = b_2 \\
                    \vdots \\
                    a_{m1} x_1 + a_{m2} x_2 + \cdots + a_{mn} x_n = b_m
                \end{matrix}
            \right. .
        \]
    Por simplicidade, diremos apenas \textit{sistema linear}, ou ainda, apenas \textit{sistema}, em vez de sistema de equações lineares.

    \textbf{(b)} O conjunto das soluções de um sistema linear é
        \[
            \bigcap_{i=1}^{m} \{ (x_1, \ldots, x_n) \in \R^n : a_{i1} x_1 + \cdots + a_{in} x_n = b_i \}.
        \]    
\end{defi}

\begin{fato}
    Todo sistema linear
        \[
            \left\{
                \begin{matrix}
                    a_{11} x_1 + a_{12} x_2 + \cdots + a_{1n} x_n = b_1 \\
                    a_{21} x_1 + a_{22} x_2 + \cdots + a_{2n} x_n = b_2 \\
                    \vdots \\
                    a_{m1} x_1 + a_{m2} x_2 + \cdots + a_{mn} x_n = b_m
                \end{matrix}
            \right. .
        \]
    pode ser representado matricialmente como $Ax = b $, onde
        \[
            A := \begin{bmatrix}
                a_{11} & a_{12} & \cdots & a_{1n} \\
                    a_{21} & a_{22} & \cdots & a_{2n} \\
                    \vdots & \vdots & \ddots &  \vdots \\
                    a_{m1} & a_{m2} & \cdots & a_{mn}
            \end{bmatrix}_{m \times n}
                \qquad    
            x := \begin{bmatrix}
                x_1 \\ x_2 \\ \vdots \\ x_n
            \end{bmatrix}_{n \times 1}
                \qquad
            b := \begin{bmatrix}
                b_1 \\ b_2 \\ \vdots \\ b_m
            \end{bmatrix}_{m \times 1}.
        \]
    Observe que $A \in \mathcal{M}_{m \times n}(\R)$, $x \in \mathcal{M}_{n \times 1}(\R)$ e $b \in \mathcal{M}_{n \times 1}(\R)$. Trabalharemos, preferencialmente, com a forma matricial dos sistemas lineares, tendo em mente que as abordagens são equivalentes.
\end{fato}

\begin{defi}
    Seja $Ax = b$ um sistema linear como acima. 
        
        \textbf{(a)} Denominamos a matriz $A$ como a \textit{matriz incompleta}, ou ainda, a \textit{matriz}, desse sistema. Ela também é chamada de \textit{matriz de coeficientes}.

        \textbf{(b)} Denominamos a matriz
            \[
                [A | b] := \begin{bmatrix}
                    a_{11} & a_{12} & \cdots & a_{1n} & b_1\\
                    a_{21} & a_{22} & \cdots & a_{2n} & b_2 \\
                    \vdots & \vdots & \ddots &  \vdots & \vdots \\
                    a_{m1} & a_{m2} & \cdots & a_{mn} & b_m
                \end{bmatrix}_{m \times (n+1)}
            \]
        como a \textit{matriz completa}, ou a \textit{matriz aumentada}, desse sistema.
\end{defi}

\begin{defi}
    Um sistema linear é dito \textit{possível} se ele tiver pelo menos uma solução. Se houver mais de uma solução, ele será dito \textit{possível e indeterminado}, no sentido do resultado a seguir; se a solução for única, então ele será dito \textit{possível e determinado}. Por fim, se não houver solução alguma, ele será dito \textit{impossível}.
\end{defi}

\begin{teo}
    Se um sistema linear $Ax=b$ tem (pelo menos) duas soluções distintas, então, na verdade, ele tem infinitas soluções distintas.
\end{teo}

\begin{proof}
    Se $x_1 \neq x_2$ são soluções, então $x_3 := \lambda x_1 + (1 - \lambda) x_2$, para qualquer $\lambda \in \R$, também é uma solução. De fato, basta observar que $Ax_3 = A [\lambda x_1 + (1 - \lambda)x_2] = b$.
\end{proof}

\begin{defi}
    Diremos que dois sistemas de equações lineares são  equivalentes se eles têm o mesmo conjunto solução.
\end{defi}

\begin{prop}
    Os sistemas lineares $Ax = b$ e $Cx=d$ são equivalentes se, e somente se, as matrizes $[A|b]$ e $[C|d]$ são equivalentes por linhas.
\end{prop}

\begin{proof}
    Reginaldo, 32.
\end{proof}

\begin{defi}
    Um sistema de equações lineares em que todos os termos independentes são nulos, isto é,
        \[
            \left\{
                \begin{matrix}
                    a_{11} x_1 + a_{12} x_2 + \cdots + a_{1n} x_n = 0 \\
                    a_{21} x_1 + a_{22} x_2 + \cdots + a_{2n} x_n = 0 \\
                    \vdots \\
                    a_{m1} x_1 + a_{m2} x_2 + \cdots + a_{mn} x_n = 0
                \end{matrix}
            \right. ,
        \]
    ou ainda, em notação matricial,
        \[
            \begin{bmatrix}
                a_{11} & a_{12} & \cdots & a_{1n} \\
                a_{21} & a_{22} & \cdots & a_{2n} \\
                \vdots & \vdots & \ddots &  \vdots \\
                a_{m1} & a_{m2} & \cdots & a_{mn}
            \end{bmatrix}
                \cdot    
            \begin{bmatrix}
                x_1 \\ x_2 \\ \vdots \\ x_n
            \end{bmatrix}
            = \begin{bmatrix}
                0 \\ 0 \\ \vdots \\ 0
            \end{bmatrix},
        \]
    é dito \textit{homogêneo}.
\end{defi}

\begin{prop}
    Todo sistema homogêneo é possível.
\end{prop}

\begin{proof}
    Basta ver que $x = 0_{n \times 1}$ é uma solução, dita \textit{trivial}.
\end{proof}

\begin{teo} \label{teo:sislinhom}
    Todo sistema linear homogêneo em que o número de incógnitas é maior que o número de equações possui uma solução não trivial (e, portanto, possui infinitas soluções).
\end{teo}

\begin{proof}
    Elon, 27
\end{proof}


















\chapter{Espaços Vetoriais}

\section{Espaços e Subespaços Vetoriais}

\begin{defi}
    Uma tripla $(V,+,\cdot)$ é um \textit{espaço vetorial real} se no conjunto $V \neq \emptyset$ existem duas operações, $+ : V \times V \to V$ e $\cdot : \R \times V \to V$, para as quais
        \begin{itemize}
            \item A1: $u + (v + w) = (u + v) + w$ para quaisquer $u,v,w \in V$;
            \item A2: $u + v = v + u$ para quaisquer $u,v \in V$;
            \item A3: existe $0_V \in V$ tal que $u + 0_V = u$ para todo $u \in V$;
            \item A4: para cada $u \in V$ existe $v \in V$ tal que $u+v=0_V$;
            \item M1: $\lambda_1 \cdot (\lambda_2 \cdot u) = (\lambda_1 \cdot \lambda_2) \cdot u$ para quaisquer $u \in V$ e $\lambda_1,\lambda_2 \in \R$;
            \item M2: $(\lambda_1 + \lambda_2) \cdot u = \lambda_1 \cdot u + \lambda_2 \cdot u$ para quaisquer $u \in V$ e $\lambda_1,\lambda_2 \in \R$;
            \item M3: $\lambda_1 \cdot (u + v) = \lambda_1 \cdot u + \lambda_1 \cdot v$ para quaisquer $u,v \in V$ e $\lambda_1 \in \R$;
            \item M4: $1 \cdot u = u$ para todo $u \in V$.
        \end{itemize}
    Os elementos de $V$ são chamados de \textit{vetores}. Para simplificar a notação, e quando não houver perigo de confusão, pomos $0 := 0_V$, $\lambda v := \lambda \cdot v =$ e $-v := w$ se $v + w = 0$. Vamos nos referir ao espaço vetorial $(V, +, \cdot)$ simplesmente como o conjunto $V$.
\end{defi}

\begin{ex}
    \leavevmode
        \begin{enumerate}[leftmargin=*, align=left, label=\textbf{(\alph*)}]
            \item O espaço $\R^n$, com soma e produto por escalar usuais, é um espaço vetorial.
            \item O conjunto $\R^X$ de todas as funções reais $f:X \subseteq \R \to \R$, com soma e produto por escalar usuais, é um espaço vetorial.
            \item O conjunto $\mathcal{P}(\R)$ de todos os polinômios em $x$, com soma e multiplicação por escalar usuais, é um espaço vetorial.
            \item O conjunto $\R_{>0}$ dos números reais positivos, com as operações $x\oplus y:=x \cdot y$ e $\alpha \odot x := x^{\alpha}$, é um espaço vetorial.
        \end{enumerate}
\end{ex}

\begin{proof}
\end{proof}

\begin{prop} \label{prop:basicas}
    Seja $(V, +, \cdot)$ um espaço vetorial. Valem as seguintes afirmações.
        \begin{enumerate}[leftmargin=*, align=left, label=\textbf{(\alph*)}]
            \item (Unicidade)
            \item (Integridade) Para quaisquer $u \in V$ e $\lambda \in \R$, temos
                \begin{enumerate}[label=\roman*.]
                    \item $\lambda \cdot 0_V = 0_V$;
                    \item $0 \cdot u = 0_V$;
                    \item se $\lambda \cdot u = 0_V$, então $\lambda = 0$ ou $u = 0_V$.
                \end{enumerate}
            \item (Sinais) Para quaisquer $u \in V$ e $\lambda \in \R$, temos
                \begin{enumerate}[label=\roman*.]
                    \item $(-1) \cdot u = -u$;
                    \item $-(-u) = u$;
                    \item $(- \lambda) \cdot u = \lambda (-u) = -(\lambda \cdot u)$.
                \end{enumerate}
            \item (Lei do Corte) Para quaisquer $u,v,w \in V$ e $\lambda, \lambda_1, \lambda_2 \in \R$, vale
                \begin{enumerate}[label=\roman*.]
                    \item $u + w = v + w \Rightarrow u = v$;
                    \item $\lambda \neq 0 \land \lambda \cdot u = \lambda \cdot v \Rightarrow u = v$; 
                    \item $v \neq 0_V \land \lambda_1 u = \lambda_2 u \Rightarrow \lambda_1 = \lambda_2$.
                \end{enumerate}
            \item Se $V \neq \{0 \}$, então $V$ é infinito.
        \end{enumerate}
\end{prop}

\begin{proof}
\end{proof}

\begin{defi}
    Um espaço vetorial $(W, +, \cdot)$ é um \textit{subespaço vetorial} de um espaço vetorial $(V, +, \cdot)$ se $W \subseteq V$.
\end{defi}

\begin{teo}
    Uma tripla $(W, +, \cdot)$ é um subespaço vetorial de $(V,+,\cdot)$ se $W \subseteq V$ e $u+v, \lambda \cdot v \in W$ para quaisquer $u,v \in W$ e $\lambda \in \R$.
\end{teo}

\begin{defi} 
    Seja $V$ um espaço vetorial. Um subconjunto $W \subseteq V$ é um \textit{subespaço vetorial} de $V$ se $u + v \in W$ e $\lambda u \in W$ para quaisquer $u, v \in W$ e $\lambda \in \R$.
\end{defi}

\begin{cor} \label{cor:1}
    Sejam $V$ um espaço vetorial. Se $u + \lambda \cdot v \in W$ para quaisquer $u, v \in W \subseteq V$ e $\lambda \in \R$, então $W$ é um subespaço vetorial de $V$.
\end{cor}

\begin{proof}
    Particularmente para $\lambda = 1$, temos $u + v \in W$. Particularmente para $v = 0_V$, temos $\lambda u \in W$. \itemproof
\end{proof}

\begin{prop}
    Se $V$ é um espaço vetorial e $W \subseteq V$ é um subespaço, então $W$ é um espaço vetorial.
\end{prop}

\begin{proof}
    Para provar que $W$ é um espaço vetorial, precisamos verificar que (i) existem operações $+$ e $\cdot$ bem definidas em $W$ e que (ii) essas operações satisfazem as propriedades A1--A4 e M1--M4 de um espaço vetorial.
        \begin{enumerate}[label=\roman*.]
            \item Como esperado, as operações $+$ e $\cdot$ de $W$ serão as mesmas de $V$: como $V$ é um espaço vetorial, as operações $+$ e $\cdot$, bem definidas em $V$, também estão bem definidas em qualquer subconjunto não vazio de $V$; em particular, também em $W$. Por exemplo, podemos tomar $w_1, w_2 \in W$ e considerar sua soma, $w_1 + w_2$, porque $W \subset V \Rightarrow w_1, w_2 \in V$ e, em $V$, a operação $+$ está bem definida.
            \item O fechamento das operações $+$ e $\cdot$ em $W$ garantem, de cara, A1--A2 e M1--M4. Falta, então, provar A3 e A4. Dado $w \in W$, temos que $0_V = 0 \cdot w \in W$; com isso, tomando $0_W := 0_V$, teremos um elemento neutro de $+$ em $W$ porque $w + 0_W = w + 0_V = w$. Por fim, dado $w \in W$, $-w = (-1) \cdot w \in W$, e então o oposto aditivo de $w \in W$ está em $W$. 
        \end{enumerate}
    Logo, $W$ é um espaço vetorial, como queríamos.
\end{proof}

\begin{ex}
    \leavevmode
        \begin{enumerate}[leftmargin=*, align=left, label=\textbf{(\alph*)}]
            \item Todo espaço vetorial $V$ tem pelo menos dois subespaços vetoriais, dito \textit{triviais}: $\{0_V\}$ e o próprio $V$.
            \item O conjunto $\mathcal{P}_n(\R)$ dos polinômios de grau menor ou igual a $n$, juntamente com o polinômio nulo, é um subespaço de $\mathcal{P}(\R)$.
        \end{enumerate}
\end{ex}

\begin{proof}
    \textbf{(a)} Elon, 10. Reginaldo, 26.
\end{proof}

\begin{prop}[Interseção de subespaços]
    Se $W_1$ e $W_2$ são dois subespaços vetoriais de um espaço vetorial $V$, então
        \begin{enumerate}[leftmargin=*, align=left, label=\textbf{(\alph*)}]
            \item $W_1 \cap W_2$ é um subespaço vetorial;
            \item $W_1 \cup W_2$ é um subespaço vetorial se, e somente se, $W_1 \subset W_2$ ou $W_2 \subset W_1$.
        \end{enumerate}
\end{prop}

\begin{proof}
    \leavevmode
        \begin{enumerate}[leftmargin=*, align=left, label=\textbf{(\alph*)}]
            \item Para quaisquer $u,v \in W_1 \cap W_2$, temos, em particular, $u,v \in W_1$ e $u,v \in W_2$. Como $W_1$ e $W_2$ são espaços vetoriais, temos que $u+\lambda v \in W_1$ e $u+ \lambda v \in W_2$ para todo $\lambda \in \R$, donde $u+\lambda v \in W_1 \cap W_2$. Logo $W_1 \cap W_2$ é um subespaço vetorial de $V$. \itemproof
            \item Por contradição, suponha que existam $w_1 \in W_1$ e $w_2 \in W_2$ tais que $w_1 \notin W_2$ e $w_2 \notin W_1$. Como, por hipótese, $W_1 \cup W_2$ é um subespaço vetorial, $ w: =w_1+w_2 \in W_1 \cup W_2$, isto é, $w \in W_1$ ou $w \in W_2$. 
                \begin{itemize} 
                    \item Se $w \in W_1$, então $w + (- w_1) = w_2 \in W_1$, absurdo!
                    \item Se $w \in W_2$, então $w + (- w_2) = w_1 \in W_2$, absurdo!
                \end{itemize}
            Logo, a prova da ida está completa. A volta é evidente. Logo, a prova está completa. 
        \end{enumerate} 
\end{proof}

\begin{cor} \label{prop:intsub}
    Seja $V$ um espaço vetorial e $I$ um conjunto de índices. Se para cada $\lambda \in I$ o conjunto $W_\lambda \subseteq V$ for um subespaço vetorial de $V$, então $\bigcap_{\lambda \in I} W_{\lambda}$ é ainda um subespaço vetorial de $V$.
\end{cor}

\begin{proof}
    Elon, 10.
\end{proof}

\begin{defi}
    Sejam $W_1$ e $W_2$ subespaços vetoriais de um espaço vetorial $V$.
        \begin{enumerate}[leftmargin=*, align=left, label=\textbf{(\alph*)}]
            \item (Soma de subespaços) Definimos $W_1 + W_2$ como sendo o conjunto de todos os vetores de $V$ que são soma de um elemento de $W_1$ com um elemento de $W_2$, isto é, 
                \[ 
                    W_1 + W_2 := \{v \in V : \exists w_1 \exists w_2 (w_1 \in W_1 \land w_2 \in W_2 \land v = w_1 + w_2\}.
                \]
            \item (Soma direta) Diremos que $W_1 \oplus W_2 := W_1 + W_2$ é a \textit{soma direta} de $W_1$ e $W_2$ se $W_1 \cap W_2 = \left\{0_V\right\}$.
        \end{enumerate}
\end{defi}

\begin{prop} 
    Nos termos da definição acima, $W_1 + W_2$ é um subespaço vetorial de $V$.
\end{prop}

\begin{proof}
    Veja inicialmente que $W_1, W_2 \subset W_1 + W_2$.
        \begin{enumerate}[label=\roman*.]
            \item Se $u,v \in W_1 + W_2$, então existem $u_1, v_1 \in W_1$ e $u_2, v_2 \in W_2$ tais que $u = u_1 + u_2$ e $v = v_1 + v_2$. Com isso,
                \[ 
                    u + v = (u_1 + u_2) + (v_1 + v_2) = \underbrace{(u_1 + v_1)}_{\in W_1} + \underbrace{(u_2+v_2)}_{\in W_2},
                \]
            de modo que $u+v$ é a soma de um vetor de $W_1$ com um vetor de $W_2$, isto é, $u + v \in W_1 + W_2$.
            \item Se $u \in W_1 + W_2$, então existem $u_1 \in W_1$ e $u_2 \in W_2$ tais que $u=u_1+u_2$. Com isso, para qualquer $\lambda \in \R$,
                \[
                    \lambda u = \lambda (u_1 + u_2) = \underbrace{\lambda u_1}_{\in W_1} + \underbrace{\lambda u_2}_{\in W_2},
                \]
             de modo que $\lambda u$ é a soma de um vetor de $W_1$ com um vetor de $W_2$, isto é, $\lambda u \in W_1 + W_2$.
        \end{enumerate}
        Assim,  $W_1 + W_2$ é um subespaço vetorial de $V$.
\end{proof}

\begin{teo}
    Sejam $W_1$ e $W_2$ dois subespaços vetoriais de um espaço vetorial $V$. Teremos $V = W_1 \oplus W_2$ se, e somente se, para cada $v \in V$ existirem únicos $w_1 \in W_1$ e $w_2 \in W_2$ tais que $v = w_1 + w_2$.
\end{teo}

\begin{proof}
    ($\Rightarrow$) Se $V = W_1 \oplus W_2$, então temos a existência da decomposição. Provemos, então, sua unicidade. Dado $v \in V$, sejam $v_1,w_1 \in W_1$ e $v_2, w_2 \in W_2$ tais que $v = v_1 + v_2 = w_1 + w_2$. Somando $[(- w_1) + (- v_2)]$, vem
            \begin{align*} 
                v = v_1 + v_2 &= w_1 + w_2 \\
                v_1 + v_2 + [(-w_1) + (-v_2)] &= w_1 + w_2 + [(-w_1) + (-v_2)] \\  
                \underbrace{v_1 + \left(-w_1\right)}_{\in W_1} &= \underbrace{w_2 + \left(-v_2\right)}_{\in W_2}. 
            \end{align*}
    Como $W_1 \cap W_2 = \{0\}$, temos então que $v_1=w_1$ e $v_2=w_2$, como queríamos.

    ($\Leftarrow$) Segue da hipótese que $V = W_1 + W_2$; provemos, então, que $W_1 \cap W_2 = \{0\}$. Como $W_1 \cap W_2 \neq \emptyset$ (pelo menos $0 \in W_1 \cap W_2$), tome $v \in W_1 \cap W_2$, para o qual, por hipótese, existem únicos $w_1 \in W_1$ e $w_2 \in W_2$. Com isso, temos
        \[
            v = \underbrace{w_1}_{\in W_1} + \underbrace{w_2}_{\in W_2} = \underbrace{(w_1 + v)}_{\in W_1} + \underbrace{(w_2 + (-v))}_{\in W_2},
        \]
    e como a decomposição é única, temos $w_1 = w_1 + v$ e $w_2 = w_2 - v$, donde $v=0$. Logo, $W_1 \cap W_2 = \{0\}$, como queríamos.
\end{proof}

\begin{ex}
    O conjunto das funções reais pares, 
        \[ 
            W_1 = \{f \in \R^{\mathcal{X}} : \forall x (x \in \mathcal{X}  \subseteq \R \rightarrow f(-x) = f(x) ) \},
        \]
    bem como o conjunto das funções reais ímpares,
        \[
            W_2 = \{f \in \R^{\mathcal{X}}:  \forall x (x \in \mathcal{X}  \subseteq \R \rightarrow f(-x) = - f(x) ) \}
        \]
    são subespaços de $\R^{\mathcal{X}}$. E ainda, temos que $\R^{\mathcal{X}} = W_1 \oplus W_2$.
\end{ex}

\begin{proof}
    Reginaldo, 35.
\end{proof}

\section{Combinações Lineares e Geradores}

\begin{defi}
    Seja $(V, +, \cdot)$ um espaço vetorial. Um vetor $u \in V$ é uma \textit{combinação linear} dos vetores $u_1, u_2 , \ldots, u_n \in V$ se existem escalares $\lambda_1, \lambda_2, \ldots, \lambda_n \in \R$ para os quais 
        \[
            u = \lambda_1 u_1 + \lambda_2 u_2 + \cdots + \lambda_n u_n = \sum_{i=1}^{n} \lambda_i u_i.
        \]
\end{defi}

\begin{teo}
    Seja $V$ um espaço vetorial e $\emptyset \neq S \subseteq V$ um subconjunto de vetores de $V$. O conjunto de todas as combinações lineares dos vetores de $S$, que denotaremos por $[S]$, é um subespaço vetorial de $V$.
\end{teo}

\begin{proof}
    Se $u, v \in [S]$, então, por definição, existem vetores e escalares
        \begin{align*}
            u_1, u_2, \ldots, u_n, v_1, v_2, \ldots, v_m \in S \\
            \lambda_1, \lambda_2, \ldots, \lambda_n, \alpha_1, \alpha_2, \ldots, \alpha_m \in \R
        \end{align*}
    para os quais $\ds u = \sum_{i=1}^{n} \lambda_i u_i$ e $\ds  v = \sum_{i=1}^{m} \alpha_i v _i$. Com isso, para qualquer $\lambda \in \R$,
        \[
            u + \lambda v = \sum_{i=1}^{n} \lambda_i  u_i + \sum_{i=1}^{m} (\lambda \alpha_i) v_i,  
        \]
    de modo que $u+\lambda v$ é uma combinação linear de vetores de $S$, isto é, $u + \lambda v \in [S]$. Logo, pelo resultado \eqref{cor:1}, $[S]$ é um subespaço vetorial de $V$.
\end{proof}

\begin{defi}
    Sejam $V$ um espaço vetorial e $\emptyset \neq S \subseteq V$ um subconjunto de vetores de $V$.
        \begin{enumerate}[leftmargin=*, align=left, label=\textbf{(\alph*)}]
            \item Diremos que $[S]$ é o \textit{subespaço gerado} por $S$, ou que $S$ \textit{gera} $[S]$. Diremos, ainda, que os elementos de $S$ são \textit{geradores} de $[S]$.
            \item Se $S$ é finito e $[S] = V$, diremos que $V$ é \textit{finitamente gerado} e que $S$ é um \textit{conjunto de geradores} para (ou de) $V$.
            \item Convenciona-se pôr $[\emptyset]= \{0\}$.
        \end{enumerate}
\end{defi}

\begin{prop}
    Sejam $V$ um espaço vetorial e $\emptyset \neq S,T \subseteq V$ subconjuntos de vetores de $V$. Valem as seguintes afirmações.
        \begin{enumerate}[leftmargin=*, align=left, label=\textbf{(\alph*)}]
            \item $S \subseteq [S]$.
            \item $[[S]] = [S]$.
            \item Se $S$ é um subespaço vetorial, então $[S]=S$.
            \item $S \subset T \Rightarrow [S] \subset [T]$.
            \item $[S \cup T ] = [S] + [T]$.
        \end{enumerate}
\end{prop}  
        
\begin{proof}
    \begin{enumerate}[leftmargin=*, align=left, label=\textbf{(\alph*)}]
        \item Se $u \in S$, então $u = 1 u \in [S]$. \itemproof

        \item Pelo item anterior, $[S] \subseteq [[S]]$. Agora, se $u \in [[S]]$, então $u$ é uma combinação linear de vetores de $[S]$, que por sua vez são combinações lineares de vetores de $[S]$, de modo que $u \in [S]$. Assim, $[[S]] \subseteq [S]$, de modo que $[[S]] = [S]$. \itemproof

        \item Se $u \in [S]$, então $u$ é uma combinação linear de elementos de $S$; como $S$ é um subespaço vetorial, temos então $u \in S$, de modo que $[S] \subseteq S$. Como $S \subseteq [S]$, temos então $[S] = S$. \itemproof
        
        \item Se $u \in S$, então existem vetores $u_1, \ldots, u_n \in S$ e escalares $\lambda_1, \ldots, \lambda_n \in \R$ tais que $u = \lambda_1 u_1 + \cdots + \lambda_n u_n$. Como $u_1, \ldots, u_n \in S$, por ser $S \subseteq T$ vem $u_1, \ldots, u_n \in T$, de modo que $\lambda_1 u_1 + \cdots + \lambda_n u_n \in T$. \itemproof

        \item 
    \end{enumerate}
\end{proof}

\section{Dependência e Independência Linear}

\begin{defi}
    Seja $(V, +, \cdot)$ um espaço vetorial.
        \begin{enumerate}[leftmargin=*, align=left, label=\textbf{(\alph*)}]
            \item Os vetores $u_1, u_2, \ldots, u_n \in V$ são \textit{linearmente independentes} (L.I.) se a equação
                \[
                    \lambda_1 u_1 + \lambda_2 u_2 + \cdots + \lambda_n u_n = 0
                \]
            possuir somente a solução trivial $\lambda_1 = \lambda_2 = \cdots = \lambda_n = 0$. Caso contrário, ou seja, se existir pelo menos uma solução com pelo menos um $\lambda_i \neq 0$, diremos que os vetores $u_1, u_2, \ldots, u_n \in V$ são \textit{linearmente dependentes} (L.D.).

            \item Um subconjunto finito $S \subseteq V$\footnote{Sendo $S$ finito, só pode ser $S = V$ se for $V = \{ 0\}$.} é L.D. (L.I.) se os vetores de $S$ são L.D. (L.I.).
                \begin{enumerate}[label=\roman*.]
                    \item Um subconjunto infinito infinito $S \subseteq V$ é L.D. se pelo menos um subconjunto finito de $S$ é L.D..
                    \item Um subconjunto infinito $S \subseteq V$ é L.I. se todo subconjunto finito de $S$ é L.I..
                \end{enumerate}
        \end{enumerate}
\end{defi}

\begin{prop} \label{prop:deplin}
    Sejam $(V,+,\cdot)$ um espaço vetorial e $S \subseteq V$ não vazio. Valem as seguintes afirmações.
        \begin{enumerate}[leftmargin=*, align=left, label=\textbf{(\alph*)}]
            \item Se $|S| = 1$ e $0_V \notin S$, então $S$ é L.I..
            
            \item $S$ é L.D. se, e somente se, pelo menos um vetor de $S$ é combinação linear de outros vetores de $S$. 
            
            Equivalentemente, pela contrapositiva, $S$ é L.I. se, e somente se, nenhum vetor de $S$ é combinação linear de outros vetores de $S$.
            
            \item Se $0_V \in S$, então $S$ é L.D.. 
            
            Equivalentemente, pela contrapositiva, se $S$ é L.I., então $0_V \notin S$.
            
            \item Suponha $S$ L.I. e seja $u \in V$. Se $S \cup \{ u\}$ é L.D., então $u \in [S]$.
            
            Equivalentemente, pela contrapositiva, se $u \notin [S]$, então $S \cup \{ u\}$ é L.I..
            
            \item Sejam $S_1,S_2 \neq \emptyset$ subconjuntos de $V$ tais que $S_1 \subseteq S_2$. Temos que
                \begin{enumerate}[label=\roman*.]
                    \item se $S_1$ é L.D., então $S_2$ também é L.D..
                    \item se $S_2$ é L.I., então $S_1$ também é L.I.. 
                \end{enumerate}
                
            \item Se $S$ é finito e L.I., então cada vetor $u \in [S]$ se escreve de maneira única como combinação linear de vetores de $S$. %Mais especificamente, se $S$ tem, digamos, $n$ vetores, então, indexando os vetores de $S$ em $[n]$ de modo que $S = \{ v_1, v_2, \ldots, v_n \}$, para cada $v \in [S]$ são únicos os $\lambda_i \in \R$, com $i \in [n]$, de modo que $v = \lambda_1 v_1 + \lambda_2 v_2 + \cdots + \lambda_n v_n$.

            \item  Se $u \in S$ é tal que $u \in [S \setminus \{u \}]$, então $[S] = [S \setminus \{u \}]$.
        \end{enumerate}
\end{prop}

\begin{proof}
    \leavevmode
        \begin{enumerate}[leftmargin=*, align=left, label=\textbf{(\alph*)}]
            \item Sendo $S = \{u\}$, se $\lambda \cdot u = 0_V$ então $\lambda \neq 0$ já que $u \neq 0_V$. Logo $S$ é L.I.. \itemproof
            
            \item ($\Rightarrow$) Se $S$ é L.D., então existem $v_1, v_2, \ldots, v_n \in S$ para os quais vale
                \[
                    \lambda_1 v_1 + \lambda_2 v_2 + \cdots + \lambda_n v_n = 0
                \]
            com $\lambda_i \neq 0$ para pelo menos um $i \in [n]$. Suponha, sem perda de generalidade, que $i=1$, isto é, que $\lambda_1 \neq 0$; com isso,
                \[
                    v_1 = \left( \dfrac{-\lambda_2}{\lambda_1} \right) v_2 + \left( \dfrac{-\lambda_3}{\lambda_1} \right) v_3 + \cdots + \left( \dfrac{-\lambda_n}{\lambda_1} \right) v_n,
                \]
            de modo que $v_1 \in S$ é combinação linear de $v_2, v_3, \ldots, v_n \in S$.

            ($\Leftarrow$) Se $v = \lambda_1 v_1 + \cdots + \lambda_n v_n$, com $v_1, \ldots, v_n \in V$ e $\lambda_1, \ldots, \lambda_n \in \R$, então
                \[
                    \lambda_1 v_1 + \cdots + \lambda_n v_n + (-1)v = 0;
                \]
            como $-1 \neq 0$, temos, por definição, que $S$ é L.D.. \itemproof
        \end{enumerate}
    
    \textbf{(c)} Basta ver que $0_V$ é combinação linear de quaisquer $v_1, v_2, \ldots, v_n \in V$: temos $0_V = 0v_1 + 0v_2 + \cdots + 0v_n$. Assim, pelo item anterior, $S$ é L.D.. \itemproof

    \textbf{(d)}

    \textbf{(e)}

    \textbf{(f)}

    \textbf{(g)} Esta afirmação nos diz que se um vetor de $S$ é combinação linear de outros vetores de $S$, então ele pode ser removido do subespaço gerado por $S$ sem alterá-lo.
\end{proof}

\section{Base e Dimensão}

\begin{lem} \label{alr.lem:fundamental}
    Seja $(V,+,\cdot)$ um espaço vetorial. Se um subconjunto finito $S \subseteq V$ é L.I., então todo subconjunto $T \subseteq [S]$ tal que $|T| = |S|+1$ é L.D..
\end{lem}

\begin{proof}
    Façamos indução em $|S|$. Se $|S| = 1$, então $S = \{u_1\}$, com $u_1 \neq 0_V$ já que $S$ é L.I.. Tomando $v_1, v_2 \in [S]$ distintos, existem escalares $\lambda_1, \lambda_2 \in \R$ tais que $v_1 = \lambda_1 u_1$ e $v_2 = \lambda_2 u_1$. Multiplicando $v_1$ por $\lambda_2$ e $v_2$ por $\lambda_1$, obtemos
        \[
            \lambda_2 v_1 - \lambda_1 v_2 = 0_V,
        \]
    isto é, uma combinação linear não trivial dos vetores de $T = \{v_1, v_2\}$, de modo que eles são L.D.. Isso completa a base de indução. Suponha então, por hipótese de indução, que o resultado vale para subconjuntos de $V$ de $n-1$ vetores. Agora, sejam $S = \{u_1, \ldots, u_n\} \subsetneq V$ e $T = \{v_1, \ldots, v_n, v_{n+1}\} \subseteq [S]$. Provemos que $T$ é L.D.. Como $v_i \in [S]$ para todo $i \in [n+1]$, existem escalares $a_{ij} \in \R$, com $i \in [n+1]$ e $j \in [n]$, tais que
        \[
            v_i = a_{i1} u_1 + \cdots + a_{in} u_n
        \]
    para todo $i \in [n+1]$.
        \begin{itemize}
            \item $a_{i1} = 0$ para todo $i \in [n+1]$. Nesse caso, temos
                \[
                    v_i = a_{i2} u_2 + \cdots + a_{in} u_n
                \]
            para todo $i \in [n+1]$, donde $T \subsetneq [S \setminus \{u_1\}]$, e como $|S \setminus \{u_1\}| = n-1$, segue da hipótese de indução que todo subconjunto de $T$ com $n$ vetores é L.D., de modo que $T$ é L.D..
            \item $a_{i1} \neq 0$ para algum $i \in [n+1]$. Nesse caso, suponha, sem perda de generalidade, que $a_{11} \neq 0$. Definindo
                \[
                    w_i := \dfrac{a_{i1}}{a_{11}} \cdot v_1 - v_i
                \]
            para cada $i \in [n+1] \setminus \{1\}$, fazendo as contas obtemos
                \[
                    w_i = \sum_{j=2}^n \left[ \left( \dfrac{a_{i1}}{a_{11}} a_{1j} - a_{ij} \right) u_j \right].
                \]
            Com isso, vemos que cada $w_i$ é uma combinação linear dos vetores $u_2, \ldots, u_n$, de modo que $T' := \{ w_2, \ldots, w_n, w_{n+1} \} \subsetneq [S \setminus \{u_1\}]$. Como $|S \setminus \{u_1\}| = n-1$ e $|T'| = n$, pela hipótese de indução temos que $T'$ é L.D., de modo que existem $n$ escalares, $\lambda_2, \ldots, \lambda_{n+1} \in \R$, com algum $\lambda_i \neq 0$, tais que
                \[
                    \sum_{i=2}^{n+1} \lambda_i w_i = 0_V.
                \]
            Daí, obtemos
                \[
                     \sum_{i=2}^{n+1} \left( \lambda_i \dfrac{a_{i1}}{a_{11}} \right) v_1 - \sum_{i=2}^{n+1} \lambda_i v_i = 0_V,
                \]
            uma combinação linear não trivial de $T$, de modo que $T$ é L.D..
        \end{itemize}
    Com isso, vemos que se o resultado vale para subconjuntos de $V$ com $n-1$ vetores, então ele também vale para subconjuntos de $V$ com $n$ vetores. Isso completa o passo indutivo e, portanto, completa a prova. \itemproof
\end{proof}

\begin{cor}
    Seja $(V,+,\cdot)$ um espaço vetorial. Se um subconjunto finito $S \subseteq V$ é L.I., então todo subconjunto $T \subseteq [S]$ tal que $|T| \geq |S|+1$ é L.D..
\end{cor}

\begin{defi}
    Seja $(V, +, \cdot)$ um espaço vetorial. Um subconjunto $\mathcal{B} \subsetneq V$ é uma \textit{base} de $V$ se $\mathcal{B}$ é L.I. e $[\mathcal{B}] = V$.
\end{defi}

\begin{teo}[Completamento]
     Todo espaço vetorial finitamente gerado possui uma base. 
\end{teo}

\begin{proof}
    Seja $(V,+,\cdot)$ um espaço vetorial finitamente gerado. Se $V = \{0\}$, então $\emptyset$ é uma base de $V$, já que os vetores de $\emptyset$ são L.I. por vacuidade e convencionamos pôr $[\emptyset] = \{0\}$. Suponha, então, $V \neq \{ 0\}$. Como $V$ é finitamente gerado, existe um subconjunto finito $S := \{v_1, v_2, \ldots, v_n\} \subsetneq V$ que gera $V$. Se $S$ for L.I., então $S$ será uma base de $V$. Se $S$ for L.D., então vale
        \[
            \lambda_1 v_1 + \lambda_2 v_2 + \cdots + \lambda_n v_n = 0,
        \]
    sendo $\lambda_i \neq 0$ para pelo menos um $i \in [n]$. Suponha, sem perda de generalidade, que $i=n$, isto é, que $\lambda_n \neq 0$; com isso,
        \[
            v_n = \left( \dfrac{-\lambda_1}{\lambda_n} \right) v_1 + \left( \dfrac{-\lambda_2}{\lambda_n} \right) v_2 + \cdots + \left( \dfrac{-\lambda_{n-1}}{\lambda_n} \right) v_{n-1},
        \]
    de modo que $v_n$ é uma combinação linear de $v_1, v_2, \ldots, v_{n-1} \in S$. Pelo item (g) de \eqref{prop:deplin}, removemos $v_n$ e obtemos $[S \setminus \{v_n\}] = [S] = V$. Repetindo esse processo (em um número finito de vezes, digamos, $m$, já que $V$ é finitamente gerado) eventualmente obteremos um subconjunto $L.I.$ de $S$ com $n-m$ vetores que continua gerando $V$; esse subconjunto vem a ser, então, a base de $V$. \itemproof
\end{proof}

\begin{proof}
    Alternativamente, suponha que $S$ é L.D.. Como $V \neq \{0\}$, existe $i \in [n]$ tal que $v_i \neq 0$. Suponha, sem perda de generalidade, que $i=1$, isto é, que $v_1 \neq 0$; se todo $v_i$, com $i \in [n] \setminus \{1\}$, puder ser escrito como combinação linear de $v_1$, então $V = [v_1]$ e $\{ v_1\}$ é uma base de $V$. Se isso não ocorre, então existe algum $v_i$, com $i \in [n] \setminus \{1\}$, que não pode ser escrito como combinação linear de $v_1$. Suponha, sem perda de generalidade, que $i=2$; se todo $v_i$, com $i \in [n] \setminus \{1,2\}$, puder ser escrito como combinação linear de $v_1$ e $v_2$, então $V = [v_1, v_2]$ e $\{ v_1, v_2\}$ é uma base de $V$. Repetindo esse processo (em um número finito de vezes, já que $V$ é finitamente gerado) eventualmente obteremos um subconjunto $L.I.$ de $S$ que gera $V$; esse subconjunto vem a ser, então, a base de $V$. \itemproof
\end{proof}

\begin{teo}[Invariância] \label{alr.teo:bases}
    Seja $V$ um espaço vetorial finitamente gerado. Se $\mathcal{B}_1$ e $\mathcal{B}_2$ são bases de $V$, então $|\mathcal{B}_1| = |\mathcal{B}_2|$.
\end{teo}

\begin{proof}
    Como $\mathcal{B}_1$ é L.I. e $[\mathcal{B}_1] = V$, pelo lema \eqref{alr.lem:fundamental} temos que $|\mathcal{B}_2| \leq |\mathcal{B}_1|$ já que $\mathcal{B}_2 \subsetneq V = [\mathcal{B}_1]$. De fato, se fosse $|\mathcal{B}_2| > |\mathcal{B}_1|$, pelo lema \eqref{alr.lem:fundamental} $\mathcal{B}_2$ seria L.D., o que contradiz a hipótese. Analogamente, como $\mathcal{B}_2$ é L.I. e $[\mathcal{B}_2] = V$, temos que $|\mathcal{B}_1| \leq |\mathcal{B}_2|$. Sendo $|\mathcal{B}_2| \leq |\mathcal{B}_1|$ e $|\mathcal{B}_1| \leq |\mathcal{B}_2|$, temos que $|\mathcal{B}_1| = |\mathcal{B}_2|$, conforme afirmado. \itemproof
\end{proof}

\begin{defi}
    A \textit{dimensão} de um espaço vetorial finitamente gerado $(V, +, \cdot)$ é o número de vetores de qualquer uma de suas bases. Mais especificamente, se $\mathcal{B}$ é uma base de $V$, a dimensão de $V$ é definida como $\dim V := |\mathcal{B}|$.
\end{defi}

\begin{teo}
    Seja $(V,+,\cdot)$ um espaço vetorial de dimensão finita $n>0$.

    \textbf{(a)} Todo subconjunto de $V$ com $n$ vetores L.I. é uma base de $V$.

    \textbf{(b)} Todo subconjunto de $V$ com $n$ vetores que gera $V$ é uma base de $V$.

    \textbf{(c)} Todo subconjunto de $V$ que gera $V$ tem pelo menos $n$ elementos.

    \textbf{(d)} Todo subconjunto de $V$ com $m < n$ vetores não é uma base de $V$.

    \textbf{(e)} Todo subconjunto de $V$ com $m < n$ vetores L.I. pode ser completado para formar uma base de $V$.

    \textbf{(f)} Se $W$ é um subespaço vetorial de $V$, então $W$ tem dimensão finita e $\dim W \leq \dim V$. Em particular, se $\dim W = \dim V$, então $W = V$.

    \textbf{(g)} Se $W_1$ e $W_2$ são subespaços vetoriais de $V$, então
        \[
            \dim{(W_1 + W_2)} = \dim{W_1} + \dim{W_2} - \dim{(W_1 \cap W_2)}.
        \]
\end{teo}

\begin{proof}
    \textbf{(a)} Reginaldo, 88.

    \textbf{(b)} Reginaldo, 88.

    \textbf{(c)} Reginaldo, 88.

    \textbf{(d)} Zani, 44.

    \textbf{(e)} Reginaldo, 88. Zani, 48.

    \textbf{(f)} Reginaldo, 87.

    \textbf{(g)} Reginaldo, 93. Zani, 49.
\end{proof}

\begin{defi}
    Seja $V$ um espaço vetorial de dimensão finita $n>0$ e $\mathcal{B} := \{v_1, v_2, \ldots, v_n\} $ uma base ordenada (indexada) de $V$. Para cada $v \in V$, diremos que os únicos $\lambda_1, \lambda_2, \cdots, \lambda_n \in \R$ tais que $v = \lambda_1 v_1 + \lambda_2 v_2 + \cdots + \lambda_n v_n$ são as \textit{coordenadas} de $v$ com relação à base $\mathcal{B}$ e denotamos isso por
        \[
            v:= 
                \begin{bmatrix}
                    \lambda_1 \\
                    \lambda_2 \\
                    \vdots \\
                    \lambda_n
                \end{bmatrix}_{\mathcal{B}} :=
                \begin{bmatrix}
                    \lambda_1 \\
                    \lambda_2 \\
                    \vdots \\
                    \lambda_n
                \end{bmatrix}_{v_1, v_2, \ldots, v_n}.
        \]
\end{defi}

\begin{teo}
    (Mudança de Base) Seja $(V, +, \cdot)$ um espaço vetorial de dimensão finita $n>0$ e $v \in V$. Sejam $\mathcal{B}$ e $\mathcal{C}$ duas bases de um espaço vetorial de dimensão finita $(V,+, \cdot)$ e $v \in V$. 
\end{teo}











\chapter{Transformações Lineares}

\begin{defi}
    Sejam $(V, +, \cdot)$ e $(W, +, \cdot)$ espaços vetoriais.
        \begin{enumerate}[leftmargin=*, align=left, label=\textbf{(\alph*)}]
            \item Uma \textit{transformação linear} de $(V, +, \cdot)$ em $(W,+,\cdot)$ é uma função $T : V \to W$ tal que $T(u + v) = T(u) + T(v)$ e $T(\lambda \cdot u) = \lambda \cdot T(u)$ para quaisquer $u, v \in V$ e $\lambda \in \R$.
            \item Um \textit{operador linear} é uma transformação linear $T : V \to V$.
        \end{enumerate}
\end{defi}

\begin{ex}
    Sejam $(V, +, \cdot)$ e $(W, +, \cdot)$ espaços vetoriais.
        \begin{enumerate}[leftmargin=*, align=left, label=\textbf{(\alph*)}]
            \item A função $0 : V \to W$ definida por $0(v) = 0_W$ para todo $v \in V$ é uma transformação linear, chamada de \textit{transformação nula}.
            \item A função $I_V : V \to V$ definida por $I_V(v) = v$ para todo $v \in V$ é um operador linear, chamada de \textit{transformação identidade}. 
        \end{enumerate}
\end{ex}

\begin{prop}
    Uma função $T : V \to W$ é uma transformação linear se, e somente se,
        \[
            T\left(\sum_{i=1}^{n} \lambda_i v_i \right) = \sum_{i=1}^{n} \lambda_i T(v_i)
        \]
    para quaisquer $n \in \N$, $v_1, \ldots, v_n \in V$ e $\lambda_1, \ldots, \lambda_n \in \R$.
\end{prop}

\begin{proof}
    
\end{proof}

\begin{prop}
    Seja $T : V \to W$ uma transformação linear. Valem as seguintes afirmações.
        \begin{enumerate}[leftmargin=*, align=left, label=\textbf{(\alph*)}]
            \item $T(0_V) = 0_W$.
        \end{enumerate}
\end{prop}

\begin{defi}
    Sejam $V$ e $W$ espaços vetoriais. O conjunto de todas as transformações lineares de $V$ em $W$ é denotado por $\mathcal{L}(V,W)$
\end{defi}

\begin{defi}
    Seja $T : V \to W$ uma transformação linear.
        \begin{enumerate}[leftmargin=*, align=left, label=\textbf{(\alph*)}]
            \item O \textit{núcleo} de $T$ é definido como
                \[
                    \ker{T} := \{ v \in V : T(v) = 0_W \}.
                \]
            \item A \textit{imagem} de $T$ é definida como
                \[
                    \Im{T} := \{ w \in W : \exists v(v \in V \land T(v) = w) \}.
                \]
        \end{enumerate}
\end{defi}

\begin{teo}
    Seja $T : V \to W$ uma transformação linear.
        \begin{enumerate}[leftmargin=*, align=left, label=\textbf{(\alph*)}]
            \item $\ker{T}$ é um subespaço vetorial de $V$.
            \item $\Im{T}$ é um subespaço vetorial de $W$.
        \end{enumerate}
\end{teo}

\section{Matrizes}

O conjunto $\mathcal{M}_{m \times n} \left(\R \right)$ de todas as matrizes $m \times n$ com entradas reais, com soma e produto por escalar usuais (vistas no capítulo 1), é um espaço vetorial.

\begin{ex}
    \textbf{(a)} Dado $(a_1, a_2, \ldots, a_n) \in \R^n$, temos que 
        \[
            W := \left\{(x_1, x_2, \ldots, x_n ) \in \R^n : a_1 x_1 + a_2 x_2 + \cdots + a_n x_n = 0 \right\}
        \] 
    é um subespaço de $\R^n$. No caso desinteressante em que $a_i = 0$ para todo $i \in [n]$, o subespaço $W$ é todo o $\R^n$. Se, por outro lado, existe pelo menos um $i \in [n]$ tal que $a_i \neq 0$, diremos que $W$ é um \textit{hiperplano} que passa pela origem.

    \textbf{(b)} Seja $m \in \N$. Para cada $i \in [m]$, sendo $(a_{i1}, a_{i2}, \ldots, a_{in}) \in \R^n$, pelo item anterior temos que cada
        \[
            W_i := \left\{(x_1, x_2, \ldots, x_n ) \in \R^n : a_{i1} x_1 + a_{i2} x_2 + \cdots + a_{in} x_n = 0 \right\}
        \]
    é um subespaço vetorial de $V$. Pela proposição \eqref{prop:intsub}, temos que $W := W_1 \cap W_2 \cap \cdots \cap W_m$ é ainda um subespaço vetorial de $V$, que é exatamente o conjunto das soluções do sistema linear homogêneo
        \[
            \left\{
                \begin{matrix}
                    a_{11} x_1 + a_{12} x_2 + \cdots + a_{1n} x_n = 0 \\
                    a_{21} x_1 + a_{22} x_2 + \cdots + a_{2n} x_n = 0 \\
                    \vdots \\
                    a_{m1} x_1 + a_{m2} x_2 + \cdots + a_{mn} x_n = 0
                \end{matrix}
            \right. .
        \]
    Outra maneira de verificar que o conjunto das soluções de um sistema linear homogêneo é um espaço vetorial é a seguinte.
\end{ex}

 O conjunto das matrizes simétricas, 
        \[
            W_1 = \{A \in \mathcal{M}_{n \times n}: A^T= A\},
        \]
    bem como o conjunto das matrizes antissimétricas, 
        \[
            W_2 = \{A \in \mathcal{M}_{n \times n}: A^T= -A \},
        \]
    são subespaços de $\mathcal{M}_{n \times n}$. E ainda, temos que $\mathcal{M}_{n \times n} = W_1 \oplus W_2$.





\chapter{Geometria Analítica}

Seguimos os capítulos 12 e 13 de \cite{apostol1}.

\begin{defi}
    \leavevmode
        \begin{enumerate}[leftmargin=*, align=left, label=\textbf{(\alph*)}]
            \item Dois vetores $A, B \in \R^n$ são \textit{paralelos} se existe $c \in \R_{\neq 0}$ tal que $B = c A$.
            \item Dois vetores $A, B \in \R^n$ têm a mesma \textit{direção} se existe $c \in \R_{>0}$ tal que $B = c A$.
            \item Dois vetores $A, B \in \R^n$ têm \textit{direções opostas} se existe $c \in \R_{<0}$ tal que $B = c A$.
        \end{enumerate}
\end{defi}

\begin{defi}
    O \textit{produto escalar} de dois vetores $A = (a_1, \ldots, a_n)$ e $B = (b_1, \ldots, b_n)$ é definido como
        \[
            A \cdot B := \sum_{i=1}^{n} a_i b_i.
        \]
\end{defi}

\begin{prop}
    
\end{prop}

\begin{proof}
    \itemproof
\end{proof}

\begin{teo}[Cauchy-Schwarz]
    Para quaisquer $A, B \in \R^n$, tem-se 
        \[
            (A \cdot B)^2 \leq (A \cdot A)(B \cdot B).
        \]
\end{teo}

\begin{proof}
    Se $A = 0$ ou $B = 0$, o resultado segue trivialmente. Suponha, então, que $A,B \in \R^n_{\neq 0}$.
    
    \itemproof
\end{proof}


\begin{defi}
    A \textit{norma} de um vetor $A \in \R^n$ é definida como
        \[
            \|A\| := \sqrt{A \cdot A}.
        \]
\end{defi}

\begin{defi}
    A \textit{projeção} de $A \in \R^n$ em $B \in \R^n_{\neq 0}$ é definida como
        \[
            \op{proj}_{B}{A} := \left(\dfrac{A \cdot B}{\| B\|^2}\right) B.
        \]
\end{defi}


\begin{defi}
    Uma \textit{reta} no $\R^n$ que passa pelo ponto $P \in \R^n$ e é paralela ao vetor $A \in \R^n_{\neq 0}$ é a imagem da função $L : \R \to \R^n$ definida por $L(t) = P + t A$. Denota-se $L(P,A):= \Im{L}$, isto é,
        \[
            L(P,A) := \{ X \in \R^n : \exists t (t \in \R \land X = P + t A) \}.
        \]
\end{defi}

\begin{prop} \label{c2.prop:retas}
    Sejam $P, Q, A, B \in \R^n$.
        \begin{enumerate}[leftmargin=*, align=left, label=\textbf{(\alph*)}]
            \item $L(P,A) = L(P,B) \Leftrightarrow A \parallel B$.
            \item $L(P,A) = L(Q,B) \Leftrightarrow Q \in L(P,A) \lor P \in L(Q,B)$.
            \item Se $P \neq Q$, então existe uma única reta $L \subsetneq \R^n$ tal que $P,Q \in L$.
        \end{enumerate}
\end{prop}

\begin{proof}
    \leavevmode
        \begin{enumerate}[leftmargin=*, align=left, label=\textbf{(\alph*)}]
            \item 
            \item 
            \item Pois tome $L = L(P, Q-P)$. Temos $P \in L$ pois $P = P + 0 \cdot (Q-P)$ e temos $Q \in L$ pois $Q = P + 1 \cdot (Q-P)$. Agora, seja $L'$ uma reta tal que $P, Q \in L'$. Como $P \in L'$, temos por definição $L' = L(P,A)$ para algum vetor $A \in \R^n_{\neq 0}$. Como $Q \in L' = L(P,A)$, existe $t \in \R$ tal que $Q = P + t A$. Daí, $Q-P = t A$, e como $Q - P \neq 0$, temos $t \neq 0$ e $Q-P \parallel A$, donde $L' = L$ pelo primeiro item. \itemproof
        \end{enumerate}
\end{proof}



\begin{defi}
    Duas retas $L(P,A)$ e $L(Q,B)$ são \textit{paralelas} se $A \parallel B$. Isso é denotado por $L(P,A) \parallel L(Q,B)$.
\end{defi}

\begin{teo}[Paralelas]
    Seja $L \subsetneq \R^n$ uma reta. Para todo ponto $Q \notin L$ existe uma única reta $L' \subsetneq \R^n$ tal que $Q \in L'$ e $L' \parallel L$. 
\end{teo}

\begin{proof}
    Seja $L = L(P,A)$ e tome $L' = L(Q,A)$. De cara, $Q \in L'$ e $L' \parallel L$. A unicidade segue imediatamente da proposição \eqref{c2.prop:retas}. \itemproof
\end{proof}

\begin{teo}
    Sejam $P, A \in \R^2$. Se $N \in \R^2$ é tal que $N \cdot A = 0$, então
        \begin{enumerate}[label=\roman*.]
            \item $\{ X \in \R^2 : (X-P) \cdot N = 0 \} = L(P,A)$;
            \item vale
                \[
                    \| X \| \geq \dfrac{|P \cdot N|}{\| N\|}
                \]
            para todo $X \in L(P,A)$, com igualdade somente se $X = \op{proj}_{N}{P}$;
            \item Se $Q \notin L(P,A)$, então
                \[
                    \| X-Q \| \geq \dfrac{|(P-Q)\cdot N|}{\|N\|}
                \]
            para todo $X \in L(P,A)$, com igualdade somente se $Q = \op{proj}_{N}{(P-Q)}$;
        \end{enumerate}
\end{teo}

\begin{proof}
    Comecemos com um lema: se $(a,b), (c,d) \in \R^2$ são tais que $(a,b) \cdot (c,d) = 0$, então $(c,d) = t \cdot (b,-a)$ para algum $t \in \R$.
\end{proof}

\begin{defi}
    Seja $n \in \N_{\geq 2}$. Um \textit{plano} no $\R^n$ que passa por um ponto $P \in \R^n$ e é gerado por vetores $u,v \in \R^n$, linearmente independentes, é a imagem da função $\alpha : \R^2 \to \R^n$ definida por $\alpha(s,t) = P + su + tv$. Denota-se $\{P + su + tv\} := \Im{\alpha}$, isto é,
        \[
             \{P + su + tv\} = \{ X \in \R^n : \exists s \exists t (s,t \in \R \land X = P + su + tv) \}.
        \]
\end{defi}

\begin{teo}
    \leavevmode
        \begin{enumerate}[leftmargin=*, align=left, label=\textbf{(\alph*)}]
            \item Se $M = \{P + sA + tB\}$ e $M' = \{P + sC + tD\}$ são planos, então $M = M'$ se, e somente se, $[A,B] = [C,D]$.
            \item Se $\alpha = \{P + su + tv\}$ e $\beta = \{Q + su + tv\}$ são planos, então $\alpha = \beta$ se, e somente se, $Q \in \alpha$ ou $P \in \beta$.
            \item Se $P, Q, R \in \R^n$ não são colineares, então existe um único plano $\alpha \subsetneq \R^n$ tal que $P, Q, R \in \alpha$.
        \end{enumerate}
\end{teo}

\begin{proof}
    \leavevmode
        \begin{enumerate}[leftmargin=*, align=left, label=\textbf{(\alph*)}]
            \item ($\Rightarrow$) bla bla

            ($\Leftarrow$)
            \item ($\Rightarrow$) bla bla

            ($\Leftarrow$)
            \item 
        \end{enumerate}
\end{proof}

\begin{defi}
    Sejam $\alpha = \{P + sA + tB\}$ e $\beta = \{Q + sC + tD\}$ planos no $\R^n$.
        \begin{enumerate}[leftmargin=*, align=left, label=\textbf{(\alph*)}]
            \item Um vetor $v \in \R^n$ é \textit{paralelo ao plano} $\alpha$ se $v \in [A,B]$.
            \item Os planos $\alpha$ e $\beta$ são \textit{paralelos} se $[A,B] = [C,D]$.
        \end{enumerate}
\end{defi}

\begin{teo}[Paralelas]
    Seja $\alpha \subsetneq \R^n$ um plano. Para todo ponto $Q \notin \alpha$ existe um único plano $\beta \subsetneq \R^n$ tal que $Q \in \beta$ e $\beta \parallel \alpha$.
\end{teo}

\begin{proof}
    \itemproof
\end{proof}


\begin{teo}
    \leavevmode
        \begin{enumerate}[leftmargin=*, align=left, label=\textbf{(\alph*)}]
            \item Dois vetores $u, v \in \R^n$ são L.I. se, e somente se, existe uma reta $r \subsetneq \R^n$ tal que $u,v,0 \in r$.
            \item Três vetores $u, v, w \in \R^n$ são L.I. se, e somente se, existe um plano $\alpha \subsetneq \R^n$ tal que $u,v,w,0 \in \alpha$.
        \end{enumerate}
\end{teo}

\begin{proof}
    \leavevmode
        \begin{enumerate}[leftmargin=*, align=left, label=\textbf{(\alph*)}]
            \item 
            \item 
        \end{enumerate}
\end{proof}

% \label{teo:importante} ~\ref{teo:importante}


\part{Cálculo II}

%!TEX root = main.tex

\chapter{Topologia do Espaço Euclidiano}

\begin{defi}
    Uma \textit{bola aberta de raio $r \in \R_{>0}$ e centro $a \in \R^n$} é definida como
        \[
            B_r(a) := \{ x \in \R^n : \|x - a\| < r \}.
        \]
\end{defi}

\begin{defi}
    \leavevmode 
        \begin{enumerate}[leftmargin=*, align=left, label=\textbf{(\alph*)}]
            \item Um ponto $a \in \R^n$ é um \textit{ponto interior} de $A \subseteq \R^n$ se existe $r \in \R_{>0}$ tal que $B_r(a) \subseteq A$.
            \item O \textit{interior} de $A \subseteq \R^n$, denotado por $\op{int}{A}$, é definido como o conjunto de todos os pontos interiores de $A$, isto é,
                \[
                    \op{int}{A} = \{ x \in A : \exists r (r \in \R_{>0} \land B_r(x) \subseteq A ) \}.
                \]
            \item Um subconjunto $A \subseteq \R^n$ é \textit{aberto} se $\op{int}{A} = A$.
        \end{enumerate}
\end{defi}

\begin{cor}
    Toda bola aberta é um conjunto aberto.
\end{cor}

\begin{proof}
    Ver \cite{guidorizzi2}, página 113. \blackproof
\end{proof}

\begin{defi}
    Um ponto $a \in \R^n$ é um \textit{ponto de acumulação} de $A \subseteq \R^n$ se
        $
            B_r(a) \cap A_{\neq a} \neq \emptyset
        $
    para todo $r \in \R_{>0}$.
\end{defi}

\begin{defi}
    \leavevmode 
        \begin{enumerate}[leftmargin=*, align=left, label=\textbf{(\alph*)}]
            \item Um ponto $a \in \R^n$ é um \textit{ponto exterior} de $A \subseteq \R^n$ se existe $r \in \R_{>0}$ tal que $B_r(a) \cap A = \emptyset $.
            \item O \textit{exterior} de $A \subseteq \R^n$, denotado por $\op{ext}{A}$, é definido como o conjunto de todos os pontos exteriores de $A$, isto é,
                \[
                    \op{ext}{A} = \{ x \in \R^n : \exists r (r \in \R_{>0} \land B_r(x) \cap A = \emptyset ) \}.
                \]
        \end{enumerate}
\end{defi}

\begin{defi}
    Um ponto $a \in \R^n$ é um \textit{ponto da fronteira} de $A \subseteq \R^n$ se $a \notin \op{int}{A}$ e $a \notin \op{ext}{A}$. O conjunto de todos os pontos da fronteira de $A$ é denotado por $\partial A$.
\end{defi}





\chapter{Caminhos}

\begin{defi}
    Uma \textit{função vetorial de variável real} é uma função $f : X \subseteq \R \to \R^n$.
\end{defi}

As funções vetoriais de variável real que nos interessam, nesse momento, são aquelas cujo domínio é um intervalo ou uma união de intervalos.

\begin{prop}
    Seja $f : X \subseteq \R \to \R^n$ uma função. Existem e são únicas as funções $f_i : X \to \R$, com $i \in [n]$, tais que
        \[
            f(t) = (f_1(t), f_2(t), \ldots, f_n(t))
        \]
    para todo $t \in X$. Isso é denotado por $f = (f_1, f_2, \ldots, f_n)$.
\end{prop}

\begin{proof}
    Trivial. \blackproof
\end{proof}

\begin{defi}
    Uma função $f : X \subseteq \R \to \R^n$ tem limite $L \in \R^n$ quando $t$ tende ao ponto $t_0 \in X'$ se para todo $\epsilon \in \R_{>0}$ existe $\delta = \delta(\epsilon, t_0) \in \R_{>0}$ tal que
        \[
            0 < |t - t_0| < \delta \rightarrow \| f(t) - L \| < \epsilon
        \]
    para todo $t \in X$. Isso é denotado por
        \[
            \lim_{t \to t_0} f(t) = L.
        \]
\end{defi}

\begin{prop}[Unicidade do limite]
    Se uma função $f : X \subseteq \R \to \R^n$ tem limites $L_1, L_2 \in \R^n$, então $L_1 = L_2$.
\end{prop}

\begin{proof}
\end{proof}

\begin{prop}
    Sejam $F : X \subseteq \R \to \R^n$, $t_0 \in X'$ e $L \in \R^n$. Vale
        \[
            \lim_{t \to t_0} F(t) = L \Leftrightarrow \lim_{t \to t_0} \|F(t) - L\| = 0.
        \]
\end{prop}

\begin{proof}
    Ver \cite{guidorizzi2}, p. 124. \blackproof
\end{proof}

\begin{teo}
    Sejam $f : X \subseteq \R \to \R^n$, $t_0 \in X'$ e $L \in \R^n$, com $f = (f_1, f_2, \ldots, f_n)$ e $L = (L_1, L_2, \ldots, L_n)$. Para todo $i \in [n]$, vale
        \[
            \lim_{t \to t_0} f(t) = L \Leftrightarrow \lim_{t \to t_0} f_i (t) = L_i.
        \]
\end{teo}

\begin{proof}
    Ver \cite{guidorizzi2}, p. 124. \blackproof
\end{proof}

\begin{prop}[Propriedades operatórias]
    Oi
\end{prop}

\begin{proof}
\end{proof}

\begin{defi}[Continuidade]
    Sejam $f : X \subseteq \R \to \R^n$ e $t_0 \in X \cap X'$.
        \begin{enumerate}[leftmargin=*, align=left, label=\textbf{(\alph*)}]
            \item $f$ é \textit{contínua em $t_0$} se
                \[
                    \lim_{t \to t_0} f(t) = f(t_0).
                \]
            \item $f$ é \textit{contínua em $Y \subseteq X$} se $f$ for contínua em todo $t_0 \in Y$.
            \item $f$ é \textit{contínua} se $f$ for contínua em $X$.
        \end{enumerate}
\end{defi}

\begin{cor}
    Sejam $f : X \subseteq \R \to \R^n$ uma função, com $f = (f_1, f_2, \ldots, f_n$), e $t_0 \in X \cap X'$. $f$ é contínua em $t_0$ se, e somente se, $f_i$ é contínua em $t_0$, para todo $i \in [n]$.
\end{cor}

\begin{defi}
    Sejam $f : X \subseteq \R \to \R^n$ e $t_0 \in X \cap X'$.
        \begin{enumerate}[leftmargin=*, align=left, label=\textbf{(\alph*)}]
            \item $f$ é \textit{derivável}, ou \textit{diferenciável}, em $t_0$, se existir o limite
                \[
                    f'(t_0) := \lim_{t \to t_0} \dfrac{f(t) - f(t_0)}{t - t_0}.
                \]
            Mais precisamente, $f$ é derivável em $t_0$ se existir o limite $\ds \lim_{t \to t_0} g(t)$, onde $g : X \setminus \{ t_0\} \to \R^n$ é a função definida por
                \[
                    g(t) := \dfrac{f(t) - f(t_0)}{t-t_0}.
                \]
            Sendo $F$ derivável em $t_0$, ou ainda, \textit{diferenciável} em $t_0$, dizemos que o limite $F'(t_0)$ é a \textit{derivada} de $F$ em $t_0$.
            
            \item Diremos que $F$ é derivável em $Y \subseteq X$ se $F$ for derivável em todo $t \in Y$; se for $Y = X$, diremos que $F$ é derivável.
        \end{enumerate}
\end{defi}

\begin{prop}
    Sejam $F : X \subseteq \R \to \R^n$ e $t_0 \in X$. Sendo $F = (f_1, f_2, \ldots, f_n$), temos que $F$ é derivável em $t_0$ se, e somente se, $f_i$ é derivável em $t_0$ para todo $i \in [n]$. Sendo $F$ derivável em $t_0$, temos que
        \[
            F'(t) = (f_1' (t), f_2'(t), \ldots, f_n'(t) ).
        \]
\end{prop}

\begin{proof}
\end{proof}

\begin{defi} %vetor e reta tangentes
    Seja $F : X \subseteq \R \to \R^n$ derivável em $t_0 \in X$, com $F'(t_0) \neq 0$.
    
        \begin{enumerate}[leftmargin=*, align=left, label=\textbf{(\alph*)}]
            \item 
        \end{enumerate}
\end{defi}

\begin{prop}[Propriedades operatórias]
\end{prop}

\begin{proof}
\end{proof}

\begin{prop}[Regra da cadeia]
\end{prop}

\begin{prop}
    Se $F : X \subseteq \R \to \R^n$ uma função vetorial derivável em $X$ tal que $\| F(t) \| = k \in \R$ para todo $t \in X$, então $F(t) \cdot F'(t) = 0$ para todo $t \in X$.
\end{prop}

\begin{proof}
\end{proof}

\begin{defi}
    Seja $f : [a,b] \to \R^n$ uma função. A soma de Riemann $S(F, P, \xi)$ tem limite $L \in \R^n$ quando $\| P \|$ tende a $0$ se para todo $\epsilon \in \R_{>0}$ existir $\delta = \delta (\epsilon) \in \R_{>0}$ tal que
        \[
            \|S(f, P, \xi) - L\| < \epsilon
        \]
    para toda partição marcada $(P, \xi)$ de $[a,b]$ com $\| P \| < \delta$. Isso é denotado por
        \[
            \lim_{\| P \| \to 0} S(f, P, \xi) = L.
        \] 
\end{defi}

\begin{prop}
    Seja $f : [a,b] \to \R^n$ uma função. O limite das somas de Riemann, quando existe, é único, isto é, se
        \[ \ds
            \lim_{\| P \| \to 0} S(f, P, \xi) = L_1 \quad \text{e} \quad \lim_{\| P \| \to 0} S(f, P, \xi) = L_2,
        \]
    então $L_1 = L_2$.
\end{prop}

\begin{proof}
\end{proof}

\begin{defi}
    Uma função $f : [a,b] \to \R^n$ é \textit{integrável em $[a,b]$ segundo Riemann} se
        $ \ds
            \lim_{\| P \| \to 0} S(f, P, \xi)
        $
    existe. Nesse caso, esse número real é chamado de \textit{integral de $f$ em $[a,b]$ segundo Riemann} e é denotado por
        \[
            \int_{a}^{b} F(x) \, dx,
        \]
    isto é,
        \[
            \int_{a}^{b} f(x) \, dx := \lim_{\| P \| \to 0} S(f, P, \xi).
        \]
\end{defi}

\begin{prop}
    Seja $f : X \subseteq \R \to \R^n$ uma função com $F = (f_1, f_2, \ldots, f_n$). $f$ é integrável em $[a,b]$ se, e somente se, $f_i$ é integrável em $[a,b]$ para todo $i \in [n]$, sendo nesse caso
        \[
            \int_{a}^{b} F(t) \, dt = \left( \int_{a}^{b} f_1 (t) \, dt, \int_{a}^{b} f_2 (t) \, dt, \ldots, \int_{a}^{b} f_n (t) \, dt \right).
        \]
\end{prop}

\begin{proof}
\end{proof}

\begin{prop}[Propriedades operatórias]
\end{prop}

\begin{teo}(Fundamental do Cálculo, parte I) \label{teo:TFC1}
     Seja $f: [a,b] \to \R^n$ uma função integrável.
        \begin{enumerate}[leftmargin=*, align=left, label=\textbf{(\alph*)}]
            \item  A função $F: [a,b] \to \R^n$ definida por
                \[
                    F(x) := \int_{a}^{x} f(t) \, dt
                \]
            é uniformemente contínua em $[a,b]$.
            
            \item Se $f$ é contínua em $x_0 \in [a,b]$, então $F$ é derivável em $x_0$ e $F'(x_0) = f(x_0)$.
        \end{enumerate} 
\end{teo}

\begin{proof}
\end{proof}

\begin{cor}
    Seja $f : [a,b] \to \R^n$ uma função contínua em $[a,b]$.
        \begin{enumerate}[leftmargin=*, align=left, label=\textbf{(\alph*)}]
            \item A função $F: [a,b] \to \R^n$ definida por
                \[
                    F(x) := \int_{a}^{x} f(t) \, dt
                \]
            é uma primitiva de $f$ em $[a,b]$.
            
            \item Se $G: [a,b] \to \R^n$ é qualquer outra primitiva de $f$, então
                \[
                    G(x) = G(a) + \int_{a}^{x} f(t) \, dt
                \]
            para todo $x \in [a,b]$. Particularmente para $x = b$, temos
                \[
                    \int_{a}^{b} f(t) \, dt = G(b) - G(a).
                \]
        \end{enumerate}
\end{cor}

\begin{proof}
\end{proof}

\begin{teo}[Fundamental do Cálculo, parte II]
    Se $f : [a,b] \to \R^n $ é uma função integrável e $F : [a,b] \to \R^n$ é uma primitiva qualquer de $f$, então
        \[
            F(x) = F(a) + \int_{a}^{x} f(t) \, dt
        \]
    para todo $x \in [a,b] $. Particularmente para $x=b$, temos
        \[
            \int_{a}^{b} f(t) \, dt = F(b) - F(a).
        \]
\end{teo}

\begin{proof}
\end{proof}

\begin{prop}
    Seja $C \in \R^n$. Se $f : [a,b] \to \R^n$ é uma função integrável, então a função $C \cdot f : [a,b] \to \R$ é integrável e
        \[
            C \cdot \left[ \int_{a}^{b} f(t) \, dt \right] = \int_{a}^{b} [C \cdot f(t)] \, dt.
        \]
\end{prop}

\begin{proof}
\end{proof}

\begin{prop}
    Se $f : [a,b] \to \R^n$ e $\| f \| : [a,b] \to \R$ são funções integráveis em $[a,b]$, então
        \[
            \left\| \int_{a}^{b} f(t) \, dt \right\| \leq \int_{a}^{b} \| f(t) \| \, dt.
        \]
\end{prop}

\section{Curvas}

No que segue, $I, J \subseteq \R$ são intervalos.

\begin{defi}
    \leavevmode
        \begin{enumerate}[leftmargin=*, align=left, label=\textbf{(\alph*)}]
            \item Uma \textit{curva} no $\R^n$ é uma função vetorial de variável real $\alpha : I \to \R^n$.
            \item Uma \textit{curva paramétrica} é uma curva $\alpha : I \to \R^n$ contínua. O \textit{traço} da curva é a imagem de $I$ por $\alpha$, isto é, o conjunto $\alpha(I)$.
            \item Uma curva paramétrica $\alpha : I \to \R^n$ é \textit{regular} se $\alpha$ é derivável em $I$ com  $\alpha'(t) \neq 0$ para todo $t \in I$.
        \end{enumerate}
\end{defi}

\begin{defi}
    Seja $\alpha : I \to \R^n$ uma curva paramétrica regular. Uma curva paramétrica $\beta : J \to \R^n$ é uma \textit{reparametrização} de $\alpha$  se $\beta(J) = \alpha(I)$ e se existe uma \textit{função de reparametrização} $\varphi : J \to I$ bijetora, derivável em $J$, com $\varphi'(t) \neq 0$, tal que $\beta(t) = \alpha(\varphi(t))$.
\end{defi}

\begin{prop}[Reparametrização conserva regularidade]
    Se uma curva paramétrica $\beta : J \to \R^n$ é uma reparametrização de uma curva paramétrica regular $\alpha : I \to \R^n$, então é $\beta$ regular.
\end{prop}

\begin{proof}
    \blackproof
\end{proof}

\begin{defi}
    Sejam $\alpha : I \to \R^n$ uma curva paramétrica regular e $\beta : J \to \R^n$ uma reparametrização de $\alpha$ por meio de uma função de reparametrização $\varphi : J \to I$.
        \begin{enumerate}[leftmargin=*, align=left, label=\textbf{(\alph*)}]
            \item $\beta$ é uma reparametrização \textit{positiva} se $\varphi'(t) > 0$ para todo $t \in J$.
            \item $\beta$ é uma reparametrização \textit{negativa} se $\varphi'(t) < 0$ para todo $t \in J$.
        \end{enumerate}
\end{defi}

\begin{defi}
    Seja $\alpha : I \to \R^n$ uma curva paramétrica derivável e com derivada integrável.
        \begin{enumerate}[leftmargin=*, align=left, label=\textbf{(\alph*)}]
            \item O \textit{comprimento do arco} de $a \in I$ até $b \in I$ é definido como
                \[
                    L_a^b(\alpha) := \int_{a}^{b} \| \alpha'(t) \| \, dt.
                \]
            \item Seja $t_0 \in I$. A \textit{função comprimento de arco} de $\alpha$ é definida como
                \[
                    L(t) := \int_{t_0}^{t} \| \alpha'(u) \| \, du.
                \]
        \end{enumerate}
\end{defi}

\begin{prop}
    Se $\alpha : [a,b] \to \R^n$ é uma curva paramétrica regular e $\beta : [c,d] \to \R^n$ é uma reparametrização de $\alpha$, então $L_{c}^{d}(\beta) = L_{a}^{b}(\alpha)$.
\end{prop}

\begin{prop}
    A função comprimento de arco de uma curva paramétrica $\alpha : I \to \R^n$ é uma função de reparametrização.
\end{prop}

\begin{defi}
    Uma curva paramétrica $\alpha : I \to \R^n$ é \textit{reparametrizada por comprimento de arco} se $\| \alpha'(t) \| = 1$ para todo $t \in I$.
\end{defi}

\begin{defi}
    \leavevmode
        \begin{enumerate}[leftmargin=*, align=left, label=\textbf{(\alph*)}]
            \item Um caminho $\alpha : [a,b] \to \R^n$ é \textit{retificável} se existe $M \in \R$ tal que
                \[
                    L_{a}^{b}(\alpha, P) \leq M
                \]
            para toda partição $P : a = t_0 < \cdots < t_k = b$ de $[a,b]$, onde
                \[
                    L_{a}^{b}(\alpha, P) := \sum_{i=1}^{k} \| \alpha(t_i) - \alpha(t_{i-1}) \|.
                \]
            \item Seja $\alpha : [a,b] \to \R^n$ um caminho retificável. O \textit{comprimento da curva} descrita por $\alpha$ é definido como
                \[
                    L_{a}^{b}(\alpha) := \sup{\{L_{a}^{b}(\alpha, P) : P \in \mathcal{P}[a,b]\}}.
                \]
        \end{enumerate}
\end{defi}

\begin{teo}
    Se um caminho $\alpha :[a,b] \to \R^n$ é de classe $C^1$ em $[a,b]$, então
        \[
            L_{a}^{b}(\alpha) = \int_{a}^{b} \| \alpha'(t) \| \, dt.
        \]
\end{teo}

\begin{proof}
    \blackproof
\end{proof}

\chapter{Campos Escalares e Vetoriais}

\begin{defi}
    Sejam $m, n \in \N$.
        \begin{enumerate}[leftmargin=*, align=left, label=\textbf{(\alph*)}]
            \item Um \textit{campo} é uma função $f : X \subseteq \R^n \to \R^m$.
            \item Um campo $f : X \subseteq \R^n \to \R^m$ é um \textit{campo escalar} se $m=1$. Ou seja, um \textit{campo escalar} é uma função $f : X \subseteq \R^n \to \R$.
            \item Um campo $f : X \subseteq \R^n \to \R^m$ é um \textit{campo vetorial} se $m \geq 2$.
        \end{enumerate}
\end{defi}

\begin{teo}
    Se $f : X \subseteq \R^n \to \R^m$ é um campo, então existem e são únicas as funções $f_i : X \to \R$, com $i \in [m]$, tais que
        \[
            f(x) = (f_1(x), f_2(x), \ldots, f_m(x))
        \]
    para todo $x \in X$. Isso é denotado por $f = (f_1, f_2, \ldots, f_m)$.
\end{teo}

\begin{proof}
    \blackproof
\end{proof}

\begin{defi}
    Uma função $f : X \subseteq \R^n \to \R^m$ tem limite $L \in \R^m$ quando $x$ tende a $a \in X'$ se para todo $\epsilon \in \R_{>0}$ existe $\delta = \delta(\epsilon, a) \in \R_{>0}$ tal que
        \[
            0 < \| x -a \| < \delta \Rightarrow \| f(x) - L \| < \epsilon
        \]
    para todo $x \in X$. Isso é denotado por 
        \[ \ds
            \lim_{x \to a} f(x) = L.
        \]
\end{defi}

\begin{prop}
    Sejam $f : X \subseteq \R^n \to \R^m$, $L \in \R^m$ e $a \in X'$. Vale
        \[
            \lim_{x \to a} f(x) = L \Leftrightarrow \lim_{\| x-a \| \to 0} \|f(x) - L \| = 0.
        \]
\end{prop}

\begin{proof}
    \blackproof
\end{proof}

\begin{prop}[Propriedades operatórias]
    %Sejam $f,g : A \subseteq \R^n \to \R^m$ e $a \in A'$. Se $\ds \lim_{x \to a} f(x) = L_1 \in \R^m$ e $\ds \lim_{x \to a} g(x) = L_2 \in \R^m$, então propriedades operatórias
\end{prop}

\begin{defi}[Continuidade]
    Seja $f : A \subseteq \R^n \to \R^m$ um campo.
        \begin{enumerate}[leftmargin=*, align=left, label=\textbf{(\alph*)}]
            \item Dizemos que $f$ é \textit{contínua em $a \in A$} se para todo $\epsilon \in \R_{>0}$ existe $\delta = \delta(\epsilon, a) \in \R_{>0}$ tal que
                \[
                    \| x - a \| < \delta \Rightarrow \| f(x) - f(a) \| < \epsilon
                \]
            para todo $x \in A$.
            \item Dizemos que $f$ é \textit{contínua em $X \subseteq A$} se $f$ é contínua em todo ponto $a \in X$. Mais especificamente, $f$ é contínua em $X$ se para cada $a \in X$ e cada $\epsilon \in \R_{>0}$ existe $\delta = \delta(\epsilon, a) \in \R_{>0}$ tal que
                \[
                    \| x - a \| < \delta \Rightarrow \| f(x) - f(a) \| < \epsilon
                \]
            para todo $x \in A$.
        \end{enumerate}
\end{defi}

\begin{teo}
    Uma função $f : A \subseteq \R^n \to \R^m$ é contínua em $a \in A \cap A'$ se, e somente se, $\ds \lim_{x \to a} f(x) = f(a)$.
\end{teo}

\begin{proof}
    \blackproof
\end{proof}

\begin{teo}
    Sejam $f : A \subseteq \R^n \to \R^m$ e $g : B \subseteq \R^m \to \R^k$ funções tais que $f(A) \subseteq B$. Se $f$ é contínua em $a \in A$ e se $g$ é contínua em $f(a) \in B$, então a função composta $g \circ f : X \subseteq \R^n \to \R^k$ é contínua em $a$.
\end{teo}

\begin{proof}
    
\end{proof}



\begin{prop}
    Sejam $f: X \subseteq \R^n \to \R$ e $a \in X \cap X'$ tais que $\ds \lim_{x \to a} f(x)$ existe. Se $\alpha : I \subseteq \R \to \R^n$ é contínua em $t_0 \in I$, com $\alpha(t_0) = a$, e $\alpha(t) \in X$ para todo $t \in I$, com $t \neq t_0 \Rightarrow \alpha(t) \neq \alpha(t_0)$, então
        \[
            \lim_{t \to t_0} f(\alpha(t)) = \lim_{x \to a} f(x).
        \]
\end{prop}

\begin{proof}
    Ver \cite{guidorizzi2}, p. 165, exemplo 4. \blackproof
\end{proof}

\begin{teo}[Compostas]
    \leavevmode
        \begin{enumerate}[leftmargin=*, align=left, label=\textbf{(\alph*)}]
            \item Sejam $f : X \subseteq \R^n \to \R$ e $g : Y \subseteq \R \to \R$ funções tais que $f(X) \subseteq Y$. Se $f$ é contínua em $a \in X$ e $g$ é contínua em $f(a) \in Y$, então a função composta $g \circ f : \R^n \to \R$ é contínua em $a$.
            \item Sejam $f : X \subseteq \R^n \to \R$ uma função e $\alpha : I \subseteq \R \to \R^n$ uma curva tais que $\alpha(t) \in X$ para todo $t \in I$. Se $\alpha$ é contínua em $a \in I$ e $f$ é contínua em $\alpha(a) \in X$, então a função composta $f \circ \alpha : \R \to \R$ é contínua em $a$.
            \item Sejam $f : X \subseteq \R^n \to \R$ e $f_1, \ldots, f_n : Y \subseteq \R^n \to \R$ funções tais que $(f_1(x),\ldots, f_n(x)) \in X$ para todo $x \in Y$. Se $f_1, \ldots, f_n$ são contínuas em $a \in Y$, e se $f$ é contínua em $(f_1(a),\ldots, f_n(a))$, então a função composta $f((f_1(x),\ldots, f_n(x)))$ é contínua em $a$.
        \end{enumerate}
\end{teo}

\begin{proof}
    Ver \cite{guidorizzi2}, pg. 170, Teorema 1. \blackproof
\end{proof}

\begin{teo}
    Sejam $f : X \subseteq \R^n \to \R$ e $\alpha : I \subseteq \R \to \R^n$ tais que $\alpha(t) \in X$ para todo $t \in I$. Se $\alpha$ é contínua em $a \in I$ e $f$ é contínua em $\alpha(a) \in X$, então a função composta $f \circ \alpha : \R \to \R$ é contínua em $a$.
\end{teo}

\begin{proof}
    
\end{proof}

\begin{defi}
    Sejam $A \subseteq \R^n$ aberto, $a \in A$ e $y \in \R^n$.
        \begin{enumerate}[leftmargin=*, align=left, label=\textbf{(\alph*)}]
            \item Um campo escalar $f: A \to \R$ é \textit{derivável} em $a$ com respeito a $y$ se existe o limite
                \[
                    f_y'(a) := \lim_{h \to 0} g(h),
                \]
            onde $g:\{h \in \R_{\neq 0} : a+hy \in A\} \to \R$ é definida por
                \[
                    g(h) := \dfrac{f(a+hy)-f(a)}{h}.
                \]
            \item Seja $f : A \to \R$ \textit{derivável} em $a$ com respeito a $y$.
                \begin{enumerate}[label=\roman*.]
                    \item Se $\| y\| = 1$, então dizemos que $f_y'(a)$ é a \textit{derivada direcional} de $f$ na \textit{direção} de $y$.
                    \item Se $y = e_k$ para algum $k \in [n]$, então dizemos que $f_{e_k}'(a)$ é a \textit{derivada parcial} de $f$ com respeito a $e_k$. Outras notações para $f_{e_k}'(a)$ são 
                        \[
                            \dfrac{\partial f}{\partial x_k} (a), \quad D_k f(a) \quad \text{e} \quad f_{x_k}'(a).
                        \]
                \end{enumerate}
        \end{enumerate}
\end{defi}

\begin{defi}
    Sejam $A \subseteq \R^n$ aberto e $f : A \to \R$ derivável em $a \in A$ com respeito a $e_k$ para todo $k \in [n]$. O \textit{gradiente} de $f$ em $a$ é definido como
        \[
            \nabla f(a) := \left(\dfrac{\partial f}{\partial x_1} (a), \dfrac{\partial f}{\partial x_2} (a), \ldots, \dfrac{\partial f}{\partial x_n} (a)  \right).
        \]
\end{defi}

\begin{defi}
    Seja $A \subseteq \R^n$ aberto. Um campo escalar $f: A \to \R$ é \textit{diferenciável} em $a \in A$ se existe uma transformação linear $T_a : \R^n \to \R$ para a qual a função $r_a : \{h \in \R^n : a+h \in A\} \to \R$ definida por
        \[
            r_a(h) := f(a+h) - f(a) - T_a(h)
        \]
    satisfaz
        $ \ds
            \lim_{h \to 0} \dfrac{|r_a(h)|}{\|h \|} = 0.
        $
    A transformação linear $T_a$ é a \textit{derivada total} de $f$ em $a$.
\end{defi}

\chapter{Integrais de Linha}

\begin{defi}
    \leavevmode
        \begin{enumerate}[leftmargin=*, align=left, label=\textbf{(\alph*)}]
            \item Um \textit{caminho} é uma função contínua $\gamma : [a,b] \to \R^n$.
            \item Um caminho $\gamma : [a,b] \to \R^n$ é \textit{suave} se $\gamma$ é de classe $C^1$ em $]a,b[$.
            \item Um caminho $\gamma : [a,b] \to \R^n$ é \textit{suave por partes} se existe uma partição
                \[
                    P : a = x_0 < x_1 < \cdots < x_k = b
                \]
            tal que $\gamma_i := \gamma |_{[x_{i-1}, x_i]}$ é suave para todo $i \in [k]$.
        \end{enumerate}
\end{defi}

\begin{defi}
    Sejam $\gamma : [a,b] \to \R^n$ um caminho suave por partes e $f: \gamma([a,b]) \subseteq \R^n \to \R^n$ um campo vetorial limitado. A \textit{integral de linha} de $f$ com respeito a $\gamma$ em $C := \gamma([a,b])$ é definida como
        \[
            \int_C f \cdot d\gamma := \int_{a}^{b} f[\gamma(t)] \cdot \gamma'(t) \, dt.
        \]
\end{defi}

\begin{defi}
    Seja $\alpha : [a,b] \to \R^n$ um caminho. Um caminho $\beta : [c,d] \to \R^n$ é uma \textit{reparametrização} de $\alpha$ se existe uma bijeção $\varphi : [c,d] \to [a,b]$, derivável em $[c,d]$, com $\varphi'(t) \neq 0$ para todo $t \in [c,d]$, tal que $\beta(t) = \alpha(\varphi(t))$ para todo $t \in [c,d]$. Os caminhos $\alpha$ e $\beta$ são \textit{equivalentes}, enquanto a bijeção $\varphi$ é uma \textit{mudança de parâmetro}.
\end{defi}

\begin{prop}
    Dois caminhos equivalentes descrevem a mesma curva.
\end{prop}

\begin{defi}
    Sejam $\alpha : [a,b] \to \R^n$ um caminho e $\beta : [c,d] \to \R^n$ uma reparametrização de $\alpha$ por meio de uma mudança de parâmetro $\varphi : [c,d] \to [a,b]$.
        \begin{enumerate}[leftmargin=*, align=left, label=\textbf{(\alph*)}]
            \item $\beta$ é uma reparametrização \textit{positiva} se $\varphi'(t) > 0$ para todo $t \in [c,d]$. Dizemos que os caminhos $\alpha$ e $\beta$ têm o \textit{mesmo sentido} e que a mudança de parâmetro \textit{conserva a orientação} do traço.
            \item $\beta$ é uma reparametrização \textit{negativa} se $\varphi'(t) < 0$ para todo $t \in [c,d]$. Dizemos que os caminhos $\alpha$ e $\beta$ têm \textit{sentidos opostos} e que a mudança de parâmetro \textit{inverte a orientação} do traço.
        \end{enumerate}
\end{defi}


\begin{teo}
    Sejam $\gamma_1 : [a,b] \to \R^n$ e $\gamma_2  : [c,d] \to \R^n$ caminhos suaves por partes equivalentes. Seja $f : C \subseteq \R^n \to \R^n$ um campo vetorial limitado, onde $C$ é o traço dos caminhos $\gamma_1$ e $\gamma_2$.
        \begin{enumerate}[leftmargin=*, align=left, label=\textbf{(\alph*)}]
            \item Se $\gamma_1$ e $\gamma_2$ têm o mesmo sentido, então
                \[
                    \int_{C} f \cdot d \gamma_1 = \int_{C} f \cdot d \gamma_2.
                \]
            \item Se $\gamma_1$ e $\gamma_2$ têm sentidos opostos, então
                \[
                    \int_{C} f \cdot d \gamma_1 = - \int_{C} f \cdot d \gamma_2.
                \]
        \end{enumerate}
\end{teo}

\begin{proof}
    \blackproof
\end{proof}

\begin{defi}
    Sejam $\gamma : [a,b] \to \R^n$ um caminho de classe $C^1$ em $[a,b]$ e $f: \gamma([a,b]) \subseteq \R^n \to \R$ um campo escalar limitado. A \textit{integral de linha} de $f$ com respeito a $\gamma$ em $C := \gamma([a,b])$ é definida como
        \[
            \int_C f \cdot ds := \int_{a}^{b} f[\gamma(t)] \cdot \|\gamma'(t) \| \, dt.
        \]
\end{defi}











\part{Probabilidade}

%!TEX root = main.tex

\chapter{Combinatória Finita}



\begin{prop}
    \leavevmode
        \begin{enumerate}[leftmargin=*, align=left, label=\textbf{(\alph*)}]
            \item (Regra da soma) Se $X_1, X_2, \ldots, X_n$ são conjuntos finitos dois a dois disjuntos, então o conjunto $\ds \bigcup_{i=1}^{n} X_i$ é finito e
                \[
                    \left| \bigcup_{i=1}^{n} X_i \right| = \sum_{i=1}^{n} |X_i|.
                \]
            \item (Regra do produto) Se $X_1, X_2, \ldots, X_n$ são conjuntos finitos, então o conjunto $X_1 \times X_2 \times \cdots \times X_n$ é finito e
                \[
                    |X_1 \times X_2 \times \cdots \times X_n| = \prod_{i=1}^{n} |X_i|.
                \]
        \end{enumerate}
\end{prop}

\begin{proof}
    \leavevmode
        \begin{enumerate}[leftmargin=*, align=left, label=\textbf{(\alph*)}]
            \item Façamos indução em $n \geq 2$. 
            
            Para $n=2$, ver \cite{cursodeanalise1}, p. 32, teorema 6, \cite{tertulianoprob}, p. 14, \cite{tme4}, p. 2.
            
            Para completar a indução, ver \cite{cursodeanalise1}, p. 33, corolário 1, \cite{tertulianoprob}, p. 15, \cite{tme4}, p. 2.

            \item Ver \cite{cursodeanalise1}, p. 33, corolário 3.
        \end{enumerate}
\end{proof}

\begin{prop}
    \leavevmode
        \begin{enumerate}[leftmargin=*, align=left, label=\textbf{(\alph*)}]
            \item Se $X$ é um conjunto finito, então $\mathcal{P} (X)$ é finito e $|\mathcal{P}(X)| = 2^{|X|}$.

            \item Se $X$ e $Y$ são conjuntos finitos, então o conjunto $X^Y$ (de todas as funções $f:X \to Y$) é finito $|X^Y| = |Y|^{|X|}$.
        \end{enumerate}
\end{prop}

\begin{proof}
    \leavevmode
        \begin{enumerate}[leftmargin=*, align=left, label=\textbf{(\alph*)}]
            \item

            \item Ver \cite{cursodeanalise1}, p. 33, corolário 3.
        \end{enumerate}
\end{proof}

\chapter{Espaços de Probabilidade}

\begin{defi}
    Seja $\Omega$ um conjunto.
        \begin{enumerate}[leftmargin=*, align=left, label=\textbf{(\alph*)}]
            \item  Uma \textit{$\sigma$-álgebra} é um subconjunto $\mathcal{F} \subseteq \mathcal{P}(\Omega)$, tal que
                \begin{enumerate}[label=\roman*.]
                    \item $\Omega \in \mathcal{F}$;
                    \item se $A \in \mathcal{F}$, então $A^C \in \mathcal{F}$;
                    \item se $\left\{ A_n \right\}_{n \in \N} \subseteq \mathcal{F}$, então $\ds \bigcup_{n \in \N} A_n \in \mathcal{F}$.
                \end{enumerate}
            \item Um \textit{espaço mensurável} é um par $(\Omega, \mathcal{F})$, onde $\mathcal{F}$ é uma $\sigma$-álgebra em $\Omega$.
        \end{enumerate}
\end{defi}

\begin{ex}
    Seja $\Omega$ um conjunto.
        \begin{enumerate}[leftmargin=*, align=left, label=\textbf{(\alph*)}]
            \item $(\Omega, \mathcal{P}(\Omega))$ é um espaço mensurável.
            \item $(\Omega, \{\emptyset, \Omega\})$ é um espaço mensurável, dito \textit{trivial}, pois $\{\emptyset, \Omega\}$ é uma $\sigma$-álgebra em $\Omega$.
        \end{enumerate}
\end{ex}

\begin{cor}
    Seja $(\Omega, \mathcal{F})$ um espaço mensurável. Valem as seguintes afirmações.
        \begin{enumerate}[leftmargin=*, align=left, label=\textbf{(\alph*)}]
            \item $\emptyset \in \mathcal{F}$.
            \item Se $\left\{ A_n \right\}_{n \in \N} \subseteq \mathcal{F}$, então $\ds \bigcap_{n \in \N} A_n \in \mathcal{F}$.
            \item Se $A, B \in \mathcal{F}$, então $A \setminus B \in \mathcal{F}$ e $B \setminus A \in \mathcal{F}$.
        \end{enumerate}
\end{cor}

\begin{proof}
    \leavevmode
        \begin{enumerate}[leftmargin=*, align=left, label=\textbf{(\alph*)}]
            \item Temos $\Omega \in \mathcal{F}$, de modo que $\Omega^{C} \in \mathcal{F}$, isto é, $\emptyset \in \mathcal{F}$. \blackproof
            \item \blackproof
        \end{enumerate}
\end{proof}

\begin{defi}
    Seja $(\Omega, \mathcal{F})$ um espaço mensurável com $\Omega \neq \emptyset$.
        \begin{enumerate}[leftmargin=*, align=left, label=\textbf{(\alph*)}]
            \item Uma \textit{medida de probabilidade} em $(\Omega, \mathcal{F})$ é uma função $\Pb : \mathcal{F} \to \R$ tal que
                \begin{enumerate}[label=\roman*.]
                    \item $\Pb(\Omega) = 1$;
                    \item $\Pb(A) \geq 0$ para todo $A \in \mathcal{F}$;
                    \item se $\left\{ A_n \right\}_{n \in \N} \subseteq \mathcal{F}$ e $A_i \cap A_j = \emptyset$ para $i \neq j$, então
                        \[
                            \Pb \left( \bigcup_{n \in \N} A_n \right) = \sum_{n \in \N} \Pb (A_n).
                        \]
                \end{enumerate}
            \item Um \textit{espaço de probabilidade} é uma terna $(\Omega, \mathcal{F}, \Pb)$, onde $\Pb : \mathcal{F} \to \R$ é uma medida de probabilidade definida em $(\Omega, \mathcal{F})$. Neste contexto, dizemos que $\Omega$ é um \textit{espaço amostral} e que os elementos de $\mathcal{F}$  (subconjuntos de $\Omega$) são \textit{eventos aleatórios}, ou simplesmente \textit{eventos}.
        \end{enumerate}
\end{defi}

\begin{prop}
    Sejam $(\Omega, \mathcal{F}, \Pb)$ um espaço de probabilidade, $\left\{ A_n \right\}_{n \in \N} \subseteq \mathcal{F}$ e $A, B \in \mathcal{F}$. Valem as seguintes afirmações.
        \begin{enumerate}[leftmargin=*, align=left, label=\textbf{(\alph*)}]
            \item $\Pb (\emptyset) = 0$;
            \item $\Pb (A^C) = 1 - \Pb (A)$;
            \item se $A \subseteq B$, então $\Pb(B \setminus A) = \Pb(B) - \Pb(A) \geq 0$;
            \item $0 \leq \Pb (A) \leq 1$;
            \item $\ds \Pb \left( \bigcup_{i \in \N} A_i \right) \leq \sum_{i \in \N} \Pb (A_i)$;
            \item $\Pb (A \cup B) = \Pb(A) + \Pb(B) - \Pb(A \cap B) $.
        \end{enumerate}
\end{prop}

\begin{proof}
    \begin{enumerate}[leftmargin=*, align=left, label=\textbf{(\alph*)}]
        \item Temos $\Pb(\emptyset) \geq 0$. Notando que $\ds \Pb(\emptyset) = \Pb\left( \bigcup_{n \in \N} \emptyset \right) = \sum_{n \in \N} \Pb(\emptyset)$, só pode ser $\Pb(\emptyset) = 0$. \blackproof
    \end{enumerate}
\end{proof}

\begin{teo}
    Seja $\Omega$ um conjunto enumerável. Se $p: \Omega \to \R$ é uma função tal que $p(\omega) \geq 0$ para todo $\omega \in \Omega$ e
        \[
            \sum_{\omega \in \Omega} p(\omega) = 1,
        \]
    então a tripla $(\Omega, \mathcal{P}(\Omega), \Pb)$, onde $\Pb : \mathcal{P}(\Omega) \to \R$ é a função definida por
        \[
            \Pb(A) := \sum_{\omega \in A} p(\omega)
        \]
    para todo $A \in \mathcal{P}(\Omega)$, é um espaço de probabilidade.
\end{teo}

\begin{proof}
    \blackproof
\end{proof}

\begin{defi}
    Sejam $(\Omega, \mathcal{F}, \Pb)$ um espaço de probabilidade, $\left\{ A_n \right\}_{n \in \N} \subseteq \mathcal{F}$ e $A \in \mathcal{F}$.
        \begin{enumerate}[leftmargin=*, align=left, label=\textbf{(\alph*)}]
            \item Denota-se
                \[
                    A_1 \subseteq A_2 \subseteq A_3 \subseteq \cdots \quad \text{e} \quad \bigcup_{n \in \N} A_n = A
                \]
            por $A_n \uparrow A$.
            \item Denota-se
                \[
                    A_1 \supseteq A_2 \supseteq A_3 \supseteq \cdots \quad \text{e} \quad \bigcap_{n \in \N} A_n = A
                \]
            por $A_n \downarrow A$.
        \end{enumerate}
\end{defi}

\begin{prop}
    Sejam $(\Omega, \mathcal{F}, \Pb)$ um espaço de probabilidade, $\left\{ A_n \right\}_{n \in \N} \subseteq \mathcal{F}$ e $A \in \mathcal{F}$.
        \begin{enumerate}[leftmargin=*, align=left, label=\textbf{(\alph*)}]
            \item Se $A_n \uparrow A$, então $\ds \lim_{n \to +\infty} \Pb(A_n) = \Pb(A)$.
            
            \item Se $A_n \downarrow A$, então $\ds \lim_{n \to +\infty} \Pb(A_n) = \Pb(A)$. 
        \end{enumerate}
\end{prop}

\begin{proof}
    \leavevmode
        \begin{enumerate}[leftmargin=*, align=left, label=\textbf{(\alph*)}]
            \item Pois tome $ A_0 := \emptyset$ e defina $\left\{ B_n \right\}_{n \in \N}$ por $B_n := A_n \setminus A_{n-1}$ para todo $n \in \N$. É fácil provar que $\left\{ B_n \right\}_{n \in \N}$ é disjunto e que $\bigcup_{n \in \N} B_n = \bigcup_{n \in \N} A_n  $. Com isso,
                \begin{align*}
                    \Pb(A) = \Pb\left( \bigcup_{n \in \N} A_n \right) &= \Pb\left( \bigcup_{n \in \N} B_n \right) = \sum_{k=1}^{\infty} \Pb(B_k) \\
                    &= \sum_{k=1}^{\infty} \Pb (A_k \setminus A_{k-1}) = \lim_{n \to +\infty} \sum_{k=1}^{n} \Pb (A_k \setminus A_{k-1}) \\
                    &= \lim_{n \to +\infty} \sum_{k=1}^{n}[\Pb(A_k) - \Pb(A_{k-1})] = \lim_{n \to +\infty} \Pb(A_n),
                \end{align*}
            como queríamos. \blackproof
            
            \item Se $A_n \downarrow A$, então
                $
                    A_1 \supseteq A_2 \supseteq A_3 \supseteq \cdots
                $
            e
                $ \ds
                    \bigcap_{n \in \N} A_n = A.
                $
            Observando que
                \[
                    A_1 \supseteq A_2 \supseteq A_3 \supseteq \cdots \Leftrightarrow A_1^C \subseteq A_2^C \subseteq A_3^C \subseteq \cdots,
                \]
            trivialmente temos $\ds A_n^C \uparrow \bigcup_{n \in \N} A_n^C = A^C$, donde $\ds \lim_{n  \to +\infty} \Pb \left(A_n^C\right) = \Pb \left(A^C\right) $ pelo item anterior. Com isso,
                \[
                    \lim_{n \to +\infty} \Pb (A_n) = \lim_{n \to +\infty} [1 - \Pb \left( A_n^C \right)] = 1 - \Pb\left(A^C\right) = \Pb(A),
                \]
            como queríamos. \blackproof
        \end{enumerate}
\end{proof}

\begin{defi}
    Seja $(\Omega, \mathcal{F}, \Pb)$ um espaço de probabilidade. \textit{A probabilidade condicional de $A \in \mathcal{F}$ dado $B \in \mathcal{F}$} é definida como
        \[
            \Pb(A|B) :=
                \begin{cases}
                    \dfrac{\Pb(A \cap B)}{\Pb (B)} & \text{ se } \Pb(B) > 0; \\
                    \Pb(A) & \text{ se } \Pb(B) = 0.
                \end{cases}
        \]
\end{defi}

\begin{prop}
    Sejam $(\Omega, \mathcal{F}, \Pb)$ um espaço de probabilidade e $B \in \mathcal{F}$. A terna $(\Omega, \mathcal{F}, \bar{\Pb})$, onde $\bar{\Pb} : \mathcal{F} \to \R$ é definida por $\bar{\Pb}(A) := \Pb(A|B)$ para todo $A \in \mathcal{F}$, é um espaço de probabilidade.
\end{prop}

\begin{proof}
    Ver \cite{rolla}, proposição 2.2. \blackproof
\end{proof}

\begin{teo}[Regra do produto]
    Sejam $(\Omega, \mathcal{F}, \Pb)$ um espaço de probabilidade e $A_1, A_2, \ldots, A_n \in \mathcal{F}$. Tem-se
        \[
            \Pb\left( \bigcap_{i = 1}^{n} A_i \right) = \Pb(A_1) \cdot \prod_{i=2}^{n} \Pb\left(A_i | \bigcap_{j=1}^{n-1} A_j \right).
        \]
\end{teo}

\begin{proof}
    Basta fazer indução em $n$. Ver \cite{rolla}, teorema 2.5. \blackproof
\end{proof}

\begin{teo}[Probabilidade total]
    Sejam $(\Omega, \mathcal{F}, \Pb)$ um espaço de probabilidade e $A \in \mathcal{F}$. Se $\left\{ B_n \right\}_{n \in \N} \subseteq \mathcal{F}$ é uma partição de $\Omega$, então
        \[
            \Pb(A) = \sum_{n \in \N} \Pb(B_n) \Pb(A | B_n).
        \]
\end{teo}

\begin{proof}
    \blackproof
\end{proof}

\begin{cor}[fórmula de Bayes]
    Sejam     
        \begin{enumerate}[leftmargin=*, align=left, label=\textbf{(\alph*)}]
            \item Oi
                \[
                    \Pb(B|A) = \dfrac{\Pb(A|B)}{\Pb(A)} \cdot \Pb(B).
                \]
            \item Se $\left\{ B_n \right\}_{n \in \N} \subseteq \mathcal{F}$ é uma partição de $\Omega$, então
                \[
                    \Pb(B_j | A) = \dfrac{\Pb(B_j) \Pb(B_j | A)}{\ds \sum_{n \in \N} \Pb(B_n) \Pb(A | B_n)}
                \]
            para todo $j \in \N$.
        \end{enumerate}
\end{cor}

\begin{proof}
    \blackproof
\end{proof}

\begin{defi}
    Dois eventos $A, B \in \mathcal{F}$ são \textit{independentes} se $\Pb(A \cap B) = \Pb(A) \Pb(B)$.
\end{defi}

\begin{prop}
    Dois eventos $A, B \in \mathcal{F}$ são independentes se , e somente se, $\Pb(A | B) = \Pb(A)$.
\end{prop}

\begin{proof}
    \blackproof
\end{proof}

\begin{defi}
    Seja $J$ um conjunto de índices.
        \begin{enumerate}[leftmargin=*, align=left, label=\textbf{(\alph*)}]
            \item Eventos independentes dois a dois.
            \item Eventos independentes
        \end{enumerate}
\end{defi}

\section{Variáveis Aleatórias}

Seja $(\Omega, \mathcal{F}, \Pb)$ um espaço de probabilidade.

\begin{defi}
    Uma \textit{variável aleatória} é uma função $X : \Omega \to \R$ tal que
        $
            X^{-1}(I) \in \mathcal{F}
        $
    para todo intervalo $I \subseteq \R$.
\end{defi}

\begin{defi}
    Seja $X : \Omega \to \R$ uma variável aleatória. A \textit{função de distribuição acumulada} de $X$ é a função $F_X : \R \to \R$ definida por $F_X(x) = \Pb(X \in \left]-\infty,x \right])$ para todo $x \in \R$.
\end{defi}

\begin{prop}
    Sejam $(\Omega, \mathcal{F}, \Pb)$ um espaço de probabilidade, $X : \Omega \to \R$ uma variável aleatória e $F_X : \R \to [0,1]$ a FDA de $X$. Valem as seguintes afirmações.
        \begin{enumerate}[leftmargin=*, align=left, label=\textbf{(\alph*)}]
            \item $F_X$ é crescente.
            \item $F_X$ é contínua à direita.
            \item $\ds \lim_{x \to - \infty} F_X (x) = 0$ e $\ds \lim_{x \to + \infty} F_X (x) = 1$.
        \end{enumerate}
\end{prop}

\begin{defi}
    Uma função $F : \R \to \R$ é uma \textit{função de distribuição} se é crescente, contínua à direita e $\ds \lim_{x \to - \infty} F(x) = 0$ e $\ds \lim_{x \to + \infty} F(x) = 1$.
\end{defi}

\begin{defi}
    Uma função $f: \R \to \R$ é uma \textit{função de densidade} se $f \geq 0$ e
        \[
            \int_{-\infty}^{+\infty} f(x) \, dx = 1.
        \]
\end{defi}

\begin{prop}
    Se $f: \R \to \R$ é uma função de densidade, então a função $F: \R \to \R$ definida por
        \[
            F(x) = \int_{-\infty}^{x} f(t) \, dt
        \]
    para todo $x \in \R$ é uma função de distribuição.
\end{prop}

\begin{proof}
    \blackproof
\end{proof}

\begin{defi}
    Duas variáveis aleatórias $X,Y: \Omega \to \R$ são \textit{independentes} se os eventos $[X \in I_1], [Y \in I_2] \in \mathcal{F}$ são independentes para quaisquer intervalos $I_1, I_2 \subseteq \R$.
\end{defi}

\subsection{Distribuições Discretas}

\begin{defi}
    Uma variável aleatória $X: \Omega \to \R$ é \textit{discreta} se existe $A \subsetneq \R$ enumerável tal que $\Pb(X \in A) = 1$.
\end{defi}

\begin{defi} \label{pb.defi:discretas}
    Seja $X : \Omega \to \R$ uma variável aleatória.
        \begin{enumerate}[leftmargin=*, align=left, label=\textbf{(\alph*)}]
            \item Dizemos que $X$ tem distribuição \textit{Bernoulli} de parâmetro $p \in \left]0,1\right[$ se $\Pb(X=1) = p$ e $\Pb(X=0) = 1-p$. Isso é denotado por $X \sim \mathrm{Bernoulli}(p)$.
            \item Dizemos que $X$ tem distribuição \textit{geométrica} de parâmetro $p \in \left] 0,1\right[$ se $\Pb(X=k) = p(1-p)^{k-1}$ para todo $k \in \N$. Isso é denotado por $X \sim \mathrm{Geom}(p)$.
            \item Dizemos que $X$ tem distribuição \textit{binomial} de parâmetros $n \in \N$ e $p \in \left] 0,1\right[$ se
                \[
                    \Pb(X=k) = \binom{n}{k}p^k(1-p)^{n-k}
                \]
            para todo $k \in \N$. Isso é denotado por $X \sim \mathrm{Binom}(n,p)$.
            \item Dizemos que $X$ tem distribuição \textit{Poisson} de parâmetro $\lambda \in \R_{>0}$ se
                \[
                    \Pb(X=k) = e^{-\lambda} \dfrac{\lambda^k}{k!}
                \]
            para todo $k \in \N$. Isso é denotado por $X \sim \mathrm{Poisson}(\lambda)$.
        \end{enumerate}
\end{defi}

\begin{prop}
    As distribuições da definição \eqref{pb.defi:discretas} são discretas.
\end{prop}

\begin{proof}
    \leavevmode
        \begin{enumerate}[leftmargin=*, align=left, label=\textbf{(\alph*)}]
            \item 
        \end{enumerate}
\end{proof}



\subsection{Distribuições Absolutamente Contínuas}

\begin{defi}
    Uma variável aleatória $X: \Omega \to \R$ é \textit{absolutamente contínua} se existe uma função contínua por partes $f: \R \to \R$ tal que $f(x) \geq 0$ para todo $x \in \R$ e
        \[
            \Pb(X \in I) = \int_{I} f
        \]
    para todo intervalo $I \subseteq \R$.
\end{defi}












\part{Outros}

%!TEX root = main.tex

\chapter{Shoenfield}

%remoção de umas provas em 18/02/2026

\begin{defi}[Linguagens de Primeira Ordem]
    \leavevmode
        \begin{enumerate}[leftmargin=*, align=left, label=\textbf{(\alph*)}]
            \item Um \textit{alfabeto} é uma coleção infinita de símbolos distintos, nenhum deles propriamente contido em outro, separados nas seguintes categorias:
                \begin{enumerate}[label=\roman*.]
                    \item Conectivos: $\lor$, $\neg$.
                    \item Quantificador existencial: $\exists$.
                    \item Variáveis, uma para cada inteiro positivo $n$: $u_1, u_2, \ldots, u_n, \ldots$.
                    \item Símbolos de função: para cada natural $n$, uma coleção de símbolos de função $n$-ários. Os símbolos de função $0$-ários são chamados de \textit{constantes}.
                    \item Símbolos de predicado: para cada natural $n$, uma coleção de símbolos de predicado $n$-ários.
                    \item Símbolo de predicado binário de igualdade $=$.
                \end{enumerate}
            Símbolos de função e de predicado distintos de $=$ são chamados de símbolos \textit{não lógicos}. Os demais são chamados de símbolos lógicos. Usamos $\mathbf{x}$, $\mathbf{y}$, $\mathbf{z}$ e $\mathbf{w}$ para denotar variáveis sintáticas que variam entre variáveis, $\mathbf{f}$ e $\mathbf{g}$ para denotar variáveis sintáticas que variam entre símbolos de funções, $\mathbf{p}$ e $\mathbf{q}$ para denotar variáveis sintáticas que variam entre símbolos de predicado e $\mathbf{e}$ para denotar uma variável sintática que varia entre constantes.
            \item Os \textit{termos} de um alfabeto são definidos do seguinte modo:
                \begin{enumerate}[label=\roman*.]
                    \item toda variável é um termo;
                    \item se $\mathbf{u}_1, \ldots, \mathbf{u}_n$ são termos e $\mathbf{f}$ é $n$-ário, então $\mathbf{f} \mathbf{u}_1 \ldots \mathbf{u}_n$ é um termo.
                \end{enumerate}
            Termos são apenas expressões que denotam indivíduos. Note que constantes também são termos. Usamos $\mathbf{a}$, $\mathbf{b}$, $\mathbf{c}$ e $\mathbf{d}$ para denotar variáveis sintáticas que variam entre termos.
            \item As \textit{fórmulas} de um alfabeto são definidas do seguinte modo:
                \begin{enumerate}[label=\roman*.]
                    \item se $\mathbf{u}_1, \ldots, \mathbf{u}_n$ são termos e $\mathbf{p}$ é $n$-ário, então $\mathbf{p} \mathbf{u}_1 \ldots \mathbf{u}_n$ é uma fórmula;
                    \item se $\mathbf{u}$ é uma fórmula, então $\neg \mathbf{u}$ é uma fórmula;
                    \item se $\mathbf{u}$ e $\mathbf{v}$ são fórmulas, então $\lor \mathbf{u} \mathbf{v}$ é uma fórmula;
                    \item se $\mathbf{u}$ é uma fórmula, então $\exists \mathbf{x} \mathbf{u}$ é uma fórmula.
                \end{enumerate}
            As fórmulas do tipo i. são chamadas de \textit{atômicas}.
            \item Uma \textit{linguagem de primeira ordem} $\mathcal{L}$ consiste num alfabeto como descrito no item (a) e termos ($\mathcal{L}$-termos) e fórmulas ($\mathcal{L}$-fórmulas) como descritos nos itens (b) e (c). Uma linguagem de primeira ordem fica então completamente determinada pelos seus símbolos não lógicos.
        \end{enumerate}
\end{defi}

\begin{defi}
    Um \textit{designador} é uma expressão que é um termo ou uma fórmula.
\end{defi}

\begin{prop}
    Todo designador tem a forma $\mathbf{u} \mathbf{v}_1 \ldots \mathbf{v}_n$, onde $\mathbf{u}$ é um símbolo do alfabeto, $\mathbf{v}_1, \ldots, \mathbf{v}_n$ são designadores e $n$ é um natural determinado por $\mathbf{u}$.  
\end{prop}

\begin{proof}
    Se $\mathbf{d}$ é um designador, então $\mathbf{d}$ 
\end{proof}

\begin{defi}
    Duas expressões são \textit{compatíveis} se uma delas puder ser obtida adicionando alguma expressão (possivelmente a expressão vazia) ao final da outra.
\end{defi}

\begin{prop}
    Sejam $\mathbf{u}$, $\mathbf{u}'$, $\mathbf{v}$ e $\mathbf{v}'$ expressões.
        \begin{enumerate}[leftmargin=*, align=left, label=\textbf{(\alph*)}]
            \item Se \( \mathbf{u} \mathbf{v} \) e \( \mathbf{u}'\mathbf{v}' \) são compatíveis, então \( \mathbf{u} \) e \( \mathbf{u}' \) são compatíveis; 
            \item se \( \mathbf{u} \mathbf{v} \) e \( \mathbf{u} \mathbf{v}' \) são compatíveis, então \( \mathbf{v} \) e \( \mathbf{v}' \) são compatíveis.
        \end{enumerate}
\end{prop}

\begin{proof}
\end{proof}

\begin{lem}
    Seja $n$ um natural. Se $\mathbf{u}_1, \ldots, \mathbf{u}_n$ e $\mathbf{u}'_1, \ldots, \mathbf{u}'_n$ são designadores, e $\mathbf{u}_1 \ldots \mathbf{u}_n$ e $\mathbf{u}'_1 \ldots \mathbf{u}'_n$ são compatíveis, então $\mathbf{u}'_i$ é $\mathbf{u}_i$, para $i = 1, \ldots, n$.
\end{lem}

\begin{proof}
    Faremos indução no comprimento de \( \mathbf{u}_1 \ldots \mathbf{u}_n \). Escreva \( \mathbf{u}_1 \) como \( \mathbf{v} \mathbf{v}_1 \ldots \mathbf{v}_k \), onde \( \mathbf{v} \) é um símbolo de índice \( k \) e \( \mathbf{v}_1, \ldots, \mathbf{v}_k \) são designadores. Como \( \mathbf{u}_1' \) começa com \( \mathbf{v} \), ele tem a forma \( \mathbf{v} \mathbf{v}_1' \ldots \mathbf{v}_k' \), onde \( \mathbf{v}_1', \ldots, \mathbf{v}_k' \) são designadores. Com isso, temos que \( \mathbf{u}_1 \) é compatível com \( \mathbf{u}_1' \), donde \( \mathbf{v}_1 \ldots \mathbf{v}_k \) é compatível com \( \mathbf{v}_1' \ldots \mathbf{v}_k' \). Daí, pela hipótese de indução, \( \mathbf{v}_i \) é \(\mathbf{v}_i' \) para \( i = 1, \ldots, k \), donde \( \mathbf{u}_1 \) é \( \mathbf{u}_1' \). Com isso, temos que \( \mathbf{u}_2 \ldots \mathbf{u}_n \) é compatível com \( \mathbf{u}_2' \ldots \mathbf{u}_n' \); assim, pela hipótese de indução, \( \mathbf{u}_i \) é \( \mathbf{u}_i' \) para \( i = 2, \ldots, n \).
\end{proof}

\begin{teo}[Formação]
    
\end{teo}

\begin{lem}
    
\end{lem}

\begin{teo}[Ocorrência]
    
\end{teo}


%%%%%%%%%%%%%%%%%%%%%%%%%%%%%%

\nocite{*}
\printbibliography[heading=bibintoc]

\end{document}