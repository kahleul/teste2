%!TEX root = main.tex

\chapter{Lógica Proposicional}

Seguimos \cite{herculespaulovich}.

\begin{defi}
    \leavevmode
    \begin{enumerate}[leftmargin=*, align=left, label=\textbf{(\alph*)}]
            \item O \textit{alfabeto proposicional} $\mathrm{Alf}$ é uma coleção infinita de símbolos distintos, nenhum deles propriamente contido em outro, separados nas seguintes categorias:
                \begin{enumerate}[label=\roman*.]
                    \item Conectivos: $\neg$, $\rightarrow$.
                    \item Parênteses: $($, $)$.
                    \item Variáveis proposicionais: $p_1, p_2, p_3 \ldots, p_i, \ldots$.
            \end{enumerate}
            \item As \textit{fórmulas} sobre $\mathrm{Alf}$ são definidas indutivamente pelas seguintes regras:
                \begin{enumerate}[label=\roman*.]
                    \item se $\mathbf{p}$ é uma variável proposicional, então $\mathbf{p}$ é uma fórmula.
                    \item se $\mathbf{A}$ e $\mathbf{B}$ são fórmulas, então $(\neg \mathbf{A})$ e $(\mathbf{A} \rightarrow \mathbf{B})$ são fórmulas;
                    \item todas as fórmulas são obtidas por um número finito de aplicações das regras acima. 
            \end{enumerate}
            O conjunto de todas as fórmulas é denotado por $\mathrm{Form}$, enquanto o conjunto de todas as fórmulas atômicas é denotado por $\mathrm{Form}_{\mathrm{At}}$ 
        \item A \textit{linguagem proposicional} é o par $\mathcal{L} := (\mathrm{Alf}, \mathrm{Form})$
    \end{enumerate}
\end{defi}

\begin{defi}
    \leavevmode
        \begin{enumerate}[leftmargin=*, align=left, label=\textbf{(\alph*)}]
            \item Um \textit{sistema de dedução proposicional} é uma tripla $(\mathcal{L}, \mathrm{Ax}, \mathrm{R})$, onde $\mathcal{L}$ é a linguagem proposicional, $\mathrm{Ax}$ é um conjunto de esquemas de axiomas e $R$ é um conjunto de regras de inferência.
            \item A Lógica Proposicional é o sistema $\mathcal{L}_{P} := (\mathcal{L}, \Lambda, \mathrm{MP})$, onde $\Lambda$ é um conjunto formado pelos esquemas de axiomas 
                \begin{enumerate}[label=$\mathrm{Ax}_{\arabic*}$.]
                    \item $(\mathbf{A} \rightarrow (\mathbf{B} \rightarrow \mathbf{A}))$
                    \item $((\mathbf{A} \rightarrow (\mathbf{B} \rightarrow \mathbf{C})) \rightarrow ((\mathbf{A} \rightarrow \mathbf{B})\rightarrow(\mathbf{A} \rightarrow \mathbf{C})))$
                    \item $(((\neg \mathbf{B}) \rightarrow (\neg \mathbf{A})) \rightarrow ( ((\neg \mathbf{B}) \rightarrow \mathbf{A}) \rightarrow \mathbf{B}))$
                \end{enumerate}
            e MP é a regra de inferência \textit{Modus Ponens}, a saber,
                \[
                    \mathrm{MP} := \{(\{\mathbf{A}, (\mathbf{A} \rightarrow \mathbf{B})\}, \mathbf{B} ) : \mathbf{A}, \mathbf{B} \in \mathrm{Form}\}.
                \]
        \end{enumerate}
\end{defi}

\begin{defi}
    Sejam $\Delta \subseteq \mathrm{Form}$ e $\mathbf{A} \in \mathrm{Form}$. 
        \begin{enumerate}[leftmargin=*, align=left, label=\textbf{(\alph*)}]
            \item Uma \textit{dedução} de $\mathbf{A}$ a partir de $\Delta$ é uma sequência $(A_1, \ldots, A_n)$ tal que $A_n \equiv \mathbf{A}$ e, para cada $k \in [n]$, vale  pelo menos uma das seguintes afirmações.
                \begin{enumerate}
                    \item $A_k \in \Lambda$.
                    \item $A_k \in \Delta$.
                    \item Existem índices $i,j < k$ tais que $A_k$ é obtida de $A_i$ e $A_j$ via $\mathrm{MP}$.
                \end{enumerate}
            Isso é denotado por $\Delta \vdash \mathbf{A}$.
            \item Dizemos que $\mathbf{A}$ é uma \textit{consequência sintática} de $\Delta$ se $\Delta \vdash \mathbf{A}$.
            \item Dizemos que $\mathbf{A}$ é um \textit{teorema} se $\emptyset \vdash \mathbf{A}$. Isso é denotado por $\vdash \mathbf{A}$.
        \end{enumerate}
\end{defi}

\begin{prop}
    $\vdash (\mathbf{A} \rightarrow \mathbf{A})$.
\end{prop}

\begin{proof}
    Pois tome:
        \[
            \begin{array}{rll}
                1. & ((\mathbf{A} \rightarrow ((\mathbf{A} \rightarrow \mathbf{A}) \rightarrow \mathbf{A})) \rightarrow ((\mathbf{A} \rightarrow (\mathbf{A} \rightarrow \mathbf{A})) \rightarrow (\mathbf{A} \rightarrow \mathbf{A}) )) & \mathrm{Ax}_2 \\
                2. & (\mathbf{A} \rightarrow ((\mathbf{A} \rightarrow \mathbf{A}) \rightarrow \mathbf{A})) & \mathrm{Ax}_1 \\
                3. & ((\mathbf{A} \rightarrow (\mathbf{A} \rightarrow \mathbf{A})) \rightarrow (\mathbf{A} \rightarrow \mathbf{A}) ) & \text{MP}(1,2) \\
                4. & (\mathbf{A} \rightarrow (\mathbf{A} \rightarrow \mathbf{A})) & \mathrm{Ax}_1 \\
                5. & (\mathbf{A} \rightarrow \mathbf{A}) & \text{MP}(4,5)
            \end{array}
        \]
\end{proof}

\begin{teo}[da Dedução] \label{teo.fund:deduçãoproposicional}
    Sejam $\Delta \subseteq \mathrm{Form}$ e $\mathbf{A}, \mathbf{B} \in \mathrm{Form}$.
        \begin{enumerate}[leftmargin=*, align=left, label=\textbf{(\alph*)}]
            \item Se $\Delta \cup \{\mathbf{A}\} \vdash \mathbf{B}$, então $\Delta \vdash (\mathbf{A} \rightarrow \mathbf{B})$.
            \item Se $\Delta \vdash (\mathbf{A} \rightarrow \mathbf{B})$, então $\Delta \cup \{\mathbf{A}\} \vdash \mathbf{B}$.
        \end{enumerate}
\end{teo}

\begin{proof}
    \leavevmode
        \begin{enumerate}[leftmargin=*, align=left, label=\textbf{(\alph*)}]
            \item Façamos indução no número de fórmulas que ocorrem na dedução de $\mathbf{B}$ a partir de $\Delta \cup \{\mathbf{A}\}$. Se $(A_1)$ é uma dedução de $\mathbf{B}$, então $A_1 \equiv \mathbf{B}$.
                \begin{enumerate}[label=\roman*.]
                    \item Se $\mathbf{B} \in \Lambda$, então $\Delta \vdash \mathbf{B}$, e como $\Delta \vdash (\mathbf{B} \rightarrow (\mathbf{A} \rightarrow \mathbf{B}))$, temos, via MP, que $\Delta \vdash (\mathbf{A} \rightarrow \mathbf{B})$.
                    \item Se $\mathbf{B} \in \Delta$, então $\Delta \vdash \mathbf{B}$, e como $\Delta \vdash (\mathbf{B} \rightarrow (\mathbf{A} \rightarrow \mathbf{B}))$, temos, via MP, que $\Delta \vdash (\mathbf{A} \rightarrow \mathbf{B})$.
                    \item Se $\mathbf{B} \equiv \mathbf{A}$, então de $\Delta \vdash (\mathbf{A} \rightarrow \mathbf{A})$ vem $\Delta \vdash (\mathbf{A} \rightarrow \mathbf{B})$.
                \end{enumerate}
            Agora, seja $(A_1, \ldots, A_n)$ uma dedução de $\mathbf{B}$ a partir de $\Delta \cup \{\mathbf{A}\}$ e suponha, por hipótese de indução, que o resultado vale para toda fórmula que pode ser deduzida a partir de $\Delta \cup \{\mathbf{A}\}$ por uma dedução com menos de $n$ fórmulas. Se $\mathbf{B} \in \Lambda$, $\mathbf{B} \in \Delta$ ou $\mathbf{B} \equiv \mathbf{A}$, então podemos deduzir $(\mathbf{A} \rightarrow \mathbf{B})$ a partir de $\Delta$ exatamente do mesmo modo que fizemos na base da indução. Suponha, então, que $\mathbf{B}$ é obtida de duas fórmulas de índices $<n$ via MP. Essas duas fórmulas têm as formas $\mathbf{C}$ e $(\mathbf{C} \rightarrow \mathbf{B})$, e como elas foram deduzidas de $\Delta \cup \{\mathbf{A}\}$ por menos de $n$ fórmulas, temos que $\Delta \vdash (\mathbf{A} \rightarrow \mathbf{C})$ e $\Delta \vdash (\mathbf{A} \rightarrow (\mathbf{C} \rightarrow \mathbf{B}))$. Com isso, podemos deduzir $(\mathbf{A} \rightarrow \mathbf{B})$ a partir de $\Delta$ do seguinte modo.
                \[
                    \begin{array}{rll}
                        & \vdots & \vdots \\
                        i. & \Delta \vdash (\mathbf{A} \rightarrow \mathbf{C}) & \\
                        & \vdots & \vdots \\
                        j. & \Delta \vdash (\mathbf{A} \rightarrow (\mathbf{C} \rightarrow \mathbf{B})) & \\[1ex]
                        k. & \Delta \vdash ((\mathbf{A} \rightarrow (\mathbf{C} \rightarrow \mathbf{B})) \rightarrow ((\mathbf{A} \rightarrow \mathbf{C}) \rightarrow (\mathbf{A} \rightarrow \mathbf{B}))) & \mathrm{Ax}_2 \\[1ex]
                        l. & \Delta \vdash ((\mathbf{A} \rightarrow \mathbf{C}) \rightarrow (\mathbf{A} \rightarrow \mathbf{B})) & \text{MP}(j, k) \\[1ex]
                        m. & \Delta \vdash (\mathbf{A} \rightarrow \mathbf{B}) & \text{MP}(i, l)
                \end{array}
                \]
            Assim, se $\Delta \cup \{\mathbf{A}\} \vdash \mathbf{B}$, então $\Delta \vdash (\mathbf{A} \rightarrow \mathbf{B})$. \blackproof
            \item Como $\Delta \vdash (\mathbf{A} \rightarrow \mathbf{B})$ e $\Delta \subseteq \Delta \cup \{\mathbf{A}\}$, temos $\Delta \cup \{\mathbf{A}\} \vdash (\mathbf{A} \rightarrow \mathbf{B})$. Daí, como $\Delta \cup \{\mathbf{A}\} \vdash \mathbf{A}$, temos que $\Delta \cup \{\mathbf{A}\} \vdash \mathbf{B}$. \blackproof
        \end{enumerate}
\end{proof}

\begin{prop} \label{prop.fund:propriedadesvdash}
    \leavevmode
        \begin{enumerate}[leftmargin=*, align=left, label=\textbf{(\alph*)}]
            \item $\{\mathbf{A} \rightarrow \mathbf{B}, \mathbf{B} \rightarrow \mathbf{C} \} \vdash \mathbf{A} \rightarrow \mathbf{C}$.
            \item $\vdash \mathbf{A} \leftrightarrow \neg \neg \mathbf{A}$.
        \end{enumerate}
\end{prop}

\begin{defi}
    \leavevmode
        \begin{enumerate}[leftmargin=*, align=left, label=\textbf{(\alph*)}]
            \item Uma \textit{valoração proposicional} é uma função $\bar{v} : \mathrm{Form}_{\mathrm{At}} \to \{\mathfrak{t}, \mathfrak{f} \}$.
            \item Uma \textit{valoração} é uma função $v: \mathrm{Form} \to \{\mathfrak{t}, \mathfrak{f} \}$ tal que
                \begin{enumerate}[label=\roman*.]
                    \item $v |_{\mathrm{Form}_{\mathrm{At}}} = \bar{v}$;
                    \item $v[(\neg \mathbf{A})] = \mathfrak{f}$ se, e somente se, $v(\mathbf{A}) = \mathfrak{t}$;
                    \item $v[(\mathbf{A} \rightarrow \mathbf{B})] = \mathfrak{f}$ se, e somente se, $v(\mathbf{A}) = \mathfrak{t}$ e $v(\mathbf{B}) = \mathfrak{f}$.
                \end{enumerate}
        \end{enumerate}
\end{defi}

\begin{defi}
    Uma fórmula $\mathbf{A}$ é \textit{válida}, ou \textit{tautológica}, se $v(\mathbf{A}) = \mathfrak{t}$ para toda valoração $v$. Isso é denotado por $\vDash \mathbf{A}$.
\end{defi}

\begin{lem} \label{lem.fund:axregrasvalidos}
    \leavevmode
        \begin{enumerate}[leftmargin=*, align=left, label=\textbf{(\alph*)}]
            \item Os axiomas de $\mathcal{L}_P$ são válidos.
            \item Se $\vDash \mathbf{A}$ e $\vDash (\mathbf{A} \rightarrow \mathbf{B})$, então $\vDash \mathbf{B}$.
        \end{enumerate}
\end{lem}

\begin{proof}
    \leavevmode
        \begin{enumerate}[leftmargin=*, align=left, label=\textbf{(\alph*)}]
            \item Suponha, por absurdo, que $\not\vDash\mathbf{A} \rightarrow (\mathbf{B} \rightarrow \mathbf{A})$. Isso só é possível se $v[(\mathbf{B} \rightarrow \mathbf{A})] = \mathfrak{f}$, o que, por sua vez, só é possível se $v(\mathbf{A}) = \mathfrak{f}$, o que contraria a hipótese. Logo, $\vDash\mathbf{A} \rightarrow (\mathbf{B} \rightarrow \mathbf{A})$. A prova de que os outros (esquemas de) axiomas são válidos segue analogamente. \blackproof
            \item Suponha, por absurdo, que $\not\vDash \mathbf{B}$. Logo, existe uma valoração $v$ tal que $v(\mathbf{B}) = \mathfrak{f}$. Para essa valoração, como $\vDash \mathbf{A}$, temos $v(\mathbf{A}) = \mathfrak{t}$, de modo que $v[(\mathbf{A} \rightarrow \mathbf{B})] = \mathfrak{f}$, o que contraria a hipótese. Com isso, $\vDash \mathbf{B}$. \blackproof
        \end{enumerate}
\end{proof}


\begin{teo}[da Correção] \label{teo.fund:correçãoproposicional}
    Se $\vdash \mathbf{A}$, então $\vDash \mathbf{A}$.    
\end{teo}

\begin{proof}
    Façamos indução no número de fórmulas que ocorrem na dedução de $\mathbf{A}$. Se $(A_1)$ é a dedução de $\mathbf{A}$, então $A_1 \equiv \mathbf{A}$ e $\mathbf{A}$ é um axioma, e como todos os axiomas são válidos (\cref{lem.fund:axregrasvalidos}), temos $\vDash \mathbf{A}$. Agora, seja $(A_1, \ldots, A_n)$ a dedução de $\mathbf{A}$ e suponha, por hipótese de indução, que o resultado vale para toda fórmula que pode ser deduzida por uma dedução com menos de $n$ fórmulas. Se $\mathbf{A}$ é um axioma, então $\vDash \mathbf{A}$. Suponha, então, que $\mathbf{A}$ é obtida de duas fórmulas de índices $<n$ via MP. Essas duas fórmulas têm as formas $\mathbf{B}$ e $(\mathbf{B} \rightarrow \mathbf{A})$, e como elas foram deduzidas por menos de $n$ fórmulas, temos que $\vDash \mathbf{B}$ e $\vDash (\mathbf{B} \rightarrow \mathbf{A})$. Daí, como a regra MP conserva validade (\cref{lem.fund:axregrasvalidos}), temos que $\vDash \mathbf{A}$. \blackproof
\end{proof}

\begin{lem}[Kalmár] \label{lem.fund:troca}
    Sejam $\mathbf{A}(p_1, \ldots, p_n) \in \mathrm{Form}$ e $v : \mathrm{Form} \to \{\mathfrak{t}, \mathfrak{f} \}$. Se
        \[
            p'_i :\equiv
                \begin{cases}
                    p_i, & \text{se } v(p_i) = \mathfrak{t} \\
                    (\neg p_i), & \text{se } v(p_i) = \mathfrak{f}
                \end{cases}
            \quad \text{ e } \quad
            \mathbf{A}' :\equiv
                \begin{cases}
                    \mathbf{A}, & \text{se } v(\mathbf{A}) = \mathfrak{t} \\
                    (\neg \mathbf{A}), & \text{se } v(\mathbf{A}) = \mathfrak{f}
                \end{cases},
        \]
    então $\{p'_1, \ldots, p'_n \} \vdash \mathbf{A}'$.
\end{lem}

\begin{proof}
    Façamos indução no número de conectivos que ocorrem em $\mathbf{A}$.
\end{proof}

\begin{teo}[da Completude] \label{teo.fund:completudeproposicional}
    Se $\vDash \mathbf{A}$, então $\vdash \mathbf{A}$.
\end{teo}

\begin{proof}
    Sejam $p_1, \ldots, p_n$ as variáveis proposicionais que ocorrem em $\mathbf{A}$. Pelo \cref{lem.fund:troca}, temos $\{p'_1, \ldots, p'_n \} \vdash \mathbf{A}'$ para toda valoração $v$, e como $\vDash \mathbf{A}$, temos $\mathbf{A}' \equiv \mathbf{A}$, de modo que $\{p'_1, \ldots, p'_n \} \vdash \mathbf{A}$. Agora, definindo
        \[
            v_1(p_i) :=
                \begin{cases}
                    v(p_i), & \text{se } i < n \\
                    \mathfrak{t}, & \text{se } i = n
                \end{cases}
            \quad \text{ e } \quad
            v_2(p_i) :=
                \begin{cases}
                    v(p_i), & \text{se } i < n \\
                    \mathfrak{f}, & \text{se } i = n
                \end{cases},
        \]
    cada $p'_i$, para $i<n$, fica bem definido. Como $v_2(p_n) = \mathfrak{f}$, vem $\{p'_1, \ldots, p'_{n-1}, (\neg p_n) \} \vdash \mathbf{A}$, de modo que, pelo \Cref{teo.fund:deduçãoproposicional}, temos $\{p'_1, \ldots, p'_{n-1}\} \vdash ((\neg p_n) \rightarrow \mathbf{A})$. Analogamente, como $v_1(p_n) = \mathfrak{t}$, então $\{p'_1, \ldots, p'_{n-1}\} \vdash (p_n \rightarrow \mathbf{A})$. Assim:
        \[
            \begin{array}{rll}
                1. & \{p'_1, \ldots, p'_{n-1}\} \vdash ((\neg p_n) \rightarrow \mathbf{A}) & p. \\
                2. & \{p'_1, \ldots, p'_{n-1}\} \vdash (p_n \rightarrow \mathbf{A}) & p. \\
                3. & \{p'_1, \ldots, p'_{n-1}\} \vdash (p_n \rightarrow \mathbf{A}) \rightarrow (((\neg p_n) \rightarrow \mathbf{A}) \rightarrow \mathbf{A}) & \eqref{prop.fund:propriedadesvdash} \\
                4. & \{p'_1, \ldots, p'_{n-1}\} \vdash (((\neg p_n) \rightarrow \mathbf{A}) \rightarrow \mathbf{A}) & \text{MP}(2,3) \\
                5. & \{p'_1, \ldots, p'_{n-1}\} \vdash \mathbf{A} & \text{MP}(1,4)
            \end{array}
        \]
    Com isso, eliminamos $p_n$. Repetindo esse processo (um número finito de vezes), eliminamos $p_{n-1}, \ldots, p_1$, obtendo por fim $\vdash \mathbf{A}$.
    \blackproof
\end{proof}


\begin{cor}[Adequação]
    $\vDash \mathbf{A}$ se, e somente se, $\vdash \mathbf{A}$.
\end{cor}

\begin{proof}
    Segue dos \cref{teo.fund:correçãoproposicional,teo.fund:completudeproposicional}. \blackproof
\end{proof}



coisas

\begin{defi}
    Sejam $\mathbf{A} \in \mathrm{Form}$ e $\Gamma \subseteq \mathrm{Form}$.
        \begin{enumerate}[leftmargin=*, align=left, label=\textbf{(\alph*)}]
            \item Um \textit{modelo} de $\mathbf{A}$ é uma valoração $v$ tal que $v(\mathbf{A}) = \mathfrak{t}$. Dizemos que $v$ \textit{satisfaz} $\mathbf{A}$. Isso é denotado por $v \vDash \mathbf{A}$.
            \item Um \textit{modelo} de $\Gamma$ é uma valoração $v$ tal que $v \vDash \mathbf{B}$ para todo $\mathbf{B} \in \Gamma$.
        \end{enumerate}
\end{defi}

\begin{defi}
    Sejam $\mathbf{A}, \mathbf{B} \in \mathrm{Form}$ e $\Gamma \subseteq \mathrm{Form}$.
        \begin{enumerate}[leftmargin=*, align=left, label=\textbf{(\alph*)}]
            \item Dizemos que $\mathbf{B}$ é uma \textit{consequência semântica} de $\mathbf{A}$ se todo modelo de $\mathbf{A}$ é também um modelo de $\mathbf{B}$. Isso é denotado por $\{\mathbf{A} \} \vDash \mathbf{B}$.
            \item Dizemos que $\mathbf{B}$ é uma \textit{consequência semântica} de $\Gamma$ se todo modelo de $\Gamma$ é também um modelo de $\mathbf{B}$. Isso é denotado por $\Gamma \vDash \mathbf{B}$.
        \end{enumerate}
\end{defi}

\begin{teo}
    Se $\Gamma \vdash \mathbf{A}$, então $\Gamma \vDash \mathbf{A}$.
\end{teo}

\begin{proof}
    O caso $\Gamma = \emptyset$ é simplesmente o teorema da correção \eqref{teo.fund:correçãoproposicional}. Suponha, então, que $\Gamma \neq \emptyset$. Se $\mathbf{C}_1, \ldots, \mathbf{C}_n$ são as fórmulas de $\Gamma$ que aparecem na dedução de $\mathbf{A}$, então $\{\mathbf{C}_1, \ldots, \mathbf{C}_n\} \vdash \mathbf{A}$, de modo que, por sucessivas aplicações do teorema da dedução \eqref{teo.fund:deduçãoproposicional}, temos $\vdash \mathbf{C}_1 \rightarrow \cdots \rightarrow \mathbf{C}_n \rightarrow \mathbf{A}$. Com isso, para toda valoração $v$ tal que $v(\mathbf{C}_i) = \mathfrak{t}$ para todo $i \in [n]$, temos $v(\mathbf{A}) = \mathfrak{t}$. Como $\{\mathbf{C}_1, \ldots, \mathbf{C}_n\} \subseteq \Gamma$, temos $v \vDash \mathbf{A}$ para toda valoração $v$ tal que $v \vDash \Gamma$, isto é, $\Gamma \vDash \mathbf{A}$. \blackproof
\end{proof}

\chapter{Teorias de Primeira Ordem}

\section{Linguagens de Primeira Ordem}

\begin{defi}[Linguagens de Primeira Ordem] 
    \leavevmode
    \begin{enumerate}[leftmargin=*, align=left, label=\textbf{(\alph*)}]
            \item Um \textit{alfabeto} é uma coleção infinita de símbolos distintos, nenhum deles propriamente contido em outro, separados nas seguintes categorias:
                \begin{enumerate}[label=\roman*.]
                    \item Conectivos: $\lor$, $\neg$.
                    \item Quantificador universal: $\forall$.
                    \item Parênteses: $($, $)$.
                    \item Variáveis, uma para cada inteiro positivo $n$: $v_1, v_2, \ldots, v_n, \ldots$. % O conjunto de símbolos de variáveis será denotado por Vars.
                    \item Símbolos de função: para cada inteiro positivo $n$, uma coleção de símbolos de função $n$-ários.
                    \item Símbolos de predicado: para cada inteiro positivo $n$, uma coleção de símbolos de predicado $n$-ários.
                    \item Símbolo predicado binário de igualdade: $=$.
                    \item Símbolos de constantes: uma coleção de símbolos.
            \end{enumerate}
            \item Os \textit{termos} correspondentes a um alfabeto são definidos do seguinte modo:
                \begin{enumerate}[label=\roman*.]
                    \item as variáveis são termos;
                    \item as constantes são termos;
                    \item se $t_1, t_2, \ldots, t_n$ são termos e $f$ é um símbolo de função $n$-ário, então $f(t_1, t_2, \ldots, t_n)$ é um termo;
                    \item todos os termos têm uma das formas acima.
                \end{enumerate}
            \item As \textit{fórmulas} correspondentes a um alfabeto são definidas do seguinte modo:
                \begin{enumerate}[label=\roman*.]
                    \item se $t_1$ e $t_2$ são termos, então $= (t_1,  t_2)$ é uma fórmula;
                    \item se $t_1, t_2, \ldots, t_n$ são termos e $R$ é um símbolo de predicado $n$-ário, então $R(t_1, t_2, \ldots, t_n)$ é uma fórmula;
                    \item se $\alpha$ e $\beta$ são fórmulas, então $(\neg \alpha)$ e $(\alpha \lor \beta)$ são fórmulas;
                    \item se $x$ é uma variável e $\alpha$ é uma fórmula, então $(\forall x)(\alpha)$ é uma fórmula;
                    \item todas as fórmulas têm uma das formas acima.
            \end{enumerate}
        As fórmulas como definidas nos itens i. e ii. são ditas \textit{atômicas}. A fórmula $\alpha$ que aparece no item iv. é chamada de \textit{escopo} do quantificador $\forall$.
        \item Uma \textit{linguagem de primeira ordem} $\mathcal{L}$ consiste num alfabeto como descrito no item (a) e termos ($\mathcal{L}$-termos) e fórmulas ($\mathcal{L}$-fórmulas) como descritos nos itens (b) e (c).
        \item Para especificar uma linguagem de primeira ordem $\mathcal{L}$, basta especificar quais são suas constantes, seus símbolos de função e seus símbolos de predicado:
            \[
                \mathcal{L} \quad \text{é} \quad \{ c_1, c_2, \ldots, f^{a(f_1)}_1, f^{a(f_2)}_2, \ldots, R^{a(R_1)}_1, R^{a(R_2)}_2, \ldots \},
            \]
        onde cada $c_i$ é um símbolo de constante, cada $f^{a(f_i)}_i$ é um símbolo de função de aridade $a(f_i)$ e cada $R^{a(R_i)}_i$ é um símbolo de predicado de aridade $a(R_i)$.
    \end{enumerate}
\end{defi}

\begin{teo}[Legibilidade única] \label{teo.fund:termform}
    Seja $\mathcal{L}$ uma linguagem de primeira ordem.
        \begin{enumerate}[leftmargin=*, align=left, label=\textbf{(\alph*)}]
            \item Todo termo tem uma, e exatamente uma, das formas i.-iii. da definição de termo.
            \item Toda fórmula tem uma, e exatamente uma, das formas i.-iv. da definição de fórmula.
        \end{enumerate}
\end{teo}

\begin{proof}
    Ver \cite{shoenfield1967}, página 16, ou ainda, \cite{learykristiansen2017logica}, página 18.
\end{proof}


\begin{defi}[Subtermos e subfórmulas]
    Sejam $t$ um $\mathcal{L}$-termo e $\varphi$ uma $\mathcal{L}$-fórmula.
        \begin{enumerate}[leftmargin=*, align=left, label=\textbf{(\alph*)}]
            \item Um \textit{subtermo} de $t$ é um $\mathcal{L}$-termo definido recursivamente do seguinte modo:
                \begin{enumerate}[label=\roman*.]
                    \item se $t$ é uma variável ou uma constante, então $t$ é o único subtermo de si mesmo;
                    \item se $t$ é da forma $ft_1 t_2 \ldots t_n$, onde $f$ é um símbolo funcional $n$-ário e $t_1, t_2, \ldots, t_n$ são $\mathcal{L}$-termos, então os subtermos de $t$ são $t$ e os subtermos de $t_1, t_2, \ldots, t_n$.
                \end{enumerate}
            \item Uma \textit{subfórmula} de $\varphi$ é uma $\mathcal{L}$-fórmula definida recursivamente do seguinte modo:
                \begin{enumerate}[label=\roman*.]
                    \item se $\varphi$ é atômica, então  $\varphi$ é a única subfórmula de si mesma;
                    \item se \(\varphi\) é da forma \((\neg\alpha)\) ou da forma \((\forall x)(\alpha)\), então as subfórmulas de \(\varphi\) são \(\varphi\) e as subfórmulas de \(\alpha\);
                    \item se \(\varphi\) é da forma \((\alpha \vee \beta)\), então as subfórmulas de \(\varphi\) são \(\varphi\) e as subfórmulas de \(\alpha\) e de \(\beta\).
                \end{enumerate}
        \end{enumerate}
\end{defi}

\begin{defi} % leary, def1.5.2.
    Sejam $\mathcal{L}$ uma linguagem de primeira ordem, $x$ uma variável e $\varphi$ uma fórmula. 
        \begin{enumerate}[leftmargin=*, align=left, label=\textbf{(\alph*)}]
            \item (Variáveis livres) Dizemos que $x$ é \textit{livre} em $\varphi$ se
                \begin{enumerate}[label=\roman*.]
                    \item $\varphi$ é atômica e $x$ ocorre em (é um símbolo) $\varphi$; ou
                    \item $\varphi$ é da forma $(\neg \alpha)$ e $x$ é livre na fórmula $\alpha$; ou  
                    \item $\varphi$ é da forma $(\alpha \lor \beta)$ e $x$ é livre em pelo menos uma das fórmulas $\alpha$ ou $\beta$; ou  
                    \item $\varphi$ é da forma $(\forall y)(\alpha)$, com $x$ diferente de $y$ e livre na fórmula $\alpha$.
                \end{enumerate}
            Equivalentemente, podemos dizer que uma ocorrência de $x$ é livre em $\varphi$ se $x$ não ocorre no escopo de uma subfórmula $(\forall x)(\alpha)$ de $\varphi$.
            \item (Variáveis ligadas) Dizemos que $x$ é \textit{ligada} em $\varphi$ se não for livre em $\varphi$. %, isto é, se ocorre no escopo de uma subfórmula $(\forall x)(\alpha)$ de $\varphi$.
            \item (Sentenças) Uma \textit{sentença} de $\mathcal{L}$, ou uma $\mathcal{L}$-\textit{sentença}, é uma $\mathcal{L}$-fórmula que não possui variáveis livres.
        \end{enumerate}
\end{defi}

\begin{defi}[Substituição]
    Sejam $\mathcal{L}$ uma linguagem de primeira ordem, $t$ um termo e $x$ uma variável.
        \begin{enumerate}[leftmargin=*, align=left, label=\textbf{(\alph*)}]
            \item Seja $u$ um termo. O termo $u_t^x$, que resulta da substituição de todas as ocorrências de $x$ em $u$ por $t$, é definido recursivamente do seguinte modo:
                \begin{enumerate}[label=\roman*.]
                    \item se $u$ é uma variável diferente de $x$, então $u_t^x$ é $u$;
                    \item se $u$ é a variável $x$, então $u_t^x$ é $t$;
                    \item se $u$ é uma constante, então $u_t^x$ é $u$;
                    \item se $u$ é da forma $f t_1 \ldots t_n$, então $u_t^x$ é $f {t_1}_t^x \ldots {t_n}_t^x$.
                \end{enumerate}
            \item Seja $\varphi$ uma fórmula. A fórmula $\varphi_t^x$, que resulta da substituição de todas as ocorrências de $x$ em $\varphi$ por $t$, é definida recursivamente do seguinte modo:
                \begin{enumerate}[label=\roman*.]
                    \item se $\varphi$ é da forma $(t_1=t_2)$, então $\varphi_t^x$ é $({t_1}_t^x={t_2}_t^x)$; 
                    \item se $\varphi$ é da forma $R t_1 \ldots t_n$, então $\varphi_t^x$ é $R {t_1}_t^x \ldots {t_n}_t^x$;
                    \item se $\varphi$ é da forma $(\neg \alpha)$, então $\varphi_t^x$ é $(\neg \alpha_t^x)$;
                    \item se $\varphi$ é da forma $(\alpha \lor \beta)$, então $\varphi_t^x$ é $(\alpha_t^x \lor \beta_t^x)$;
                    \item se $\varphi$ é da forma $(\forall y) (\alpha)$, então $\varphi_t^x$ é
                        \[
                            \begin{cases}
                                \varphi, & \text{ se } y \text{ é } x; \text{ ou } \\
                                (\forall y)(\alpha_t^x), & \text{ caso contrário. }
                            \end{cases}
                        \]
                \end{enumerate}
        \end{enumerate}
\end{defi}

\begin{defi}[Substituibilidade]
    Sejam $\mathcal{L}$ uma linguagem de primeira ordem, $\varphi$ uma fórmula, $t$ um termo e $x$ uma variável. Dizemos que $x$ é \textit{substituível} por $t$ em $\varphi$ se
        \begin{enumerate}[label=\roman*.]
            \item $\varphi$ é atômica; ou
            \item $\varphi$ é da forma $(\neg \alpha)$ e $x$ é substituível por $t$ em $\alpha$;
            \item $\varphi$ é da forma $(\alpha \lor \beta)$ e $x$ é substituível por $t$ em $\alpha$ e em $\beta$;
            \item $\varphi$ é da forma $(\forall y)(\alpha)$ e, exclusivamente, ou $x$ é ligada em $\varphi$, ou $y$ não ocorre em $t$ e $x$ é substituível por $t$ em $\alpha$.
        \end{enumerate}
\end{defi}

\section{Estruturas}

\begin{defi}
    Seja $\mathcal{L}$ uma linguagem de primeira ordem. Uma $\mathcal{L}$-\textit{estrutura} $\mathfrak{A}$ consiste num conjunto $A$, chamado de \textit{universo} de $\mathfrak{A}$, tal que
        \begin{enumerate}[label=\roman*.]
            \item para cada símbolo de constante $c$ de $\mathcal{L}$, há um elemento $c^{\mathfrak{A}}$ em $A$;
            \item para cada símbolo de função $n$-ário $f$ de $\mathcal{L}$, há uma função $f^{\mathfrak{A}} : A^n \to A$;
            \item para cada símbolo de relação $n$-ário $R$ de $\mathcal{L}$, há uma relação $R^{\mathfrak{A}}$ em $A$ (isto é, $R^{\mathfrak{A}} \subseteq A^n$).
        \end{enumerate}
\end{defi}

\begin{defi}
    Seja $\mathfrak{A}$ uma $\mathcal{L}$-estrutura de universo $A$. 
        \begin{enumerate}[leftmargin=*, align=left, label=\textbf{(\alph*)}]
            \item Uma \textit{valoração} é qualquer função $s : \mathrm{Vars} \to A$.
            \item Sejam $s$ uma valoração, $x$ uma variável e $a$ um elemento de $A$. Uma \textit{$x$-modificação de $s$} é definida como
                \[
                    s[x|a](v) := \begin{cases}
                        s(v) & \text{ se } v \text{ é uma variável diferente de } x \\
                        a & \text{ se } v \text{ é a variável } x
                    \end{cases}.
                \]
            \item Seja $s : \mathrm{Vars} \to A$ uma valoração. Uma \textit{valoração de termos gerada por $s$} é uma função $\bar{s} : \mathrm{Term} \to A$ definida recursivamente do seguinte modo:
                \begin{enumerate}[label=\roman*.]
                    \item se $t$ é uma variável, então $\bar{s}(t) = s(t)$;
                    \item se $t$ é um símbolo de constante $c$, então $\bar{s}(t) = c^{\mathfrak{A}}$;
                    \item se $t$ é da forma $ft_1 \ldots t_n$, onde $f$ é um símbolo funcional $n$-ário e $t_1, \ldots, t_n$ são termos, então $\bar{s}(t) = f^{\mathfrak{A}}(\bar{s}(t_1), \ldots, \bar{s}(t_n))$.
                \end{enumerate}
            \item Sejam $\varphi$ uma $\mathcal{L}$-fórmula e $s : \mathrm{Vars} \to A$ uma valoração. Dizemos que $\mathfrak{A}$ \textit{satisfaz $\varphi$ com relação a $s$}, denotando isso por $\mathfrak{A} \models  \varphi[s]$, se
                \begin{enumerate}[label=\roman*.]
                    \item $\varphi$ é da forma $=t_1 t_2$ e $\bar{s}(t_1)$ coincide com $\bar{s}(t_2)$; ou
                    \item $\varphi$ é da forma $Rt_1 \ldots t_n$ e $(\bar{s}(t_1), \ldots, \bar{s}(t_n))$ é um elemento de $R^{\mathfrak{A}}$; ou
                    \item $\varphi$ é da forma $(\neg \alpha)$ e $\mathfrak{A} \not\models \alpha[s]$; ou
                    \item $\varphi$ é da forma $(\alpha \lor \beta)$ e $\mathfrak{A} \models \alpha[s]$ ou $\mathfrak{A} \models \beta[s]$; ou
                    \item $\varphi$ é da forma $(\forall x)(\alpha)$ e $\mathfrak{A} \models \alpha [s[x|a]]$ para cada elemento $a$ de $A$.
                \end{enumerate}
            Se $\Gamma$ é um conjunto de $\mathcal{L}$-fórmulas, dizemos que $\mathfrak{A}$ satisfaz $\Gamma$ com relação a $s$, escrevendo $\mathfrak{A} \models \Gamma[s]$, se $\mathfrak{A} \models \gamma[s]$ para cada fórmula $\gamma$ em $\Gamma$.
        \end{enumerate}
\end{defi}

\begin{teo}
    Seja $\mathfrak{A}$ uma $\mathcal{L}$-estrutura.
        \begin{enumerate}[leftmargin=*, align=left, label=\textbf{(\alph*)}]
            \item Se $s_1$ e $s_2$ são valorações tais que $s_1 (v) = s_2 (v)$ para toda variável $v$ que ocorre num termo $t$, então $\bar{s}_1(t) = \bar{s}_2(t)$.
            \item Se $s_1$ e $s_2$ são valorações tais que $s_1 (v) = s_2 (v)$ para toda variável livre $v$ que ocorre na fórmula $\varphi$, então $\mathfrak{A} \models \varphi[s_1]$ se, e somente se, $\mathfrak{A} \models \varphi[s_2]$.
            \item Se $\psi$ é uma sentença, então ou $\mathfrak{A} \models \psi[s]$ para todas as valorações $s$, ou $\mathfrak{A} \models \psi[s]$ para nenhuma valoração $s$. 
        \end{enumerate}
\end{teo}

\begin{proof}
    Ver \cite{learykristiansen2017logica}, seção 1.7.
\end{proof}

\begin{defi}
    Seja $\mathfrak{A}$ uma $\mathcal{L}$-estrutura.
        \begin{enumerate}[leftmargin=*, align=left, label=(\alph*)]
            \item Seja $\varphi$ uma fórmula. Diremos que $\mathfrak{A}$ é um \textit{modelo} de $\varphi$, denotando isso por $\mathfrak{A} \models \varphi$, se $\mathfrak{A} \models \varphi[s]$ para toda função de atribuição de variável $s$.
            \item Seja $\Phi$ um conjunto de fórmulas. Diremos que $\mathfrak{A}$ \textit{modela} $\Phi$, denotando isso por $\mathfrak{A} \models \Phi$, se  $\mathfrak{A} \models \varphi$ para cada fórmula $\varphi$ de $\Phi$.
        \end{enumerate}
\end{defi}

\chapter{Teoria dos Conjuntos}

A linguagem (de primeira ordem) da teoria dos conjuntos, denotada por $\mathcal{L}_{ST}$, consiste em somente um símbolo de predicado binário dito de \textit{pertencimento} $\in$. A seguir apresentamos os axiomas de ZFC que constituem a teoria de primeira ordem da teoria dos conjuntos. Na primeira seção apresentamos e discutimos os axiomas básicos, deixando os axiomas do infinito (que garante a existência do conjunto dos números naturais $\omega$), da escolha (que tem muitas equivalências) e da substituição (fundamental para a teoria dos ordinais), os mais importantes, e mais complicados, para serem tratados nas próximas seções.

\section{Primeiros Axiomas} \label{sec1}

\subsection{O Axioma da Extensão}

\begin{ax}[da Extensão] \label{ax:1}
    Dois conjuntos são iguais se, e somente se, eles têm os mesmos elementos.
        \[
            \boxed{
                \forall x \forall y ( (x=y) \leftrightarrow \forall z((z \in x) \leftrightarrow (z \in y)))
            }
        \]
\end{ax}

\begin{defi}[Inclusão]
    Um conjunto $x$ está \textit{contido} num conjunto $y$, ou é um \textit{subconjunto} de $y$, se todo elemento de $x$ é um elemento de $y$:
        \[ \ds
            (x \subseteq y) \overset{\text{def}}{\leftrightarrow} \forall z ((z \in x) \rightarrow (z \in y)).
        \]
\end{defi}

\begin{obs}
    De maneira análoga podemos definir $\subsetneq$, $\not\subseteq$, $\supseteq$, etc. Além disso, com a definição de $\subseteq$, o axioma da extensão pode ser enunciado assim:
        \[
            \forall x \forall y ( (x=y) \leftrightarrow ((x \subseteq y) \land (y \subseteq x))).
        \] 
\end{obs}

\begin{prop}\footnote{Conforme a definição \eqref{defi.fund:ordemparcial}, isso significa que a relação $\subseteq$ é uma relação de ordem parcial.} \label{prop.fund:inclusaoparcial}
    \leavevmode
        \begin{enumerate}[leftmargin=*, align=left, label=\textbf{(\alph*)}]
            \item $\forall x (x \subseteq x)$.
            \item $\forall x \forall y ((x \subseteq y) \land (y \subseteq x) \rightarrow (x=y))$.
            \item $\forall x \forall y \forall z (((x \subseteq y) \land (y \subseteq z)) \rightarrow (x \subseteq z))$.
        \end{enumerate}
\end{prop}

\begin{proof}
    \leavevmode
        \begin{enumerate}[leftmargin=*, align=left, label=\textbf{(\alph*)}]
            \item Trivialmente, a implicação $(z \in x) \rightarrow (z \in x)$ é verdadeira para todo $z$. Logo, pela definição de inclusão $\subseteq$, temos $x \subseteq x$. \blackproof
            \item Se $x \subseteq y$ e $y \subseteq x$, então todo elemento de $x$ pertence a $y$ e todo elemento de $y$ pertence a $x$. Pelo axioma da extensão \eqref{ax:1}, $x=y$. \blackproof
            \item Suponha que $x \subseteq y$ e $y \subseteq z$. Seja $w \in x$. Como $x \subseteq y$, temos $w \in y$. Daí, como $y \subseteq z$, de $w \in y$ segue que $w \in z$. Assim, todo elemento de $x$ é elemento de $z$, isto é, $x \subseteq z$. \blackproof
        \end{enumerate}
\end{proof}

\subsection{O Axioma do Vazio}

\begin{defi}
    Um conjunto $x$ é \textit{vazio} se $\forall y (y \notin x)$.
\end{defi}

\begin{ax}[do Vazio] \label{ax:2}
    Existe um conjunto vazio.
        \[
            \boxed{
                \exists x \forall y (y \notin x)
            }
        \]
    Onde $(y \notin x) \overset{\text{def}}{\leftrightarrow} (\neg (y \in x))$.
\end{ax}

\begin{prop} \label{prop.fund:extensaounico1}
    Quaisquer dois conjuntos vazios são iguais.
        \[
            \forall x_1 \forall x_2 ( (\forall y (y \notin x_1) \land \forall y (y \notin x_2)) \rightarrow (x_1 = x_2) ).
        \]
\end{prop}

\begin{proof}
    Se $x_1 \neq x_2$, então ou existe $z \in x_1$ tal que $z \notin x_2$, ou existe $z \in x_2$ tal que $z \notin x_1$. Em ambos os casos, $x_1$ e $x_2$ não são vazios, uma contradição. Logo $x_1 = x_2$. \blackproof
\end{proof}

\begin{obs}
    O axioma do vazio \eqref{ax:2}, junto com a proposição \eqref{prop.fund:extensaounico1}, nos permite estabelecer que existe um único conjunto vazio:
        \[
            \exists x (\forall y  (y \notin x) \land \forall z(\forall y ( y \notin z ) \rightarrow (z=x) )).
        \]
    Podemos então falar \textit{do} conjunto vazio (em vez de \textit{de um}). Ele é denotado por $\emptyset$.
\end{obs}

\begin{prop}
    O conjunto vazio está contido em qualquer conjunto.
        \[
          \forall x (\emptyset \subseteq x)
        \]
     %\forall E \left( \forall y (y \notin E) \rightarrow \forall x (E \subseteq x) \right)$
\end{prop}

\begin{proof}
    Suponha que existe $x$ tal que $\emptyset \not\subseteq x$. Então existe $y \in \emptyset$ tal que $y \notin x$, uma contradição pois $\forall y (y \notin \emptyset)$. Logo $\forall x (\emptyset \subseteq x)$. \blackproof
\end{proof}

\begin{proof}
    Pela definição de $\subseteq$, precisamos provar que $\forall x (\forall y ( (y \in \emptyset) \rightarrow (y \in x) ) )$. Como a fórmula $y \in \emptyset$ é sempre falsa, $(y \in \emptyset) \rightarrow (y \in x)$ é sempre verdadeira, donde $\forall y ( (y \in \emptyset) \rightarrow (y \in x) )$ é sempre verdadeira, donde $\forall x (\forall y ( (y \in \emptyset) \rightarrow (y \in x) ) )$ é sempre verdadeira. Isto prova que a fórmula $\forall x (\emptyset \subseteq x)$ é sempre verdadeira. \blackproof
\end{proof}

\subsection{O Axioma do Par}

\begin{ax}[do Par] \label{ax:3}
    Para quaisquer conjuntos $x$ e $y$, existe um conjunto cujos elementos são $x$ e $y$.
        \[
            \boxed{
                \forall x \forall y \exists z \forall w ((w \in z) \leftrightarrow ((w = x) \lor (w = y)))
            }
        \]
\end{ax}

\begin{prop} \label{prop.fund:extensaounico2}
    O conjunto $z$ do axioma do par é único. Notação: $z := \{x,y\}$. %As chaves podem ser entendidas como um símbolo funcional binário (futuramente, $n$-ário), apesar de terem uma sintaxe um pouco diferente.
\end{prop}

\begin{proof}
    Pelo axioma do par, $z$ é tal que
        \[
            \forall w ((w \in z) \leftrightarrow ((w = x) \lor (w = y))).
        \]
    Se $z'$ é tal que
        \[
            \forall w ( (w \in z') \leftrightarrow ((w=x) \lor (w=y)) ),
        \]
    então $\forall w ((w \in z') \leftrightarrow (w \in z) )$, de modo que $z' = z$ pelo axioma da extensão. \blackproof
\end{proof}

\begin{prop}
    $\forall x \forall y((x \in y) \leftrightarrow \{x\} \subseteq y)$.
\end{prop}

\begin{proof}
    Por um lado ($\Rightarrow$), suponha que $x \in y$. Pela definição de inclusão, precisamos provar que $\forall z ((z \in \{x\}) \rightarrow (z \in y))$. Se $z \in \{x\}$, então $z = x$. Como $x \in y$ por hipótese e $z=x$, temos $z \in y$. Logo $\{x\} \subseteq y$. Por outro lado ($\Leftarrow$), suponha que $\{x\} \subseteq y$. Temos que $x \in \{x\}$. Como $\{x\} \subseteq y$, pela definição de inclusão, todo elemento de $\{x\}$ pertence a $y$. Logo $x \in y$. \blackproof
\end{proof}

\subsection{O Axioma da União}

\begin{ax}[da União] \label{ax:4}
    Para todo conjunto $x$ existe o conjunto de todos os conjuntos que pertencem a algum elemento de $x$.
        \[
            \boxed{
                \forall x \exists y \forall z ((z \in y) \leftrightarrow \exists w ( (z \in w) \land (w \in x)))
            }
        \]
\end{ax}

\begin{prop} \label{prop.fund:extensaounico3}
    O conjunto $y$ do axioma da união é único. Notação: $y := \bigcup x $.
\end{prop}

\begin{proof}
    Pelo axioma da união, $y$ é tal que
        \[
            \forall z ((z \in y) \leftrightarrow \exists w ( (z \in w) \land (w \in x))).
        \]
    Se $y'$ é tal que 
        \[
            \forall z ((z \in y') \leftrightarrow \exists w ( (z \in w) \land (w \in x))),
        \]
    então $\forall z ((z \in y') \leftrightarrow (z \in y)) $, donde $y' = y$ pelo axioma da extensão. \blackproof
\end{proof}

\begin{teo}
    Para quaisquer conjuntos $x$ e $y$, existe o conjunto dos conjuntos que pertencem a $x$ ou a $y$.
        \[
            \forall x \forall y \exists z \forall w ((w \in z) \leftrightarrow ((w \in x) \lor (w \in y)))
        \]
    Ademais, esse conjunto é único, sendo denotado por $x \cup y$.
\end{teo}

\begin{proof}
    Provemos que $z := \bigcup \{x,y\}$, que existe pelos axiomas do par e da união, funciona. De fato, para todo $w$, temos $w \in z$ se, e somente se, existe $u \in \{x,y \}$ tal que $w \in u$. Mas $u \in \{ x,y \}$ se, e somente se, $u = x$ ou $u=y$, de modo que $w \in x$ ou $w \in y$, como queríamos. A unicidade de $z$ segue do axioma da extensão, de modo que podemos denotar $x \cup y := z$. \blackproof
\end{proof}

\begin{proof}
    Uma prova alternativa é a seguinte. Precisamos provar que
        \[
            \forall w \left( \left(w \in \bigcup \{ x,y \}  \right) \leftrightarrow ((w \in x) \lor (w \in y))\right). \tag{$\lozenge$}
        \]
    Pelos axiomas do par e da união, temos
        \begin{align*}
            w \in \bigcup \{ x,y \} &\leftrightarrow \exists u ((u \in \{x,y \}) \land (w \in u) ) \\
            &\leftrightarrow \exists u ( ( (u=x) \lor (u=y) ) \land (w \in u) ) \\
            &\leftrightarrow \exists u ( ( (u=x) \land (w \in u) )  \lor ( (u=y) \land ( w \in u ) ) ) \\
            &\leftrightarrow \exists u ( (w \in x) \lor (w \in y) ) \\
            &\leftrightarrow (w \in x) \lor (w \in y).
        \end{align*}
    Com isso, temos $\lozenge$, como queríamos. \blackproof
\end{proof}

\subsection{O Axioma das Partes}

\begin{ax}[das Partes] \label{ax:5}
    Para todo conjunto $x$, existe o conjunto dos subconjuntos de $x$.
        \[
            \boxed{
                \forall x \exists y \forall z ( (z \in y) \leftrightarrow (z \subseteq x) )
            }
        \]
\end{ax}

\begin{prop} \label{prop.fund:extensaounico4}
    O conjunto $y$ do axioma das partes é único. Notação: $y:=\mathcal{P}(x)$.
\end{prop}

\begin{proof}
    Pelo axioma das partes, o conjunto $y$ cumpre $\forall z ( (z \in y) \leftrightarrow (z \subseteq x) )$. Se $y'$ cumpre $\forall z ( (z \in y') \leftrightarrow (z \subseteq x) )$, então $\forall z ((z \in y' )\leftrightarrow(z \in y))$, de modo que $y' = y$ pelo axioma da extensão. \blackproof
\end{proof}

\subsection{O Esquema de Axiomas da Separação}

\begin{ax}[da Separação] \label{ax:6}
    Para cada fórmula $P$ em que $z$ não ocorre livre, a fórmula
        \[
            \boxed{
                \forall y \exists z \forall x ( (x \in z) \leftrightarrow ((x \in y) \land P) )
            }    
        \]
    é um axioma.
\end{ax}

\begin{obs}
    O conjunto $y$ é o ``universo'' da discussão.  O axioma da separação também é chamado de axioma da compreensão ou axioma da especificação.
\end{obs}

\begin{prop}
    O conjunto $z$ do axioma da separação \eqref{ax:6} é único. Notação: $z:=\{x \in y : P(x) \}$.
\end{prop}

\begin{proof}
    Segue do axioma da extensão. \blackproof
\end{proof}

\begin{teo}[Paradoxo de Russell] Não existe o conjunto de todos os conjuntos.
        \[
            \forall x \exists y (y \notin x)
        \]
\end{teo}

\begin{proof}
    Suponha que $\exists x \forall y (y \in x)$. Pelo axioma da separação com universo $x$ e a fórmula $y \notin y$, existe $z$ tal que $\forall y ( (y \in z) \leftrightarrow ((y \in x) \land (y \notin y)))$ (note que $z$ não ocorre livre em $y \notin y$). Como $\forall y (y \in x)$, temos $\forall y ((y \in z) \leftrightarrow (y \notin y))$. Particularmente para $y = z$, temos $((z \in z) \leftrightarrow (z \notin z))$, uma contradição. Logo $\forall x \exists y (y \notin x)$. \blackproof
\end{proof}

\begin{teo} \label{teo.fund:intfamilia}
    Para todo conjunto $x \neq \emptyset$ existe o conjunto de todos os conjuntos que pertencem simultaneamente a todos os elementos de $x$.
        \[
            \forall x ((x \neq \emptyset) \rightarrow \exists y \forall z ((z \in y) \leftrightarrow \forall w ( (w \in x) \rightarrow (z \in w))))
        \]
    Ademais, esse conjunto é único, sendo denotado por $\bigcap x$.
\end{teo}

\begin{proof}
    Precisamos provar que existe $y$ tal que
        \[
           \forall z ( (z \in y) \leftrightarrow \forall w ( (w \in x) \rightarrow (z \in w)) ). \tag{$\lozenge$}
        \]
    Observe inicialmente que o axioma da separação pode ser escrito como
        \[
            \forall x \exists y \forall z  ( (z \in y) \leftrightarrow ( (z \in x) \land P ) ),
        \]
    onde $P$ é uma fórmula em que $y$ não ocorre livre. Agora, como $x \neq \emptyset$, tome $v \in x$. Pelo axioma da separação com universo $v$ e a fórmula $\forall w ( (w \in x) \rightarrow (z \in w))$, onde $y$ não ocorre livre, existe $y$ tal que
        \[
            \forall z ( (z \in y) \leftrightarrow ( (z \in v) \land  (\forall w ( (w \in x) \rightarrow (z \in w))) ) ),
        \]
    isto é, existe 
        \[
            y := \{ z \in v : \forall w ( (w \in x) \rightarrow (z \in w)) \}.
        \]
    Afirmamos que vale  ($\lozenge$) nesse $y$. De fato,
        \begin{itemize}
            \item por um lado ($\Rightarrow$), se $z \in y$, então trivialmente $\forall w ((w \in x) \rightarrow (z \in w))$;
            \item por outro lado ($\Leftarrow$), se $z$ é tal que $\forall w ((w \in x) \rightarrow (z \in w))$, então, particularmente para $w = v$, temos $v \in x \rightarrow z \in v$, e como $v \in x$, temos $z \in v$. Como $z \in v$ e $\forall w ((w \in x) \rightarrow (z \in w))$, temos que $z \in y$.
        \end{itemize}
    Logo, existe $y$ tal que $\lozenge$. A unicidade de $y$ segue do axioma da extensão, de modo que podemos denotar $\bigcap x := y$.  \blackproof
\end{proof}

\begin{defi} \label{defi.fund:capminus}
    Sejam $x$ e $y$ conjuntos.
        \begin{enumerate}[leftmargin=*, align=left, label=\textbf{(\alph*)}]
            \item A \textit{interseção} entre $x$ e $y$ é definida como
                \[
                    x \cap y :=  \{ z \in x : z \in y\}.
                \]
            Dizemos que $x$ e $y$ são \textit{disjuntos} se $x \cap y = \emptyset$. 
            \item A \textit{diferença} entre $x$ e $y$ é definida como
                \[
                    x \setminus y := \{z \in x : z \notin y \}.
                \]
            Dizemos que $x \setminus y$ é o \textit{complementar} de $y$ relativo a $x$ se $y \subseteq x$. Isso é denotado por $y^{C} := x \setminus y$.
            \item A \textit{diferença simétrica} entre $x$ e $y$ é definida como
                \[
                    x \Delta y := \{ z \in x \cup y :z \notin x \cap y \}.
                \]
        \end{enumerate}
\end{defi}

\begin{obs}
    As definições \eqref{defi.fund:capminus} se dão pelo axioma da separação. Vejamos como isso é feito, por exemplo, na definição de $x \cap y$. Sendo $x$ o universo, o axioma da separação é a fórmula $\forall x \exists w \forall z ((z \in w) \leftrightarrow ((z \in x) \land P) )$, onde $P$ é uma fórmula em que $w$ não ocorre livre. Se $P$ é a fórmula $z \in y$, então existe $w$ que cumpre $\forall z ( (z \in w) \leftrightarrow (z \in x) \land (x \in y) )$, isto é, $w = \{z \in x : z \in y \}$. Denotamos esse $w$ por $x \cap y$.
\end{obs}

\subsection{Propriedades Algébricas}

\begin{prop}[Propriedades da União]
    \leavevmode
        \begin{enumerate}[leftmargin=*, align=left, label=\textbf{(\alph*)}]
            \item $\forall x (x \cup x = x)$.
            \item $\forall x (x \cup \emptyset = x)$.
            \item $\forall x \forall y (x \cup y = y \cup x)$.
            \item $\forall x \forall y \forall z (x \cup (y \cup z) = (x \cup y) \cup z)$.
            \item $\forall x \forall y (x \cup y = y \leftrightarrow x \subseteq y)$.
            \item $\forall x \forall y ( (x \subseteq x \cup y) \land (y \subseteq x \cup y))$.
            \item $\forall x \forall y \forall z (x \subseteq y \rightarrow x \cup z \subseteq y \cup z )$.
            \item $\forall x \forall y (x \subseteq y \rightarrow \bigcup x \subseteq \bigcup y)$.
        \end{enumerate}
\end{prop}

\begin{proof}
    Trivial. \blackproof
\end{proof}

\begin{prop}[Propriedades da Interseção]
    \leavevmode
        \begin{enumerate}[leftmargin=*, align=left, label=\textbf{(\alph*)}]
            \item $\forall x (x \cap x = x)$.
            \item $\forall x (x \cap \emptyset = \emptyset)$.
            \item $\forall x \forall y (x \cap y = y \cap x)$.
            \item $\forall x \forall y \forall z (x \cap (y \cap z) = (x \cap y) \cap z)$.
            \item $\forall x \forall y (x \cap y = x \leftrightarrow x \subseteq y)$.
            \item $\forall x \forall y ( (x \cap y \subseteq x) \land (x \cap y \subseteq y))$.
            \item $\forall x \forall y \forall z (x \subseteq y \rightarrow x \cap z \subseteq y \cap z)$.
            \item $\forall x \forall y (x \subseteq y \land x \neq \emptyset \rightarrow \bigcap y \subseteq \bigcap x)$.
        \end{enumerate}
\end{prop}

\begin{proof}
    Trivial. \blackproof
\end{proof}

\begin{prop}[Distributividade]
    \leavevmode
        \begin{enumerate}[leftmargin=*, align=left, label=\textbf{(\alph*)}]
            \item $\forall x \forall y \forall z (x \cap (y \cup z) = (x \cap y) \cup (x \cap z))$.
            \item $\forall x \forall y \forall z (x \cup (y \cap z) = (x \cup y) \cap (x \cup z))$.
        \end{enumerate}
\end{prop}

\begin{proof}
    Trivial. \blackproof
\end{proof}

\begin{prop}[Propriedades da Diferença]
    \leavevmode
        \begin{enumerate}[leftmargin=*, align=left, label=\textbf{(\alph*)}]
            \item (Imediatas).
                \begin{enumerate}[label=\roman*.]
                    \item $\forall x (x \setminus \emptyset = x)$;
                    \item $\forall x (x \setminus x = \emptyset)$;
                    \item $\forall x (\emptyset \setminus x = \emptyset)$.
                \end{enumerate}
            \item \leavevmode
                \begin{enumerate}[label=\roman*.]
                    \item $\forall x \forall y (x \setminus y = x \leftrightarrow x \cap y = \emptyset)$;
                    \item $\forall x \forall y (x \setminus y = \emptyset \leftrightarrow x \subseteq y)$.
                \end{enumerate}
            \item (Leis de De Morgan).
                \begin{enumerate}[label=\roman*.]
                    \item $\forall x \forall y \forall z ( x \setminus (y \cup z) = (x \setminus y) \cap (x \setminus z) )$;
                    \item $\forall x \forall y \forall z ( x \setminus (y \cap z) = (x \setminus y) \cup (x \setminus z) )$.
                \end{enumerate}
            \item $\forall x \forall y \forall z ( x \setminus (y \setminus z) = (x \setminus y) \cup (x \cap z) )$.
            \item (Diferenças entre interseções).
                \begin{enumerate}[label=\roman*.]
                    \item $\forall x \forall y \forall z ( x \cap (y \setminus z) = (x \cap y) \setminus (x \cap z) )$;
                    \item $\forall x \forall y \forall z ( (x \setminus y) \cap z = (x \cap z) \setminus y )$.
                \end{enumerate}
            \item (Monotonocidade da diferença).
                \begin{enumerate}[label=\roman*.]
                    \item $\forall x \forall y \forall z (x \subseteq y \rightarrow x \setminus z \subseteq y \setminus z)$.
                    \item $\forall x \forall y \forall z (y \subseteq z \rightarrow x \setminus z \subseteq x \setminus y)$.
                \end{enumerate}
        \end{enumerate}
\end{prop}

\begin{proof}
    Trivial. \blackproof
\end{proof}

\begin{prop}[Propriedades do Complemento Relativo]
    \leavevmode
        \begin{enumerate}[leftmargin=*, align=left, label=\textbf{(\alph*)}]
            \item $\forall x \forall y (y \subseteq x \rightarrow (y^C)^C = y)$.
            \item $\forall x \forall y (y \subseteq x \rightarrow (y \cup y^C = x) \land (y \cap y^C = \emptyset))$.
            \item $\forall x \forall y \forall z (z \subseteq y \subseteq x \rightarrow y^C \subseteq z^C)$.
            \item $\forall x \forall y \forall z (y \subseteq x \land z \subseteq x \rightarrow y \setminus z = y \cap z^C)$.
            \item (Leis de De Morgan para complementos).
                \begin{enumerate}[label=\roman*.]
                    \item $\forall x \forall y \forall z (y \subseteq x \land z \subseteq x \rightarrow (y \cup z)^C = y^C \cap z^C)$.
                    \item $\forall x \forall y \forall z (y \subseteq x \land z \subseteq x \rightarrow (y \cap z)^C = y^C \cup z^C)$.
                \end{enumerate}
        \end{enumerate}
\end{prop}

\begin{proof}
    Trivial. \blackproof
\end{proof}




\begin{prop}
    \leavevmode
        \begin{enumerate}[leftmargin=*, align=left, label=\textbf{(\alph*)}]
            \item $\forall x \forall y \forall z ( ((x \in y) \land (y \in z)) \rightarrow ( ( x \in \bigcup z) \land (y \subseteq \bigcup z)))$.
        \end{enumerate}
\end{prop}

\begin{prop}
    $\forall A (\bigcup \mathcal{P}{(A)} = A)$.
\end{prop}



\subsection{O Axioma da Regularidade}

\begin{ax}[da Regularidade]
    Para todo conjunto $x \neq \emptyset$ existe $y \in x$ tal que $x \cap y = \emptyset$.
        \[
            \boxed{
                \forall x ((x \neq \emptyset) \rightarrow \exists y ( (y \in x) \land (x \cap y = \emptyset)))
            }
        \]
\end{ax}

\begin{prop} \label{prop.fund:xinyeyinxabs}
    Não existem conjuntos $x$ e $y$ tais que $x \in y$ e $y \in x$.
        \[
            \neg (\exists x \exists y ((x \in y) \land (y \in x)))
        \]
\end{prop}

\begin{proof}
    Basta provar que $\forall x \forall y ((x \notin y) \lor (y \notin x))$. Pelo axioma do par, tome $z := \{ x,y \}$. Como $z \neq \emptyset$, pelo axioma da regularidade existe $w \in z$ tal que $w \cap z = \emptyset$. Se $w = x$, então $y \notin x$, porque se fosse $y \in x$ teríamos $x \cap z = \{ y\} \neq \emptyset$, uma contradição. Analogamente, se $w = y$, então $x \notin y$. \blackproof
\end{proof}

\begin{cor} \label{cor.fund:xnotinx}
    Não existe $x$ tal que $x \in x$.
        \[
            \forall x(x \notin x)
        \]
\end{cor}

\begin{proof}
    Segue do teorema anterior com $y = x$. \blackproof
\end{proof}

\section{Relações}

\subsection{Produto Cartesiano}

\begin{defi}[Par ordenado]
    Sejam $a$ e $b$ conjuntos. O \textit{par ordenado} $(a,b)$ é definido como o conjunto $\{ \{ a \}, \{a,b \} \}$, isto é,
        \[
            (a,b) := \{ \{ a \}, \{a,b \} \}.
        \]
\end{defi}

\begin{prop}
    Dois pares ordenados $(a,b)$ e $(c,d)$ são iguais se, e somente se, $a = c$ e $b=d$.
        \[
            \forall a \forall b \forall c \forall d ( ((a,b) = (c,d)) \leftrightarrow ((a=c) \land (b=d)))
        \]
\end{prop}

\begin{proof}
    Ver \cite{fajardo2024conjuntos}, teorema 4.2, página 95. Ver \cite{hercules}, teorema 4.2, página 50. \blackproof
\end{proof}

\begin{teo} \label{teo.fund:prodcart}
    Para quaisquer conjuntos $A$ e $B$ existe o conjunto de todos os pares ordenados $(a,b)$ tais que $a \in A$ e $b \in B$.
        \[
            \forall A \forall B \exists C \forall x (x \in C \leftrightarrow \exists a \exists b (a \in A \land b \in B \land x = (a,b))) 
        \]
    Ademais, esse conjunto é único, sendo denotado por $A \times B$.
\end{teo}

\begin{proof}
    Pelos axiomas da união e das partes, considere o conjunto $\mathcal{P}{(\mathcal{P}{(A \cup B)})}$; pelo axioma da separação, considere o conjunto
        \[
            C := \{x \in \mathcal{P}{(\mathcal{P}{(A \cup B)})} : \exists a \exists b (a \in A \land b \in B \land x = (a,b)) \}.
        \]
    Afirmamos que $C$ cumpre as condições do enunciado. Se $x \in C$, então pela definição de $C$ existem $a \in A$ e $b \in B$ tais que $x = (a,b)$. Provemos então que se existem $a \in A$ e $b \in B$ tais que $x = (a,b)$, então $x \in C$. Para isso, basta provar que $x \in \mathcal{P}(\mathcal{P}(A \cup B))$. Qualquer que seja o par ordenado $(a,b)$, onde $a \in A$ e $b \in B$,
        \begin{align*}
            (a,b) \in \mathcal{P}{(\mathcal{P}{(A \cup B)})} &\leftrightarrow
            \{ \{ a\}, \{ a,b\} \} \in \mathcal{P}{(\mathcal{P}{(A \cup B)})} \\ &\leftrightarrow \{ \{ a\}, \{ a,b\} \} \subseteq \mathcal{P}{(A \cup B)} \\
            &\leftrightarrow \{a\} \in \mathcal{P}{(A \cup B)} \land \{ a,b\} \in \mathcal{P}{(A \cup B)} \\
            &\leftrightarrow \{a\} \subseteq (A \cup B) \land \{ a,b\} \subseteq (A \cup B),
        \end{align*}
    o que sabemos ser verdade. A unicidade de $C$ segue do axioma da extensão, de modo que podemos denotar $A \times B := C$. \blackproof
\end{proof}

\begin{defi}
    O \textit{produto cartesiano} dos conjuntos $A$ e $B$ é definido como $A \times B$.
\end{defi}

Defina $\pi_1 : A \times B \to A$ e $\pi_2 : A \times B \to B$ por $\pi_1(a,b) = a$ e $\pi_2(a,b) = b$ para quaisquer $(a,b) \in A \times B$.

\subsection{Relações}

\begin{defi}
    \leavevmode
        \begin{enumerate}[leftmargin=*, align=left, label=\textbf{(\alph*)}]
            \item Uma \textit{relação binária}, ou simplesmente uma \textit{relação}, é um conjunto de pares ordenados.
            \item Um símbolo de predicado para ``$R$ é relação'', onde $R$ ocorre livre, é
                \[
                    \operatorname{Rel}(R) \overset{\text{def}}{\leftrightarrow} \forall x \left(x \in R \rightarrow \exists a \exists b \left(x = (a,b)\right)\right).
                \]
            Denotamos $(a,b) \in R$ por $aRb$.
            %\forall x \left(x \in R \rightarrow \exists a \exists b \forall y (y \in x \leftrightarrow (\forall z (z \in y \leftrightarrow z = a) \lor \forall z (z \in y \leftrightarrow (z = a \lor z = b) ) ) )\right).
            \item Dizemos que $R$ é uma relação de $A$ em $B$ se $R \subseteq A \times B$. Dizemos que $R$ é uma relação em $A$ se $R \subseteq A \times A$.
        \end{enumerate}
\end{defi}

\begin{teo}
    Um conjunto $R$ é uma relação se, e somente se, existem conjuntos $A$ e $B$ tais que $R \subseteq A \times B$.
        \[
            \forall R (\operatorname{Rel}{(R)} \leftrightarrow \exists A \exists B(R \subseteq A \times B))
        \]
\end{teo}

\begin{proof}
     ($\Rightarrow$) Sendo $R$ uma relação, usando os axiomas da união e da separação, defina
        \begin{align*}
             A &:= \left\{ a \in \bigcup \bigcup R : \exists b \left(b \in \bigcup \bigcup R \land aRb \right) \right\} \\
             B &:= \left\{ b \in \bigcup \bigcup R : \exists a \left(a \in \bigcup \bigcup R \land aRb \right) \right\}
        \end{align*}
    Seja $x \in R$. Então existem $a$ e $b$ tais que $x = (a,b)$. Se $\{ \{a \}, \{a,b \} \} \in R$, então $\{ \{a \}, \{a,b \} \} \subseteq \bigcup R$, donde $\{a,b \} \in \bigcup R$, donde $\{ a,b \} \subseteq \bigcup \bigcup R$, donde $a,b \in \bigcup \bigcup R$. Como $aRb$, pelas definições de $A$ e $B$, temos $a \in A$ e $b \in B$. Como $x = (a,b)$ e $a \in A$ e $b \in B$, pelo teorema \eqref{teo.fund:prodcart} temos $x \in A \times B$, donde, por fim, segue que $R \subseteq A \times B$.

    ($\Leftarrow$) Os elementos de $A \times B$ são pares ordenados; logo, qualquer subconjunto de $A \times B$ terá pares ordenados como elementos. \blackproof
\end{proof}

\begin{defi}
    Seja $R$ uma relação.
        \begin{enumerate}[leftmargin=*, align=left, label=\textbf{(\alph*)}]
            \item O \textit{domínio} de $R$ é definido como
                \[
                    \Dom{(R)} := \left\{ a \in \bigcup \bigcup R : \exists b ((a,b) \in R) \right\}.
                \]
            \item A \textit{imagem} de $R$ é definida como
                \[
                    \Im{(R)} := \left\{ b \in \bigcup \bigcup R : \exists a ((a,b) \in R) \right\}.
                \]
            \item A \textit{relação inversa} de $R$ é definida como
                \[
                    R^{-1} := \{ (b,a) \in \Im{(R)} \times \Dom{(R)} : (a,b) \in R \}.
                \]
            \item A \textit{imagem de um conjunto $X$ por $R$} é definida como
                \[
                    R[X] := \{ b \in \bigcup \bigcup R : \exists a (a \in X \land (a,b) \in R) \}.
                \]
            \item A \textit{imagem inversa de um conjunto $Y$ por $R$} é definida como
                \[
                    R^{-1}[Y] := \left\{ a \in \bigcup \bigcup R : \exists b (b \in Y \land (a,b) \in R) \right\}
                \]
            Equivalentemente, $R^{-1}[Y]$ é a imagem de $Y$ pela relação $R^{-1}$. 
            \item A \textit{restrição de $R$ a $X$} é definida como
                \[
                    R \restriction_X := \{ (a,b) \in R : a \in X \}.
                \]
            \item A \textit{composição de $R$ e $S$} é definida como 
                \[
                    S \circ R := \{ (a,c) \in \Dom{(R)} \times \Im{(S)} : \exists b (b \in \Im{(R)} \cap \Dom{(S)} \land (aRb \land bSc)) \}.
                \]
        \end{enumerate}
\end{defi}

\begin{prop} \label{prop.fund:domeimdeRAB}
    Sejam $R$ e $S$ relações e $A$, $B$, $C$ e $D$ conjuntos tais que $R \subseteq A \times B$ e $S \subseteq C \times D$.
        \begin{enumerate}[leftmargin=*, align=left, label=\textbf{(\alph*)}]
            \item $\Dom{(R)} = \{x \in A : \exists y (y \in B \land xRy)\}$.
            \item $\Im{(R)} = \{ y \in B : \exists x (x \in A \land xRy)\}$.
            \item $S \circ R \subseteq A \times D$.
        \end{enumerate}
\end{prop}

\begin{proof}
    \leavevmode
        \begin{enumerate}[leftmargin=*, align=left, label=\textbf{(\alph*)}]
            \item Se $x \in \Dom{(R)}$, então existe $y \in \Im{(R)}$ tal que $(x,y) \in R$. Como $R \subseteq A \times B$, temos $(x,y) \in A \times B$, donde $x \in A$ e $y \in B$. Assim, $x \in A$ e existe $y \in B$ tal que $xRy$, o que prova a inclusão $\subseteq$. Reciprocamente, se $x \in A$ e existe $y \in B$ tal que $(x,y) \in R$, então $x \in \Dom{(R)}$, o que prova a inclusão $\supseteq$. \blackproof
            \item Se $y \in \Im{(R)}$, então existe $x \in \Dom{(R)}$ tal que $(x,y) \in R$. Como $R \subseteq A \times B$, temos $(x,y) \in A \times B$, donde $x \in A$ e $y \in B$. Assim, $y \in B$ e existe $x \in A$ tal que $xRy$, o que prova a inclusão $\subseteq$. Reciprocamente, se $y \in B$ e existe $x \in A$ tal que $(x,y) \in R$, então $y \in \Im{(R)}$, o que prova a inclusão $\supseteq$. \blackproof
            \item Se $(x,z) \in S \circ R$, então $x \in \Dom{(R)}$ e $z \in \Im{(S)}$. Como $\Dom{(R)} \subseteq A$ (primeiro item) e $\Im{(S)} \subseteq D$ (segundo item), temos $x \in A$ e $z \in D$, de modo que $(x,z) \in A \times D$. Com isso, $S \circ R \subseteq A \times D$. \blackproof
        \end{enumerate}
\end{proof}

\begin{prop} \label{prop.fund:relacoes}
    Sejam $R$, $S$ e $T$ relações. Valem as seguintes afirmações.
        \begin{enumerate}[leftmargin=*, align=left, label=\textbf{(\alph*)}]
            \item 
                \begin{enumerate}[label=\roman*.]
                    \item $(R^{-1})^{-1} = R$.
                    \item $\Dom{R^{-1}} = \Im{(R)}$.
                    \item $\Im{(R^{-1})} = \Dom{(R)}$
                \end{enumerate}
            \item $T \circ (S \circ R) = (T \circ S) \circ R$.
            \item $(S \circ R)^{-1} = R^{-1} \circ S^{-1}$.
            \item Se $\Im{(R)} \subseteq \Dom{(S)}$, então $\Dom{(S \circ R)} = \Dom{(R)}$.
            \item Se $\Dom{(S)} \subseteq \Im{(R)}$, então $\Im{(S \circ R)} = \Im{(S)}$.
        \end{enumerate}
\end{prop}

\begin{proof}
    \leavevmode
        \begin{enumerate}[leftmargin=*, align=left, label=\textbf{(\alph*)}]
            \item 
                \begin{enumerate}[label=\roman*.]
                    \item Temos $(a,b) \in R$ se, e somente se, $(b,a) \in R^{-1}$, o que é equivalente a $(a,b) \in (R^{-1})^{-1}$. Logo $R = (R^{-1})^{-1}$.
                    \item 
                    \item 
                \end{enumerate}
            \item Se $(a,d) \in T \circ (S \circ R)$, então $a \in \Dom{(S \circ R)}$, $d \in \Im{(T)}$ e existe $c \in \Im{(S \circ R)} \cap \Dom{(T)}$ tal que $(a,c) \in S \circ R$ e $(c,d) \in T$. De $(a,c) \in S \circ R$ segue que $a \in \Dom{(R)}$, $c \in \Im{(S)}$ e existe $b \in \Im{(R)} \cap \Dom{(S)}$ tal que $(a,b) \in R$ e $(b,c) \in S$. Como $b \in \Dom{(S)}$, $d \in \Im{(T)}$ e existe $c \in \Dom{(T)} \cap \Im{(S)}$ tal que $(b,c) \in S$ e $(c,d) \in T$, temos que $(b,d) \in T \circ S$. Com isso, $b \in \Dom{(T \circ S)}$ e $d \in \Im{(T \circ S)}$. Como $a \in \Dom{(R)}$, $d \in \Im{(T \circ S)}$ e existe $b \in \Dom{(T \circ S)} \cap \Im{(R)}$ tal que $(a,b) \in R$ e $(b,d) \in T \circ S$, temos que $(a,d) \in (T \circ S) \circ R$. Com isso, $T \circ (S \circ R) \subseteq (T \circ S) \circ R$. A prova de que $(T \circ S) \circ R \subseteq T \circ (S \circ R)$ é completamente análoga, de modo que $T \circ (S \circ R) = (T \circ S) \circ R$. \blackproof
            \item Se $(c,a) \in (S \circ R)^{-1}$, então $(a,c) \in S \circ R$, $a \in \Dom{(R)}$, $c \in \Im{(S)}$ e existe $b \in \Im{(R)} \cap \Dom{(S)}$ tal que $(a,b) \in R$ e $(b,c) \in S$ isto é, $(c,b) \in S^{-1}$, $(b,a) \in R^{-1}$, com $c \in \Dom{(S^{-1})}$, $a \in \Im{(R^{-1})}$ e $b \in \Im{(S^{-1})} \cap \Dom{(R^{-1})}$. Com isso, $(c,a) \in R^{-1} \circ S^{-1}$, de modo que $(S \circ R)^{-1} \subseteq R^{-1} \circ S^{-1}$. A prova de que $R^{-1} \circ S^{-1} \subseteq (S \circ R)^{-1}$ é completamente análoga, de modo que $(S \circ R)^{-1} = R^{-1} \circ S^{-1}$. \blackproof
            \item Se $a \in \Dom{(R)}$, então existe $b \in \Im{(R)}$ tal que $(a,b) \in R$. Se $\Im{(R)} \subseteq \Dom{(S)}$, então $b \in \Dom{(S)}$, de modo que existe $c \in \Im{(S)}$ tal que $(b,c) \in S$. Assim, $(a,c) \in S \circ R$, de modo que $a \in \Dom{(S \circ R)}$. Com isso, $\Dom{(R)} \subseteq \Dom{(S \circ R)}$. Agora, se $a \in \Dom{(S \circ R)}$, então existe $c \in \Im{(S \circ R)}$ tal que $(a,c) \in S \circ R$, donde $a \in \Dom{(R)}$ e $\Dom{(S \circ R)} \subseteq \Dom{(R)}$. Com isso, $\Dom{(R)} = \Dom{(S \circ R)}$. \blackproof
            \item Se $c \in \Im{(S)}$, então existe $b \in \Dom{(S)}$ tal que $(b,c) \in S$. Se $\Dom{(S)} \subseteq \Im{(R)}$, então $b \in \Im{(R)}$, de modo que existe $a \in \Dom{(R)}$ tal que $(a,b) \in R$. Assim, $(a,c) \in S \circ R$, de modo que $c \in \Im{(S \circ R)}$. Com isso, $\Im{(S)} \subseteq \Im{(S \circ R)}$. Agora, se $c \in \Im{(S \circ R)}$, então existe $a \in \Dom{(S \circ R)}$ tal que $(a,c) \in S \circ R$, donde $c \in \Im{(S)}$ e $\Im{(S \circ R)} \subseteq \Im{(S)}$. Com isso, $\Im{(S)} = \Im{(S \circ R)}$. \blackproof
        \end{enumerate}
\end{proof}

\subsection{Relações de Ordem}

\begin{defi}
    Seja $R$ uma relação em $X$.
        \begin{enumerate}[leftmargin=*, align=left, label=\textbf{(\alph*)}]
            \item Dizemos que $R$ é \textit{reflexiva} se
                \[
                    \forall x (x \in X \rightarrow (x,x) \in R).
                \]
            \item Dizemos que $R$ é \textit{irreflexiva} se
                \[
                    \forall x (x \in X \rightarrow (x,x) \notin R).
                \]
            \item Dizemos que $R$ é \textit{simétrica} se
                \[
                    \forall x \forall y (x,y \in X \rightarrow (xRy \rightarrow yRx)).
                \]
            \item Dizemos que $R$ é \textit{antissimétrica} se
                \[
                    \forall x \forall y (x,y \in X \rightarrow (xRy \land yRx \rightarrow x=y)).
                \]
            \item Dizemos que $R$ é \textit{transitiva} se
                \[
                    \forall x \forall y \forall z (x,y,z \in X \rightarrow (xRy \land yRz \rightarrow xRz)).
                \]
        \end{enumerate}
\end{defi}

\begin{defi}[Ordem parcial] \label{defi.fund:ordemparcial}
    \leavevmode
        \begin{enumerate}[leftmargin=*, align=left, label=\textbf{(\alph*)}]
            \item Uma \textit{relação de ordem parcial} em $X$ é uma relação $\leq  \, \, \subseteq X \times X$ que tem as seguintes propriedades.
                \begin{enumerate}[label=\roman*.]
                    \item Reflexividade: $\forall x(x \in X \rightarrow x \leq x)$;
                    \item Antissimetria: $\forall x \forall y (x,y \in X \rightarrow (x \leq y \land y \leq x \rightarrow x=y))$;
                    \item Transitividade: $\forall x \forall y \forall z (x,y,z \in X \rightarrow (x \leq y \land y \leq z \rightarrow x \leq z))$. 
                \end{enumerate}
            Dizemos que $X$ é o \textit{domínio} de $\leq$.
            \item Um \textit{conjunto parcialmente ordenado} é um par $(X, \leq)$ onde $\leq  \, \, \subseteq X \times X$ é uma relação de ordem parcial.
        \end{enumerate}
\end{defi}

\begin{nota}
    Sendo $\leq$ uma ordem parcial, abreviaremos $y \leq x$ por $x \geq y$, $x \leq y$ e $x \neq y$ por $x < y$ e $x<y$ por $y>x$. Quando não houver perigo de confusão, podemos escrever somente \textit{ordem} em vez de ordem parcial.
\end{nota}

\begin{ex}
    A relação de inclusão $\subseteq$ é uma relação de ordem parcial \eqref{prop.fund:inclusaoparcial}.
\end{ex}

\begin{defi}
    Dois conjuntos parcialmente ordenados $(X_1, \leq_1)$ e $(X_2, \leq_2)$ são \textit{ordem-isomorfos} se existe uma bijeção $f:X_1 \to X_2$ tal que 
        \[
          \forall x \forall y (x,y \in X_1 \rightarrow (x \leq_1 y \leftrightarrow f(x) \leq_2 f(y))).  
        \]
    Dizemos que a função $f$ é um \textit{isomorfismo de ordens parciais}.
\end{defi}

\begin{teo}
    Se $(X, \leq)$ é um conjunto ordenado, então existe um conjunto ordenado $(Y, \preceq)$ ordem-isomorfo a $(X,\leq)$ tal que 
        \[
            \preceq \, \, = \{ (x,y) \in Y \times Y : x \subseteq y \}.
        \]
\end{teo}

\begin{proof}
    Definindo $f : X \to \mathcal{P}{(X)}$ por $f(x) := \{y \in X : y \leq x \}$, temos que $f$ é bijetiva em relação a $Y := \Im{(f)}$. De fato, se $f(x) = f(y)$, então $x \in f(y)$ e $y \in f(x)$ já que $x \in f(x)$ e $y \in f(y)$; daí, pela definição de $f$, vem $x \leq y$ e $y \leq x$, de modo que $x = y$ e $f$ é injetiva. Provemos então que $x \leq y$ se, e somente se, $f(x) \subseteq f(y)$, para quaisquer $x,y \in X$. Se $z \in f(x)$, então $z \leq x$, e se $x \leq y$, então $z \leq y$, de modo que $z \in f(y)$ e $f(x) \subseteq f(y)$. Agora, se $x \in f(x) \subseteq f(y)$, então $x \in f(y)$, donde $x \leq y$. Com isso, $(X, \leq)$ é ordem-isomorfo a $(Y, \subseteq)$, como havíamos afirmado. \blackproof
\end{proof}

\begin{defi}
    Sejam $(X, \leq)$ um conjunto parcialmente ordenado e $S$ um subconjunto não vazio de $X$.
        \begin{enumerate}[leftmargin=*, align=left, label=\textbf{(\alph*)}]
            \item Dizemos que $M \in X$ é
                \begin{enumerate}[label=\roman*.]
                    \item uma \textit{cota superior} de $S$ se
                        \[
                            \forall x (x \in S \rightarrow x \leq M).
                        \]
                    Nesse caso, dizemos que $S$ é \textit{limitado superiormente} em $X$.
                    \item o \textit{elemento máximo} de $S$ se $M$ é uma cota superior de $S$ e $M \in S$. Isso é denotado por $\max{S} := M$.
                    \item o \textit{supremo} de $S$ se
                        \[
                            \forall x (x \in S \rightarrow x \leq M) \land \forall y (y \in X \land \forall x (x \in S \rightarrow x \leq y) \rightarrow M \leq y),
                        \]
                    isto é, se $M$ é a menor cota superior de $S$. Isso é denotado por $\sup{S} := M$.
                \end{enumerate}
            \item Dizemos que $m \in X$ é
                \begin{enumerate}[label=\roman*.]
                    \item uma \textit{cota inferior} de $S$ se
                        \[
                            \forall x (x \in S \rightarrow m \leq x).
                        \]
                    Nesse caso, dizemos que $S$ é \textit{limitado inferiormente} em $X$.
                    \item o \textit{elemento mínimo} de $S$ se $m$ é uma cota inferior de $S$ e $m \in S$. Isso é denotado por $\min{S} := m$.
                    \item o \textit{ínfimo} de $S$ se
                        \[
                            \forall x (x \in S \rightarrow m \leq x) \land \forall y (y \in X \land \forall x (x \in S \rightarrow m \leq y) \rightarrow y \leq m),
                        \]
                    isto é, se $m$ é a maior cota inferior de $S$. Isso é denotado por $\inf{S} := m$.
                \end{enumerate}
        \end{enumerate}
\end{defi}

\begin{obs}
    É fácil ver que o máximo de $S$, quando existe, é único. De fato, se $x, y \in S$ são máximos de $S$, então $x \leq y$ e $y \leq x$, de modo que $x = y$. Essa unicidade também vale para o mínimo, o ínfimo e o supremo de $S$, quando existem. Isso justifica o uso do artigo ``o'' (em \textit{o} máximo, em vez de \textit{um} máximo, por exemplo) e nos permite denotar esses elementos por $\max{S}$, $\min{S}$, $\inf{S}$ e $\sup{S}$, respectivamente.
\end{obs}

\begin{defi}
    Seja $(X, \leq)$ um conjunto parcialmente ordenado.
        \begin{enumerate}[leftmargin=*, align=left, label=\textbf{(\alph*)}]
            \item Dizemos que $\leq$ é uma relação de ordem \textit{total}\footnote{O termo \textit{ordem linear} também costuma ser usado. Nesse caso, dizemos que o par $(X,\leq)$ é um \textit{conjunto linearmente ordenado}.}, e que o par $(X,\leq)$ é um \textit{conjunto totalmente ordenado}, se 
                \[
                    \forall x \forall y (x,y \in X \rightarrow (x \leq y \lor y \leq x)).
                \]
            \item Dizemos que $\leq$ é uma \textit{boa ordem} em $X$, e que o par $(X,\leq)$ é um \textit{conjunto bem-ordenado}, se todo subconjunto não vazio de $X$ possui um elemento mínimo.
            \item Dizemos que $\leq$ é um \textit{reticulado} se para quaisquer $x,y \in X$, o conjunto $\{ x,y \}$ possui supremo e ínfimo.
            \item Dizemos que $\leq$ é uma \textit{árvore} se, para todo $x \in X$, o conjunto $S = \{ y \in X : y \leq x\}$ é tal que $(S, \leq \cap \ S^2)$ é um conjunto bem-ordenado.
        \end{enumerate}
\end{defi}

\begin{prop}
    \leavevmode
        \begin{enumerate}[leftmargin=*, align=left, label=\textbf{(\alph*)}]
            \item Toda boa ordem é uma ordem total.
            \item Toda boa ordem é uma árvore.
            \item Toda ordem total é um reticulado.
        \end{enumerate}
\end{prop}

\begin{proof}
    \leavevmode
        \begin{enumerate}[leftmargin=*, align=left, label=\textbf{(\alph*)}]
            \item Sejam $x, y \in X$. Como $\leq$ é uma boa ordem, todo subconjunto não vazio de $X$ possui elemento mínimo. Assim, o conjunto $\{x,y\} \subseteq X$ possui um elemento mínimo. Se $\min\{x,y\} = x$, então $x \leq y$. Se $\min\{x,y\} = y$, então $y \leq x$. Em qualquer caso, vale $x \leq y$ ou $y \leq x$, de modo que a ordem $\leq$ é total. \blackproof
            \item Pela definição de árvore, precisamos mostrar que, para todo $x \in X$, o conjunto $S = \{ y \in X : y \leq x\}$ é bem-ordenado pela ordem induzida. Seja $A \subseteq S$ um subconjunto não vazio. Como $S \subseteq X$, temos que $A \subseteq X$. Como $X$ é bem-ordenado, $A$ possui um elemento mínimo. Com isso, todo subconjunto não vazio de $S$ possui um elemento mínimo, isto é, $S$ é bem-ordenado. \blackproof
            \item Sejam $x, y \in X$. Como a ordem $\leq$ é total, temos $x \leq y$ ou $y \leq x$. Suponha, sem perda de generalidade, que $x \leq y$. Como $x \leq x$ e $x \leq y$, temos que $x$ é uma cota inferior de $\{x,y\}$. Se $z$ uma cota inferior qualquer de $\{x,y\}$, então, em particular, $z \leq x$. Logo, $x$ é a maior das cotas inferiores, isto é, $x = \inf\{x,y\}$. Como $y \geq x$ e $y \geq y$, temos que $y$ é uma cota superior de $\{x,y\}$. Se $w$ uma cota superior qualquer de $\{x,y \}$, então, em particular, $w \geq y$. Logo, $y$ é a menor das cotas superiores, isto é, $y = \sup\{x,y\}$. Portanto, $\{x,y\}$ possui supremo e ínfimo. \blackproof
        \end{enumerate}
\end{proof}

\begin{prop}
    Se $(X,\leq)$ é parcialmente ordenado e $Y \subseteq X$, então $(Y, \leq \cap \ Y^2)$ é parcialmente ordenado.
\end{prop}

\begin{proof}
    Defina $\leq_Y \ := \ \leq \cap \ Y^2$.
        \begin{enumerate}[label=\roman*.]
            \item (Reflexividade) Dado $y \in Y$, como $Y \subseteq X$, temos $y \in X$, donde $y \leq y$. Como $y \in Y$, vem $(y,y) \in \leq_Y$.
            \item (Antissimetria) Sejam $x, y \in Y$ tais que $(x,y) \in \leq_Y$ e $(y,x) \in \leq_Y$. Então $x \leq y$ e $y \leq x$. Pela antissimetria de $\leq$, vem $x=y$.
            \item (Transitividade) Sejam $x, y, z \in Y$ tais que $(x,y) \in \leq_Y$ e $(y,z) \in \leq_Y$. Então $x \leq y$ e $y \leq z$. Pela transitividade de $\leq$, vem $x \leq z$. Como $x, z \in Y$, vem $(x,z) \in \leq_Y$.
        \end{enumerate}
    Com isso, $\leq_Y \ := \ \leq \cap \ Y^2$ é uma ordem. \blackproof
\end{proof}

\begin{defi}
    Sejam $(X,\leq)$ parcialmente ordenado e $Y \subseteq X$. 
        \begin{enumerate}[leftmargin=*, align=left, label=\textbf{(\alph*)}]
            \item Dizemos que $\leq_Y \ := \ \leq \cap \ Y^2$ é uma \textit{subordem} de $\leq$.
            \item Dizemos que $(Y, \leq_Y)$ é um \textit{subconjunto parcialmente ordenado} de $(X, \leq)$. Isso é denotado por $(Y, \leq)$.
        \end{enumerate}
\end{defi}

\begin{prop}
    Sejam $(X,\leq)$ parcialmente ordenado e $Y \subseteq X$.
        \begin{enumerate}[leftmargin=*, align=left, label=\textbf{(\alph*)}]
            \item Se $\leq$ é uma ordem total, então $\leq_Y$ é uma ordem total.
            \item Se $\leq$ é uma boa ordem, então $\leq_Y$ é uma boa ordem.
            \item Se $\leq$ é uma árvore, então $\leq_Y$ é uma árvore.
        \end{enumerate}
\end{prop}

\begin{proof}
    \leavevmode
        \begin{enumerate}[leftmargin=*, align=left, label=\textbf{(\alph*)}]
            \item Sejam $x, y \in Y$. Como $Y \subseteq X$, temos $x, y \in X$, e como $\leq$ é total em $X$, temos $x \leq y$ ou $y \leq x$, de modo que $x \leq_Y y$ ou $y \leq_Y x$. \blackproof
            \item Seja $A \in \mathcal{P}(Y)_{\neq \emptyset}$. Como $Y \subseteq X$, temos $A \subseteq X$, e como $\leq$ é uma boa ordem em $X$, existe $\min{A} \in A$. Como para todo $z \in A \subseteq Y$ temos $\min{A} \leq z$, segue que $\min{A}$ é o mínimo de $A$ com relação a $\leq_Y$. \blackproof
            \item Sendo $x \in Y$ e $S := \{y \in Y : y \leq_Y x \}$, provemos que $(S, \leq_Y \cap \ S^2)$ é bem ordenado. Pondo $T := \{y \in X : y \leq x \}$, temos $S = T \cap Y$, e como $x \in X$ e $\leq$ é uma árvore, temos que $(T, \leq \cap \ T^2)$ é bem ordenado. Como $S \subseteq T$, temos que $(S, (\leq \cap \ T^2) \cap S^2 )$ é também uma boa ordem. Observando que
                \begin{align*}
                    \leq_Y \cap \ S^2 &= (\leq \cap \ Y^2) \cap (T \cap Y)^2 \\
                    &= \leq \cap \ Y^2 \cap T^2 \\
                    &= (\leq \cap \ T^2) \cap (T^2 \cap Y^2) \\
                    &= (\leq \cap \ T^2) \cap S^2, 
                \end{align*}
            segue que $(S, \leq_Y \cap \ S^2)$ é bem ordenado. \blackproof
        \end{enumerate}
\end{proof}

\subsection{Relações de Equivalência}

\begin{defi} (Relações de equivalência)
    \begin{enumerate}[leftmargin=*, align=left, label=\textbf{(\alph*)}]
        \item Uma \textit{relação de equivalência} em $X$ é uma relação $\sim \, \, \subseteq X \times X$ que tem as seguintes propriedades.
        \begin{enumerate}[label=\roman*.]
            \item Reflexividade: $\forall x(x \in X \rightarrow x \sim x)$;
            \item Simetria: $\forall x \forall y (x,y \in X \rightarrow (x \sim y \rightarrow y \sim x))$;
            \item Transitividade: $\forall x \forall y \forall z (x,y,z \in X \rightarrow (x \sim y \land y \sim z \rightarrow x \sim z))$.
        \end{enumerate}
        \item A \textit{classe de equivalência} de $a \in X$ por $\sim$ é definida como
            \[
                [a]_{\sim} := \{ x \in X : x \sim a \}. 
            \]
        \item O conjunto das classes de equivalência de $\sim$ é definido como
            \[
                X / \sim \ := \{ Y \in \mathcal{P} (X) : \exists x \forall y (y \in Y \leftrightarrow x \sim y) \} = \{ [x]_{\sim} : x \in X \}.
            \]
    \end{enumerate}
\end{defi}

\begin{prop}
    Seja $\sim$ uma relação de equivalência num conjunto $X$. As seguintes afirmações são equivalentes.
        \begin{enumerate}[leftmargin=*, align=left, label=\textbf{(\alph*)}]
            \item $a \sim b$.
            \item $a \in [b]$.
            \item $b \in [a]$.
            \item $[a] = [b]$.
        \end{enumerate}
\end{prop}

\begin{proof}
    \leavevmode
        \begin{enumerate}[leftmargin=*, align=left]
            \item[\textbf{(a)} $\Rightarrow$ \textbf{(b)}:] Por definição, $[b] = \{ x \in X : x \sim b \}$, e como $a \sim b$, segue $a \in [b]$. \blackproof
            \item[\textbf{(b)} $\Rightarrow$ \textbf{(c)}:] Se $a \in [b]$, então $a \sim b$, isto é, $b \in [a]$. \blackproof
            \item[\textbf{(c)} $\Rightarrow$ \textbf{(d)}:] Se $b \in [a]$, então $b \sim a$. Se $x \in [a]$, então $x \sim a$, de modo que $x \sim b$, isto é, $x \in [b]$. Com isso, $[a] \subseteq [b]$. Analogamente temos $[b] \subseteq [a]$, de modo que $[a] = [b]$. \blackproof
            \item[\textbf{(d)} $\Rightarrow$ \textbf{(a)}:] Se $a \in [a] = [b]$, então $a \sim b$. \blackproof  
        \end{enumerate}
\end{proof}

\begin{defi}
    Uma \textit{partição} de um conjunto $X \neq \emptyset$ é um subconjunto $\mathcal{P} \subseteq \mathcal{P}{(X)}$ que tem as seguintes propriedades.
        \begin{enumerate}[label=\roman*.]
            \item $\emptyset \notin \mathcal{P}$;
            \item $\bigcup \mathcal{P} = X$;
            \item $A \cap B = \emptyset$ para quaisquer $A, B \in \mathcal{P}$ tais que $A \neq B$.
        \end{enumerate}
    % tal que $\emptyset \notin \mathcal{P}$, $A \cap B = \emptyset$ para quaisquer $A, B \in \mathcal{P}$ tais que $A \neq B$ e $\bigcup \mathcal{P} = X$.
\end{defi}

\begin{teo}
    Se $\sim$ é uma relação de equivalência num conjunto $X$, então $X / \sim$ é uma partição de $X$, isto é, valem as seguintes afirmações.
        \begin{enumerate}[leftmargin=*, align=left, label=\textbf{(\alph*)}]
            \item $\emptyset \notin X / \sim$.
            \item $\bigcup X / \sim = X$.
            \item $\forall Y\forall Z((Y, Z \in X / \sim)\rightarrow (Y=Z \lor Y \cap Z = \emptyset))$.
        \end{enumerate}
\end{teo}

\begin{proof}
    \leavevmode
        \begin{enumerate}[leftmargin=*, align=left, label=\textbf{(\alph*)}]
            \item Se $\emptyset \in X/ \sim$, então existiria $x \in X$ tal que $\emptyset = [x]$, mas $x \in [x]$, uma contradição. \blackproof
            \item Se $y \in \bigcup X / \sim$, então existe $Y \in X / \sim$ tal que $y \in Y$. Como $Y \in X / \sim$, existe $x \in X$ tal que $Y = [x]$. Como $[x] \subseteq X$, vem $y \in X$, de modo que $\bigcup X / \sim \subseteq X$. Agora, se $x \in X$, então $x \sim x$ e $x \in [x]$, e como $[x] \in X / \sim$, vem $x \in \bigcup X / \sim$, de modo que $X \subseteq \bigcup X / \sim$. Logo $\bigcup X / \sim = X$. \blackproof
            \item Se $Y \cap Z = \emptyset$, nada há de ser provado. Se $Y \cap Z \neq \emptyset$, então existe $x \in X$ tal que $x \in Y \cap Z$. Sendo $y_0, z_0 \in X$ tais que $Y = [y_0]$ e $Z = [z_0]$, temos $x \sim y_0$ e $x \sim z_0$, de modo que $[y_0] = [z_0]$, isto é, $Y = Z$. \blackproof
        \end{enumerate}
\end{proof}

\begin{teo}
    Se $\mathcal{P}$ é uma partição de um conjunto $X \neq \emptyset$, então existe uma relação de equivalência $R$ em $X$ tal que $X / R = \mathcal{P}$.
\end{teo}

\begin{proof}
    Pois tome $R := \{(x,y) \in X \times X : \exists A (A \in \mathcal{P} \land x,y \in A) \}$. \blackproof
\end{proof}

\section{Funções}

\begin{defi}[Função] \label{defi.fund:função}
    \leavevmode
        \begin{enumerate}[leftmargin=*, align=left, label=\textbf{(\alph*)}]
            \item Uma relação $f$ é uma \textit{função} se $(a,b) \in f$ e $(a,c) \in f$ implicam $b=c$. A \textit{imagem} de $a \in \Dom{(f)}$ por $f$ é denotada por $f(a)$.
            \item Uma \textit{função parcial de $A$ em $B$} é uma função $f$ tal que $\Dom{(f)} \subseteq A$ e $\Im{(f)} \subseteq B$.
            \item Uma \textit{função total de $A$ em $B$} é uma função $f$ tal que $\Dom{(f)} = A$ e $\Im{(f)} \subseteq B$. Isso é denotado por $f : A \to B$. O conjunto de todas as funções de $A$ em $B$ é denotado por $B^{A}$, isto é,
                \begin{align*}
                    B^{A} := \{f \in \mathcal{P}{(A \times B)} : & \forall a \forall b \forall c ((a,b) \in f \land (a,c) \in f \rightarrow b=c) \\ 
                    & \land \forall x(x \in A \rightarrow \exists y((x,y) \in f)) \}.
                \end{align*}
        \end{enumerate}
\end{defi}

\begin{obs}
    Escrevemos apenas ``função de $A$ em $B$'', omitindo o ``total''.
\end{obs}

\begin{defi}
    A \textit{função identidade} de um conjunto $A$ é a função $\id_A : A \to A$ definida por $\id_A{(x)} = x$ para todo $x \in A$.
\end{defi}

\begin{prop}
    Se $f: A \to B$ é uma função, então $\id_B \circ f = B$ e $f \circ \id_A = A$.
\end{prop}

\begin{proof}
    Teste só para ver se está funcionando. \blackproof
\end{proof}

\begin{prop}
    Sejam $f$ e $g$ funções e $X$ um conjunto.
        \begin{enumerate}[leftmargin=*, align=left, label=\textbf{(\alph*)}]
            \item $f \restriction_X$ é uma função e seu domínio é $\Dom{(f)} \cap X$.
            \item $g \circ f$ é uma função.
        \end{enumerate}
\end{prop}

\begin{proof}
    \leavevmode
        \begin{enumerate}[leftmargin=*, align=left, label=\textbf{(\alph*)}]
            \item Por definição, $f\restriction_X = \{ (a,b) \in f : a \in X \}$, isto é, $f$ é um conjunto de pares ordenados e, portanto, uma relação. Se $(a,b) \in f\restriction_X$ e $(a,c) \in f\restriction_X$, então, pela definição de $f\restriction_X$, $(a,b) \in f$, $(a,c) \in f$ e $a \in X$; como $f$ é função, $b=c$, de modo que $f\restriction_X$ é também uma função. Provemos, por fim, que $\Dom{(f\restriction_X)} =  \Dom{(f)} \cap X$. Se $a \in \Dom{(f\restriction_X)}$, então existe $b \in \Im{(f\restriction_X)}$ tal que $(a,b) \in f\restriction_X$; pela definição de $f\restriction_X$, vem $(a,b) \in f$ e $a \in X$, e de $(a,b) \in f$ vem $a \in \Dom{(f)}$. Com isso, $a \in \Dom{(f)} \cap X$, de modo que $\Dom{(f\restriction_X)} \subseteq \Dom{(f)} \cap X$. Por outro lado, se $a \in \Dom{(f)} \cap X$, então de $a \in \Dom{(f)}$ segue que existe $b \in \Im{(f)}$ tal que $(a,b) \in f$, e como $a \in X$, vem $(a,b) \in f\restriction_X$, de modo que $\Dom{(f)} \cap X \subseteq \Dom{(f\restriction_X)}$. Logo $\Dom{(f\restriction_X)} = \Dom{(f)} \cap X$. Em particular, temos $f\restriction_X = f \restriction_{\Dom{(f)} \cap X}$. \blackproof
            \item Se $(a,x) \in g \circ f$ e $(a,y) \in g \circ f$, então, por definição, existe $b \in \Im{(f)} \cap \Dom{(g)}$ tal que $(a,b) \in f$ e $(b,x) \in g$ e existe $c \in \Im{(f)} \cap \Dom{(g)}$ tal que $(a, c) \in f$ e $(c,y) \in g$. Como $f$ é função, vem $b=c$; daí, vem $(b,x) \in g$ e $(b,y) \in g$, e como $g$ é função, vem $x=y$. \blackproof
        \end{enumerate}
\end{proof}


\subsection{Funções Injetivas}

\begin{defi}
    Uma função $f$ é \textit{injetiva} se
        \[
            \forall x \forall y (x,y \in \Dom{(f)} \rightarrow (x \neq y \rightarrow f(x) \neq f(y))).
        \]
\end{defi}

\begin{prop} \label{prop.fund:feginj}
    Sejam $f$ e $g$ funções.
        \begin{enumerate}[leftmargin=*, align=left, label=\textbf{(\alph*)}]
            \item Se $f$ e $g$ são injetivas, então $g \circ f$ é injetiva.
            \item Se $g \circ f$ é injetiva e $\Im{(f)} \subseteq \Dom{(g)}$, então $f$ é injetiva.
        \end{enumerate}
\end{prop}

\begin{proof}
    \leavevmode
        \begin{enumerate}[leftmargin=*, align=left, label=\textbf{(\alph*)}]
            \item Sejam $x,y \in \Dom{(g \circ f)}$. Se $g(f(x)) = g(f(y))$, então $f(x) = f(y)$ pela injetividade de $g$. Se $f(x) = f(y)$, então $x = y$ pela injetividade de $f$. Logo $g \circ f$ é injetiva. \blackproof
            \item Sejam $x,y \in \Dom{(f)}$ tais que $f(x) = f(y)$. Se $\Im{(f)} \subseteq \Dom{(g)}$, então $f(x), f(y) \in \Dom{(g)}$, e como $f(x) = f(y)$, temos $g (f(x)) = g(f(y))$. Daí, como $\Dom{(f)} = \Dom{(g \circ f)}$ (proposição \eqref{prop.fund:relacoes}) e $g \circ f$ é injetiva, vem $x=y$, de modo que $f$ é injetiva. \blackproof
        \end{enumerate}
\end{proof}

\begin{teo} \label{teo.fund:inj}
    Seja $f$ uma função.
        \begin{enumerate}[leftmargin=*, align=left, label=\textbf{(\alph*)}]
            \item Se a relação $f^{-1}$ é uma função, então $f^{-1}$ é injetiva.
            \item A relação $f^{-1}$ é uma função se, e somente se,
                \begin{enumerate}[label=\roman*.]
                    \item $f$ é injetiva.
                    \item $f^{-1} \circ f = \id_{\Dom{(f)}}$.
                \end{enumerate}
        \end{enumerate}
\end{teo}

\begin{proof}
    \leavevmode
        \begin{enumerate}[leftmargin=*, align=left, label=\textbf{(\alph*)}]
            \item Se $(y,x) \in f^{-1}$ e $(z,x) \in f^{-1}$, então $(x,y) \in f$ e $(x,z) \in f$, e como $f$ é função vem $y=z$, de modo que $f^{-1}$ é uma função injetiva. \blackproof
            \item A equivalência mais importante é com $f$ ser injetiva.
                \begin{enumerate}[label=\roman*.]
                    \item Se $f^{-1}$ é uma função, então $(x,y) \in f^{-1}$ e $(x,z) \in f^{-1}$ implicam $y=z$. Daí, como $(y,x) \in f$ e $(z,x) \in f$, sendo $y=z$ segue que $f$ é injetiva. Agora, se $f$ é injetiva, então $(y,x) \in f$ e $(z,x) \in f$ implicam $y=z$, e como $(x,y) \in f^{-1}$ e $(x,z) \in f^{-1}$, segue que $f^{-1}$ é uma função.
                \end{enumerate}
            Provemos que $f$ é injetiva se, e somente se, $f^{-1} \circ f = \id_{\Dom{(f)}}$.
                \begin{enumerate}[label=\roman*., resume]
                    \item ($\Rightarrow$) Se $(x,z) \in f^{-1} \circ f$, então existe $y \in \Im{(f)} \cap \Dom{(f^{-1})}$ tal que $(x,y) \in f$ e $(y,z) \in f^{-1}$. Com isso, $(z,y) \in f$, e como $f$ é injetiva vem $z=x$, de modo que $(x,x) \in \id_{\Dom{(f)}}$, isto é, $f^{-1} \circ f \subseteq \id_{\Dom{(f)}}$. Agora, se $(x,x) \in \id_{\Dom{(f)}}$, então existe $y \in \Im{(f)}$ tal que $(x, y) \in f$, isto é, $(y,x) \in f^{-1}$. Com isso, $(x,x) \in f^{-1} \circ f$, de modo que $\id_{\Dom{(f)}} \subseteq f^{-1} \circ f$. Com isso, vem $f^{-1} \circ f = \id_{\Dom{(f)}}$.
                    
                    ($\Leftarrow$) Sendo $(x,y) \in f$ e $(z,y) \in f$, temos $(y,x) \in f^{-1}$, de modo que $(z,x) \in f^{-1} \circ f$, e como $f^{-1} \circ f = \id_{\Dom{(f)}}$, vem $x=z$, o que prova a injetividade de $f$.
                \end{enumerate}
            Com isso, todas as equivalências foram provadas. \blackproof
        \end{enumerate}
\end{proof}

\begin{defi}
    Uma função $f$ é \textit{invertível à esquerda} se existe uma função $g$ tal que $g \circ f = \id_{\Dom{(f)}}$. Dizemos que $g$ é uma \textit{inversa à esquerda} de $f$.
\end{defi}

\begin{teo} \label{teo.fund:invesquerda}
    Uma função $f$ é invertível à esquerda se, e somente se, $f$ é injetiva.
\end{teo}

\begin{proof}
    Se $f$ é injetiva, então pelo teorema \eqref{teo.fund:inj} $f^{-1}$ é uma função e $f^{-1} \circ f = \id_{\Dom{(f)}}$, de modo que $f$ é invertível à esquerda. Agora, sendo $(x,y) \in f$ e $(z,y) \in f$, provemos que $x=z$. Como $f$ é invertível à esquerda, existe uma função $g$ tal que $g \circ f = \id_{\Dom(f)}$. Como $(x,x) \in g \circ f$, existe $w \in \Im{(f)} \cap \Dom{(g)}$ tal que $(x,w) \in f$ e $(w,x) \in g$. Como $f$ é uma função, vem $w=y$, de modo que $(y,x) \in g$. Analogamente temos $(y,z) \in g$, e como $g$ é uma função vem $x=z$, de modo que $f$ é injetiva. \blackproof
\end{proof}

\subsection{Funções Sobrejetivas}

\begin{defi}
    Uma função $f : A \to B$ é \textit{sobrejetiva $B$} se $\Im{(f)} = B$.
\end{defi}

\begin{prop}
    Uma função $f: A \to B$ é sobrejetiva em $B$ se, e somente se, para todo $y \in B$ existe $x \in A$ tal que $(x,y) \in f$.
\end{prop}

\begin{proof}
    Segue da proposição \eqref{prop.fund:domeimdeRAB}. \blackproof    
\end{proof}

\begin{lem} \label{lem.fund:fegsob}
    Sejam $f: A \to B$ e $g: C \to D$ funções.
        \begin{enumerate}[leftmargin=*, align=left, label=\textbf{(\alph*)}]
            \item $\Dom{(g \circ f)} = \{ x \in A : f(x) \in C\}$.
            \item $\Dom{(g \circ f)} = A$ se, e somente se, $f(A) \subseteq C$.
        \end{enumerate}
\end{lem}

\begin{proof}
    \leavevmode
        \begin{enumerate}[leftmargin=*, align=left, label=\textbf{(\alph*)}]
            \item Pela proposição \eqref{prop.fund:domeimdeRAB}, $\Dom{(g \circ f)} = \{ x \in A : \exists y(y \in D \land (x,y \in g \circ f))\}$.
                \begin{enumerate}[label=\roman*.]
                    \item Se $x \in \Dom{(g \circ f)}$, então existe $y \in D$ tal que $(x,y) \in g \circ f$. Logo existe $z \in \Im{(f)} \cap \Dom{(g)} \subseteq B \cap C$ tal que $(x,z) \in f$ e $(z,y) \in g$, isto é, $z = f(x)$ e $y = g(z)$. Com isso, $x \in A$ e $f(x) \in C$, de modo que $x \in \{ x \in A : f(x) \in C\}$, isto é, $\Dom{(g \circ f)} \subseteq \{ x \in A : f(x) \in C\}$.
                    \item Se $x \in \{ x \in A : f(x) \in C\}$, então $x \in A$ e $f(x) \in C$, isto é, existe (um único) $z \in C$ tal que $(x,z) \in f$. Como $z \in C$, existe $y \in \Im{g} \subseteq D$ tal que $(z,y) \in g$. Com isso, $x \in A$ e existe $y \in D$ tal que $(x,y) \in g \circ f$, de modo que $x \in \Dom{(g \circ f)}$, isto é, $\{ x \in A : f(x) \in C\} \subseteq \Dom{(g \circ f)}$.
                \end{enumerate}
            Logo $\Dom{(g \circ f)} = \{ x \in A : f(x) \in C\}$. \blackproof
            \item A volta ($\Leftarrow$) já foi provada (proposição \eqref{prop.fund:relacoes}). Agora, se $y \in f(A)$, então existe $x \in A$ tal que $y = f(x)$, e como $A = \Dom{(g \circ f)}$, vem $f(x) \in C$, isto é, $y \in C$. Logo $f(A) \subseteq C$. \blackproof
        \end{enumerate}
\end{proof}

\begin{prop} \label{prop.fund:fegsob}
    Sejam $f : A \to B$ e $g : C \to D$ funções.
        \begin{enumerate}[leftmargin=*, align=left, label=\textbf{(\alph*)}]
            \item Se $f$ e $g$ são sobrejetivas e $B=C$, então $g \circ f$ é sobrejetiva.
            \item Se $g \circ f$ é sobrejetiva e $f(A) \subseteq C$, então $g$ é sobrejetiva.
        \end{enumerate} 
\end{prop}

\begin{proof}
    \leavevmode
        \begin{enumerate}[leftmargin=*, align=left, label=\textbf{(\alph*)}]
            \item Se $B=C$, então $\Dom{(g \circ f)} = A$ pelo lema \eqref{lem.fund:fegsob}. Se $g$ é sobrejetiva, então para todo $z \in D$ existe $y \in C = B$ tal que $(y,z) \in g$. Se $f$ é sobrejetiva, então para esse $y \in B$ existe $x \in A$ tal que $(x,y) \in f$. Com isso, para todo $z \in D$ existe $x \in A$ tal que $(x,z) \in g \circ f$, o que prova a sobrejetividade de $g \circ f$. \blackproof
            \item Se $f(A) \subseteq C$, então $\Dom{(g \circ f)} = A$ pelo lema \eqref{lem.fund:fegsob}. Se $g \circ f$ é sobrejetiva, então para todo $z \in D$ existe $x \in A$ tal que $(x,z) \in g \circ f$. Com isso, existe $y \in f(A) \cap C = C$ tal que $(x,y) \in f$ e $(y,z) \in g$. o que prova a sobrejetividade de $g$. \blackproof
        \end{enumerate}
\end{proof}

\begin{lem}
    Para qualquer função $f$, tem-se $f \circ f^{-1} = \id_{\Im{(f)}}$.
\end{lem}

\begin{proof}
    Provemos que $f \circ f^{-1} \subseteq \id_{\Im{(f)}}$ e $f \circ f^{-1} \supseteq \id_{\Im{(f)}}$. Se $(y,z) \in f \circ f^{-1}$, então existe $x \in \Dom{(f)} \cap \Im{(f^{-1})}$ tal que $(y,x) \in f^{-1}$ e $(x,z) \in f$. Daí, $(x,y) \in f$, e como $f$ é uma função, vem $y=z$. Logo $f \circ f^{-1} \subseteq \id_{\Im{(f)}}$. Por outro lado, se $(y,y) \in \id_{\Im{(f)}}$, então existe $x \in \Dom{(f)}$ tal que $(x,y) \in f$. Logo $(y,x) \in f^{-1}$, de modo que $(y, y) \in f \circ f^{-1}$, isto é, $f \circ f^{-1} \supseteq \id_{\Im{(f)}}$. Com isso, vem $f \circ f^{-1} = \id_{\Im{(f)}}$, como queríamos. \blackproof
\end{proof}

\begin{teo} \label{teo.fund:sob}
    Uma função $f : A \to B$ é sobrejetiva se, e somente se, $f \circ f^{-1} = \id_B$.
\end{teo}

\begin{proof}
    Se $f$ é sobrejetiva em $B$, então $\Im{(f)} = B$, de modo que $f \circ f^{-1} = \id_B$. Por outro lado, se $f \circ f^{-1} = \id_B$, então $f$ é sobrejetiva em $B$ porque, como também $f \circ f^{-1} = \id_{\Im{(f)}}$, vem $\Im{(f)} = B$. \blackproof
\end{proof}

\begin{defi}
    Uma função $f:A \to B$ é \textit{invertível à direita} se existe uma função $g : B \to A$ tal que $f \circ g = \id_{B}$. Dizemos que $g$ é uma \textit{inversa à direita} de $f$.
\end{defi}

\begin{teo} \label{teo.fund:invdireita}
    Uma função $f:A \to B$ é invertível à direita se, e somente se, $f$ é sobrejetiva em $B$.
\end{teo}

\begin{obs}
    A prova da volta ($\Leftarrow$) deste teorema depende do axioma da escolha. Mais precisamente, de um enunciado equivalente ao axioma da escolha: para toda relação $R$ existe uma função $f \subseteq R$ tal que $\Dom{(f)} = \Dom{(R)}$. Ainda assim, enunciamos este resultado aqui por uma questão de organização didática.  
\end{obs}

\begin{proof}
    \blackproof
\end{proof}

\subsection{Funções Bijetivas e Funções Inversas}

\begin{defi}
    Uma função $f: A \to B$ é \textit{bijetiva em relação a $B$} se é injetiva e sobrejetiva em $B$.
\end{defi}

\begin{prop} \label{prop.fund:defbij}
    Uma função $f: A \to B$ é bijetiva em $B$ se, e somente se, para todo $y \in B$ existe um único $x \in A$ tal que $(x,y) \in f$.
\end{prop}

\begin{proof}
    Segue imediatamente da definição. \blackproof
\end{proof}

\begin{prop}
    Se $f : A \to B$ e $g : B \to C$ são funções bijetivas, então a função $g \circ f : A \to C$ é uma função bijetiva.
\end{prop}

\begin{proof}
    Segue como corolário imediato das proposições \eqref{prop.fund:feginj} e \eqref{prop.fund:fegsob}. \blackproof
\end{proof}

\begin{teo} \label{teo.fund:bij1}
    Seja $f : A \to B$ uma função.
        \begin{enumerate}[leftmargin=*, align=left, label=\textbf{(\alph*)}]
            \item Se a relação $f^{-1}$ é uma função de $B$ em $A$, então $f^{-1}$ é bijetiva.
            \item A relação $f^{-1}$ é uma função de $B$ em $A$ se, e somente se,
                \begin{enumerate}[label=\roman*.]
                    \item $f$ é bijetiva em $B$.
                    \item $f^{-1} \circ f = \id_{A}$ e $f \circ f^{-1} = \id_{B}$.
                \end{enumerate}
        \end{enumerate}
\end{teo}

\begin{proof}
    \leavevmode
        \begin{enumerate}[leftmargin=*, align=left, label=\textbf{(\alph*)}]
            \item Como $f$ é função, para todo $x \in A$ existe um único $y \in B$ tal que $(x,y) \in f$, isto é, $(y,x) \in f^{-1}$. Daí, pela proposição \eqref{prop.fund:defbij}, $f^{-1}$ é bijetiva em $A$. \blackproof
            \item A equivalência que mais importa é com $f$ ser bijetiva em $B$.
                \begin{enumerate}[label=\roman*.]
                    \item Se $f^{-1} \subseteq B \times A$ é uma função tal que $\Dom{(f^{-1})} = B$, então para todo $y \in B$ existe um único $x \in A$ tal que $(y,x) \in f^{-1}$, isto é, $(x,y) \in f$. Daí, pela proposição \eqref{prop.fund:defbij}, temos que $f$ é bijetiva. Agora, pela mesma proposição, se $f$ é bijetiva, então para todo $y \in B$ existe um único $x \in A$ tal que $(x,y) \in f$, isto é, $(y,x) \in f^{-1}$. Daí, pela definição de função, $f^{-1}$ é uma função de $B$ em $A$. \blackproof
                \end{enumerate}
            Provemos que $f$ é bijetiva em $B$ se, e somente se, $f^{-1} \circ f = \id_A$ e $f \circ f^{-1} = \id_{B}$.
                \begin{enumerate}[label=\roman*., resume]
                    \item Se $f$ é bijetiva em $B$, então $f$ é injetiva e sobrejetiva, de modo que, pelos teoremas \eqref{teo.fund:inj} e \eqref{teo.fund:sob}, vem $f^{-1} \circ f = \id_A$ e $f \circ f^{-1} = \id_B$, respectivamente. Agora, se $f^{-1} \circ f = \id_A$ e $f \circ f^{-1} = \id_B$, então pelos mesmos teoremas $f$ é injetiva e sobrejetiva em $B$, isto é, $f$ é bijetiva em $B$. \blackproof
                \end{enumerate}
            Com isso, todas as equivalências foram provadas. \blackproof
        \end{enumerate}
\end{proof}

\begin{defi}
    Uma função $f: A \to B$ é \textit{invertível} se existe uma função $g : B \to A$ tal que $g \circ f = \id_A$ e $f \circ g = \id_B$. Dizemos que $g$ é a \textit{inversa} de $f$.
\end{defi}

\begin{prop}
    A função inversa de uma função invertível é única.\footnote{Note que é esta proposição que nos permite dizer ``a função inversa'' em vez de ``uma função inversa''. }
\end{prop}

\begin{proof}
    Seja $f : A \to B$ uma função invertível. Sejam $g_1, g_2 : B \to A$ funções inversas de $f$. Provemos que $g_1 = g_2$. De fato,
        \[
            g_1 = g_1 \circ \id_B = g_1 \circ (f \circ g_2) = (g_1 \circ f) \circ g_2 = \id_A \circ g_2 = g_2.    
        \]
    Logo, a função inversa da função $f : A \to B$, quando existe, é única. \blackproof
\end{proof}

\begin{teo} \label{teo.fund:bij2}
    Seja $f: A \to B$ uma função.
        \begin{enumerate}[leftmargin=*, align=left, label=\textbf{(\alph*)}]
            \item $f$ é invertível se, e somente se, $f$ é bijetiva.
            \item Se $f$ é invertível, então a função inversa de $f$ é a relação inversa $f^{-1}$.
        \end{enumerate}
\end{teo}

\begin{proof}
    \leavevmode
        \begin{enumerate}[leftmargin=*, align=left, label=\textbf{(\alph*)}]
            \item Se $f$ é invertível, então existe uma função $g: B \to A$ tal que $g \circ f = \id_A$ e $f \circ g = \id_B$, de modo que $f$ é invertível à esquerda e à direita, isto é, $f$ é injetiva (teorema \eqref{teo.fund:invesquerda}) e sobrejetiva em $B$ (teorema \eqref{teo.fund:invdireita}), isto é, $f$ é bijetiva em $B$. Por outro lado, se $f$ é bijetiva em $B$, então pelo teorema \eqref{teo.fund:bij1} sua relação inversa $f^{-1} \subseteq B \times A$ é uma função de $B$ em $A$ e, mais ainda, satisfaz $f^{-1} \circ f = \id_{A}$ e $f \circ f^{-1} = \id_{B}$, de modo que $f$ é invertível. \blackproof   
            \item Segue imediatamente do primeiro item. \blackproof
        \end{enumerate}
\end{proof}

\begin{cor}
    Uma função é invertível se, e somente se, é invertível à esquerda e à direita.
\end{cor}

\begin{proof}
    A ida ($\Rightarrow$) decorre imediatamente das definições. Provemos então a volta ($\Leftarrow$). Se $f$ é invertível à esquerda e à direita, então pelos teoremas \eqref{teo.fund:invesquerda} e \eqref{teo.fund:invdireita} $f$ é bijetiva em $B$, de modo que, pelo teorema \eqref{teo.fund:bij2}, $f$ é invertível. \blackproof
\end{proof}

\begin{obs}
    Vamos resumir o que está acontecendo. O resultado mais importante é a equivalência entre invertibilidade e bijetividade: por um lado, se $f : A \to B$ é bijetiva, então a relação inversa $f^{-1} \subseteq B \times A$ é uma função de $B$ em $A$ tal que $f^{-1} \circ f = \id_{A}$ e $f \circ f^{-1} = \id_{B}$, o que prova que $f$ é invertível. Por outro lado, se $f$ é invertível, então existe $f^{-1} : B \to A$ tal que $f^{-1} \circ f = \id_{A}$ e $f \circ f^{-1} = \id_{B}$, o que prova que $f$ é bijetiva.
\end{obs}

\section{O Axioma do Infinito e os Números Naturais}

\begin{defi}
    O \textit{sucessor} de um conjunto $x$ é definido como $x^+ := x \cup \{x\}$.
\end{defi}

\begin{prop}
    Valem as seguintes afirmações sobre o sucessor.
        \begin{enumerate}[leftmargin=*, align=left, label=\textbf{(\alph*)}]
            \item $\forall x \forall y ((y \in x^+) \leftrightarrow ((y \in x) \lor (y = x)))$
            \item $\forall x (x \in x^+)$
            \item $\forall x (x \subseteq x^+)$
        \end{enumerate}
\end{prop}

\begin{proof}
    \leavevmode
        \begin{enumerate}[leftmargin=*, align=left, label=\textbf{(\alph*)}]
            \item Pela definição de sucessor, $x^+ = x \cup \{x\}$. Pela definição de união, $y \in x \cup \{x\}$ se, e somente se, $y \in x$ ou $y \in \{x\}$. Como $y \in \{x\}$ equivale a $y = x$, temos que $y \in x^+$ se, e somente se, $y \in x$ ou $y = x$. \blackproof
            \item Tomando $y=x$ no item anterior, obtemos $(x \in x^+) \leftrightarrow ((x \in x) \lor (x = x))$. Como vale $x = x$, vale também a disjunção $(x \in x) \lor (x = x)$, de modo que $x \in x^+$. \blackproof
            \item Se $z \in x$, então $(z \in x) \lor (z = x)$, de modo que $z \in x^+$.
            Com isso, $\forall z (z \in x \rightarrow z \in x^+)$, isto é, $x \subseteq x^+$. \blackproof
        \end{enumerate}
\end{proof}

\begin{defi}
    Um conjunto $x$ é \textit{indutivo} se
        \[
            (\emptyset \in x) \land \forall y ( (y \in x) \rightarrow (y^+ \in x)).
        \]
    Isso é denotado por $\ind{(x)}$.
\end{defi}

\begin{ax}[Infinito]
    Existe um conjunto indutivo.
        \[
            \boxed{
                \exists x ( (\emptyset \in x) \land \forall y ((y \in x) \rightarrow (y^+ \in x)) )
            }
        \]
\end{ax}

\begin{teo} \label{teo.fund:omega}
    Seja $I$ um conjunto indutivo. Defina
        \[
            \omega(I) := \bigcap \{ x \in \mathcal{P}{(I)} : \ind{(x)}\}.
        \]
        \begin{enumerate}[leftmargin=*, align=left, label=\textbf{(\alph*)}]
            \item Para todo conjunto $I$, se $I$ é indutivo, então $\omega(I)$ é indutivo.
                \[
                    \forall I ( \ind{(I)} \rightarrow \ind{(\omega(I))})
                \]
            \item Para quaisquer conjuntos $I$ e $J$, se $I$ e $J$ são indutivos, então $\omega(I) = \omega(J)$.
                \[
                    \forall I \forall J ( \ind{(I)} \land \ind{(J)} \rightarrow \omega(I) = \omega(J) )
                \] 
        \end{enumerate}  
\end{teo}

\begin{proof} 
    Provemos que $\omega(I)$ está bem definido. Pelo axioma das partes, existe o conjunto $\mathcal{P}{(I)}$. Pelo axioma da separação com o conjunto $\mathcal{P}{(I)}$ e a fórmula $\ind{(x)}$, existe $z(I) :=  \{ x \in \mathcal{P}{(I)} : \ind{(x)} \}$. Pelo \Cref{teo.fund:intfamilia}, como $z(I) \neq \emptyset$ já que $I \in z(I)$, existe $\omega(I) := \bigcap z(I)$.
        \begin{enumerate}[leftmargin=*, align=left, label=\textbf{(\alph*)}]
            \item \blackproof
            \item \blackproof
        \end{enumerate}
\end{proof}

\begin{obs}
    O \Cref{teo.fund:omega} nos diz que $\omega(I)$ é a interseção da família de todos os conjuntos indutivos e que o parâmtro $I$ pode ser suprimido. A seguinte definição só é possível devido a esse teorema.
\end{obs}

\begin{defi}
    O conjunto dos números naturais é definido como a interseção de todos os conjuntos indutivos. Ele é denotado por $\omega$.
\end{defi}

\begin{teo}[Indução]
    Se $A \subseteq \omega$ é indutivo, então $A = \omega$.
\end{teo}

\begin{proof}
    Pelo teorema \eqref{teo.fund:omega}, se $A$ é indutivo, então $\omega \subseteq A$. Daí, como $A \subseteq \omega$, vem $A = \omega$. \blackproof
\end{proof}

\begin{teo}[Axiomas de Peano] \label{teo.fund:peano}
    O conjunto $\omega$ dos números naturais satisfaz os axiomas de Peano, isto é, valem as seguintes afirmações.
        \begin{enumerate}[leftmargin=*, align=left, label=\textbf{(\alph*)}]
            \item $\forall x \forall y (x,y \in \omega  \rightarrow  (\neg (x=y) \rightarrow \neg (x^+ = y^+ )))$.
            \item $\forall x (x \in \omega \rightarrow (\neg (x^+ = \emptyset) ))$.
            \item Para toda fórmula $P$,
                \[
                    P(\emptyset) \land  \forall x (P(x) \rightarrow P(x^+)) \rightarrow \forall x (x \in \omega \rightarrow P(x)).
                \]
        \end{enumerate}
\end{teo}

\begin{proof}
    Ver \cite{fajardo2024conjuntos}, teorema 3.20, página 89. \blackproof
\begin{comment}
    \leavevmode
        \begin{enumerate}[leftmargin=*, align=left, label=\textbf{(\alph*)}]
            \item Suponha, por absurdo, que $x \neq y$ e $x^+ = y^+$. Então $x^+ = y \cup \{y \}$, e como $x \in x^+$, vem $x \in y \cup \{y\}$. Como $x \neq y$, então $x \in y$. Analogamente, $y \in x$, o que contraria a proposição \eqref{prop.fund:xinyeyinxabs}. Logo, se $x \neq y$, então $x^+ \neq y^+$. \blackproof
        \end{enumerate}
    Note que não usamos a hipótese de ser $x,y \in \omega$, ou seja, vale a afirmação mais forte $\forall x \forall y (\neg(x=y) \rightarrow \neg (x^+ = y^+) )$.
        \begin{enumerate}[leftmargin=*, align=left, label=\textbf{(\alph*)}, resume]
            \item Sabemos que $\forall x (x \in x^+)$. Se fosse $x^+ = \emptyset$ para algum $x$, seria $x \in \emptyset$, o que não é. \blackproof
        \end{enumerate}
    Novamente, não usamos a hipótese de ser $x \in \omega$, ou seja, vale a afirmação mais forte $\forall x (\neg(x^+ = \emptyset))$.
        \begin{enumerate}[leftmargin=*, align=left, label=\textbf{(\alph*)}, resume]
            \item Pelo axioma da separação, defina $A := \{ x \in \omega : P(x) \}$. Então $A \subseteq \omega$. Afirmamos que $A$ é indutivo. Como $P(\emptyset)$, então $\emptyset \in A$. Se $x \in A$, então $x \in \omega$ e $P(x)$, e como $P(x^+)$ e $x^+ \in \omega$, vem $x^+ \in A$. Com isso, $A$ é indutivo, de modo que $\omega \subseteq A$, e como $A \subseteq \omega$, vem $A = \omega$. \blackproof
        \end{enumerate}.
\end{comment}
\end{proof}

\begin{defi}
    Um conjunto $A$ é \textit{transitivo} se
        \[
            \forall x \forall y (x \in y \land y \in A \rightarrow x \in A).
        \]
\end{defi}

\begin{prop}
    Um conjunto $A$ é transitivo se, e somente se,
        \begin{enumerate}[label=\roman*.]
            \item $\forall a (a \in A \rightarrow a \subsetneq A)$.
            \item $\bigcup A \subseteq A$.
            \item $\bigcup A^+ = A$.
            \item $A \subseteq \mathcal{P}(A)$.
        \end{enumerate}    
\end{prop}

\begin{proof}
    \leavevmode
        \begin{enumerate}[leftmargin=*, align=left, label=\textbf{(\alph*)}]
            \item Por um lado ($\Rightarrow$), se $A$ é transitivo e $a \in A$, então para todo $x \in a$ temos $x \in A$, de modo que $a \subseteq A$. Se fosse $a = A$, teríamos $a \in a$ ou $A \in A$, o que contraria o corolário \eqref{cor.fund:xnotinx}. Com isso, $a \neq A$, donde $a \subsetneq A$. Por outro lado ($\Leftarrow$), se $a \subsetneq A$ para todo $a \in A$, então $a \subseteq A$. Logo, se $x \in y$ e $y \in A$, então $y \subseteq A$, de modo que $x \in A$. \blackproof
        \end{enumerate}
\end{proof}

\begin{teo}
    \leavevmode
        \begin{enumerate}[leftmargin=*, align=left, label=\textbf{(\alph*)}]
            \item Todo número natural é um conjunto transitivo.
            \item O conjunto $\omega$ é transitivo.
        \end{enumerate}
\end{teo}

\begin{proof}
    \leavevmode
        \begin{enumerate}[leftmargin=*, align=left, label=\textbf{(\alph*)}]
            \item A prova se dará por indução em $n$. O conjunto vazio $\emptyset$ é trivialmente transitivo (por vacuidade). Agora, supondo que $n \in \omega$ é transitivo, provemos que $n^+ \in \omega$ também é transitivo. Se $x \in n^+$, então $x \in n$ ou $x = n$. 
                \begin{itemize}
                    \item Se $x \in n$, então, dado $y \in x$, como $n$ é transitivo, vem $y \in n$, e como $n \subsetneq n^+$, vem $y \in n^+$, de modo que $n^+$ é transitivo.
                    \item Se $x = n$, então, dado $y \in x$, temos $y \in n$, e como $n \subsetneq n^+$, vem $y \in n^+$, de modo que $n^+$ é transitivo. 
                \end{itemize}
            Com isso, todo número natural é transitivo. \blackproof
            \item Provemos que $\forall n (n \in \omega \rightarrow n \subsetneq \omega)$ por indução. Trivialmente $\emptyset \subsetneq \omega$. Agora, se $n \in \omega$, então $\{n\} \subsetneq \omega$, e como $n \subsetneq \omega$ (hipótese de indução), temos $n \cup \{n\} \subseteq \omega$, isto é, $n^+ \subsetneq \omega$, pois $n^+ \neq \omega$. \blackproof
        \end{enumerate}
\end{proof}

\begin{lem}
    Para quaisquer $m,n \in \omega$, valem as seguintes afirmações.
        \begin{enumerate}[leftmargin=*, align=left, label=\textbf{(\alph*)}]
            \item Se $m \in n$, então $m^+ \in n$ ou $m^+=n$.
            \item Se $m \in n$, então $m \subsetneq n$.
        \end{enumerate}
\end{lem}

\begin{proof}
    \leavevmode
        \begin{enumerate}[leftmargin=*, align=left, label=\textbf{(\alph*)}]
            \item A prova se dará por indução em $n$. Se $m \in \emptyset$, então a conclusão segue trivialmente por vacuidade. Agora, supondo que $m \in n$ implica $m^+ \in n$ ou $m^+=n$, provemos que $m \in n^+$ implica $m^+ \in n^+$ ou $m^+ = n^+$. Se $m \in n^+$, então $m \in n$ ou $m=n$. Se $m \in n$, então $m^+ \in n$ ou $m^+=n$ (hipótese de indução). Como $n \subsetneq n^+$ e $n \in n^+$, vem $m^+ \in n^+$. Se $m=n$, então $m^+=n^+$. \blackproof
            \item A prova se dará por indução em $n$. Se $m \in \emptyset$, então a conclusão segue trivialmente por vacuidade. Agora, supondo que $m \in n$ implica $m \subsetneq n$, provemos que $m \in n^+$ implica $m \subsetneq n^+$. Se $m \in n^+$, então $m \in n$ ou $m=n$. Se $m \in n$, então $m \subsetneq n$ (hipótese de indução), e como $n \subsetneq n^+$, vem $m \subsetneq n^+$. Se $m = n$, então $m \subsetneq n^+$ pois $n \subsetneq n^+$. \blackproof
        \end{enumerate}
\end{proof}

\begin{teo} \label{teo.fund:tricotomiadopertence}
    Para quaisquer $m,n \in \omega$, valem as seguintes afirmações.
        \begin{enumerate}[leftmargin=*, align=left, label=\textbf{(\alph*)}]
            \item $m \in n$ se, e somente se, $m^+ \in n^+$.
            \item Ou $m \in n$, ou $n \in m$, ou $m = n$.
        \end{enumerate}
\end{teo}

\begin{proof}
    \leavevmode
        \begin{enumerate}[leftmargin=*, align=left, label=\textbf{(\alph*)}]
            \item Por um lado ($\Rightarrow$), se $m \in n$, então $m^+ \in n$ ou $m^+ = n$. Como $n \subsetneq n^+$ e $n \in n^+$, em ambos os casos temos $m^+ \in n^+$. Por outro lado ($\Leftarrow$), se $m^+ \in n^+$, então $m^+ \in n$ ou $m^+ = n$. Se $m^+ \in n$, então $m^+ \subsetneq n$, e como $m \in m^+$, segue que $m \in n$. Se $m^+ = n$, como $m \in m^+$, temos imediatamente $m \in n$. \blackproof


            \item A prova se dará por indução em $n$. Se $n = \emptyset$, não pode ser $m \in \emptyset$. Provemos rapidamente que ou $\emptyset \in m$ ou $m = \emptyset$.
                \begin{itemize}
                    \item Se $m = \emptyset$, a afirmação é trivialmente verdadeira. Supondo que $m = \emptyset$ ou $\emptyset \in m$, provemos que $m^+ = \emptyset$ ou $\emptyset \in m^+$. Como $m \in m^+$, não pode ser $m^+ = \emptyset$. Se $m = \emptyset$, então $m^+ = \{\emptyset\}$, de modo que $\emptyset \in m^+$. Se $\emptyset \in m$, então $\emptyset \in m^+$ pois $m \subseteq m^+$.
                \end{itemize}
            Com isso, o resultado vale para $n = \emptyset$. Agora, supondo que o resultado vale para $n$, provemos que ele também vale para $n^+$. Se $m \in \omega$, então, pela hipótese de indução,
                \begin{itemize}
                    \item se $m \in n$, como $n \subsetneq n^+$, temos $m \in n^+$;
                    \item se $m = n$, como $n \in n^+$, temos $m \in n^+$;
                    \item se $n \in m$, então $n^+ \in m$ ou $n^+ = m$.
                \end{itemize}
            Com isso, em qualquer caso, pelo menos uma das afirmações $m \in n^+$, $n^+ \in m$ ou $m = n^+$ é verdadeira. A exclusividade é garantida pela \Cref{prop.fund:xinyeyinxabs} e pelo \Cref{cor.fund:xnotinx}. \blackproof
        \end{enumerate}
\end{proof}

\begin{cor} \label{cor.fund:bigcupomega}
    \leavevmode
        \begin{enumerate}[leftmargin=*, align=left, label=\textbf{(\alph*)}]
            \item Para quaisquer $m,n \in \omega$, se $m \subsetneq n$, então $m \in n$.
            \item Para quaisquer $m,n \in \omega$, se $m \subsetneq n^+$, então $m \subseteq n$.
            \item $\bigcup \omega = \omega$.
        \end{enumerate}
\end{cor}

\begin{proof}
    \leavevmode
        \begin{enumerate}[leftmargin=*, align=left, label=\textbf{(\alph*)}]
            \item Se $m \subsetneq n$, então $m \neq n$. Pelo \Cref{teo.fund:tricotomiadopertence}, ou $m \in n$, ou $n \in m$. Se fosse $n \in m$, teríamos $n \in n$, uma contradição. Logo, só pode ser $m \in n$. \blackproof
            \item Como $m \subsetneq n^+$, pelo item anterior temos $m \in n^+$. Pela definição de sucessor, $m \in n \cup \{n\}$, o que implica $m \in n$ ou $m = n$. Se $m \in n$, então $m \subseteq n$ (pois todo número natural é um conjunto transitivo). Se $m = n$, trivialmente $m \subseteq n$. Em ambos os casos, $m \subseteq n$. \blackproof
            \item Se $x \in \bigcup \omega$, então existe $y \in \omega$ tal que $x \in y$, e como $y \subseteq \omega$, vem $x \in \omega$. Com isso, $\bigcup \omega \subseteq \omega$. Agora, se $x \in \omega$, então existe $x^+ \in \omega$ tal que $x \in x^+$, de modo que $x \in \bigcup \omega$. Com isso, $\bigcup \omega \supseteq \omega$. Logo, $\bigcup \omega = \omega$. \blackproof
        \end{enumerate}
\end{proof}

\begin{teo}
    \leavevmode
        \begin{enumerate}[leftmargin=*, align=left, label=\textbf{(\alph*)}]
            \item Para todo $n \in \omega$, o conjunto $(n, \subseteq)$ é bem ordenado.
            \item $(\omega, \subseteq)$ é um conjunto bem ordenado.
        \end{enumerate}
\end{teo}

\begin{proof}
    \leavevmode
        \begin{enumerate}[leftmargin=*, align=left, label=\textbf{(\alph*)}]
            \item A prova se dará por indução em $n$. Trivialmente $\emptyset$ é bem ordenado por $\subseteq$ (por vacuidade: não existem subconjuntos não vazios de $\emptyset$). Agora, suponha que $n \in \omega$ seja bem ordenado por $\subseteq$. Tome $S \in \mathcal{P}(n^+)_{\neq \emptyset}$ e defina $S' := S \setminus \{n\}$. Se $S' = \emptyset$, então $S = \{n\}$ e $S$ tem um elemento mínimo (nesse caso, $n = \min{S}$). Se $S' \neq \emptyset$, então, como $S' \subseteq n$, existe $\min{S'} \in S'$. Provemos que $\min{S'} \subseteq x$ para todo $x \in S$. Se $x \in S$, então $x \in n^+$, de modo que $x \in n$ ou $x = n$. Se $x \in n$, então, como $x \in S$ e $x \neq n$, vem $x \in S'$, donde $\min{S'} \subseteq x$. Se $x=n$, então, como $\min{S'} \in n$ já que $S' \subseteq n$, da transitividade de $n$ vem $\min{S'} \subseteq n$, de modo que $\min{S'} \subseteq x$. Com isso, $\min{S'}$ é o elemento mínimo de $S$, de modo que $n^+$ é bem ordenado por $\subseteq$. \blackproof
            \item Sejam $S \in \mathcal{P}(\omega)_{\neq \emptyset}$ e $n_0 \in S$. Como $n_0 \in S \cap n^+_0$, temos $S \cap n^+_0 \neq \emptyset$, e como $S \cap n^+_0 \subseteq n^+_0$, existe $m := \min{S \cap n^+_0}$. Agora, sendo $n \in S$, provemos que $m \subseteq n$. Se $n \in S$, então ou $n \in n_0$, ou $n = n_0$, ou $n_0 \in n$, pela tricotomia de $\in$. Se $n \in n_0$, então, como $n_0 \subseteq n^+_0$, vem $n \in S \cap n^+_0$, de modo que $m \subseteq n$. Se $n = n_0$, então $n \in n^+_0$, e como $n_0 \subseteq n^+_0$, vem $n \in S \cap n^+_0$, de modo que $m \subseteq n$. Se $n_0 \in n$, então $n_0 \subseteq n$ (transitividade de $n$), e como $m \subseteq n_0$ já que $n_0 \in S \cap n^+_0$, vem $m \subseteq n$. Com isso, $m = \min{S \cap n^+_0}$ é o elemento mínimo de $S$, de modo que $\omega$ é bem-ordenado por $\subseteq$. \blackproof
        \end{enumerate}
\end{proof}

\subsection{O Teorema da Recursão}

Para definir funções de domínio $\omega$ recursivamente, precisamos
    \begin{enumerate}
        \item estabelecer o valor da função em 0;
        \item estabelecer uma ``regra'' para definir o valor da função em $n^+$ uma vez que se conheça o seu valor em $n$.
    \end{enumerate}

\begin{teo}[da recursão finita] \label{teo.fund:recfin}
    Sejam $X$ um conjunto, $x_0 \in X$ e $f:X \to X$. Existe uma única função $\varphi : \omega \to X$ tal que
        \begin{itemize}
            \item $\varphi(0) = x_0$;
            \item $\varphi(n^+) = f(\varphi(n))$, para todo $n \in \omega$.
        \end{itemize}
\end{teo}

\begin{proof}
    A ideia é considerar todas as relações de $\omega$ em $X$ que têm as propriedades desejadas e provar que a interseção de todas elas resulta em uma única função de domínio $\omega$. Defina
        \[
            \mathcal{C} := \left\{ R \in \mathcal{P}(\omega \times X) : (0,x_0) \in R \land \forall n \forall x ((n,x) \in R \to (n^{+},f(x)) \in R) \right\}.
        \]
    Como $\omega \times X \in \mathcal{C}$, temos $\mathcal{C} \neq \emptyset$, de modo que, pelo \Cref{teo.fund:intfamilia}, existe $\varphi := \bigcap \mathcal{C}$. Essa é a função procurada. Provemos isso.
        \begin{claim}
           $\varphi \in \mathcal{C}$.
        \end{claim}
        \begin{midproof}
            Temos $(0,x_0) \in \varphi$ porque $(0,x_0) \in R$ para toda $R \in \mathcal{C}$. Analogamente, se $(n,x) \in \varphi$, então $(n,x) \in R$ para toda $R \in \mathcal{C}$, de modo que $(n^{+},f(x)) \in R$ para toda $R \in \mathcal{C}$, o que prova que $(n^{+},f(x)) \in \varphi$ e que $\varphi \in \mathcal{C}$. \whiteproof
        \end{midproof}
        \begin{claim}
            $\Dom{(\varphi)} = \omega$.
        \end{claim}
        \begin{midproof}
            Claramente, $\Dom{(\varphi)} \subseteq \omega$. Como $(0,x_0) \in \varphi$, temos $0 \in \Dom{(\varphi)}$. Agora, se $n \in \Dom{(\varphi)}$, então existe $x \in X$ tal que $(n,x) \in \varphi$. Com isso, vem $(n^+,f(x)) \in \varphi$, donde $n^+ \in \Dom{(\varphi)}$. Assim, $\Dom{(\varphi)}$ é indutivo, e como $\Dom{(\varphi)} \subseteq \omega$, temos $\Dom{(\varphi)} = \omega$. \whiteproof
        \end{midproof}
        \begin{claim}
            A relação $\varphi \subseteq \omega \times X$ é uma função.
        \end{claim}
        \begin{midproof}
            Provemos que $(n,x) \in \varphi$ e $(n,y) \in \varphi$ implicam $x=y$ por indução em $n$.
                \begin{itemize}
                    \item Base de indução. Como $(0, x_0) \in \varphi$, se existisse $y \in X$ tal que $(0, y) \in \varphi$ e $y \neq x_0$, então $(0, x_0) \in \varphi \setminus \{(0,y)\}$, e se $(m,z) \in \varphi \setminus \{(0,y)\}$, então $(m^+, f(z)) \in \varphi$, e como $m^+ \neq 0$ para todo $m \in \omega$, teríamos $(m^+, f(z)) \in \varphi \setminus \{(0,y)\}$, de modo que $\varphi \setminus \{(0,y)\} \in \mathcal{C}$, uma contradição: teríamos $\varphi \subseteq \varphi \setminus \{(0,y)\}$, mas como $\varphi \setminus \{(0,y)\} \subseteq \varphi$, seria $\varphi \setminus \{(0,y)\} = \varphi$, isto é, $(0,y) \notin \varphi$, o que contraria a hipótese $(0,y) \in \varphi$. Com isso, se $(0,y) \in \varphi$, então $y=x_0$.
                    \item Passo indutivo. Se $n \in \Dom{(\varphi)}$, então existe $x \in X$ tal que $(n,x) \in \varphi$. Suponha, por hipótese de indução, que $(n,y) \in \varphi$ implica $y=x$. De $(n,x) \in \varphi$, vem $(n^+,f(x)) \in \varphi$. Suponha que existe $y \in X$ tal que $(n^+,y) \in \varphi$ e $y \neq f(x)$. Como $n^+ \neq 0$, teríamos $(0,x_0) \in \varphi \setminus \{(n^+,y)\}$, e se $(m,z) \in \varphi \setminus \{(n^+,y)\}$, então $(m^+,f(z)) \in \varphi$. Se $m^+ = n^+$, então $m=n$, e pela hipótese de indução, teríamos $z=x$, de modo que $f(z) = f(x) \neq y$, o que implicaria $(m^+, f(z)) \in \varphi \setminus \{(n^+,y)\}$. Por outro lado, se $m \neq n$, então $m^+ \neq n^+$, de modo que $(m^+, f(z)) \in \varphi \setminus \{(n^+,y)\}$. Em qualquer caso, teríamos $\varphi \setminus \{(n^+,y)\} \in \mathcal{C}$, uma contradição (pelo mesmo motivo visto na base de indução). Logo, se $(n^+,y) \in \varphi$, então $y=f(x)$.
                \end{itemize}
            Com isso, temos que $\varphi$ é uma função de $\omega$ em $X$. \whiteproof
        \end{midproof}
    Assim, fica provada a existência de uma função $\varphi: \omega \to X$ que tem as propriedades do enunciado. Provemos, por fim, sua unicidade.
        \begin{claim}
            A função $\varphi : \omega \to X$ é única.
        \end{claim}
        \begin{midproof}
            A prova se dará por indução. Se $\Phi : \omega \to X$ é uma função tal que $\Phi(0) = x_0$ e $\Phi(n^+) = f(\Phi(n))$ para todo $n \in \omega$, então $\Phi(0) = \varphi(0)$ e, se $\Phi(n) = \varphi(n)$, então
                \[
                    \Phi(n^+) = f(\Phi(n)) = f(\varphi(n)) = \varphi(n^+),
                \]  
            de modo que $\Phi = \varphi$. \whiteproof
        \end{midproof}
    Logo, existe uma única função $\varphi: \omega \to X$ tal que $\varphi(0) = x_0$ e $\varphi(n^+) = f(\varphi(n))$ para todo $n \in \omega$. \blackproof 
\end{proof}

\begin{teo}[da recursão com parâmetro] \label{teo.fund:recpar}
    Sejam $X$ um conjunto, $x_0 \in X$ e $f : \omega \times X \to X$. Existe uma única função $\varphi : \omega \to X$ tal que
        \begin{itemize}
            \item $\varphi(0) = x_0$;
            \item $\varphi(n^+) = f(n, \varphi(n))$, para todo $n \in \omega$.
        \end{itemize}
\end{teo}

\begin{proof}
    Defina $g : \omega \times X \to \omega \times X$ por $g(n, y) = (n^+, f(n, y))$. Pelo \Cref{teo.fund:recfin}, existe uma única função $\psi : \omega \rightarrow \omega \times X$ tal que $ \psi(0) = (0, x_0)$ e $\psi(n^+) = g(\psi(n))$ para todo $n \in \omega$.
        \begin{claim}
            $\pi_1 (\psi(n)) = n$ para todo $n \in \omega$.
        \end{claim}
        \begin{midproof}
            A prova se dará por indução em $n$. Para $n=0$, como $\psi(0) = (0, x_0)$, temos que $\pi_1(\psi(0)) = 0$. Suponha, por hipótese de indução, que $\pi_1(\psi(n)) = n$. Com isso, $\psi(n) = (n, \pi_2(\psi(n)))$. Pelas definições de $\psi$ e $g$, temos 
                \[
                    \psi(n^+) = g(\psi(n)) = g(n, \pi_2(\psi(n))) = (n^+, f(n, \pi_2(\psi(n)))),
                \]
            de modo que $\pi_1(\psi(n^+)) = n^+$. \whiteproof
        \end{midproof}
    Agora, defina $\varphi := \pi_2 \circ \psi : \omega \to X$, isto é, $\varphi(n) := \pi_2(\psi(n))$ para todo $n \in \omega$. Essa é a função procurada. Provemos isso.
        \begin{claim}
            $\varphi(0) = x_0$ e $\varphi(n^+) = f(n, \varphi(n))$, para todo $n \in \omega$.
        \end{claim}
        \begin{midproof}
            Como $\psi(0) = (0,x_0)$, temos $\varphi(0) = \pi_2(0,x_0) = x_0$. Pela definição de $\varphi$, temos $\psi(n) = (n, \varphi(n))$ para todo $n \in \omega$, de modo que
                \[
                    \psi(n^+) = (n^+, f(n, \pi_2(\psi(n)))) = (n^+, f(n, \varphi(n))).
                \]
            Com isso, vem
                \[
                    \varphi(n^+) = \pi_2(\psi(n^+)) = \pi_2(n^+, f(n, \varphi(n))) = f(n,\varphi(n))
                \]
            para todo $n \in \omega$. \whiteproof 
        \end{midproof}
    Assim, fica provada a existência de uma função $\varphi: \omega \to X$ que tem as propriedades do enunciado. Provemos, por fim, sua unicidade.
        \begin{claim}
            A função $\varphi: \omega \to X$ é única.
        \end{claim}
        \begin{midproof}
            A prova se dará por indução. Se $\Phi : \omega \to X$ é uma função tal que $\Phi(0) = x_0$ e $\Phi(n^+) = f(n, \Phi(n))$ para todo $n \in \omega$, então $\Phi(0) = \varphi(0)$ e, se $\Phi(n) = \varphi(n)$, então
                \[
                    \Phi(n^+) = f(n, \Phi(n)) = f(n, \varphi(n)) = \varphi(n^+),
                \]
            de modo que $\Phi = \varphi$. \whiteproof
        \end{midproof}
    Logo, existe uma única função $\varphi: \omega \to X$ tal que $\varphi(0) = x_0$ e $\varphi(n^+) = f(n,\varphi(n))$ para todo $n \in \omega$. \blackproof
\end{proof}

\begin{defi}
    O conjunto de todas as funções de um certo $n \in \omega$ em $X$ é denotado por $X^{< \omega}$, isto é,
        \[
            X^{< \omega} := \{f \in \mathcal{P}(\omega \times X) : \exists n (n \in \omega \land f \in X^n) \}.
        \]
\end{defi}

\begin{teo}[da recursão completa] \label{teo.fund:reccom}
    Sejam $X$ um conjunto e $f :  X^{<\omega} \to X$ uma função. Existe uma única função $\varphi : \omega \to X$ tal que $\varphi(n) = f(\varphi \restriction_n)$ para todo $n \in \omega$.
\end{teo}

\begin{proof}
    Defina $h : X^{< \omega} \to X^{< \omega}$ por $h(g) := g \cup \{(\Dom{(g)}, f(g)) \}$. Pelo \Cref{teo.fund:recfin}, existe uma única função $\psi : \omega \to X^{< \omega}$ tal que $\psi(0) = \emptyset$ e $\psi(n^+) = h(\psi(n))$.
        \begin{claim} \label{fund:reccom.claim.dompsin}
            $\psi(n) \in X^n$ para todo $n \in \omega$.
        \end{claim}
        \begin{midproof}
            Dado $n \in \omega$, como $\psi(n) \in X^{< \omega}$, existe $m \in \omega$ para o qual $\psi(n) : m \to X$. Com isso, $\Im{(\psi(n))} \subseteq X$ e a afirmação equivale a provar que $m = n$, isto é, basta provar que $\Dom{(\psi(n))} = n$ para todo $n \in \omega$. Façamos isso por indução em $n$. Para $n=0$, como $\psi(0) = \emptyset$, temos $\Dom{(\psi(0))} = \Dom{(\emptyset)} = \emptyset = 0$. Suponha, por hipótese de indução, que $\Dom{(\psi(n))} = n$. Como
                \[
                    \psi(n^+) = h(\psi(n)) = \psi(n) \cup \{ (\Dom{(\psi(n))}, f(\psi(n)))\},
                \]
            obtemos
                \[
                    \Dom{(\psi(n^+))} = \Dom{(\psi(n))} \cup \{ \Dom{(\psi(n))} \},
                \]
            e como $\Dom{(\psi(n))} = n$, vem $\Dom{(\psi(n^+))} = n \cup \{ n \} = n^+$. Isso completa o passo indutivo. Como $\Dom{(\psi(n))} = n$ e $\Im{\psi(n)} \subseteq X$, vem $\psi(n) : n \to X$, isto é, $\psi(n) \in X^n$, para todo $n \in \omega$. \whiteproof
        \end{midproof}
        \begin{claim} \label{fund:reccom.claim.restricao}
            Para quaisquer $m,n \in \omega$, se $m \subseteq n$, então $\psi(n) \restriction_m = \psi(m)$.
        \end{claim}
        \begin{midproof}
            A prova se dará por indução em $n$. Para $n=0$, obtemos $\psi(0) \restriction_m = \emptyset \restriction_m = \emptyset$, e como $m \subseteq \emptyset$, vem $m = \emptyset$, de modo que $\psi(0) \restriction_{\emptyset} = \emptyset = \psi(\emptyset) = \psi(0)$. Supondo, por hipótese de indução, que $\psi(n) \restriction_m = \psi(m)$ para todo $m \subseteq n$, provemos que $\psi(n^+) \restriction_m = \psi(m)$ para todo $m \subseteq n^+$. Se $m = n^+$, então $\psi(n^+) \restriction_{n^+} = \psi(n^+)$ trivialmente. Se $m \neq n^+$, então $m \subsetneq n^+$, de modo que $m \subseteq n$ (\Cref{cor.fund:bigcupomega}). 
                \begin{itemize}
                    \item Por um lado ($\subseteq$), seja $(k,x) \in \psi(n^+) \restriction_m$. Então $(k,x) \in \psi(n^+)$, com $k \in m$. Como $\psi(n^+) = \psi(n) \cup \{(n, f(\psi(n)))\}$, vem $(k,x) \in \psi(n)$ ou $(k,x) = (n, f(\psi(n)))$. Se fosse $(k,x) = (n, f(\psi(n)))$, então seria $k=n$, de modo que $n \in n$ pois $k \in m \subseteq n$, uma contradição. Logo, só pode ser $(k,x) \in \psi(n)$, e como $k \in m$, vem $(k,x) \in \psi(m)$ pois $\psi(n) \restriction_m = \psi(m)$ pela hipótese de indução. Assim, fica provado que $\psi(n^+) \restriction_m \subseteq \psi(m)$.
                    \item Por outro lado ($\supseteq$), seja $(k,x) \in \psi(m)$. Então, como $\psi(m) = \psi(n) \restriction_m$ pela hipótese de indução, temos $(k,x) \in \psi(n) \restriction_m$. Como $\psi(n) \subseteq \psi(n^+)$, temos $(k,x) \in \psi(n^+)$, como $k \in m$, temos $(k,x) \in \psi(n^+) \restriction_m$. Assim, fica provado que $\psi(n^+) \restriction_m \supseteq \psi(m)$.
                \end{itemize}
            Com isso, vem $\psi(n^+) \restriction_m = \psi(m)$ para todo $m \subseteq n^+$. Isso completa o passo indutivo. \whiteproof
        \end{midproof}
    Agora, defina $\varphi := \bigcup \Im{(\psi)}$. Essa é a função procurada. Provemos isso.
        \begin{claim} \label{fund:reccom.claim.varphifuncao}
            $\varphi$ é uma função de $\omega$ em $X$.
        \end{claim}
        \begin{midproof}
            Comecemos com $\varphi$ ser função. Como $\varphi \subseteq \omega \times X$, provemos que se $(k,x) \in \varphi$ e $(k,y) \in \varphi$, então $x=y$. Se $(k,x), (k,y) \in \varphi = \bigcup \Im{(\psi)}$, então existem $n,m \in \omega$ tais que $(k,x) \in \psi(n)$ e $(k,y) \in \psi(m)$. Como a relação $\subseteq$ é total em $\omega$, suponha, sem perda de generalidade, que $m \subseteq n$. Com isso, $\psi(n) \restriction_m = \psi(m)$, e como $k \in m$ pois $(k,y) \in \psi(m)$, vem $(k,x) \in \psi(m)$. Como $\psi(m)$ é uma função, vem $x=y$, de modo que $\varphi$ é uma função. Agora, vejamos que $\Dom{(\varphi)} = \omega$. Como
                \[
                    \varphi = \bigcup \Im{(\psi)} = \bigcup \{\psi(n) : n \in \omega \} = \bigcup_{n \in \omega} \psi(n),
                \]
            obtemos
                \[
                    \Dom{(\varphi)} = \Dom{\left( \bigcup_{n \in \omega} \psi(n) \right)} = \bigcup_{n \in \omega} \Dom{(\psi(n))} = \bigcup_{n \in \omega} n = \bigcup \omega = \omega,
                \]
            pois $\Dom{(\psi(n))} = n$ para todo $n \in \omega$ (\Cref{fund:reccom.claim.dompsin}) e $\bigcup \omega = \omega$ (\Cref{cor.fund:bigcupomega}). Por fim, verifiquemos que $\Im{(\varphi)} \subseteq X$. Para isso, basta ver que
                \[
                    \Im{(\varphi)} = \Im{\left(\bigcup_{n \in \omega} \psi(n) \right)} = \bigcup_{n \in \omega} \Im{(\psi(n))} \subseteq X,
                \]
            pois $\Im{(\psi(n))} \subseteq X$ para todo $n \in \omega$. Com isso, $\varphi : \omega \to X$. \whiteproof
        \end{midproof}
        \begin{claim} \label{fund:reccom.claim.propriedaderecursiva}
            $\varphi(n) = f(\varphi \restriction_n)$ para todo $n \in \omega$.
        \end{claim}
        \begin{midproof}
            Seja $n \in \omega$. Temos $(n, f(\psi(n))) \in \psi(n^+)$ pois 
                \[
                    \psi(n^+) = \psi(n) \cup \{ (n, f(\psi(n)))\},
                \]
            e como $\varphi = \bigcup \Im{(\psi)}$, temos $\psi(n^+) \subseteq \varphi$, de modo que $(n, f(\psi(n))) \in \varphi$; como $\varphi : \omega \to X$ é uma função (\Cref{fund:reccom.claim.varphifuncao}), vem  $\varphi(n) = f(\psi(n))$. Para finalizarmos, vejamos que $\psi(n) = \varphi \restriction_n$. Por um lado ($\subseteq$), se $(k,x) \in \psi(n)$, então, como $\psi(n) \in \Im{(\psi)}$, temos $(k,x) \in \bigcup \Im{(\psi)} = \varphi$. Com isso, como $k \in n$ pois $\Dom{(\psi(n))} = n$ (\Cref{fund:reccom.claim.dompsin}), vem $(k,x) \in \varphi \restriction_n$. Por outro lado ($\supseteq$), se $(k,x) \in \varphi \restriction_n$, então $(k,x) \in \varphi$ com $k \in n$ e existe $m \in \omega$ tal que $(k,x) \in \psi(m)$. Se $n \subseteq m$, então $\psi(m) \restriction_n = \psi(n)$ (\Cref{fund:reccom.claim.restricao}), e como $k \in n$, temos $(k,x) \in \psi(m) \restriction_n$, de modo que $(k,x) \in \psi(n)$. Se $m \subseteq n$, então $\psi(m) = \psi(n) \restriction_m$, e como $k \in m$, temos $(k,x) \in \psi(n) \restriction_m$, isto é, $(k,x) \in \psi(n)$. Com isso, temos $\psi(n) = \varphi \restriction_n$ para todo $n \in \omega$, de modo que, por fim, como $\varphi(n) = f(\psi(n))$, temos $\varphi(n) = f(\varphi \restriction_n)$ para todo $n \in \omega$. \whiteproof
        \end{midproof}
    Com isso, fica provada a existência de uma função $\varphi: \omega \to X$ que tem as propriedades do enunciado. Provemos, por fim, sua unicidade.
        \begin{claim}
            A função $\varphi : \omega \to X$ é única.
        \end{claim}
        \begin{midproof}
            Seja $\Phi : \omega \to X$ tal que $\Phi(n) = f(\Phi \restriction_n)$ para todo $n \in \omega$. Provemos que $\Phi \restriction_n = \varphi \restriction_n$ para todo $n \in \omega$ por indução em $n$. Para $n=0$, trivialmente $\Phi \restriction_{0} = \emptyset = \varphi \restriction_{0}$. Suponha, por hipótese de indução, que $\Phi \restriction_n = \varphi \restriction_n$. Com isso,
                \begin{align*}
                    \Phi \restriction_{n^+} &= \Phi \restriction_n \cup \{ (n, \Phi(n)) \} \\
                    &= \varphi \restriction_n \cup \{ (n, f(\Phi \restriction_n)) \} \\
                    &= \varphi \restriction_n \cup \{ (n, f(\varphi \restriction_n)) \} \\
                    &= h(\varphi \restriction_n) = h(\psi(n)) \\
                    &= \psi(n^+) = \varphi \restriction_{n^+},
                \end{align*}
            o que completa o passo indutivo. Logo
                \[
                    \Phi(n) = \Phi \restriction_{n^+}(n) = \varphi \restriction_{n^+}(n) = \varphi(n)
                \]
            para todo $n \in \omega$, de modo que $\Phi = \varphi$. \whiteproof
        \end{midproof}
    Logo, existe uma única função $\varphi : \omega \to X$ tal que $\varphi(n) = f(\varphi \restriction_n)$ para todo $n \in \omega$. \blackproof
\end{proof}

\subsection{Aritmética dos Números Naturais}

\section{O Axioma da Escolha}

\begin{ax}[da Escolha]
    Para todo conjunto $x$ de conjuntos não vazios existe uma função $\varphi : x \to \bigcup x$ tal que $\varphi(y) \in y$ para todo $y \in x$.
        \[
            \boxed{
                \forall x \left(\emptyset \notin x \rightarrow \exists \varphi \left( \varphi : x \to \bigcup x \land \forall y (y \in x \rightarrow \varphi(y) \in y) \right)\right)
            }        
        \]
\end{ax}

\begin{obs}
    Usamos a sigla \textsf{AC} (do inglês \textit{Axiom of Choice}) para nos referirmos ao axioma da escolha. Por seu caráter não construtivo, o axioma da escolha é o axioma mais controverso da matemática, evitado por uns e usado indiscriminadamente por outros. Desastres acontecem com e sem \textsf{AC}: por exemplo, sem \textsf{AC}, muitos resultados matemáticos fundamentais falham, sendo equivalentes em \textsf{ZF} a \textsf{AC} ou a alguma forma fraca de \textsf{AC}.
\end{obs}

\begin{prop}
    O axioma da escolha é equivalente à seguinte afirmação.
        \[
            \forall x \exists \varphi \left( \varphi : x \setminus \{\emptyset\} \to \bigcup x \land \forall y (y \in x \setminus \{\emptyset\} \rightarrow \varphi(y) \in y) \right).
        \]
\end{prop}

\begin{proof}
    \blackproof
\end{proof}

\begin{defi}
    Uma \textit{sequência} em $X$ \textit{indexada} $I$ é uma função $x : I \to X$. Isso é denotado por $\left( x_i \right)_{i \in I}$.
        \begin{enumerate}[leftmargin=*, align=left, label=\textbf{(\alph*)}]
            \item Denotamos por $x_i$ a imagem de $i \in I$ pela sequência $x : I \to X$, isto é, $x_i := x(i)$.
            \item Denotamos por $\left( x_i \right)_{i \in I}$ a sequência $x : I \to X$.
            \item Denotamos por $\{x_i : i \in I \}$ a imagem da sequência $\left( x_i \right)_{i \in I}$.
            \item Denotamos por $\bigcup_{i \in I} x_i$ a união da imagem da sequência $\left( x_i \right)_{i \in I}$, isto é, $\bigcup_{i \in I} x_i := \bigcup \left\{x_i : i \in I \right\}$ 
        \end{enumerate}
\end{defi}

\begin{defi}
    O \textit{produto cartesiano} de uma sequência $\left(x_i \right)_{i \in I}$ é definido como
        \[
           \prod_{i \in I} x_i := \left\{ f \in \left(\bigcup_{i \in I} x_i\right)^{I} : \forall i(i \in I \rightarrow f(i) \in x_i) \right\},
        \]
    isto é, $\prod_{i \in I} x_i$ é definido como o conjunto de todas as funções $f$ de domínio $I$ tais que $f(i) \in x_i$ para todo $i \in I$.
\end{defi}

\begin{prop}
    O axioma da escolha é equivalente à seguinte afirmação: se $\left(X_i \right)_{i \in I}$ é uma sequência com $X_i \neq \emptyset$ para todo $i \in I$, então $\prod_{i \in I} X_i \neq \emptyset$.
\end{prop}

\begin{proof}
    Usemos a notação usual de função escrevendo $g = (X_i)_{i \in I}$.

    ($\Rightarrow$) Como $g(i) \neq \emptyset$ para todo $i \in I$, temos $\emptyset \notin \Im{(g)}$, de modo que, pelo axioma da escolha, existe uma função $\varphi : \Im{(g)} \to \bigcup \Im{(g)}$ tal que $\varphi(y) \in y$ para todo $y \in \Im{(g)}$. Tomando $f := \varphi \circ g : I \to \bigcup \Im{g}$, temos, para todo $i \in I$,
        \[
            f(i) = (\varphi \circ g)(i) = \varphi[g(i)] \in g(i),
        \]
    isto é, $f(i) \in g(i)$. Como $g(i) := X_i$, vem $f(i) \in X_i$ para todo $i \in I$, de modo que $f \in \prod_{i \in I} X_i$, isto é, $\prod_{i \in I} X_i \neq \emptyset$.

    $(\Leftarrow)$ Agora, seja $x \neq \emptyset$ tal que $\emptyset \notin x$. Defina $g = (X_i)_{i \in I}$ assim: $I = x$ e $g := \id_I$. Como $X_i = g(i) = \id_I(i) = i \in I = x$ e $\emptyset \notin x$, temos $X_i \neq \emptyset$ para todo $i \in I$, de modo que $\prod_{i \in I} X_i \neq \emptyset$. Com isso, existe $\varphi \in \prod_{i \in I} X_i$ tal que $\varphi(i) \in X_i$ para todo $i \in I$, e como $X_i = i$ e $I = x$, vem $\varphi(i) \in i$ para todo $i \in x$, de modo que $\varphi : x \to \bigcup x$ é uma função de escolha em $x$. \blackproof
\end{proof}

\begin{defi}
    Sejam $(X, \leq)$ um conjunto parcialmente ordenado e $S$ um subconjunto não vazio de $X$.
        \begin{enumerate}[leftmargin=*, align=left, label=\textbf{(\alph*)}]
            \item Dizemos que $x \in S$ é \textit{maximal} em $S$ se não existe $y \in S$ tal que $x < y$.
            \item Dizemos que $x \in S$ é \textit{minimal} em $S$ se não existe $y \in S$ tal que $y < x$.
            \item Dizemos que $S$ é uma cadeia em $X$ se
                \[
                    \forall x \forall y (x,y \in S \rightarrow (x \leq y \lor y \leq x)).
                \]
        \end{enumerate}
\end{defi}

\begin{teo}[Lema de Zorn]
    Seja $(X, \leq)$ um conjunto parcialmente ordenado. Se toda cadeia em $X$ é limitada superiormente em $X$, então $X$ possui um elemento maximal. 
\end{teo}

\begin{proof}
    \blackproof
\end{proof}