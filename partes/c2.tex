%!TEX root = main.tex

\chapter{Topologia do Espaço Euclidiano}

\begin{defi}
    Uma \textit{bola aberta de raio $r \in \R_{>0}$ e centro $a \in \R^n$} é definida como
        \[
            B_r(a) := \{ x \in \R^n : \|x - a\| < r \}.
        \]
\end{defi}

\begin{defi}
    \leavevmode 
        \begin{enumerate}[leftmargin=*, align=left, label=\textbf{(\alph*)}]
            \item Um ponto $a \in \R^n$ é um \textit{ponto interior} de $A \subseteq \R^n$ se existe $r \in \R_{>0}$ tal que $B_r(a) \subseteq A$.
            \item O \textit{interior} de $A \subseteq \R^n$, denotado por $\op{int}{A}$, é definido como o conjunto de todos os pontos interiores de $A$, isto é,
                \[
                    \op{int}{A} = \{ x \in A : \exists r (r \in \R_{>0} \land B_r(x) \subseteq A ) \}.
                \]
            \item Um subconjunto $A \subseteq \R^n$ é \textit{aberto} se $\op{int}{A} = A$.
        \end{enumerate}
\end{defi}

\begin{cor}
    Toda bola aberta é um conjunto aberto.
\end{cor}

\begin{proof}
    Ver \cite{guidorizzi2}, página 113. \blackproof
\end{proof}

\begin{defi}
    Um ponto $a \in \R^n$ é um \textit{ponto de acumulação} de $A \subseteq \R^n$ se
        $
            B_r(a) \cap A_{\neq a} \neq \emptyset
        $
    para todo $r \in \R_{>0}$.
\end{defi}

\begin{defi}
    \leavevmode 
        \begin{enumerate}[leftmargin=*, align=left, label=\textbf{(\alph*)}]
            \item Um ponto $a \in \R^n$ é um \textit{ponto exterior} de $A \subseteq \R^n$ se existe $r \in \R_{>0}$ tal que $B_r(a) \cap A = \emptyset $.
            \item O \textit{exterior} de $A \subseteq \R^n$, denotado por $\op{ext}{A}$, é definido como o conjunto de todos os pontos exteriores de $A$, isto é,
                \[
                    \op{ext}{A} = \{ x \in \R^n : \exists r (r \in \R_{>0} \land B_r(x) \cap A = \emptyset ) \}.
                \]
        \end{enumerate}
\end{defi}

\begin{defi}
    Um ponto $a \in \R^n$ é um \textit{ponto da fronteira} de $A \subseteq \R^n$ se $a \notin \op{int}{A}$ e $a \notin \op{ext}{A}$. O conjunto de todos os pontos da fronteira de $A$ é denotado por $\partial A$.
\end{defi}





\chapter{Caminhos}

\begin{defi}
    Uma \textit{função vetorial de variável real} é uma função $f : X \subseteq \R \to \R^n$.
\end{defi}

As funções vetoriais de variável real que nos interessam, nesse momento, são aquelas cujo domínio é um intervalo ou uma união de intervalos.

\begin{prop}
    Seja $f : X \subseteq \R \to \R^n$ uma função. Existem e são únicas as funções $f_i : X \to \R$, com $i \in [n]$, tais que
        \[
            f(t) = (f_1(t), f_2(t), \ldots, f_n(t))
        \]
    para todo $t \in X$. Isso é denotado por $f = (f_1, f_2, \ldots, f_n)$.
\end{prop}

\begin{proof}
    Trivial. \blackproof
\end{proof}

\begin{defi}
    Uma função $f : X \subseteq \R \to \R^n$ tem limite $L \in \R^n$ quando $t$ tende ao ponto $t_0 \in X'$ se para todo $\epsilon \in \R_{>0}$ existe $\delta = \delta(\epsilon, t_0) \in \R_{>0}$ tal que
        \[
            0 < |t - t_0| < \delta \rightarrow \| f(t) - L \| < \epsilon
        \]
    para todo $t \in X$. Isso é denotado por
        \[
            \lim_{t \to t_0} f(t) = L.
        \]
\end{defi}

\begin{prop}[Unicidade do limite]
    Se uma função $f : X \subseteq \R \to \R^n$ tem limites $L_1, L_2 \in \R^n$, então $L_1 = L_2$.
\end{prop}

\begin{proof}
\end{proof}

\begin{prop}
    Sejam $F : X \subseteq \R \to \R^n$, $t_0 \in X'$ e $L \in \R^n$. Vale
        \[
            \lim_{t \to t_0} F(t) = L \Leftrightarrow \lim_{t \to t_0} \|F(t) - L\| = 0.
        \]
\end{prop}

\begin{proof}
    Ver \cite{guidorizzi2}, p. 124. \blackproof
\end{proof}

\begin{teo}
    Sejam $f : X \subseteq \R \to \R^n$, $t_0 \in X'$ e $L \in \R^n$, com $f = (f_1, f_2, \ldots, f_n)$ e $L = (L_1, L_2, \ldots, L_n)$. Para todo $i \in [n]$, vale
        \[
            \lim_{t \to t_0} f(t) = L \Leftrightarrow \lim_{t \to t_0} f_i (t) = L_i.
        \]
\end{teo}

\begin{proof}
    Ver \cite{guidorizzi2}, p. 124. \blackproof
\end{proof}

\begin{prop}[Propriedades operatórias]
    Oi
\end{prop}

\begin{proof}
\end{proof}

\begin{defi}[Continuidade]
    Sejam $f : X \subseteq \R \to \R^n$ e $t_0 \in X \cap X'$.
        \begin{enumerate}[leftmargin=*, align=left, label=\textbf{(\alph*)}]
            \item $f$ é \textit{contínua em $t_0$} se
                \[
                    \lim_{t \to t_0} f(t) = f(t_0).
                \]
            \item $f$ é \textit{contínua em $Y \subseteq X$} se $f$ for contínua em todo $t_0 \in Y$.
            \item $f$ é \textit{contínua} se $f$ for contínua em $X$.
        \end{enumerate}
\end{defi}

\begin{cor}
    Sejam $f : X \subseteq \R \to \R^n$ uma função, com $f = (f_1, f_2, \ldots, f_n$), e $t_0 \in X \cap X'$. $f$ é contínua em $t_0$ se, e somente se, $f_i$ é contínua em $t_0$, para todo $i \in [n]$.
\end{cor}

\begin{defi}
    Sejam $f : X \subseteq \R \to \R^n$ e $t_0 \in X \cap X'$.
        \begin{enumerate}[leftmargin=*, align=left, label=\textbf{(\alph*)}]
            \item $f$ é \textit{derivável}, ou \textit{diferenciável}, em $t_0$, se existir o limite
                \[
                    f'(t_0) := \lim_{t \to t_0} \dfrac{f(t) - f(t_0)}{t - t_0}.
                \]
            Mais precisamente, $f$ é derivável em $t_0$ se existir o limite $\ds \lim_{t \to t_0} g(t)$, onde $g : X \setminus \{ t_0\} \to \R^n$ é a função definida por
                \[
                    g(t) := \dfrac{f(t) - f(t_0)}{t-t_0}.
                \]
            Sendo $F$ derivável em $t_0$, ou ainda, \textit{diferenciável} em $t_0$, dizemos que o limite $F'(t_0)$ é a \textit{derivada} de $F$ em $t_0$.
            
            \item Diremos que $F$ é derivável em $Y \subseteq X$ se $F$ for derivável em todo $t \in Y$; se for $Y = X$, diremos que $F$ é derivável.
        \end{enumerate}
\end{defi}

\begin{prop}
    Sejam $F : X \subseteq \R \to \R^n$ e $t_0 \in X$. Sendo $F = (f_1, f_2, \ldots, f_n$), temos que $F$ é derivável em $t_0$ se, e somente se, $f_i$ é derivável em $t_0$ para todo $i \in [n]$. Sendo $F$ derivável em $t_0$, temos que
        \[
            F'(t) = (f_1' (t), f_2'(t), \ldots, f_n'(t) ).
        \]
\end{prop}

\begin{proof}
\end{proof}

\begin{defi} %vetor e reta tangentes
    Seja $F : X \subseteq \R \to \R^n$ derivável em $t_0 \in X$, com $F'(t_0) \neq 0$.
    
        \begin{enumerate}[leftmargin=*, align=left, label=\textbf{(\alph*)}]
            \item 
        \end{enumerate}
\end{defi}

\begin{prop}[Propriedades operatórias]
\end{prop}

\begin{proof}
\end{proof}

\begin{prop}[Regra da cadeia]
\end{prop}

\begin{prop}
    Se $F : X \subseteq \R \to \R^n$ uma função vetorial derivável em $X$ tal que $\| F(t) \| = k \in \R$ para todo $t \in X$, então $F(t) \cdot F'(t) = 0$ para todo $t \in X$.
\end{prop}

\begin{proof}
\end{proof}

\begin{defi}
    Seja $f : [a,b] \to \R^n$ uma função. A soma de Riemann $S(F, P, \xi)$ tem limite $L \in \R^n$ quando $\| P \|$ tende a $0$ se para todo $\epsilon \in \R_{>0}$ existir $\delta = \delta (\epsilon) \in \R_{>0}$ tal que
        \[
            \|S(f, P, \xi) - L\| < \epsilon
        \]
    para toda partição marcada $(P, \xi)$ de $[a,b]$ com $\| P \| < \delta$. Isso é denotado por
        \[
            \lim_{\| P \| \to 0} S(f, P, \xi) = L.
        \] 
\end{defi}

\begin{prop}
    Seja $f : [a,b] \to \R^n$ uma função. O limite das somas de Riemann, quando existe, é único, isto é, se
        \[ \ds
            \lim_{\| P \| \to 0} S(f, P, \xi) = L_1 \quad \text{e} \quad \lim_{\| P \| \to 0} S(f, P, \xi) = L_2,
        \]
    então $L_1 = L_2$.
\end{prop}

\begin{proof}
\end{proof}

\begin{defi}
    Uma função $f : [a,b] \to \R^n$ é \textit{integrável em $[a,b]$ segundo Riemann} se
        $ \ds
            \lim_{\| P \| \to 0} S(f, P, \xi)
        $
    existe. Nesse caso, esse número real é chamado de \textit{integral de $f$ em $[a,b]$ segundo Riemann} e é denotado por
        \[
            \int_{a}^{b} F(x) \, dx,
        \]
    isto é,
        \[
            \int_{a}^{b} f(x) \, dx := \lim_{\| P \| \to 0} S(f, P, \xi).
        \]
\end{defi}

\begin{prop}
    Seja $f : X \subseteq \R \to \R^n$ uma função com $F = (f_1, f_2, \ldots, f_n$). $f$ é integrável em $[a,b]$ se, e somente se, $f_i$ é integrável em $[a,b]$ para todo $i \in [n]$, sendo nesse caso
        \[
            \int_{a}^{b} F(t) \, dt = \left( \int_{a}^{b} f_1 (t) \, dt, \int_{a}^{b} f_2 (t) \, dt, \ldots, \int_{a}^{b} f_n (t) \, dt \right).
        \]
\end{prop}

\begin{proof}
\end{proof}

\begin{prop}[Propriedades operatórias]
\end{prop}

\begin{teo}(Fundamental do Cálculo, parte I) \label{teo:TFC1}
     Seja $f: [a,b] \to \R^n$ uma função integrável.
        \begin{enumerate}[leftmargin=*, align=left, label=\textbf{(\alph*)}]
            \item  A função $F: [a,b] \to \R^n$ definida por
                \[
                    F(x) := \int_{a}^{x} f(t) \, dt
                \]
            é uniformemente contínua em $[a,b]$.
            
            \item Se $f$ é contínua em $x_0 \in [a,b]$, então $F$ é derivável em $x_0$ e $F'(x_0) = f(x_0)$.
        \end{enumerate} 
\end{teo}

\begin{proof}
\end{proof}

\begin{cor}
    Seja $f : [a,b] \to \R^n$ uma função contínua em $[a,b]$.
        \begin{enumerate}[leftmargin=*, align=left, label=\textbf{(\alph*)}]
            \item A função $F: [a,b] \to \R^n$ definida por
                \[
                    F(x) := \int_{a}^{x} f(t) \, dt
                \]
            é uma primitiva de $f$ em $[a,b]$.
            
            \item Se $G: [a,b] \to \R^n$ é qualquer outra primitiva de $f$, então
                \[
                    G(x) = G(a) + \int_{a}^{x} f(t) \, dt
                \]
            para todo $x \in [a,b]$. Particularmente para $x = b$, temos
                \[
                    \int_{a}^{b} f(t) \, dt = G(b) - G(a).
                \]
        \end{enumerate}
\end{cor}

\begin{proof}
\end{proof}

\begin{teo}[Fundamental do Cálculo, parte II]
    Se $f : [a,b] \to \R^n $ é uma função integrável e $F : [a,b] \to \R^n$ é uma primitiva qualquer de $f$, então
        \[
            F(x) = F(a) + \int_{a}^{x} f(t) \, dt
        \]
    para todo $x \in [a,b] $. Particularmente para $x=b$, temos
        \[
            \int_{a}^{b} f(t) \, dt = F(b) - F(a).
        \]
\end{teo}

\begin{proof}
\end{proof}

\begin{prop}
    Seja $C \in \R^n$. Se $f : [a,b] \to \R^n$ é uma função integrável, então a função $C \cdot f : [a,b] \to \R$ é integrável e
        \[
            C \cdot \left[ \int_{a}^{b} f(t) \, dt \right] = \int_{a}^{b} [C \cdot f(t)] \, dt.
        \]
\end{prop}

\begin{proof}
\end{proof}

\begin{prop}
    Se $f : [a,b] \to \R^n$ e $\| f \| : [a,b] \to \R$ são funções integráveis em $[a,b]$, então
        \[
            \left\| \int_{a}^{b} f(t) \, dt \right\| \leq \int_{a}^{b} \| f(t) \| \, dt.
        \]
\end{prop}

\section{Curvas}

No que segue, $I, J \subseteq \R$ são intervalos.

\begin{defi}
    \leavevmode
        \begin{enumerate}[leftmargin=*, align=left, label=\textbf{(\alph*)}]
            \item Uma \textit{curva} no $\R^n$ é uma função vetorial de variável real $\alpha : I \to \R^n$.
            \item Uma \textit{curva paramétrica} é uma curva $\alpha : I \to \R^n$ contínua. O \textit{traço} da curva é a imagem de $I$ por $\alpha$, isto é, o conjunto $\alpha(I)$.
            \item Uma curva paramétrica $\alpha : I \to \R^n$ é \textit{regular} se $\alpha$ é derivável em $I$ com  $\alpha'(t) \neq 0$ para todo $t \in I$.
        \end{enumerate}
\end{defi}

\begin{defi}
    Seja $\alpha : I \to \R^n$ uma curva paramétrica regular. Uma curva paramétrica $\beta : J \to \R^n$ é uma \textit{reparametrização} de $\alpha$  se $\beta(J) = \alpha(I)$ e se existe uma \textit{função de reparametrização} $\varphi : J \to I$ bijetora, derivável em $J$, com $\varphi'(t) \neq 0$, tal que $\beta(t) = \alpha(\varphi(t))$.
\end{defi}

\begin{prop}[Reparametrização conserva regularidade]
    Se uma curva paramétrica $\beta : J \to \R^n$ é uma reparametrização de uma curva paramétrica regular $\alpha : I \to \R^n$, então é $\beta$ regular.
\end{prop}

\begin{proof}
    \blackproof
\end{proof}

\begin{defi}
    Sejam $\alpha : I \to \R^n$ uma curva paramétrica regular e $\beta : J \to \R^n$ uma reparametrização de $\alpha$ por meio de uma função de reparametrização $\varphi : J \to I$.
        \begin{enumerate}[leftmargin=*, align=left, label=\textbf{(\alph*)}]
            \item $\beta$ é uma reparametrização \textit{positiva} se $\varphi'(t) > 0$ para todo $t \in J$.
            \item $\beta$ é uma reparametrização \textit{negativa} se $\varphi'(t) < 0$ para todo $t \in J$.
        \end{enumerate}
\end{defi}

\begin{defi}
    Seja $\alpha : I \to \R^n$ uma curva paramétrica derivável e com derivada integrável.
        \begin{enumerate}[leftmargin=*, align=left, label=\textbf{(\alph*)}]
            \item O \textit{comprimento do arco} de $a \in I$ até $b \in I$ é definido como
                \[
                    L_a^b(\alpha) := \int_{a}^{b} \| \alpha'(t) \| \, dt.
                \]
            \item Seja $t_0 \in I$. A \textit{função comprimento de arco} de $\alpha$ é definida como
                \[
                    L(t) := \int_{t_0}^{t} \| \alpha'(u) \| \, du.
                \]
        \end{enumerate}
\end{defi}

\begin{prop}
    Se $\alpha : [a,b] \to \R^n$ é uma curva paramétrica regular e $\beta : [c,d] \to \R^n$ é uma reparametrização de $\alpha$, então $L_{c}^{d}(\beta) = L_{a}^{b}(\alpha)$.
\end{prop}

\begin{prop}
    A função comprimento de arco de uma curva paramétrica $\alpha : I \to \R^n$ é uma função de reparametrização.
\end{prop}

\begin{defi}
    Uma curva paramétrica $\alpha : I \to \R^n$ é \textit{reparametrizada por comprimento de arco} se $\| \alpha'(t) \| = 1$ para todo $t \in I$.
\end{defi}

\begin{defi}
    \leavevmode
        \begin{enumerate}[leftmargin=*, align=left, label=\textbf{(\alph*)}]
            \item Um caminho $\alpha : [a,b] \to \R^n$ é \textit{retificável} se existe $M \in \R$ tal que
                \[
                    L_{a}^{b}(\alpha, P) \leq M
                \]
            para toda partição $P : a = t_0 < \cdots < t_k = b$ de $[a,b]$, onde
                \[
                    L_{a}^{b}(\alpha, P) := \sum_{i=1}^{k} \| \alpha(t_i) - \alpha(t_{i-1}) \|.
                \]
            \item Seja $\alpha : [a,b] \to \R^n$ um caminho retificável. O \textit{comprimento da curva} descrita por $\alpha$ é definido como
                \[
                    L_{a}^{b}(\alpha) := \sup{\{L_{a}^{b}(\alpha, P) : P \in \mathcal{P}[a,b]\}}.
                \]
        \end{enumerate}
\end{defi}

\begin{teo}
    Se um caminho $\alpha :[a,b] \to \R^n$ é de classe $C^1$ em $[a,b]$, então
        \[
            L_{a}^{b}(\alpha) = \int_{a}^{b} \| \alpha'(t) \| \, dt.
        \]
\end{teo}

\begin{proof}
    \blackproof
\end{proof}

\chapter{Campos Escalares e Vetoriais}

\begin{defi}
    Sejam $m, n \in \N$.
        \begin{enumerate}[leftmargin=*, align=left, label=\textbf{(\alph*)}]
            \item Um \textit{campo} é uma função $f : X \subseteq \R^n \to \R^m$.
            \item Um campo $f : X \subseteq \R^n \to \R^m$ é um \textit{campo escalar} se $m=1$. Ou seja, um \textit{campo escalar} é uma função $f : X \subseteq \R^n \to \R$.
            \item Um campo $f : X \subseteq \R^n \to \R^m$ é um \textit{campo vetorial} se $m \geq 2$.
        \end{enumerate}
\end{defi}

\begin{teo}
    Se $f : X \subseteq \R^n \to \R^m$ é um campo, então existem e são únicas as funções $f_i : X \to \R$, com $i \in [m]$, tais que
        \[
            f(x) = (f_1(x), f_2(x), \ldots, f_m(x))
        \]
    para todo $x \in X$. Isso é denotado por $f = (f_1, f_2, \ldots, f_m)$.
\end{teo}

\begin{proof}
    \blackproof
\end{proof}

\begin{defi}
    Uma função $f : X \subseteq \R^n \to \R^m$ tem limite $L \in \R^m$ quando $x$ tende a $a \in X'$ se para todo $\epsilon \in \R_{>0}$ existe $\delta = \delta(\epsilon, a) \in \R_{>0}$ tal que
        \[
            0 < \| x -a \| < \delta \Rightarrow \| f(x) - L \| < \epsilon
        \]
    para todo $x \in X$. Isso é denotado por 
        \[ \ds
            \lim_{x \to a} f(x) = L.
        \]
\end{defi}

\begin{prop}
    Sejam $f : X \subseteq \R^n \to \R^m$, $L \in \R^m$ e $a \in X'$. Vale
        \[
            \lim_{x \to a} f(x) = L \Leftrightarrow \lim_{\| x-a \| \to 0} \|f(x) - L \| = 0.
        \]
\end{prop}

\begin{proof}
    \blackproof
\end{proof}

\begin{prop}[Propriedades operatórias]
    %Sejam $f,g : A \subseteq \R^n \to \R^m$ e $a \in A'$. Se $\ds \lim_{x \to a} f(x) = L_1 \in \R^m$ e $\ds \lim_{x \to a} g(x) = L_2 \in \R^m$, então propriedades operatórias
\end{prop}

\begin{defi}[Continuidade]
    Seja $f : A \subseteq \R^n \to \R^m$ um campo.
        \begin{enumerate}[leftmargin=*, align=left, label=\textbf{(\alph*)}]
            \item Dizemos que $f$ é \textit{contínua em $a \in A$} se para todo $\epsilon \in \R_{>0}$ existe $\delta = \delta(\epsilon, a) \in \R_{>0}$ tal que
                \[
                    \| x - a \| < \delta \Rightarrow \| f(x) - f(a) \| < \epsilon
                \]
            para todo $x \in A$.
            \item Dizemos que $f$ é \textit{contínua em $X \subseteq A$} se $f$ é contínua em todo ponto $a \in X$. Mais especificamente, $f$ é contínua em $X$ se para cada $a \in X$ e cada $\epsilon \in \R_{>0}$ existe $\delta = \delta(\epsilon, a) \in \R_{>0}$ tal que
                \[
                    \| x - a \| < \delta \Rightarrow \| f(x) - f(a) \| < \epsilon
                \]
            para todo $x \in A$.
        \end{enumerate}
\end{defi}

\begin{teo}
    Uma função $f : A \subseteq \R^n \to \R^m$ é contínua em $a \in A \cap A'$ se, e somente se, $\ds \lim_{x \to a} f(x) = f(a)$.
\end{teo}

\begin{proof}
    \blackproof
\end{proof}

\begin{teo}
    Sejam $f : A \subseteq \R^n \to \R^m$ e $g : B \subseteq \R^m \to \R^k$ funções tais que $f(A) \subseteq B$. Se $f$ é contínua em $a \in A$ e se $g$ é contínua em $f(a) \in B$, então a função composta $g \circ f : X \subseteq \R^n \to \R^k$ é contínua em $a$.
\end{teo}

\begin{proof}
    
\end{proof}



\begin{prop}
    Sejam $f: X \subseteq \R^n \to \R$ e $a \in X \cap X'$ tais que $\ds \lim_{x \to a} f(x)$ existe. Se $\alpha : I \subseteq \R \to \R^n$ é contínua em $t_0 \in I$, com $\alpha(t_0) = a$, e $\alpha(t) \in X$ para todo $t \in I$, com $t \neq t_0 \Rightarrow \alpha(t) \neq \alpha(t_0)$, então
        \[
            \lim_{t \to t_0} f(\alpha(t)) = \lim_{x \to a} f(x).
        \]
\end{prop}

\begin{proof}
    Ver \cite{guidorizzi2}, p. 165, exemplo 4. \blackproof
\end{proof}

\begin{teo}[Compostas]
    \leavevmode
        \begin{enumerate}[leftmargin=*, align=left, label=\textbf{(\alph*)}]
            \item Sejam $f : X \subseteq \R^n \to \R$ e $g : Y \subseteq \R \to \R$ funções tais que $f(X) \subseteq Y$. Se $f$ é contínua em $a \in X$ e $g$ é contínua em $f(a) \in Y$, então a função composta $g \circ f : \R^n \to \R$ é contínua em $a$.
            \item Sejam $f : X \subseteq \R^n \to \R$ uma função e $\alpha : I \subseteq \R \to \R^n$ uma curva tais que $\alpha(t) \in X$ para todo $t \in I$. Se $\alpha$ é contínua em $a \in I$ e $f$ é contínua em $\alpha(a) \in X$, então a função composta $f \circ \alpha : \R \to \R$ é contínua em $a$.
            \item Sejam $f : X \subseteq \R^n \to \R$ e $f_1, \ldots, f_n : Y \subseteq \R^n \to \R$ funções tais que $(f_1(x),\ldots, f_n(x)) \in X$ para todo $x \in Y$. Se $f_1, \ldots, f_n$ são contínuas em $a \in Y$, e se $f$ é contínua em $(f_1(a),\ldots, f_n(a))$, então a função composta $f((f_1(x),\ldots, f_n(x)))$ é contínua em $a$.
        \end{enumerate}
\end{teo}

\begin{proof}
    Ver \cite{guidorizzi2}, pg. 170, Teorema 1. \blackproof
\end{proof}

\begin{teo}
    Sejam $f : X \subseteq \R^n \to \R$ e $\alpha : I \subseteq \R \to \R^n$ tais que $\alpha(t) \in X$ para todo $t \in I$. Se $\alpha$ é contínua em $a \in I$ e $f$ é contínua em $\alpha(a) \in X$, então a função composta $f \circ \alpha : \R \to \R$ é contínua em $a$.
\end{teo}

\begin{proof}
    
\end{proof}

\begin{defi}
    Sejam $A \subseteq \R^n$ aberto, $a \in A$ e $y \in \R^n$.
        \begin{enumerate}[leftmargin=*, align=left, label=\textbf{(\alph*)}]
            \item Um campo escalar $f: A \to \R$ é \textit{derivável} em $a$ com respeito a $y$ se existe o limite
                \[
                    f_y'(a) := \lim_{h \to 0} g(h),
                \]
            onde $g:\{h \in \R_{\neq 0} : a+hy \in A\} \to \R$ é definida por
                \[
                    g(h) := \dfrac{f(a+hy)-f(a)}{h}.
                \]
            \item Seja $f : A \to \R$ \textit{derivável} em $a$ com respeito a $y$.
                \begin{enumerate}[label=\roman*.]
                    \item Se $\| y\| = 1$, então dizemos que $f_y'(a)$ é a \textit{derivada direcional} de $f$ na \textit{direção} de $y$.
                    \item Se $y = e_k$ para algum $k \in [n]$, então dizemos que $f_{e_k}'(a)$ é a \textit{derivada parcial} de $f$ com respeito a $e_k$. Outras notações para $f_{e_k}'(a)$ são 
                        \[
                            \dfrac{\partial f}{\partial x_k} (a), \quad D_k f(a) \quad \text{e} \quad f_{x_k}'(a).
                        \]
                \end{enumerate}
        \end{enumerate}
\end{defi}

\begin{defi}
    Sejam $A \subseteq \R^n$ aberto e $f : A \to \R$ derivável em $a \in A$ com respeito a $e_k$ para todo $k \in [n]$. O \textit{gradiente} de $f$ em $a$ é definido como
        \[
            \nabla f(a) := \left(\dfrac{\partial f}{\partial x_1} (a), \dfrac{\partial f}{\partial x_2} (a), \ldots, \dfrac{\partial f}{\partial x_n} (a)  \right).
        \]
\end{defi}

\begin{defi}
    Seja $A \subseteq \R^n$ aberto. Um campo escalar $f: A \to \R$ é \textit{diferenciável} em $a \in A$ se existe uma transformação linear $T_a : \R^n \to \R$ para a qual a função $r_a : \{h \in \R^n : a+h \in A\} \to \R$ definida por
        \[
            r_a(h) := f(a+h) - f(a) - T_a(h)
        \]
    satisfaz
        $ \ds
            \lim_{h \to 0} \dfrac{|r_a(h)|}{\|h \|} = 0.
        $
    A transformação linear $T_a$ é a \textit{derivada total} de $f$ em $a$.
\end{defi}

\chapter{Integrais de Linha}

\begin{defi}
    \leavevmode
        \begin{enumerate}[leftmargin=*, align=left, label=\textbf{(\alph*)}]
            \item Um \textit{caminho} é uma função contínua $\gamma : [a,b] \to \R^n$.
            \item Um caminho $\gamma : [a,b] \to \R^n$ é \textit{suave} se $\gamma$ é de classe $C^1$ em $]a,b[$.
            \item Um caminho $\gamma : [a,b] \to \R^n$ é \textit{suave por partes} se existe uma partição
                \[
                    P : a = x_0 < x_1 < \cdots < x_k = b
                \]
            tal que $\gamma_i := \gamma |_{[x_{i-1}, x_i]}$ é suave para todo $i \in [k]$.
        \end{enumerate}
\end{defi}

\begin{defi}
    Sejam $\gamma : [a,b] \to \R^n$ um caminho suave por partes e $f: \gamma([a,b]) \subseteq \R^n \to \R^n$ um campo vetorial limitado. A \textit{integral de linha} de $f$ com respeito a $\gamma$ em $C := \gamma([a,b])$ é definida como
        \[
            \int_C f \cdot d\gamma := \int_{a}^{b} f[\gamma(t)] \cdot \gamma'(t) \, dt.
        \]
\end{defi}

\begin{defi}
    Seja $\alpha : [a,b] \to \R^n$ um caminho. Um caminho $\beta : [c,d] \to \R^n$ é uma \textit{reparametrização} de $\alpha$ se existe uma bijeção $\varphi : [c,d] \to [a,b]$, derivável em $[c,d]$, com $\varphi'(t) \neq 0$ para todo $t \in [c,d]$, tal que $\beta(t) = \alpha(\varphi(t))$ para todo $t \in [c,d]$. Os caminhos $\alpha$ e $\beta$ são \textit{equivalentes}, enquanto a bijeção $\varphi$ é uma \textit{mudança de parâmetro}.
\end{defi}

\begin{prop}
    Dois caminhos equivalentes descrevem a mesma curva.
\end{prop}

\begin{defi}
    Sejam $\alpha : [a,b] \to \R^n$ um caminho e $\beta : [c,d] \to \R^n$ uma reparametrização de $\alpha$ por meio de uma mudança de parâmetro $\varphi : [c,d] \to [a,b]$.
        \begin{enumerate}[leftmargin=*, align=left, label=\textbf{(\alph*)}]
            \item $\beta$ é uma reparametrização \textit{positiva} se $\varphi'(t) > 0$ para todo $t \in [c,d]$. Dizemos que os caminhos $\alpha$ e $\beta$ têm o \textit{mesmo sentido} e que a mudança de parâmetro \textit{conserva a orientação} do traço.
            \item $\beta$ é uma reparametrização \textit{negativa} se $\varphi'(t) < 0$ para todo $t \in [c,d]$. Dizemos que os caminhos $\alpha$ e $\beta$ têm \textit{sentidos opostos} e que a mudança de parâmetro \textit{inverte a orientação} do traço.
        \end{enumerate}
\end{defi}


\begin{teo}
    Sejam $\gamma_1 : [a,b] \to \R^n$ e $\gamma_2  : [c,d] \to \R^n$ caminhos suaves por partes equivalentes. Seja $f : C \subseteq \R^n \to \R^n$ um campo vetorial limitado, onde $C$ é o traço dos caminhos $\gamma_1$ e $\gamma_2$.
        \begin{enumerate}[leftmargin=*, align=left, label=\textbf{(\alph*)}]
            \item Se $\gamma_1$ e $\gamma_2$ têm o mesmo sentido, então
                \[
                    \int_{C} f \cdot d \gamma_1 = \int_{C} f \cdot d \gamma_2.
                \]
            \item Se $\gamma_1$ e $\gamma_2$ têm sentidos opostos, então
                \[
                    \int_{C} f \cdot d \gamma_1 = - \int_{C} f \cdot d \gamma_2.
                \]
        \end{enumerate}
\end{teo}

\begin{proof}
    \blackproof
\end{proof}

\begin{defi}
    Sejam $\gamma : [a,b] \to \R^n$ um caminho de classe $C^1$ em $[a,b]$ e $f: \gamma([a,b]) \subseteq \R^n \to \R$ um campo escalar limitado. A \textit{integral de linha} de $f$ com respeito a $\gamma$ em $C := \gamma([a,b])$ é definida como
        \[
            \int_C f \cdot ds := \int_{a}^{b} f[\gamma(t)] \cdot \|\gamma'(t) \| \, dt.
        \]
\end{defi}









