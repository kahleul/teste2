%!TEX root = main.tex

\chapter{Sequências}

\begin{defi}
    Uma \textit{sequência numérica} é qualquer função $x : \N \to \R$, que associa a cada número natural $n$ um número real $x_n := x(n)$ (isto é, $n \mapsto x_n$), que será chamado de $n$-\textit{ésimo} termo da sequência. Escreveremos $(x_1, x_2, \ldots, x_n, \ldots)$, $\left(x_n\right)_{n \in \N}$ ou $(x_n)$ para indicar a sequência $x : \N \to \R$ cujo $n$-ésimo termo é $x_n \in \R$.
\end{defi}

\begin{defi}
    Uma sequência $(x_n)$ é
        \begin{enumerate}[label=\roman*.]
            \item \textit{crescente} se $n > m \Rightarrow x_n \geq x_m$;
            \item \textit{decrescente} se $n > m \Rightarrow x_n \leq x_m$;
            \item \textit{estritamente crescente} se $n > m \Rightarrow x_n > x_m$;
            \item \textit{estritamente decrescente} se $n > m \Rightarrow x_n < x_m$;
            \item \textit{monótona} se cumprir exatamente uma das condições acima.
        \end{enumerate}
\end{defi}

\begin{defi}
    Uma sequência $(x_n)$ é
    \begin{enumerate}[label=\roman*.]
        \item \textit{limitada superiormente} se existe $M \in \R$ tal que $x_n \leq M$ para todo $n \in \N$;
        \item \textit{limitada inferiormente} se existe $m \in \R$ tal que $m \leq x_n$ para todo $n \in \N$;
        \item \textit{limitada} se é limitada superiormente e limitada inferiormente.
        \item \textit{ilimitada} se não é limitada.
    \end{enumerate}
\end{defi}

\begin{prop}
    Uma sequência $(x_n)$ é limitada se, e somente se, existe $L \in \R_{>0}$ tal que $|x_n| \leq L$ para todo $n \in \N$.
\end{prop}

\begin{proof}
    Absolutamente trivial. \itemproof
\end{proof}

\begin{ex}
    A sequência $(x_n)$ é limitada se, e somente se, a sequência $\{|x_n|\}$ é limitada.
\end{ex}

\begin{proof}
    Elão, início da seção 4.1.
\end{proof}

\begin{defi}
    \leavevmode
        \begin{enumerate}[leftmargin=*, align=left, label=\textbf{(\alph*)}]
            \item Uma sequência $(x_n)$ é \textit{convergente} e \textit{converge} para $a \in \R$ se para todo $\epsilon \in \R_{>0}$ existe $n_0 \in \N$ tal que $n > n_0 \Rightarrow |x_n - a| < \epsilon$. Isso é denotado por
                \[ \ds
                    \lim_{n \to +\infty} x_n = a.
                \]
            \item Uma sequência $(x_n)$ é \textit{divergente} se não for convergente.
        \end{enumerate}
\end{defi}

\begin{obs}
    As notações ``$\ds \lim_{n \in \N} x_n = a$'', ``$\lim x_n =a$'', ``$x_n \to a$ quando $n \to +\infty$'' e ``$x_n \to a$'' também são frequentemente usadas para indicar que $\ds \lim_{n \to +\infty} x_n = a$.
\end{obs}

\begin{prop}[Unicidade]
    Uma sequência convergente converge para um único limite.
\end{prop}

\begin{proof}
    Provemos que se a sequência $(x_n)$ converge para $a \in \R$ e para $b \in \R$, então $a=b$.
\end{proof}

\begin{prop}
    Toda sequência convergente é limitada.
\end{prop}

\begin{proof}
\end{proof}

\begin{teo}[Convergência monótona] \label{teo.anal:convmon}
    \leavevmode
        \begin{enumerate}[leftmargin=*, align=left, label=\textbf{(\alph*)}]
            \item Toda sequência crescente e limitada superiormente é convergente.
            \item Toda sequência decrescente e limitada inferiormente é convergente.
            \item Toda sequência monótona e limitada é convergente.
        \end{enumerate}
\end{teo}

\begin{proof}
    \textbf{(a)} Seja $\left(x_n\right)_{n \in \N}$ uma sequência crescente e limitada superiormente. Como o conjunto $X := \{x_n \mid n \in \N\}$ é, por hipótese, não vazio e limitado superiormente, pela propriedade do supremo existe $\sup X$. Como, para todo $\epsilon \in \R_{>0}$, $\sup{X} - \epsilon$ não é uma cota superior de $X$, existe $n_0 \in \N$ tal que $\sup{X} - \epsilon < x_{n_0} \leq \sup{X}$. Como $\left(x_n\right)_{n \in \N}$ é crescente, para todo $n \in \N$, se $n > n_0$, então $\sup{X} - \epsilon < x_{n_0} \leq x_n \leq \sup{X} < \sup{X} + \epsilon$. Temos então que $x_n \to \sup{X}$, isto é, $\left(x_n\right)_{n \in \N}$ converge para $\sup{X}$. \itemproof

    \textbf{(b)} Segue analogamente: sendo $\left(x_n\right)_{n \in \N}$ uma sequência decrescente e limitada inferiormente, basta provar que $x_n \to \inf{\{x_n :n \in \N \}}$.
\end{proof}

\begin{defi} \label{defi:subseq}
    Uma \textit{subsequência} de uma sequência $x : \N \to \R$ dada por $n \mapsto x_n$ é qualquer composição $x \circ n : \N \to \R$, onde $n : \N \to \N$ dada por $k \mapsto n_k$ é uma sequência estritamente crescente de números naturais. A subsequência de $\left( x_n \right)_{n \in \N}$ definida por $\left( n_k \right)_{k \in \N}$ será denotada por $\left(x_{n_k} \right)_{k \in \N}$.
\end{defi}

\begin{prop} \label{prop.anal:subseqconv}
    Se uma sequência $\left(x_n\right)_{n \in \N}$ converge para $a \in \R$, então toda subsequência $\left(x_{n_k}\right)_{k \in \N}$ converge para $a$.
\end{prop}

\begin{proof}
\end{proof}

\begin{teo} \label{teo.anal:bolzanoweierstrass}
    (Bolzano-Weierstrass) Toda sequência limitada possui uma subsequência convergente.
\end{teo}

\begin{proof}
    Pelo teorema \eqref{teo.anal:convmon}, basta mostrar que toda sequência possui uma subsequência monótona. Seja $\left( x_n \right)_{n \in \N}$ uma sequência limitada. Um índice $k \in \N$ é dito \textit{básico} quando $x_p \geq x_k$ para todo $p > k$, isto é, $x_k$ é menor ou igual aos termos que o sucedem. 
        \begin{itemize}
            \item Se existem infinitos índices básicos $n_1 < n_2 < n_3 < \cdots$, então $x_{n_1} \leq x_{n_2} \leq x_{n_3} \leq \cdots$, de modo que a subsequência $\left(x_{n_k} \right)_{k \in \N}$ é crescente; como ela é limitada, pelo teorema \eqref{teo.anal:convmon}, ela é convergente.
            \item Por outro lado, se o número de índices básicos é finito, seja $n_1 \in \N$ maior que todos eles (se o número de índices básicos for 0, qualquer $n_1$ funciona). Como $n_1$ não é um índice básico, existe um índice $n_2 \in \N_{>n_1}$ tal que $x_{n_2} < x_{n_1}$. Como $n_2$ não é um índice básico, existe um índice $n_3 \in \N_{>n_2}$ tal que $x_{n_3} < x_{n_2}$. Prosseguindo deste modo, obtemos uma subsequência $\left(x_{n_k} \right)_{k \in \N}$ estritamente decrescente; como ela é limitada, pelo teorema \eqref{teo.anal:convmon}, ela é convergente.
        \end{itemize}
    Com isso, vemos que toda sequência possui uma subsequência monótona, e como a sequência original é limitada, a subsequência monótona também é (proposição \eqref{prop.anal:subseqconv}), sendo, portanto, convergente.
\end{proof}







\chapter{Limites e Continuidade}

\section{Topologia da Reta}

\begin{defi}
    Uma \textit{vizinhança} de $a \in \R$ de raio $r \in \R_{>0}$ é definida como
        \[
            V_r(a) := \{x \in \R : |x-a|<r \}.
        \]
\end{defi}

\begin{cor}
    Para quaisquer $a \in \R$ e $r \in \R_{>0}$, temos $V_r(a) = (a-r, a+r)$.
\end{cor}

\begin{defi}
    Um ponto $a \in \R$ é um \textit{ponto de acumulação} de $A \subseteq \R$ se 
        \[
            V_{\delta}(a) \cap A_{\neq a} \neq \emptyset
        \]
    para todo $\delta \in \R_{>0}$. O conjunto de todos os pontos de acumulação de $A$ é denotado por $A'$.
\end{defi}

\begin{prop}
    Seja $A \subseteq \R$.
        \begin{enumerate}[leftmargin=*, align=left, label=\textbf{(\alph*)}]
            \item $a \in A'$ se, e somente se, para todo $\delta \in \R_{>0}$ existe $x \in A$ tal que $0 < |x - a| < \delta$.
            \item $a \in A'$ se, e somente se, $0 \in B'$, onde $B := \{h \in \R_{\neq 0} : a+h \in A \}$.
        \end{enumerate}
\end{prop}

\begin{proof}
    \leavevmode
        \begin{enumerate}[leftmargin=*, align=left, label=\textbf{(\alph*)}]
            \item ($\Rightarrow$) blabla.
            
            ($\Leftarrow$) blabla. \itemproof
            \item ($\Rightarrow$) Para que $0$ seja um ponto de acumulação de $B$, para todo $\delta \in \R_{>0}$ deve existir $h \in B$ tal que $0 < |h-0| < \delta$. Bem, como $a$ é um ponto de acumulação de $A$, para todo $\delta \in \R_{>0}$ existe $x \in A$ tal que $0 < |x-a| < \delta$. Pois tomando $h := x-a$, temos que $x = a+h$, e como $x \in A$, temos $a+h \in A$, de modo que $h \in B$. Daí, segue a conclusão.

            ($\Leftarrow$) Para que $a$ seja um ponto de acumulação de $A$, para todo $\delta \in \R_{>0}$ deve existir $x \in A$ tal que $0 < |x-a| < \delta$. Bem, como $0$ é ponto de acumulação de $B$, para todo $\delta \in \R_{>0}$ existe $h \in B$ tal que $0<|h|<\delta$. Pois tome $x:=a+h$: como $h \in B$, temos que $a+h \in A$, de modo que $x \in A$. Daí, segue a conclusão. \itemproof
        \end{enumerate}
\end{proof}

\begin{defi}
    \leavevmode
        \begin{enumerate}[leftmargin=*, align=left, label=\textbf{(\alph*)}]
            \item Diremos que $a \in \R$ é um \textit{ponto de acumulação à direita} de $A \subseteq \R$ se $(a, a+ \delta) \cap A \neq \emptyset$ para todo $\delta \in \R_{>0}$.
            \item Diremos que $a \in \R$ é um \textit{ponto de acumulação à esquerda} de $A \subseteq \R$ se $(a - \delta, a) \cap A \neq \emptyset$ para todo $\delta \in \R_{>0}$.
        \end{enumerate}
\end{defi}

\section{Limites}

\begin{defi}[Limite]
    Uma função $f : A \subseteq \R \to \R$ tem \textit{limite} $L \in \R$ quando $x$ \textit{tende} a $a \in A'$ se para todo $\epsilon \in \R_{>0}$ existe $\delta = \delta(\epsilon, a) \in \R_{>0}$ tal que
        \[
            0 < | x-a | < \delta \Rightarrow | f(x) - L| < \epsilon
        \]
    para todo $x \in A$. Isso é denotado por 
        \[ \ds
            \lim_{x \to a} f(x) = L.
        \]
\end{defi}

\begin{prop}[Unicidade]
\end{prop}

\begin{proof}
    \itemproof
\end{proof}


\begin{defi}[Limites laterais]
    \leavevmode
        \begin{enumerate}[leftmargin=*, align=left, label=\textbf{(\alph*)}]
            \item Diremos que uma função $f : A \subseteq \R \to \R$ tem \textit{limite lateral à direita} $L \in \R$ quando $x$ \textit{tende} ao ponto de acumulação à direita $a \in \R$ de $A$, indicando isso por 
                \[ \ds
                    \lim_{x \to a^+} f(x) = L,
                \]
            se para todo $\epsilon \in \R_{>0}$ existir $\delta = \delta(\epsilon, a) \in \R_{>0}$ tal que
                \[
                    a < x < a+ \delta \Rightarrow | f(x) - L| < \epsilon
                \]
            para todo $x \in A$.
            \item Diremos que uma função $f : A \subseteq \R \to \R$ tem \textit{limite lateral à esquerda} $L \in \R$ quando $x$ \textit{tende} ao ponto de acumulação à esquerda $a \in \R$ de $A$, indicando isso por
                \[ \ds
                    \lim_{x \to a^-} f(x) = L,
                \]
            se para todo $\epsilon \in \R_{>0}$ existir $\delta = \delta(\epsilon, a) \in \R_{>0}$ tal que
                \[
                    a - \delta < x  < a \Rightarrow | f(x) - L| < \epsilon
                \]
            para todo $x \in A$.
        \end{enumerate}
\end{defi}

\begin{prop}[Unicidade]
\end{prop}

\begin{proof}
    \itemproof
\end{proof}

\begin{teo}[Bilateral $\Leftrightarrow$ Laterais]
\end{teo}

\begin{proof}
    \itemproof
\end{proof}

\begin{defi} (Limites no infinito)

    \textbf{(a)} Diremos que uma função $f : A \subseteq \R \to \R$, onde $A$ é ilimitado superiormente, tem limite $L \in \R$ quando $x$ \textit{cresce indefinidamente}, ou \textit{tende ao infinito positivo}, indicando isso por 
        \[
            \ds \lim_{x \to +\infty} f(x) = L,
        \]
    se para todo $\epsilon \in \R_{>0}$ existir $\delta = \delta(\epsilon) \in \R_{>0}$ tal que
        \[
            x > \delta \Rightarrow |f(x) - L| < \epsilon
        \]
    para todo $x \in A$.

    \textbf{(b)} Diremos que uma função $f : A \subseteq \R \to \R$, onde $A$ é ilimitado inferiormente, tem limite $L \in \R$ quando $x$ \textit{decresce indefinidamente}, ou \textit{tende ao infinito negativo}, indicando isso por 
        \[
            \ds \lim_{x \to -\infty} f(x) = L,
        \]
    se para todo $\epsilon \in \R_{>0}$ existir $\delta = \delta(\epsilon) \in \R_{>0}$ tal que
        \[
            x < - \delta \Rightarrow |f(x) - L| < \epsilon
        \]
    para todo $x \in A$.
\end{defi}

\begin{teo} \label{teo:caracterizacao}
    \textbf{(a)} (Unicidade do Limite) Seja $f$ uma função. O limite de $f$ quando $x \to p, p^{\pm}, \pm \infty$, quando existe, é único, isto é, se $\ds \lim f(x) = L_1$ e $\ds \lim f(x) = L_2$, então $L_1 = L_2$.

    \textbf{(b)} (Bilateral $\Leftrightarrow$ Laterais) Sejam $f$ uma função e $p$ um número real. Se existem números reais $a$ e $b$, com $a < p < b$, tais que $ \left]a, p\right[ , \ \left]p, b\right[ \subset D_f$, então
        \[ \ds
            \lim_{x \to p} f(x) = L \in \R \Leftrightarrow \lim_{x \to p^+} f(x) = L = \lim_{x \to p^-} f(x).
        \]
    \textbf{(c)} (Cálculo de Limites) Sejam $f$ e $g$ funções para as quais existe $r>0$ tal que $f(x) = g(x)$ sempre que $0 < |x-p| < r$ (caso $x \to p$), ou $p<x<p+r$ (caso $x \to p^+$), ou $p-r<x<p$ (caso $x \to p^-$), ou $x>r$ (caso $x \to + \infty$), ou $x < -r$ (caso $x \to - \infty$). Nestas condições, se $\ds \lim f(x) = L \in \R$, então $\ds \lim g(x) = L$. \begin{comment}
        Sejam $f$ e $g$ duas funções para as quais existe $r > 0$ tal que $f(x) = g(x)$ sempre que $0 < \left| x - p \right| < r$. Nestas condições, se $\ds \lim_{x \to p} f(x) \in \R$, então $\ds \lim_{x \to p} g(x) = \lim_{x \to p} f(x)$.
    \end{comment}

    \textbf{(d)} (do \textit{Confronto}) Sejam $f$, $g$ e $h$ funções para as quais existe $r>0$ tal que $f(x) \leq g(x) \leq h(x)$ sempre que $0 < |x-p| < r$ (caso $x \to p$), ou $p<x<p+r$ (caso $x \to p^+$), ou $p-r<x<p$ (caso $x \to p^-$), ou $x>r$ (caso $x \to + \infty$), ou $x < -r$ (caso $x \to - \infty$). Nestas condições, se $\ds \lim f(x) = \lim h(x) = L \in \R$, então $\ds \lim g(x) = L$. \begin{comment}
    Nestas condições, se $\ds \lim f(x) = \lim h(x) = L \in \R$, então $\ds \lim g(x) = L$.
    
    Sejam $f$, $g$ e $h$ funções tais que $\ds \lim f(x) = \lim h(x) = L$, com $x \to p, p^{\pm}, \pm \infty$. Se existe $r>0$ tal que $f(x) \leq g(x) \leq h(x)$ sempre que 
    
    \begin{itemize} 
        \item $0 < |x-p| < r$, se $x \to p$; 
        \item ou $p<x<p+r$, se $x \to p^+$;
        \item ou $p-r<x<p$, se $x \to p^-$; 
        \item ou $x>r$, se $x \to + \infty$;
        \item ou $x < -r$, se $x \to - \infty$;
    \end{itemize}

    então $\ds \lim g(x) = L$.

    Sejam $f$, $g$ e $h$ funções tais que $\ds \lim f(x) = \lim h(x) = L$, com $x \to p, p^{\pm}, \pm \infty$. Se existe $r>0$ tal que $f(x) \leq g(x) \leq h(x)$ sempre que $0 < |x-p| < r$, se $x \to p$, ou $p<x<p+r$, se $x \to p^+$, ou $p-r<x<p$, se $x \to p^-$, ou $x>r$, se $x \to + \infty$, ou $x < -r$, se $x \to - \infty$, então $\ds \lim g(x) = L$.
    
    Sejam $f$, $g$ e $h$ funções para as quais existe $r>0$ tal que $0 < |x-p| < r \Rightarrow f(x) \leq g(x) \leq h(x)$. Nestas condições, se $\ds \lim_{x \to p} f(x) = L = \lim_{x \to p} h(x)$, então $\ds \lim_{x \to p} g(x) = L$. \end{comment}

    \textbf{(e)} (Limites Básicos) Dados $a,p \in \R$, temos que \begin{comment} $\ds \lim_{x \to p} a = \lim_{x \to \pm \infty} a = a$, $\ds \lim_{x \to p} x = p$ e $\ds \lim_{x \to \pm \infty} \dfrac{1}{x} = 0$. \end{comment}
        \[
            \ds \lim_{x \to p} a = \lim_{x \to \pm \infty} a = a;
            \qquad \lim_{x \to p} x = p;
            \qquad \lim_{x \to \pm \infty} \dfrac{1}{x} = 0.
        \]  
\end{teo}

\begin{proof}
    \textbf{(a)} Consideremos o caso em que $x \to p$. Como $\ds \lim_{x \to p} f(x) = L_1$ e $\ds \lim_{x \to p} f(x) = L_2$, temos por definição que para todo $\epsilon > 0$ existem $\delta_1, \delta_2 > 0$ para os quais
    \begin{align*}  
            0 < | x - p | < \delta_1 \Rightarrow | f(x) - L_1 | < \dfrac{\epsilon}{2}; \\
            0 < | x - p | < \delta_2 \Rightarrow | f(x) - L_2 | < \dfrac{\epsilon}{2}.
        \end{align*}
    Tomando $\delta := \min \{\delta_1, \delta_2\}$, temos que para todo $\epsilon > 0$ existe $\delta > 0$ tal que
        \[
            0 < | x - p | < \delta \Rightarrow |f(x) - L_1| + |f(x) - L_2| < \epsilon.
        \]
    Com isso, temos que, para todo $\epsilon > 0$,
        \begin{align*}
            | L_1 - L_2 | &= | L_1 - f(x) + f(x) - L_2 | \\
            &\leq | L_1 - f(x) | + | f(x) - L_2 | \\
            &= |f(x) - L_1| + |f(x) - L_2| \\
            &< \epsilon,
        \end{align*}
    donde $L_1 = L_2$. \itemproof

    \textbf{(b)}

    \textbf{(c)}

    \textbf{(d)} Consideremos o caso em que $x \to p$. Como, por hipótese, $\ds \lim_{x \to p} f(x) = L = \lim_{x \to p} h(x)$, temos
        \begin{align*}
            \forall \epsilon > 0, \exists \delta_1 > 0: 0 < \left| x - p \right| < \delta_1 \Rightarrow L - \epsilon < f(x) < L + \epsilon; \\
            \forall \epsilon > 0, \exists \delta_2 > 0: 0 < \left| x - p \right| < \delta_2 \Rightarrow L - \epsilon < h(x) < L + \epsilon.
        \end{align*}
    Pois tome $\delta = \min \left\{ \delta_1, \delta_2, r \right\}$; daí, vem
        \[
            \forall \epsilon > 0, \exists \delta > 0 : 0 < \left| x-p \right| < \delta \Rightarrow L - \epsilon < f(x) \leq g(x) \leq h(x) < L + \epsilon,
        \]
    e então
        \[
            \forall \epsilon > 0, \exists \delta > 0 : 0 < \left| x-p \right| < \delta \Rightarrow L - \epsilon < g(x) < L + \epsilon, 
        \]
    donde $\ds \lim_{x \to p} g(x) = L$.
\end{proof}

\begin{teo} (Propriedades Operatórias) \label{teo:proplimites}
    Se $f_1$, $f_2$, $\ldots$, $f_n$ são funções tais que $\ds \lim f_1(x) = L_1$, $\ds \lim f_2(x) = L_2$,  $\ldots$, $\ds \lim f_n(x) = L_n$, em que $x \to p, p^{\pm}, \pm \infty$, então:
    
    \textbf{(a)} O limite da soma é igual à soma dos limites:
        \[ \ds
            \lim \left[ \sum_{i=1}^{n} f_i(x) \right] = \sum_{i=1}^{n} \left[ \lim f_i(x) \right] = \sum_{i=1}^{n} L_i = L_1 + L_2 + \ldots + L_n.
        \]  
    \textbf{(b)} O limite do produto é igual ao produto dos limites:
        \[ \ds
            \lim \left[ \prod_{i=1}^{n} f_i(x) \right] = \prod_{i=1}^{n} \left[ \lim f_i(x) \right] = \prod_{i=1}^{n} L_i =  L_1 \cdot L_2 \cdot \ldots \cdot L_n.
        \]
    \textbf{(c)} O limite do quociente é igual ao quociente dos limites, desde que o denominador seja diferente de $0$:
        \[ \ds
            \lim \dfrac{f_1(x)}{f_2(x)} = \dfrac{\lim f_1(x)}{\lim f_2(x)} = \dfrac{L_1}{L_2}. \quad (L_2 \neq 0)
        \]
\end{teo}

\begin{proof}
\end{proof}

\begin{teo} (Composição de Limites) \label{teo:complimites}
    Sejam $f$ e $g$ funções tais que $Im_f \subset D_g$ e $\ds \lim f(x) := a$, com $x \to p, \pm \infty$.  

    \textbf{(a)} Se $\ds \lim_{u \to a} g(u) = g(a)$, então
        \[ \ds
            \lim g[f(x)] = \lim_{u \to a} g(u),
        \]
    sendo $u := f(x)$.

    \textbf{(b)} Se $\ds \lim_{u \to a} g(u) := L$ e $a \notin D_g$, então
        \[ \ds
            \lim g[f(x)] = \lim_{u \to a} g(u),
        \]
    sendo $u := f(x)$.
\end{teo}

\begin{proof}
\end{proof}

\begin{obs}
    Para o item (a) do teorema acima, como, por hipótese, $\ds \lim_{u \to a} g(u) = g(a)$, podemos expressar o teorema como
        $ \ds
            \lim g \left[ f(x) \right] = g \left[ \lim f(x) \right].
        $
\end{obs}

\begin{cor} \label{consin1}
    \textbf{(a)} (Conservação do sinal) Se $\ds \lim_{x \to p} f(x) := L \neq 0$, então existe $\delta > 0$ tal que, para todo $x \in D_f$, temos $0 < |x-p| < \delta \Rightarrow f(x) \neq 0$.
    
    \textbf{(b)} Temos 
        \begin{align*}
             \lim_{x \to p} f(x) = L &\Leftrightarrow \lim_{h \to 0} f(p+h) = L \\
             &\Leftrightarrow \lim_{x \to p} [f(x) - L] = 0 \\ &\Leftrightarrow \lim_{x \to p} |f(x) - L| = 0.
        \end{align*}
\end{cor}

\begin{proof}
    \textbf{(a)} Basta tomar $\epsilon = L$. \itemproof

    \textbf{(b)}
\end{proof}

\section{Continuidade}

\begin{defi}[Continuidade] \label{ar1.defi:continuidade}
    Seja $f : A \subseteq \R \to \R$ uma função.
        \begin{enumerate}[leftmargin=*, align=left, label=\textbf{(\alph*)}]
            \item $f$ é \textit{contínua} em $a \in A$ se para todo $\epsilon \in \R_{>0}$ existe $\delta = \delta(\epsilon, a) \in \R_{>0}$ tal que
                \[
                    \forall x (x \in A \cap V_\delta(a) \Rightarrow f(x) \in V_\epsilon(f(a))),
                \]
            ou ainda, equivalentemente,
                \[
                    \forall x (x \in A \land |x-a| < \delta \Rightarrow |f(x) - f(a)| < \epsilon).
                \]
            \item $f$ é \textit{contínua} em $X \subseteq A$ se $f$ é contínua em todos os pontos de $X$, isto é, se para cada $a \in X$ e todo $\epsilon \in \R_{>0}$ existe $\delta = \delta(\epsilon, a) \in \R_{>0}$ tal que
                \[
                    |x-a| < \delta \Rightarrow |f(x) - f(a)| < \epsilon
                \]
            para todo $x \in A$.
            \item $f$ é \textit{uniformemente contínua} em $A$ se para todo $\epsilon \in \R_{>0}$ existe $\delta = \delta(\epsilon) \in \R_{>0}$ tal que
                \[
                    |x-y| < \delta \Rightarrow |f(x) - f(y)| < \epsilon
                \]
            para quaisquer $x,y \in A$.
        \end{enumerate}
\end{defi}

\begin{teo}
    Uma função $f : A \subseteq \R \to \R$ é contínua em $a \in A \cap A'$ se, e somente se, $\ds \lim_{x \to a} f(x) = f(a)$.
\end{teo}

\begin{proof}
\end{proof}

\begin{teo}
    Se $f$ e $g$ são funções contínuas em $p$, então as funções $f+g$ e $f \cdot g$ são contínuas em $p$; e se $g(p) \neq 0$, então a função $\dfrac{f}{g}$ é contínua em $p$.
\end{teo}

\begin{proof}
    Segue como corolário do Teorema \eqref{teo:proplimites}.
\end{proof}

\begin{prop}
    \textbf{(a)} Seja $a \in \R$. A função constante $f(x) := a$ é contínua.

    \textbf{(b)} A função identidade $f(x) := x$ é contínua.

    \textbf{(c)} Toda função polinomial é contínua. E ainda, toda função racional é contínua.
\end{prop}

\begin{proof}
\end{proof}

\begin{teo}
    Sejam $f : A \subseteq \R \to \R$ e $g : B \subseteq \R \to \R$ funções tais que $f(A) \subseteq B$. Se $f$ é contínua em $a \in A$ e se $g$ é contínua em $f(a) \in B$, então a função composta $g \circ f : X \subseteq \R \to \R$ é contínua em $a$.    
\end{teo}

\begin{teo}
    Sejam $f$ e $g$ funções tais que $Im_f \subset D_g$. Se $f$ é contínua em $p$ e $g$ é contínua em $f(p)$, então a função composta $(g \circ f)(x) = g[f(x)]$ é contínua em $p$.
\end{teo}

\begin{proof}
    Pois tome:
        $
            \ds \lim_{x \to p} g[f(x)] =  g \left[ \lim_{x \to p} f(x) \right] = g[f(p)].
        $
\end{proof}

\begin{teo} (Intervalos) \label{teo:intervalos}
     Sejam $f$ uma função e $p \in D_f$ um número real.
        
    \textbf{(a)} Se para todo $\epsilon > 0$ existir um intervalo aberto $\left]a,b \right[$, com $p \in \left] a,b \right[$, tal que
        $ \ds
           \forall x \in D_f : x \in \left]a,b \right[ \Rightarrow | f(x) - f(p) | < \epsilon,
        $        
    então $f$ é contínua em $p$.

    \textbf{(b)} Seja $r>0$. Se para todo $0 < \epsilon < r$ existir um intervalo aberto $I$ (como no item anterior, com $p \in I$) tal que
            $ \ds
               \forall x \in D_f : x \in I \Rightarrow | f(x) - f(p) | < \epsilon,
            $        
    então $f$ é contínua em $p$.
\end{teo}

\begin{proof}
    \textbf{(a)} Segue imediatamente do seguinte fato: para todo intervalo $]a, b[$, existe $\delta > 0$ tal que $\left]p - \delta, p + \delta \right[ \subset \left] a,b \right[$. Com efeito, basta tomar $\delta = \min \{ b-p, p-a\}$. Com isso, escolhendo esse $\delta$, temos que 
        \[
            x \in \left] p - \delta, p + \delta \right[ \Rightarrow x \in \left] a,b \right[.
        \]
    Como, por hipótese, $x \in \left] a,b \right[ \Rightarrow | f(x) - f(p) | < \epsilon$, temos então que
        \[
            x \in \left] p - \delta, p + \delta \right[ \Rightarrow |f(x) - f(p)| < \epsilon.
        \]
    Como $ x \in \left] p - \delta, p + \delta \right[ \Leftrightarrow |x-p| < \delta $, vemos que para todo $\epsilon>0$ existe um $\delta > 0$ tal que $|x-p| < \delta \Rightarrow |f(x) - f(p)| < \epsilon$, isto é, $f$ é contínua em $p$. \itemproof

    \textbf{(b)} Pelo item anterior, se $\epsilon < r$, então nada há de ser provado. Temos que provar o resultado para todos os $\epsilon$'s, isto é, falta provar o caso $\epsilon \geq r$.

    Pois tome $\epsilon_1 < r$. Para esse $\epsilon_1$, existe (por hipótese) um intervalo aberto $I$ tal que $x \in I \Rightarrow | f(x) - f(p)| < \epsilon_1$. Como $\epsilon_1 < \epsilon$, também vale 
        \[
            x \in I \Rightarrow | f(x) - f(p)| < \epsilon,
        \]
    o que completa a prova.
\end{proof}

\begin{prop}[Conservação do sinal] \label{consin2}
    Seja $f : A \subseteq \R \to \R$ uma função contínua em $a \in A$.
        \begin{enumerate}[leftmargin=*, align=left, label=\textbf{(\alph*)}]
            \item Se $f(a) > 0$, então existe $\delta \in \R_{>0}$ tal que $f(x) > 0$ para todo $x \in V_{\delta}(a) \cap A$.
            \item Se $f(a) < 0$, então existe $\delta \in \R_{>0}$ tal que $f(x) < 0$ para todo $x \in V_{\delta}(a) \cap A$.
        \end{enumerate}
\end{prop}

\begin{proof}
    \leavevmode
        \begin{enumerate}[leftmargin=*, align=left, label=\textbf{(\alph*)}]
            \item Para $\epsilon = f(a)$ na definição de continuidade, existe $\delta \in \R_{>0}$ tal que
                \[
                    \forall x (x \in V_\delta(a) \cap A \Rightarrow |f(x) - f(a)| < f(a)).
                \]
            Observando que
                \[
                    |f(x) - f(a)| < f(a) \Leftrightarrow 0 = f(a) - f(a) < f(x) < 2f(a),
                \]
            a conclusão segue. \itemproof
            \item Tomando $\epsilon = -f(a)$, segue analogamente. \itemproof
        \end{enumerate}
\end{proof}


\section{Limites Infinitos}

\begin{defi}
    (Limites infinitos quando $x \to \pm \infty$) Seja $f$ uma função.
    
    \textbf{(a)} Suponha que existe um número real $a$ tal que $\left] a, +\infty \right[ \subset D_f$.
    
    \begin{enumerate}[label=\roman*.]
        \item Diremos que $f$ \textit{cresce indefinidamente}, ou \textit{tende ao infinito positivo}, quando $x$ tende ao infinito positivo, indicando
            $ \ds
                \lim_{x \to + \infty} f(x) = + \infty,
            $
        se para todo $\epsilon > 0$ existir $\delta > 0$, com $\delta > a$, tal que $x > \delta \Rightarrow f(x) > \epsilon$.
        \item Diremos que $f$ \textit{decresce indefinidamente}, ou \textit{tende ao infinito negativo}, quando $x$ tende ao infinito positivo, indicando
            $ \ds
                \lim_{x \to + \infty} f(x) = - \infty,
            $
        se para todo $\epsilon > 0$ existir $\delta > 0$, com $\delta > a$, tal que $x > \delta \Rightarrow f(x) < - \epsilon$.
    \end{enumerate}
    
    \textbf{(b)} Suponha que existe um número real $a$ tal que $\left] -\infty, a \right[ \subset D_f$. 
    
    \begin{enumerate}[label=\roman*.]
        \item Diremos que $f$ tende ao infinito positivo, quando $x$ tende ao infinito negativo, indicando
            $ \ds
                \lim_{x \to - \infty} f(x) = + \infty,
            $
        se para todo $\epsilon > 0$ existir $\delta > 0$, com $- \delta < a$, tal que $x < - \delta \Rightarrow f(x) > \epsilon$.
        \item Diremos que $f$ tende ao infinito negativo, quando $x$ tende ao infinito negativo, indicando
            $ \ds
                \lim_{x \to - \infty} f(x) = - \infty,
            $
        se para todo $\epsilon > 0$ existir $\delta > 0$, com $- \delta < a$, tal que $x < - \delta \Rightarrow f(x) < - \epsilon$.
    \end{enumerate}
\end{defi}

\begin{defi}
    (Limites infinitos quando $x \to p^{\pm}$) Seja $f$ uma função e $p$ um número real.

    \textbf{(a)} Suponha que existe um número real $b$ tal que $\left]p,b\right[ \subset D_f$.
    
    \begin{enumerate}[label=\roman*.]
        \item Diremos que $f$ tende ao infinito positivo, quando $x$ tende a $p$, pela direita, indicando
            $ \ds
                \lim_{x \to p^+} f(x) = + \infty,
            $
        se para todo $\epsilon > 0$ existir $\delta > 0$, com $p+ \delta < b$, tal que $p < x < p + \delta \Rightarrow f(x) > \epsilon$.
        \item Diremos que $f$ tende ao infinito negativo, quando $x$ tende a $p$, pela direita, indicando
            $ \ds
                \lim_{x \to p^+} f(x) = - \infty,
            $
        se para todo $\epsilon > 0$ existir $\delta > 0$, com $p+ \delta < b$, tal que $p < x < p + \delta \Rightarrow f(x) < - \epsilon$.
    \end{enumerate}

    \textbf{(b)} Suponha que existe um número real $a$ tal que $\left]a,p\right[ \subset D_f$.

    \begin{enumerate}[label=\roman*.]
        \item Diremos que $f$ tende ao infinito positivo, quando $x$ tende a $p$, pela esquerda, indicando
            $ \ds
                \lim_{x \to p^-} f(x) = + \infty,
            $
        se para todo $\epsilon > 0$ existir $\delta > 0$, com $a< p - \delta$, tal que $p - \delta < x < p \Rightarrow f(x) > \epsilon$.
        \item Diremos que $f$ tende ao infinito negativo, quando $x$ tende a $p$, pela esquerda, indicando
            $ \ds
                \lim_{x \to p^-} f(x) = - \infty,
            $
        se para todo $\epsilon > 0$ existir $\delta > 0$, com $a< p - \delta$, tal que $p - \delta < x < p \Rightarrow f(x) < - \epsilon$.
    \end{enumerate}
\end{defi}

\begin{defi}
    (Limites infinitos quando $x \to p$) Seja $f$ uma função e $p$ um número real. Suponha que existem números reais $a$ e $b$, com $a < p < b$, tais que $ \left]a, p\right[ , \ \left]p, b\right[ \subset D_f$.

    \begin{enumerate}[label=\roman*.]
        \item Diremos que $f$ tende ao infinito positivo, quando $x$ tende a $p$, indicando
            $ \ds
                \lim_{x \to p} f(x) = + \infty,
            $
        se para todo $\epsilon > 0$ existir $\delta > 0$, com $a < p - \delta$ e $p + \delta < b$, tal que $0 < |x-p| < \delta \Rightarrow f(x) > \epsilon$.
        \item Diremos que $f$ tende ao infinito negativo, quando $x$ tende a $p$, indicando
            $ \ds
                \lim_{x \to p} f(x) = - \infty,
            $
        se para todo $\epsilon > 0$ existir $\delta > 0$, com $a < p - \delta$ e $p + \delta < b$, tal que $0 < |x-p| < \delta \Rightarrow f(x) < -\epsilon$.
    \end{enumerate}
\end{defi}

\begin{teo}
    Seja $f$ uma função e $p$ um número real. Se existem números reais $a$ e $b$, com $a < p < b$, tais que $ \left]a, p\right[ , \ \left]p, b\right[ \subset D_f$, então
        \[ \ds
            \lim_{x \to p} f(x) = \pm \infty \Leftrightarrow \lim_{x \to p^+} f(x) = \pm \infty = \lim_{x \to p^-} f(x).
        \]
\end{teo}

\begin{proof}
\end{proof}

\begin{teo}
    Os resultados a seguir valem para $x \to p$, $x \to p^{\pm}$ e $x \to \pm \infty$.

    \textbf{(a)} Se $\lim f(x) = \lim g(x) = \pm \infty$, então $\lim [f(x) + g(x)] = \pm \infty$ e $\lim [f(x)g(x)] = + \infty$.
    
    \textbf{(b)} Se $\lim f(x) = - \infty$ e $\lim g(x) = + \infty$, então $\lim [f(x)g(x)] = - \infty$.

    \textbf{(c)} Seja $\lim f(x) = L$. Se $\lim g(x) = \pm \infty$, então $\lim [f(x) + g(x)] = \pm \infty$.

    \textbf{(d)} Seja $\lim f(x) = L > 0$. Se $\lim g(x) = \pm \infty$, então $\lim [f(x)g(x)] = \pm \infty$.

    \textbf{(e)} Seja $\lim f(x) = L < 0$. Se $\lim g(x) = \pm \infty$, então $\lim [f(x)g(x)] = \mp \infty$.
\end{teo}

\begin{proof}
\end{proof}

\begin{prop}
    \textbf{(a)} Seja $\ds \lim f(x) = 0$, com $x \to p^{\pm}$. Se existe $r>0$ tal que $f(x) > 0$ sempre que $p<x<p+r$, se $x \to p^+$, ou $p-r<x<p$, se $x \to p^-$, então $\ds \lim \dfrac{1}{f(x)} = + \infty$.

    \textbf{(b)} Sejam $\ds \lim f(x) = L \neq 0$ e $\ds \lim g(x) = 0$, com $x \to p^{\pm}$. Se existe $r>0$ tal que $f(x) > 0$ sempre que $p<x<p+r$, se $x \to p^+$, ou $p-r<x<p$, se $x \to p^-$, então ou $\ds \lim \dfrac{f(x)}{g(x)} = + \infty$, ou $\ds \lim \dfrac{f(x)}{g(x)} = - \infty$, ou $\ds \lim \dfrac{f(x)}{g(x)}$ não existe.
\end{prop}

\begin{proof}
\end{proof}

\section{Limites e Sequências}

\begin{defi}
    
    \textbf{(c)} Diremos que $\left(x_n\right)_{n \in \N}$
        \begin{enumerate}
            \item \textit{diverge} para $+ \infty$, indicando
            \[ \ds
                \lim_{n \to +\infty} x_n = + \infty,
            \]
        se para todo $\epsilon \in \R_{>0}$ existir $n_0 \in \N$ tal que $n > n_0 \Rightarrow x_n > \epsilon$.
        \item \textit{diverge} para $- \infty$, indicando
            \[ \ds
                \lim_{n \to +\infty} x_n = - \infty,
            \]
        se para todo $\epsilon \in \R_{>0}$ existir $n_0 \in \N$ tal que $n > n_0 \Rightarrow x_n < - \epsilon$.
        \end{enumerate}
\end{defi}

\begin{obs}
    \textbf{(a)} Note que as definições acima são análogas àquelas que demos aos limites no infinito de funções. Assim, os resultados sobre os limites da forma $\ds \lim_{x \to + \infty} f(x)$ também são válidos para os limites da forma $\ds \lim_{x \to + \infty} x_n$.

    \textbf{(b)} A notação ``$x_n \to a$ quando $n \to +\infty$'' também é frequentemente usada para indicar $\ds \lim_{n \to +\infty} x_n = a$. Quando não houver confusão, podemos escrever simplesmente $x_n \to a$. Analogamente, também podemos escrever $x_n \to \pm \infty$ quando $n \to \infty$ ou, simplesmente, $x_n \to \pm \infty$.
\end{obs}

\begin{teo} \label{teo:seqmonlim}
    (Convergência Monótona)

    \textbf{(a)} Toda sequência crescente e limitada superiormente é convergente.

    \textbf{(b)} Toda sequência decrescente e limitada inferiormente é convergente.
\end{teo}

\begin{proof}
    \textbf{(a)} Seja $\left(x_n\right)_{n \in \N}$ uma sequência crescente e limitada superiormente. Como o conjunto $X := \{x_n \mid n \in \N\}$ é, por hipótese, não vazio e limitado superiormente, pela propriedade do supremo existe $\sup X$. Como, para todo $\epsilon \in \R_{>0}$, $\sup{X} - \epsilon$ não é uma cota superior de $X$, existe $n_0 \in \N$ tal que $\sup{X} - \epsilon < x_{n_0} \leq \sup{X}$. Como $\left(x_n\right)_{n \in \N}$ é crescente, para todo $n \in \N$, se $n > n_0$, então $\sup{X} - \epsilon < x_{n_0} \leq x_n \leq \sup{X} < \sup{X} + \epsilon$. Temos então que $x_n \to \sup{X}$, isto é, $\left(x_n\right)_{n \in \N}$ converge para $\sup{X}$. \itemproof

    \textbf{(b)} Segue analogamente: sendo $\left(x_n\right)_{n \in \N}$ uma sequência decrescente e limitada inferiormente, basta provar que $x_n \to \inf{\{x_n :n \in \N \}}$.
    \begin{comment}
    
    para todo $\epsilon \in \R_{>0}$ existe $n_0 \in \N$ tal que 
        \[
            n > n_0 \Rightarrow \sup{X} - \epsilon < x_n < \sup{X} + \epsilon
        \]
    Em particular, isso significa que para todo $\epsilon \in \R_{>0}$ 
    
    existe $n_0 \in \N$ tal que $n > n_0 \Rightarrow \sup{A} - \epsilon < a_n$. Como $a_n \leq \sup A < \sup A + \epsilon$, temos então que $\ds \lim_{n \to +\infty} a_n = \sup A$, como havíamos afirmado.

    Provaremos que se a sequência $\left(a_n\right)_{n \geq 0}$ for crescente e limitada superiormente, então
        $ \ds
            \lim_{n \to +\infty} a_n = \sup \{a_n \mid n \geq 0\}.
        $
    (É possível provar, de forma análoga, que se a sequência $\left(a_n\right)_{n \geq 0}$ for decrescente e limitada inferiormente, então 
        $ \ds
            \lim_{n \to +\infty} a_n = \inf \{a_n \mid n \geq 0\}.)
        $    \end{comment}
\end{proof}

\begin{teo} \label{teo.calc:bolzanoweierstrass}
    (Bolzano-Weierstrass) Toda sequência limitada possui uma subsequência convergente.
\end{teo}

\begin{proof}
    Pelo teorema \eqref{teo:seqmonlim}, basta mostrar que toda sequência possui uma subsequência monótona. Seja $\left( x_n \right)_{n \in \N}$ uma sequência limitada. Um índice $k \in \N$ é dito \textit{básico} quando $x_p \geq x_k$ para todo $p > k$, isto é, $x_k$ é menor ou igual aos termos que o sucedem. 
        \begin{itemize}
            \item Se existem infinitos índices básicos $n_1 < n_2 < n_3 < \cdots$, então $x_{n_1} \leq x_{n_2} \leq x_{n_3} \leq \cdots$, de modo que a subsequência $\left(x_{n_i} \right)_{i \in \N}$ é crescente; como ela é limitada, pelo teorema \eqref{teo:seqmonlim}, ela é convergente.
            \item Por outro lado, se o número de índices básicos é finito, seja $n_1 \in \N$ maior que todos eles (se o número de índices básicos for 0, qualquer $n_1$ funciona). Como $n_1$ não é um índice básico, existe um índice $n_2 \in \N_{>n_1}$ tal que $x_{n_2} < x_{n_1}$. Como $n_2$ não é um índice básico, existe um índice $n_3 \in \N_{>n_2}$ tal que $x_{n_3} < x_{n_2}$. Prosseguindo deste modo, obtemos uma subsequência $\left(x_{n_i} \right)_{i \in \N}$ estritamente decrescente; como ela é limitada, pelo teorema \eqref{teo:seqmonlim}, ela é convergente.
        \end{itemize}
    Com isso, toda sequência possui uma subsequência monótona, e como a sequência original é limitada, a subsequência monótona também é, sendo, portanto, convergente.
\end{proof}

\begin{teo} \label{teo.calc:contseq}
    Seja $f : A \subseteq \R \to \R$ uma função e $a \in A$. As seguintes afirmações são equivalentes:
        \begin{itemize}
            \item $f$ é contínua em $a$;
            \item toda sequência $\left( x_n \right)_{n \in \N}$, com $x_n \in A$ para todo $n \in \N$, satisfaz 
                \[
                    x_n \to a \Rightarrow f(x_n) \to f(a).
                \]
        \end{itemize}
\end{teo}

\begin{proof}
\end{proof}

\begin{teo} \label{teo.calc:contunifocont}
    Toda função $f :[a,b] \to \R$ contínua em $[a,b]$ é uniformemente contínua em $[a,b]$.
\end{teo}

\begin{proof}
    Suponha que exista uma função $f$ contínua em $[a,b]$ que não seja uniformemente contínua em $[a,b]$. Negando a definição de continuidade uniforme \eqref{ar1.defi:continuidade}, isso significa que existe $\epsilon_0 \in \R_{>0}$ tal que, para todo $\delta \in \R_{>0}$, existem $x,y \in [a,b]$ tais que $|x-y| < \delta$ e $|f(x) - f(y)| \geq \epsilon_0$. Em particular, escolhendo $\delta_n = \frac{1}{n}$ para cada $n \in \N$, existem $x_n, y_n \in [a,b]$ tais que $|x_n-y_n| < \frac{1}{n}$ e $|f(x_n) - f(y_n)| \geq \epsilon_0$; definimos, assim, duas sequências $\left(x_n \right)_{n \in \N}$ e $\left(y_n \right)_{n \in \N}$. Como $x_n \in [a,b]$ para todo $n \in \N$, a sequência $\left(x_n \right)_{n \in \N}$ é limitada, de modo que, pelo teorema de Bolzano-Weierstrass \eqref{teo.calc:bolzanoweierstrass}, existe uma subsequência $\left(x_{n_k} \right)_{k \in \N}$ que converge para algum $L \in [a,b]$, isto é, $x_{n_k} \to L$. Considerando a subsequência correspondente $\left(y_{n_k} \right)_{k \in \N}$, temos, para todo $k \in \N$,
        \[
            | x_{n_k} - y_{n_k} | < \dfrac{1}{n_k}.
        \]
    Como $n_k \to +\infty$ (pois $\left(n_k\right)_{k \in \N}$ é uma sequência de índices), temos que $\frac{1}{n_k} \to 0$, de modo que
        \[
            \lim_{k \to + \infty} |x_{n_k} - y_{n_k} | = 0.
        \]
    Com isso, sendo $x_{n_k} \to L$, só pode ser $y_{n_k} \to L$. Como $f$ é contínua em $L \in [a,b]$, pelo teorema \eqref{teo.calc:contseq} temos que
        \[
            \lim_{k \to \infty} f(x_{n_k}) = f(L) \quad \text{e} \quad \lim_{k \to \infty} f(y_{n_k}) = f(L),
        \]
    de modo que
        $ \ds
            \lim_{k \to + \infty} |f(x_{n_k}) - f(y_{n_k})| = 0,
        $
    o que contraria a hipótese de ser $|f(x_{n_k}) - f(y_{n_k})| \geq \epsilon_0 > 0$ para todo $k \in \N$. Assim, uma tal função $f$ não pode existir.
\end{proof}

\begin{teo} \label{teo:intenc}
    Se $\left(a_n\right)_{n \geq 0}$ e $\left(b_n\right)_{n \geq 0}$ são sequências tais que $\ds \lim_{n \to + \infty} (b_n - a_n) = 0$ e, para todo $n \in \N$, $a_n \leq a_{n+1} \leq b_{n+1} \leq b_{n}$, então existe um único $\alpha \in \R$ tal que, para todo $n \in \N$, $a_n \leq \alpha \leq b_n$.
\end{teo}

\begin{proof}
    (Existência) A segunda condição nos diz que $\left(a_n\right)_{n \geq 0}$ é crescente (pois $a_n \leq a_{n+1}$) e limitada superiormente (pois todo $b_n$ é uma cota superior dessa sequência). Analogamente, $\left(b_n\right)_{n \geq 0}$ é decrescente e limitada inferiormente. Assim, pelo Teorema \eqref{teo:seqmonlim}, existem $\ds \alpha := \lim_{n \to + \infty} a_n$ e $\ds \beta := \lim_{n \to + \infty} b_n$, e então $\ds 0 = \lim_{n \to + \infty} (b_n - a_n) = \alpha - \beta$, donde $\alpha = \beta$. Ainda pelo Teorema \eqref{teo:seqmonlim}, $\alpha = \sup \{a_n \mid n \in \N \}$, e então, para todo $n \in \N$, $a_n \leq \alpha$. Analogamente, $\alpha = \beta = \inf{\{b_n \mid n \in \N \}}$, e então, para todo $n \in \N$, temos que $\alpha \leq b_n$. Logo, existe um $\alpha \in \R$ tal que, para todo $n \in \N$, $a_n \leq \alpha \leq b_n$.

    (Unicidade) Suponha que existe $\alpha_1 \in \R$ para o qual também vale $a_n \leq \alpha_1 \leq b_n$. Daí, $0 \leq \alpha_1 - a_n \leq b_n - a_n$; observando que $\ds \lim_{n \to + \infty} 0 = 0$ e $\ds \lim_{n \to + \infty} (b_n - a_n) = 0$, temos, pelo Teorema do Confronto, que $\ds 0 = \lim_{n \to + \infty} (\alpha_1  - a_n) = \alpha_1 - \alpha$, isto é, $\alpha_1 = \alpha$. Logo, é único o $\alpha \in \R$ tal que, para todo $n \in \N$, $a_n \leq \alpha \leq b_n$.
\end{proof}

\begin{cor} \label{teo:intenc2}
    (Intervalos Encaixantes) Se $\left([a_n, b_n]\right)_{n \geq 0}$ for uma sequência de intervalos fechados em que, para todo $n \in \N$, $[a_n, b_n] \supset [a_{n+1}, b_{n+1}]$, e $\ds \lim_{n \to + \infty} (b_n - a_n) = 0$, então o conjunto
        $
            \bigcap_{n=0}^{\infty} [a_n, b_n]
        $
    é unitário.
\end{cor}

\begin{proof}
    Este enunciado é equivalente ao enunciado do Teorema \eqref{teo:intenc}.
\end{proof}

\begin{cor} \label{teo:intenc3}
    Se $\left([a_n, b_n]\right)_{n \geq 0}$ for uma sequência de intervalos encaixantes, com $a_n, b_n \geq 0$, então $\left([a^m_n, b^m_n]\right)_{n \geq 0}$, com $m \geq 2$ natural, também será uma sequência de intervalos encaixantes.
\end{cor}

\begin{proof}
    Basta ver que, para todo $n \in \N$,
        \begin{align*}
            [a_n, b_n] \supset [a_{n+1}, b_{n+1}] &\Leftrightarrow a_n \leq a_{n+1} \leq b_{n+1} \leq b_n \\ 
            &\Leftrightarrow a^m_n \leq a^m_{n+1} \leq b^m_{n+1} \leq b^m_n \\
            &\Leftrightarrow [a^m_n, b^m_n] \supset [a^m_{n+1}, b^m_{n+1}],
        \end{align*}
    e ainda,
        \begin{align*}
            \lim_{n \to + \infty} (b^m_n - a^m_n) &= \left( \lim_{n \to + \infty} b_n \right)^m - \left( \lim_{n \to + \infty} a_n \right)^m \\ 
            &= \left( \lim_{n \to + \infty} a_n \right)^m - \left( \lim_{n \to + \infty} a_n \right)^m = 0.
        \end{align*}
    Logo, $\left([a^m_n, b^m_n]\right)_{n \geq 0}$ é de intervalos encaixantes.

    Ademais, se $\alpha$ é o real que satisfaz, para todo $k \in \N$, $a_k \leq \alpha \leq b_k$, então $\alpha^m$ é o real que satisfaz, para todo $k \in \N$, $a^m_k \leq \alpha^m \leq b^m_k$.\footnote{Prove! O argumento é semelhante ao argumento da unicidade no Teorema \eqref{teo:intenc}.}
\end{proof}

\section{Teoremas do Valor Intermediário e de Weierstrass}

\begin{teo}[Bolzano]
    Seja $f : [a,b] \to \R$ uma função contínua em $[a,b]$. Se $f(a) \cdot f(b) < 0$, então existe $c \in (a,b)$ tal que $f(c) = 0$.
\end{teo}

\begin{proof}
    Suponha, sem perda de generalidade, que $f(a) < 0$ e $f(b) > 0$. Pois tome
        \[
            S := \{x \in [a,b] : f(x) \leq 0\}.
        \]
    Temos $S \neq \emptyset$ pois $a \in S$ já que $f(a) < 0$. $S$ é limitado pois $S \subsetneq [a,b]$. Assim, existe $p := \sup{S}$. Provemos que $f(p) = 0$. De fato, pela tricotomia de $<$ em $\R$, ou $f(p) > 0$, ou $f(p) < 0$, ou $f(p) = 0$. 
        \begin{itemize}
            \item Se fosse $f(p) > 0$, pela conservação do sinal existiria $\delta \in \R_{>0}$ tal que $f(x) > 0$ para todo $x \in (p-\delta,p+\delta) \cap [a,b]$. Com isso, se $x \in S$, então $x \leq p - \delta$, de modo que $p - \delta$ é uma cota superior de $S$, uma contradição pois $p - \delta < p = \sup{S}$.
            \item Se fosse $f(p) < 0$, pela conservação do sinal existiria $\delta \in \R_{>0}$ tal que $f(x) < 0$ para todo $x \in (p-\delta,p+\delta) \cap [a,b]$. Em particular, se $x \in (p,p+\delta) \cap [a,b]$, então $f(x) > 0$ e $x \in S$, uma contradição pois $x > p = \sup{S}$.
        \end{itemize}
    Logo, só pode ser $f(p) = 0$. Além disso, como $p \in [a,b]$ e $f(a) < 0$ e $f(b) > 0$, temos que $p \in (a,b)$. \itemproof
\end{proof}

\begin{proof}
    Suponha, sem perda de generalidade, que $f(a) < 0$ e $f(b) > 0$. Construamos uma sequência de intervalos $\left([a_n, b_n]\right)_{n \geq 0}$ recursivamente do seguinte modo: $a_0 := a$, $b_0 := b$ e
        \[
            \begin{cases}
                a_{n+1} := \dfrac{a_n + b_n}{2} \text{ e } b_{n+1} := b_n, & \text{ se } f \left( \dfrac{a_n + b_n}{2} \right) < 0 \\
                a_{n+1} := a_n \text{ e } b_{n+1} := \dfrac{a_n + b_n}{2}, & \text{ se } f \left( \dfrac{a_n + b_n}{2} \right) \geq 0.
            \end{cases}
        \]
    É fácil ver que, para todo $n \in \N$, temos $a_n \leq a_{n+1} \leq b_{n+1} \leq b_n$ e $\ds \lim_{n \to +\infty} (b_n - a_n) = 0$. Com isso,  $\left([a_n, b_n]\right)_{n \geq 0}$ é uma sequência de intervalos encaixantes, de modo que existe um único $c \in [a,b]$ tal que, para todo $n \in \N$, $a_n \leq c \leq b_n$. Em particular, temos que $f(a_n) < 0 \leq f(b_n)$ para todo $n \in \N$.

    Pela continuidade de $f$, $\ds \lim_{n \to + \infty} f(a_n) = f(c)$ e $\ds \lim_{n \to + \infty} f(b_n) = f(c)$, e como $f(a_n) < 0 \leq f(b_n)$ para todo $n \in \N$, temos, pelo Teorema do Confronto, que $f(c) =0$.
\end{proof}

\begin{teo}[Valor Intermediário]
    Seja $f : [a,b] \to \R$ uma função contínua em $[a,b]$.
        \begin{enumerate}[leftmargin=*, align=left, label=\textbf{(\alph*)}]
            \item Se $f(a) \leq f(b)$, então para todo $\gamma \in [f(a),f(b)]$ existe $c \in [a,b]$ tal que $f(c) = \gamma$.
            \item Se $f(b) \leq f(a)$, então para todo $\gamma \in [f(b),f(a)]$ existe $c \in [a,b]$ tal que $f(c) = \gamma$.
            \item Para todo $\gamma \in [\min{\{f(a),f(b)\}}, \max{\{f(a),f(b)\}}]$ existe $c \in [a,b]$ tal que $f(c) = \gamma$.
        \end{enumerate}

\end{teo}

\begin{proof}
    Pois tome $g(x) := f(x) - \alpha$, com $x \in [a,b]$. Como $f$ é contínua em $[a,b]$, $g$ também o é. Em particular, $g(a) = f(a) - \alpha < 0$ e $g(b) = f(b) - \alpha > 0$, de modo que, pelo Teorema do Anulamento, existe $c \in [a,b]$ tal que $g(c) = 0$, isto é, $f(c) = \alpha$.
\end{proof}

\begin{teo} \label{teo.calc:limitacao}
    (Limitação) Se uma função $f : [a,b] \to \R$ é contínua em $[a,b]$, então $f$ é limitada em $[a,b]$.
\end{teo}

\begin{proof}
    Suponhamos, por absurdo, que $f$ não seja limitada em $[a,b]$. Colocando $a_0 := a$ e $b_0 := b$, existe $x_0 \in [a_0, b_0]$ tal que $|f(x_0)| > 0$. Suponha, indutivamente, que $[a_n,b_n] \subset [a_0, b_0]$ esteja bem definido, sendo $f$ não limitada em $[a_n,b_n]$. Em particular, existe $x_n \in [a_n, b_n]$ tal que $|f(x_n)| > n$. Agora, defina $\ds a_{n+1} := a_n$ e $b_{n+1} := \dfrac{a_n + b_n}{2}$, se $f$ não for limitada em $\ds \left[a_n, \dfrac{a_n + b_n}{2} \right]$, ou $a_{n+1} := \dfrac{a_n + b_n}{2}$ e $b_{n+1} := b_n$ se $f$ não for limitada em $\ds \left[\dfrac{a_n + b_n}{2}, b_n \right]$. No intervalo em que $f$ não for limitada, existirá $x_{n+1}$ nesse intervalo tal que $|f(x_{n+1})|>n+1$. 
    
    Assim, fica construída uma sequência $\left([a_n, b_n]\right)_{n \geq 0}$ de intervalos encaixantes tal que, para todo $n \in \N$, existe $x_n \in [a_n,b_n]$ com $|f(x_n)| > n$. Em particular, isso significa que $\ds \lim_{n \to + \infty} |f(x_n)| = + \infty$. Agora, sendo $c$ o único real tal que $a_n \leq c \leq b_n$ para todo $n \in \N$, pelo Teorema do Confronto temos que $x_n \to c$, e sendo $f$ contínua em $c$, temos que $\ds \lim_{n \to + \infty} |f(x_n)| = |f(c)|$, absurdo! Logo, $f$ não ser limitada em $[a,b]$ nos leva a uma contradição, de modo que $f$ é, então, limitada em $[a,b]$.
\end{proof}

\begin{teo}[Valor Extremo, ou Weierstrass] \label{teo.calc:weierstrass}
    Se uma função $f: [a,b] \to \R$ é contínua em $[a,b]$, então existem $x_1, x_2 \in [a,b]$ tais que $f(x_1) \leq f(x) \leq f(x_2)$ para todo $x \in [a,b]$.
\end{teo}

\begin{proof}
    Pelo teorema da limitação \eqref{teo.calc:limitacao}, $f$ é limitada em $[a,b]$, de modo que o conjunto $A:= \{ f(x) : x \in [a,b]\}$ admite $M := \sup A$ e $m := \inf A$. Isto significa que $m \leq f(x) \leq M$ para todo $x \in [a,b]$. Afirmamos que existe $x_2 \in [a,b]$ para o qual $M = f(x_2)$. De fato, se um tal $x_2$ não existisse, seria $f(x) < M$ para todo $x \in [a,b]$, de modo que a função $\ds g(x) := \dfrac{1}{M - f(x)}$, com $x \in [a,b]$, seria contínua, mas não limitada, em $[a,b]$, o que é uma contradição (se $g$ fosse limitada, então existiria $\gamma > 0$ tal que $\ds 0< \dfrac{1}{M-f(x)}<\gamma$, donde $f(x) < M - \dfrac{1}{\gamma}$, de modo que $M$ não seria supremo de $A$). Assim, não pode ser $f(x) < M$, e como $f(x) \leq M$, existirá $x_2 \in [a,b]$ para o qual $f(x_2) =M$. Analogamente, prova-se que existe $x_1 \in [a,b]$ para o qual $f(x_1) = m$.
\end{proof}
\section{Algumas Funções Transcendentais}

\subsection{Trigonometria, parte I}

\begin{teo} \label{teo:sencos}
    Existe um único par de funções, $s,c : \R \to \R$, para as quais
        \begin{itemize}
            \item $s(0)=0$ e $c(0)=1$;
            \item $\forall x \forall y : s(x-y)=s(x)c(y) - s(y)c(x) \text{ e } c(x-y) = c(x)c(y) + s(x)s(y)$;
            \item $\exists r>0 : 0<x<r \Rightarrow 0 < s(x)<x<\dfrac{s(x)}{c(x)}.$
        \end{itemize}
    A função $s$ é chamada de \textit{seno} e será indicada por $\sin{x}$, enquanto $c$ é chamada de \textit{cosseno} e será indicada por $\cos{x}$.
\end{teo}

\begin{proof}
\end{proof}

\begin{prop}
    \textbf{(a)} (Identidade Fundamental) Temos $\sin^2{x} + \cos^2{x} = 1$ para todo $x \in \R$.

    \textbf{(b)} $\sin$ é uma função ímpar, isto é, $\sin{-x} = - \sin{x}$ para todo $x \in \R$, enquanto $\cos$ é uma função par, isto é, $\cos{-x} =  \cos{x}$ para todo $x \in \R$.

    \textbf{(c)} Temos, para todos $x,y \in \R$,
        \begin{align*}
            \sin{(x+y)} &= \sin{x}\cos{y} + \sin{y}\cos{x} \\ \cos{(x+y)} &= \cos{x}\cos{y} - \sin{x}\sin{y}
        \end{align*}
    \textbf{(d)} Temos $\sin{2x} = 2 \sin{x} \cos{x}$ e $\cos{2x} = \cos^2{x} - \sin^2{x}$ para todo $x \in \R$.

    \textbf{(e)} Temos $\sin^2{x} = \dfrac{1}{2} - \dfrac{1}{2} \cos{2x}$ e $\cos^2{x}  = \dfrac{1}{2} + \dfrac{1}{2} \cos {2x}$ para todo $x \in \R$.
\end{prop}

\begin{proof}
\end{proof}

\begin{teo}
    As funções $\sin$ e $\cos$ são contínuas em $\R$.
\end{teo}

\begin{proof}
    Pelo terceiro item no resultado \eqref{teo:sencos}, existe $r>0$ tal que $|x| < r \Rightarrow |\sin{x}| \leq |x|$. Usaremos isso para provar que $|x-p|<2r \Rightarrow |\sin{x} - \sin{p}| \leq |x-p|$. Pois tome:
        \begin{align*}
            |\sin{x} - \sin{p}| &= \left|2 \sin{\left(\dfrac{x-p}{2}\right)} \cos{\left(\dfrac{x+p}{2}\right)} \right| \\ &= 2 \left| \sin{\left(\dfrac{x-p}{2}\right)}  \right| \left| \cos{\left(\dfrac{x+p}{2}\right)}  \right|;
        \end{align*}
    como $\ds \left| \cos{\left( \dfrac{x+p}{2} \right)} \right| \leq 1$, temos que 
        \[
            |\sin{x} - \sin{p}| \leq 2 \left| \sin{\left(\dfrac{x-p}{2}\right)}  \right|,
        \]
    e então, pelo fato mencionado acima, vem 
        \[
            |x-p| < 2r \Rightarrow \left| \sin{\left(\dfrac{x-p}{2}\right)}  \right| \leq \left| \dfrac{x-p}{2} \right|,
        \]
    donde $|x-p|<2r \Rightarrow |\sin{x} - \sin{p}| \leq |x-p|$. De maneira completamente análoga, prova-se que $|x-p|<2r \Rightarrow |\cos{x} - \cos{p}| \leq \cos{x} - \cos{p}$.

    Com isso, $|x-p| < 2r \Rightarrow 0 \leq |\sin{x} - \sin{p}| \leq |x-p|$, e como $\ds \lim_{x \to p} |x-p|=0$, pelo Teorema do Confronto vem $\ds \lim_{x \to p} |\sin{x} - \sin{p} | = 0$, donde $\ds \lim_{x \to p} \sin{x} = \sin{p}$. De modo completamente análogo, prova-se que $\ds \lim_{x \to p} \cos{x} = \cos{p}$. Logo, $\sin$ e $\cos$ são contínuas em todo $p \in \R$.  
\end{proof}

\begin{teo}
     Temos $\ds \lim_{x \to 0} \dfrac{\sin{x}}{x}=1$ e $\ds \lim_{x \to 0} \dfrac{1 - \cos x}{x} = 0$.
\end{teo}

\begin{proof}
\end{proof}

\subsection{Exponencial e Logaritmo}

\subsubsection{Expoentes Racionais}

O intuito aqui é definir $a^x$ quando $x \in \Q$. A referência é \cite{guidorizzi1}.

\begin{teo} \label{teo:raizes}
    \leavevmode
        \begin{enumerate}[leftmargin=*, align=left, label=\textbf{(\alph*)}]
            \item Para quaisquer $a \in \R_{>0}$ e $n \in \N_{\geq 2}$ existe um único $x \in \R_{>0}$ tal que $x^n=a$.
            \item Para quaisquer $a \in \R$ e $n \in \N$ ímpar existe um único $x \in \R$ tal que $x^n=a$.
        \end{enumerate}
\end{teo}

\begin{proof}
    \textbf{(a)} Iremos construir duas sequências, $\left(a_k\right)_{k \geq 0}$ e $\left(b_k\right)_{k \geq 0}$, no sentido dos Teoremas \eqref{teo:intenc} e \eqref{teo:intenc2}.

    Seja $A_0$ o maior natural tal que $A^n_0 \leq a < (A_0 + 1)^n$. Em particular, isso nos diz que, se o real $x>0$ existe, então ele satisfaz $A_0 \leq x < A_0 + 1$. Agora, para cada $k \geq 1$, seja $A_k$ um elemento do conjunto $\{ 0, 1, \ldots, 9\}$ (isto é, $A_k$ é um \textit{dígito}, ou \textit{algarismo}), e defina $\left(a_k\right)_{k \geq 0}$ e $\left(b_k\right)_{k \geq 0}$ por
        \begin{gather*}
            a_k := \max \left\{ \sum_{i=0}^{k} \dfrac{A_k}{10^k} \mid \left( \sum_{i=0}^{k} \dfrac{A_k}{10^k} \right)^n \leq a \right\},  \forall k \geq 0 \\
            b_k := 
                \begin{cases}
                    A_0 + 1, & \text{se } k = 0 \\
                    a_{k-1} + \dfrac{A_k + 1}{10^k}, & \text{se } k \geq 1
                \end{cases}
        \end{gather*}
    É imediato que, para todo $k \in \N$, $a^n_k \leq a < b^n_k$. Em particular, isso nos diz que, se o real $x>0$ existe, então ele satisfaz $a_k \leq x < b_k$. Provemos agora que $a_k \leq a_{k+1} \leq b_{k+1} \leq b_k$:
    
    \begin{enumerate}[label=\roman*.]
        \item $a_k \leq a_{k+1}$: para ver isso, basta ver que
            \[
                a_{k+1} = \sum_{i=0}^{k+1} \dfrac{A_k}{10^k} = \sum_{i=0}^{k} \dfrac{A_k}{10^k} + \dfrac{A_{k+1}}{10^k} = a_k + \dfrac{A_{k+1}}{10^k}.
            \]
        Daí, $a_{k+1} - a_k = \dfrac{A_{k+1}}{10^k} \geq 0 \Rightarrow a_k \leq a_{k+1}$.
        \item $a_{k+1} \leq b_{k+1}$: releia uma das afirmações ditas acima.
        \item $b_{k+1} \leq b_k$: basta ver que
            \begin{align*}
                b_k - b_{k+1} &= \left( a_{k-1} + \dfrac{A_k + 1}{10^k} \right) - \left( a_k + \dfrac{A_{k+1} + 1}{10^{k+1}} \right) \\ 
                &= \left( a_{k-1} + \frac{A_k + 1}{10^k} \right) - \left( a_{k-1} + \frac{A_k}{10^k} + \frac{A_{k+1} + 1}{10^{k+1}} \right) \\ &= \frac{A_k + 1}{10^k} - \frac{A_k}{10^k} - \frac{A_{k+1} + 1}{10^{k+1}} \\ &= \frac{1}{10^k} - \frac{A_{k+1} + 1}{10^{k+1}} = \frac{9 - A_{k+1}}{10^{k+1}}.
            \end{align*}
        Como $A_{k+1} \in \{1, 2, \ldots, 9\}$, temos que $\dfrac{9 - A_{k+1}}{10^{k+1}} \geq 0$, donde $b_{k+1} \leq b_k$.
    \end{enumerate}

    Por fim, provemos que $\ds \lim_{k \to + \infty} (b_k - a_k) = 0$. De fato, veja que
        \begin{align*}
            b_k - a_k &= a_{k-1} + \dfrac{A_k + 1}{10^k} - \sum_{i=0}^{k} \dfrac{A_k}{10^k} \\ &= \sum_{i=0}^{k-1} \dfrac{A_k}{10^k} + \dfrac{A_k}{10^k} + \dfrac{1}{10^k} - \sum_{i=0}^{k} \dfrac{A_k}{10^k} \\
            &= \sum_{i=0}^{k} \dfrac{A_k}{10^k} + \dfrac{1}{10^k} - \sum_{i=0}^{k} \dfrac{A_k}{10^k} = \dfrac{1}{10^k},
        \end{align*}
    e então $\ds \lim_{k \to + \infty} (b_k - a_k) = \lim_{k \to + \infty} \dfrac{1}{10^k} = 0$.

    Com isso, a sequência $\left([a_k, b_k]\right)_{k \geq 0}$ é de intervalos encaixantes, e então existe um único $x \in \R$ tal que, para todo $k \in \N$, $a_k \leq x \leq b_k$. Pelo resultado \eqref{teo:intenc3}, a sequência $\left([a^n_k, b^n_k]\right)_{k \geq 0}$ também é de intervalos encaixantes; temos então que, para todo $k \in \N$, $a^n_k \leq x^n \leq b^n_k$. No entanto, vimos isso também vale para o real $a>0$: para todo $k \in \N$, $a^n_k \leq a \leq b^n_k$. Daí, $x^n = a$.

    \textbf{(b)} Se $a>0$, então, pelo item (a), existe $x>0$ tal que $x^n=a$. Por outro lado, se $a<0$, então $-a>0$, e pelo item (a) existe $x>0$ tal que $x^n=-a$; daí, $(-x)^n=a$.
\end{proof}

\begin{defi}
    Seja $n \geq 1$ um natural.
    
    \textbf{(a)} Para cada real $a$, o único real $x$ tal que $x^n=a$ será chamado de \textit{raíz $n$-ésima} de $a$ e será denotado por $\sqrt[n]{a}$. Assim, temos que $\ds (\sqrt[n]{a})^n=a$.

    \textbf{(b)} Como para todo $x \in \R_+$ existe um único $\sqrt[n]{x} \in \R_+$, a relação $\{ (x,y) \in \R_+ \times \R_+ : y = \sqrt[n]{x} \}$ é uma função, que será chamada de \textit{função raiz}.
\end{defi}

\begin{cor}
    Se $a,b >0$ são reais, $m,n \geq 1$ são naturais e $p$ é um inteiro, então

    \textbf{(a)} $\ds \sqrt[n]{a^p} = \sqrt[nm]{a^{pm}}$;

    \textbf{(b)} $\ds \sqrt[n]{\sqrt[m]{a}} = \sqrt[nm]{a}$;

    \textbf{(c)} $\ds \sqrt[n]{a} \cdot \sqrt[n]{b} = \sqrt[n]{ab}$;

    \textbf{(d)} $\ds a<b \Leftrightarrow \sqrt[n]{a} < \sqrt[n]{b}$.
        \begin{comment} \[ \ds
            \textbf{(a)} \ \sqrt[n]{a^p} = \sqrt[nm]{a^{pm}};
            \quad \textbf{(b)} \ \sqrt[n]{\sqrt[m]{a}} = \sqrt[nm]{a};
            \quad \textbf{(c)} \ \sqrt[n]{a} \cdot \sqrt[n]{b} = \sqrt[n]{ab};
            \quad \textbf{(d)} \ a<b \Leftrightarrow \sqrt[n]{a} < \sqrt[n]{b}.
        \] \end{comment}
\end{cor}

\begin{proof}
\end{proof}

\begin{prop}
    A função $f(x) := \sqrt[n]{x}$ é contínua em todo seu domínio.
\end{prop}

\begin{proof}
\end{proof}

\begin{defi}
    (Expoente Racional) Seja $a>0$ um real. Para cada racional $r$ (isto é, $r := m/n$, com $m \in \Z$ e $n \in \N \backslash \{0 \}$), definimos $\ds a^r = a^{\frac{m}{n}} := \sqrt[n]{a^m}$.\footnote{Como $\sqrt[n]{a^p} = \sqrt[nm]{a^{pm}}$, a definição acima não depende da escolha da fração $m/n$. Em particular, $f: \Q \to \R_+^*$ tal que $f(r) = a^r$ fica bem definida como função.}
\end{defi}

\begin{prop}
    Para quaisquer $a,b >0$ reais e $r,s$ racionais, temos que
    
    \textbf{(a)} $a^r \cdot a^s = a^{r+s}$;
    
    \textbf{(b)} $(a^r)^s = a^{rs}$;
    
    \textbf{(c)} $(ab)^r = a^r b^r$;
    
    \textbf{(d)} $\dfrac{a^r}{a^s} = a^{r-s}$;
    
    \textbf{(e)} $\left( \dfrac{a}{b} \right)^r =  \dfrac{a^r}{b^r}$;
    
    \textbf{(f)} Se $1<a$ e $r<s$, então $a^r < a^s$;

    \textbf{(g)} Se $0 < a < 1$ e $r<s$, então $a^s < a^r$.

    \begin{comment} \textbf{(a)} $\ds a^r \cdot a^s = a^{r+s}$, $\ds (a^r)^s = a^{rs}$, $\ds (ab)^r = a^r b^r$, $\ds \dfrac{a^r}{a^s} = a^{r-s}$, $\ds \left( \dfrac{a}{b} \right)^r =  \dfrac{a^r}{b^r}$.\end{comment}
\end{prop}

\begin{proof}
\end{proof}

\subsubsection{Expoentes Reais}

O intuito aqui é definir $a^x$ quando $x \in \R$.

\begin{teo}
    \textbf{(a)} Se $f:\R \to \R$ é contínua em $\R$ e $f(x) = 0$ para todo $x \in \Q$, então $f(x) = 0$ para todo $x \in \R$.

    \textbf{(b)} Se $f,g:\R \to \R$ são contínuas em $\R$ e $f(x) = g(x)$ para todo $x \in \Q$, então $f(x) = g(x)$ para todo $x \in \R$.\footnote{Isto nos diz que se duas funções contínuas em $\R$ coincidem em $\Q$, então elas são iguais.}

    \textbf{(c)} Se $f,g:\R \to \R$ são contínuas em $\R$ e existe $0 < a \neq 1$ tal que $f(x) = a^x$ e $g(x)=a^x$ para todo $x \in \Q$, então $f(x) = g(x)$ para todo $x \in \R$.\footnote{Isto significa que poderá existir no máximo uma função definida e contínua em $\R$ que coincide com $a^x$ para todo $x \in \Q$.}
\end{teo}

\begin{proof}
    \textbf{(a)} Segue como corolário da conservação do sinal \eqref{consin2}. \itemproof

    \textbf{(b)} Basta aplicar o resultado do item (a) na função $h(x) := f(x) - g(x)$. \itemproof

    \textbf{(c)} Segue como corolário do item (b) acima.
\end{proof}

\begin{teo} \label{teo:expirra}
    \textbf{(a)} Se $a>1$ é um real, então para todo $\epsilon >0$ existe um natural $n$ tal que $a^{\frac{1}{n}} -1 < \epsilon$. \begin{comment} Simbolicamente,
        \[
            \forall a (a \in \R \land a>1 \to \forall \epsilon (\epsilon \in \R \land \epsilon > 0 \to \exists n (n \in \N \land a^{\frac{1}{n}} -1 < \epsilon)) ).
        \] \end{comment}
    
    \textbf{(b)} Se $a>1$ e $x$ são reais. então para todo $\epsilon >0$ existem racionais $r$ e $s$ tais que $r<x<s$ e $a^s-a^r<\epsilon$. \begin{comment} Simbolicamente,
        \[
            \forall a \forall x (a,x \in \R \land a>1 \to \forall \epsilon (\epsilon \in \R \land \epsilon > 0 \to \exists r \exists s (r,s \in \Q \land r<x<s \land a^s - a^r < \epsilon) ) ).
        \] \end{comment}
    
    \textbf{(c)} Se $a>1$ é um real, então para todo $x$ real existe um único real $\gamma$ tal que $a^r < \gamma < a^s$ para todos os racionais $r$ e $s$ com $r<x<s$. \begin{comment} Simbolicamente,
        \[
            \forall a \forall x (a,x \in \R \land a>1 \to \exists ! \gamma (\gamma \in \R \land \forall r \forall s (r,s \in \Q \land r < x < s \to a^r < \gamma < a^s ) ) ).
        \] \end{comment}
    
    \textbf{(d)} Se $0 < a \neq 1$ é um real, então existe uma única função definida e contínua em $\R$ tal que $f(r) = a^r$ para todo $x \in \Q$.
\end{teo}

\begin{proof} 
    \textbf{(a)} Sabemos que $(1+\epsilon)^n \geq 1 + n\epsilon$ para todo natural $n \geq 1$. Pois tome $n$ tal que $1 + n\epsilon > a$ (basta que $ n > \frac{a-1}{\epsilon}$); daí, $(1+\epsilon)^n>a$, donde $a^{\frac{1}{n}} - 1 < \epsilon$. \itemproof

    \textbf{(b)} Para racionais $t>x$ temos $a^r<a^t$ para todo racional $r<x$. Pelo item anterior, existe um natural $n$ para o qual $a^{\frac{1}{n}} - 1 < \epsilon \cdot a^{-t}$, donde $a^t (a^{\frac{1}{n}}-1) < \epsilon$. Tomando racionais $r$ e $s$, com $r<x<s$, para os quais $s-r < 1/n$, temos $a^s - a^r = a^r (a^{s-r} -1) < a^t (a^{\frac{1}{n}} - 1) < \epsilon$. \itemproof

    \textbf{(c)} O conjunto $A := \{a^r : r \in \Q \land r<x \}$ é não vazio e limitado superiormente (por todo $a^s$, com $s>x$). Assim, existe $\gamma := \sup A$. Claramente, $a^r \leq \gamma \leq a^s$, mas, mais geralmente, $a^r < \lambda < a^s$ (prove!). Provemos, agora, a unicidade de $\gamma$. Se $\gamma'$ for tal que $a^r < \gamma' < a^s$ para quaisquer racionais $r$ e $s$, com $r<x<s$, então $|\gamma - \gamma'| < a^s - a^r$. Pelo item anterior, para todo $\epsilon > 0$ existem $r_0$ e $s_0$, com $r_0 < x < s_0$, para os quais $a^{s_0} - a^{r_0} < \epsilon$; logo, temos $|\gamma - \gamma'| < \epsilon$ para todo $\epsilon > 0$, donde $\gamma = \gamma'$. \itemproof

    \textbf{(d)} Pelo item anterior, para quaisquer $a>1$ e $x$ reais existe um único $\gamma$; assim, basta tomar $f(x) := \gamma$. Antes de provar a continuidade de $f$, provemos que $f$ é estritamente crescente. De fato, tomando reais $x_1 < x_2$ temos que $a^{r_1} < f(x_1) < a^{s_1}$ e $a^{r_2} < f(x_2) < a^{s_2}$ para todos os racionais $r_1$, $r_2$, $s_1$ e $s_2$ tais que $r_1 < x_1 < s_1$ e $r_2 < x_2 < s_2$. Como existe $s$ racional tal que $x_1 < s < x_2$, temos $f(x_1)<a^s<f(x_2$, donde $f$ é estritamente crescente.

    Agora, sendo $p \in \R$, pelo item (b) deste teorema, para todo $\epsilon > 0$ existem racionais $r$ e $s$, com $r<x<s$, para os quais $a^s-a^r < \epsilon$. Em particular, para todo $x \in \left]r,s\right[$, temos $a^r<f(x)<a^s$, e como também $a^r<f(p)<a^s$, temos $|f(x) - f(p)| < a^s - a^r < \epsilon$. Assim, pelo Teorema \eqref{teo:intervalos}, $f$ é contínua em $p$. Como $p$ foi tomado de modo arbitrário, segue que $f$ é contínua em $\R$. 

    Por outro lado, se $0<a<1$, então $f(x) := \left( \frac{1}{a} \right)^{-x}$ está bem definida em $\R$, é contínua em $\R$ e coincide com $a^r$ nos racionais.
\end{proof}

\begin{defi}
    Seja $0<a\neq1$ um real. Para todo $x \in \R$, definimos $a^x := f(x)$, em que $f$ é a função a que se refere o item (d) do Teorema \eqref{teo:expirra}.
\end{defi}

\begin{prop}
    Para quaisquer $0 < a,b \neq 1$ reais e $x,y$ reais, temos que
    
    \textbf{(a)} $a^x \cdot a^y = a^{x+y}$;
    
    \textbf{(b)} $(a^x)^y = a^{xy}$;
    
    \textbf{(c)} $(ab)^x = a^x b^x$;
    
    \textbf{(d)} $\dfrac{a^x}{a^y} = a^{x-y}$;
    
    \textbf{(e)} $\left( \dfrac{a}{b} \right)^x =  \dfrac{a^x}{b^x}$;
    
    \textbf{(f)} Se $1<a$ e $x<y$, então $a^x < a^y$;

    \textbf{(g)} Se $0 < a < 1$ e $x<y$, então $a^y < a^x$.
\end{prop}

\begin{proof}
\end{proof}

\subsubsection{Logaritmos}

\begin{teo}
    Para quaisquer reais $0<a\neq 1$ e $b>0$, existe um único real $\gamma := \log_{a}{b}$ tal que $a^{\gamma} = a^{\log_{a}{b}} = b$. Em particular, $f:\R_+ \to \R$ tal que $f(x) := \log_{a}{x}$ fica bem definida. 
\end{teo}

\begin{proof}    
\end{proof}

\begin{teo}
    Para quaisquer $0<a,b \neq 1$ e $x,y>0$, temos que
        \begin{align*}
            \log_{a}{xy} &= \log_{a}{x} + \log_{a}{y} \\
            \log_{a}{x^y} &= y \log_{a}{x} \\
            \log_{a}{\dfrac{x}{y}} &= \log_{a}{x} - \log_{a}{y} \\
            \log_{a}{x} &= \dfrac{\log_{b}{x}}{\log_{b}{a}}
        \end{align*}
    E ainda, se $a>1$ e $x<y$, então $\ds \log_{a}{x} < \log_{a}{y}$ (isto é, se $a>1$ então $f(x) := \log_{a}{x}$ é crescente), e se $0<a<1$ e $x<y$, então $\ds \log_{a}{y} < \log_{a}{x}$ (isto é, se $0<a<1$, então $f(x) := \log_{a}{x}$ é decrescente).
\end{teo}

\begin{proof}    
\end{proof}

\begin{teo}
    Se $a>1$, então $\ds \lim_{x \to + \infty} \log_{a}{x} = \infty$ e $\lim_{x \to 0^+} \log_{a}{x} = - \infty$; se $0<a<1$, então $\lim_{x \to + \infty} \log_{a}{x} = -\infty$ e $\lim_{x \to 0^+} \log_{a}{x} = +\infty$.
\end{teo}

\begin{proof}
\end{proof}

\begin{teo}
    A função logarítmica $f(x) := \log_{a}{x}$ é contínua em todo seu domínio.
\end{teo}

\begin{proof}
\end{proof}

\chapter{Derivadas}

\section{Definições e Resultados Iniciais}

\begin{defi} \label{defi.calc:derivada}
    Uma função $f : A \subseteq \R \to \R$ é \textit{derivável} em $a \in A \cap A'$, se existe o limite
        \[
            f'(a) := \lim_{x \to a} \dfrac{f(x) - f(a)}{x-a}.
        \]
    Noutros termos, $f$ é derivável em $a$ se existe o limite $\ds \lim_{x \to a} \dfrac{\Delta{f}}{\Delta{x}}(a)$, onde $\dfrac{\Delta{f}}{\Delta{x}}(a) : A \setminus \{ a\} \to \R$ é a \textit{função quociente de diferenças}, definida por
        \[
            \dfrac{\Delta{f}}{\Delta{x}}(a) := \dfrac{f(x) - f(a)}{x-a}.
        \]
    Sendo $f$ derivável em $a$, o limite $f'(a)$ é a \textit{derivada} de $f$ em $a$.
\end{defi}

\begin{obs}
    O objetivo das proposições seguintes é estabelecer precisamente a equivalência
        \[ \ds
            \lim_{x \to a} \dfrac{f(x) - f(a)}{x-a} = L \Leftrightarrow \lim_{h \to 0} \dfrac{f(a+h) - f(a)}{h} = L,
        \]
    por vezes assumida sem mais explicações.
\end{obs}

\begin{prop} \label{prop.ar1:xmenosa}
    Sejam $f : A \subseteq \R \to \R$ e $a \in A'$. Tem-se $\ds \lim_{x \to a} f(x) = L$ para algum $L \in \R$ se, e somente se, $\ds \lim_{h \to 0} g(h) = L$, onde $g: \{h \in \R_{\neq 0} : a+h \in A \} \to \R$ é definida por $g(h) := f(a+h)$.
\end{prop}

\begin{proof}
    A proposição (???) estabelece que $a \in A' \Leftrightarrow 0 \in \{h \in \R_{\neq 0} : a+h \in A \}'$. 
    
    Daí, $\ds \lim_{x \to a} f(x) = L$ se, e somente se, para todo $\epsilon \in \R_{>0}$ existe $\delta \in \R_{>0}$ tal que $0<|x-a|<\delta \Rightarrow |f(x) - L| < \epsilon$ para todo $x \in A$. Tomando $h := x-a$, temos $x = a+h \in A$, de modo que $h \in B$. Assim, para todo $\epsilon \in \R_{>0}$ existe $\delta \in \R_{>0}$ tal que $0<|h|<\delta \Rightarrow |f(a+h) - L| < \epsilon$ para todo $h \in B$, de modo que $\ds \lim_{h \to 0} g(h) = L$.
\end{proof}

\begin{cor}
    Uma função $f: A \subseteq \R \to \R$ é derivável em $a \in A \cap A'$, com derivada $L \in \R$, se, e somente se, $\ds \lim_{h \to 0} g(h) = L$, onde $g : \{ h \in \R_{\neq 0} : a+h \in A \setminus \{ a\} \} \to \R$ é definida por $g(h) := \dfrac{\Delta{f}}{\Delta{x}}(a+h)$.
\end{cor}

\begin{proof}
    Por definição, $f$ ser derivável em $a$ com derivada $L$ significa que $\ds \lim_{x \to a} g(x) = L$. Definindo $\varphi : \{ h \in \R_{\neq 0} : a+h \in A \setminus \{a \} \} \to \R$ por
        \[
            \varphi(h) = g(a+h) = \dfrac{f(a+h) - f(a)}{(a+h) - a} = \dfrac{f(a+h) - f(a)}{h},
        \]
    pela proposição \eqref{prop.ar1:xmenosa} temos $\ds \lim_{x \to a} g(x) = L \Leftrightarrow \lim_{h \to 0} \varphi(h) = L$, como havíamos afirmado.
\end{proof}

\begin{comment}
\begin{defi}
    Seja $f$ uma função e $p$ um ponto no domínio de $f$.

    \textbf{(a)} O limite
        $ \ds
            \lim_{x \to p} \dfrac{f(x) - f(p)}{x-p},
        $
    ou, equivalentemente,
        $ \ds
            \lim_{h \to 0} \dfrac{f(x+h) - f(x)}{h},            
        $
    quando existe, denomina-se \textit{derivada} de $f$ em $p$ e indica-se por $f'(p)$. Diremos que $f$ é \textit{derivável}, ou \textit{diferenciável}, em $p$.

    \textbf{(b)} Definimos a \textit{reta tangente} ao gráfico de $f$ no ponto $(p, f(p))$ como sendo a reta de equação $y - f(p) = f'(p)(x-p)$. Assim, a derivada de $f$ em $p$ é a \textit{inclinação} da reta tangente ao gráfico de $f$ no ponto $(p, f(p))$.
\end{defi}
\end{comment}

\begin{defi}
    Seja $f$ uma função e $A \subseteq D_f$ o conjunto dos $x \in D_f$ para os quais existe $f'(x)$. A função $f': A \to \R$ dada por $x \to f'(x)$ denomina-se \textit{função derivada} ou, simplesmente, \textit{derivada} de $f$. Diremos, ainda, que $f'$ é a \textit{derivada de 1ª ordem} de $f$, que também pode ser denotada por $f^{(1)}$. Por fim, definimos, indutivamente, $f^{(n+1)} := \left[f^{(n)}\right]'$.
\end{defi}

\begin{defi}
    Seja $f$ uma função, sendo $y := f(x)$. O símbolo $\ds \dfrac{dy}{dx}$, que se lê ``derivada de $y$ em relação a $x$'', denota a derivada de $f$ em $x$, isto é, $\ds \dfrac{dy}{dx} := f'(x)$. Já $\ds \dfrac{d^ny}{dx^n}$ denota a $n$-ésima derivada de $f$ em $x$: $\ds \dfrac{d^ny}{dx^n} := f^{(n)}(x)$. E ainda, $\ds \dfrac{df}{dx}$ denota a função derivada de $y=f(x)$: $\ds \dfrac{df}{dx} := f'$. Naturalmente, então, $\ds \dfrac{df}{dx} (x) := f'(x)$. A derivada de $y=f(x)$ no ponto $p$ é denotada por $\ds \dfrac{dy}{dx} \bigg|_{x = p}$.
    
    \begin{comment} \textbf{(a)} $\ds \dfrac{dy}{dx}$, que se lê ``derivada de $y$ em relação a $x$'', denota a derivada de $f$ em $x$: $\ds \dfrac{dy}{dx} := f'(x)$.
    
    \textbf{(b)} $\ds \dfrac{d^ny}{dx^n}$ denota a $n$-ésima derivada de $f$ em $x$: $\ds \dfrac{d^ny}{dx^n} = f^{(n)}(x)$. 

    \textbf{(c)} $\ds \dfrac{df}{dx}$ denota a função derivada de $y=f(x)$: $\ds \dfrac{df}{dx} := f'$. Naturalmente, então, $\ds \dfrac{df}{dx} (x)$ denota a derivada de $f$ em $x$: $\ds \dfrac{df}{dx} (x) := f'(x)$.
    
    \textbf{(c)} A derivada de $y=f(x)$ no ponto $p$ é denotada por $\ds \dfrac{dy}{dx} \bigg|_{x = p} := f'(p)$. \end{comment}
\end{defi}

\begin{teo}
    Se $f$ for derivável em $p$, então $f$ será contínua em $p$.
\end{teo}

\begin{proof}
\end{proof}

\begin{teo}
    Se $f$ e $g$ são funções deriváveis em $p$, então

    \textbf{(a)} a função $f+g$ é derivável em $p$ e
        \[
            (f+g)'(p) = f(p) + g(p);
        \]
    \textbf{(b)} a função $f \cdot g$ é derivável em $p$ e 
        \[
            (f \cdot g)'(p) = f'(p) \cdot g(p) + g'(p) \cdot f(p);
        \]
    \textbf{(c)} a função $\dfrac{f}{g}$ é derivável em $p$, desde que $g(p) \neq 0$, sendo
        \[
            \left( \dfrac{f}{g} \right)'(p) = \dfrac{f'(p) \cdot g(p) - g'(p) \cdot f(p)}{[g(p)]^2}.
        \]
\end{teo}

\begin{proof}
\end{proof}

\begin{cor}
    derivada de n funções, derivada de kf.
\end{cor}

\begin{lem}
    Seja $f : D_f \subseteq \R \to \R$ uma função derivável em $p \in D_f$. Definindo $\rho : D_f \backslash \{p\} \to \R$ de modo que
        \[
            f(x) = f(p) + f'(p) (x-p) + \rho (x) (x-p),
        \]
    temos que $\ds \lim_{x \to p} \rho (x) = 0$.
\end{lem}

\begin{proof}
\end{proof}

\begin{teo}
    (Regra da Cadeia) Se $f : D_f \subseteq \R \to \R$ e $g : D_g \subseteq \R \to \R$ são funções deriváveis, com $Im_g \subseteq D_f$, então a função composta $h:D_g \to \R$ dada por $h(x) = f(g(x))$ é derivável e
        \[
            h'(x) = f'(g(x)) \cdot g'(x).
        \]
    Sendo $y = f(u)$ e $u = g(x)$, a regra da cadeia nos diz que
        \[
            \dfrac{dy}{dx} = \dfrac{dy}{du} \cdot \dfrac{du}{dx},
        \]
    em que $\dfrac{dy}{du}$ deve ser calculada em $u = g(x)$.
\end{teo}

\begin{proof}
\end{proof}

\begin{teo}
    (Derivada de Função Inversa) Seja $f$ uma função inversível e $g$ a função inversa de $f$. Se $f$ for derivável em $q = g(p)$, com $f'(q) \neq 0$, e se $g$ for contínua em $p$, então $g$ será derivável em $p$ e
        \[
            g'(p) = \dfrac{1}{f'(g(p))}.
        \]
\end{teo}

\begin{proof}
\end{proof}

\section{Teoremas de Rolle, do Valor Médio e de Cauchy}

\begin{defi}
    Sejam $f:D_f \subseteq \R \to \R$ uma função e $p \in D_f$ um ponto no domínio de $f$.
    
    \textbf{(a)} Diremos que $f(p)$ é o \textit{valor máximo global} de $f$, ou que $p$ é um \textit{ponto de máximo global} de $f$, se $f(x) \leq f(p)$ para todo $x \in D_f$. Diremos que $f(p)$ é o \textit{valor mínimo global} de $f$, ou que $p$ é um \textit{ponto de mínimo global} de $f$, se $f(x) \geq f(p)$ para todo $x \in D_f$.

    \textbf{(b)} Suponha ainda que $p \in A \subseteq D_f$. Diremos que $f(p)$ é o \textit{valor máximo} de $f$ em $A$, ou que $p$ é um \textit{ponto de máximo} de $f$ em $A$, se $f(x) \leq f(p)$ para todo $x \in A$. Diremos que $f(p)$ é o \textit{valor mínimo} de $f$ em $A$, ou que $p$ é um \textit{ponto de mínimo} de $f$ em $A$, se $f(x) \geq f(p)$ para todo $x \in A$.

    \textbf{(c)} Diremos que $f(p)$ é o \textit{valor máximo local} de $f$, ou que $p$ é um \textit{ponto de máximo local} de $f$, se existir $r>0$ tal que $f(x) \leq f(p)$ para todo $x \in \left]p-r, p+r\right[ \cap D_f$. Diremos que $f(p)$ é o \textit{valor mínimo local} de $f$, ou que $p$ é um \textit{ponto de mínimo local} de $f$, se existir $r>0$ tal que $f(x) \geq f(p)$ para todo $x \in \left]p-r, p+r\right[ \cap D_f$.
\end{defi}

\begin{defi}
    Dada uma função $f$, diremos que o ponto $p$ é \textit{interior} a $D_f$ se existir um intervalo aberto $I \subset D_f$ tal que $p \in I$.
\end{defi}

\begin{teo} \label{teo:maxlocal}
    Seja $f:D_f \subseteq \R \to \R$ uma função derivável no ponto interior $p \in D_f$. Se $p$ é ponto de máximo (mínimo) local de $f$, então $f'(p)=0$.
\end{teo}

\begin{proof}
    Como $p$ é ponto de máximo local de $f$, existe $r_1 >0$ tal que $f(x) \leq f(p)$ para todo $x \in \left]p-r_1, p+r_1\right[ \cap D_f$. Como $p$ é um interior a $D_f$, existe $r_2 > 0$ tal que $\left] p-r_2, p+r_2 \right[ \subseteq D_f$. Sendo $r := \min{\{ r_1, r_2\}}$, temos $f(x) \leq f(p)$ para todo $x \in \left]p-r, p+r\right[$. Como $f$ é derivável em $p$, temos que
        $ \ds
            f'(p) = \lim_{x \to p^+} \dfrac{f(x) - f(p)}{x-p} = \lim_{x \to p^-} \dfrac{f(x) - f(p)}{x-p}.
        $
    Sendo $p < x < p+r$, temos $\ds \dfrac{f(x) - f(p)}{x-p} \leq 0$; daí, pela conservação do sinal, $\ds \lim_{x \to p^+} \dfrac{f(x) - f(p)}{x-p} \leq 0$.  Analogamente, sendo $p-r<x<p$, temos $\ds \dfrac{f(x) - f(p)}{x-p} \geq 0$; daí, pela conservação do sinal, $\ds \lim_{x \to p^+} \dfrac{f(x) - f(p)}{x-p} \geq 0$. Com isso, temos $f'(p) \leq 0$ e $f'(p) \geq 0$, donde, $f'(p) = 0$. O caso em que $p$ é ponto de mínimo local de $f$ segue de forma completamente análoga.
\end{proof}

$\subseteq D_f$ e derivável em $]a,b[$.

\begin{teo} \label{teo:rolletvm}
    Seja $f: [a,b] \to \R$ contínua em $[a,b]$ e derivável em $]a,b[$.
        \begin{enumerate}[leftmargin=*, align=left, label=\textbf{(\alph*)}]
            \item (Rolle) Se $f(a) = f(b)$, então existe $c \in \left]a,b\right[$ tal que $f'(c) = 0$.
            \item (Valor Médio) Existe $c \in \left]a,b\right[$ tal que
                \[
                    f(b) - f(a) = f'(c) \cdot (b-a).
                \]
        \end{enumerate}
\end{teo}

\begin{proof}
    \textbf{(a)} Se $f$ for constante em $[a,b]$, então $f'(x) = 0$ para todo $x \in \left]a,b \right[$. Suponha, então, que $f$ não seja constante em $[a,b]$. Sendo $f$ contínua em $[a,b]$, pelo Teorema de Weierstrass \eqref{teo.calc:weierstrass}, existem $x_1, x_2 \in [a,b]$ tais que $f(x_1) \leq f(x) \leq f(x_2)$ para todo $x \in [a,b]$. Se fosse $f(x_1) = f(x_2)$, então $f$ seria constante; logo, $f(x_1) \neq f(x_2)$. E como, por hipótese, $f(a) = f(b)$, temos que $x_1$ ou $x_2$ estão em $]a,b[$. O $x_i \in \left] a,b \right[$ é um ponto de máximo local de $f$; daí, pelo Teorema \eqref{teo:maxlocal}, $f'(x_i)=0$. \itemproof

    \textbf{(b)} Defina $S: [a,b] \to \R$ por
        \[
            S(x) := f(a) + \dfrac{f(b) - f(a)}{b-a} (x-a).
        \]
    Observe que o gráfico de $S$ é a reta que passa pelos pontos $(a, f(a))$ e $(b, f(b))$. Agora, defina $g : [a,b] \to \R$ por $g(x) := f(x) - S(x)$. Note que $g$ é contínua em $[a,b]$ e derivável em $]a,b[$; daí, pelo Teorema de Rolle, existe $c \in \left] a,b \right[$ tal que $g'(c) = 0$. Como
        \[
            g'(x) = f'(x) - \dfrac{f(b) - f(a)}{b-a},
        \]
    temos que
        \[ \ds
            g'(c) = f'(c) - \dfrac{f(b) - f(a)}{b-a} = 0,
        \]
    donde $f(b) - f(a) = f'(c) (b-a)$.
\end{proof}

\begin{teo}
    (Cauchy) Se $f : D_f \subseteq \R \to \R$ e $g : D_g \subseteq \R \to \R$ são funções contínuas em $\left[ a,b \right] \subseteq D_f \cap D_g$ e deriváveis em $]a,b[$, então existe pelo menos um ponto $c \in \left] a,b \right[$ tal que
        \[
            [f(b) - f(a)] g'(c) = [g(b) - g(a)] f'(c).
        \]
    Em particular, se $g'(x) \neq 0$ para todo $x \in \left] a,b \right[$, então
        \[
            \dfrac{f(b) - f(a)}{g(b) - g(a)} = \dfrac{f'(c)}{g'(c)}.
        \]
\end{teo}

\begin{proof}
    Defina $h:[a,b] \to \R$ por
        \[
            h(x) := [f(b) - f(a)] g(x) -[g(b) - g(a)]f(x).
        \]
    É fácil ver que $h$ é contínua em $[a,b]$, derivável em $]a,b[$ e $h(a) = h(b)$. Daí, pelo Teorema de Rolle $\eqref{teo:rolletvm}$, existe $c \in \left] a,b \right[$ tal que
        \[
            [f(b) - f(a)] g(c) -[g(b) - g(a)]f(c) = 0,
        \]
    donde
        \[
            [f(b) - f(a)] g'(c) = [g(b) - g(a)] f'(c).
        \]
    Em particular, se $g'(x) \neq 0$ para todo $x \in \left] a,b \right[$, então
        \[
            \dfrac{f(b) - f(a)}{g(b) - g(a)} = \dfrac{f'(c)}{g'(c)},
        \]
    pois o Teorema do Valor Médio \eqref{teo:rolletvm} aplicado à função $g$ nos diz que existe $\tilde{c} \in \left] a,b \right[$ tal que $ g(b) - g(a) = g'(\tilde{c})(b-a)$; como $g'(\tilde{c}) \neq 0$ e $b \neq a$, temos $g(b) - g(a) \neq 0$.
    \end{proof}

\section{Gráficos de Funções}

\begin{teo}
    Seja $f: D_f \subseteq \R \to \R$ uma função derivável no intervalo aberto $I \subset D_f$.

    \textbf{(a)} Se $f'(x)>0$ para todo $x \in I$ interior, então $f$ será estritamente crescente em $I$.

    \textbf{(b)} Se $f'(x)<0$ para todo $x \in I$ interior, então $f$ será estritamente decrescente em $I$.
\end{teo}

\begin{proof}
    \textbf{(a)} Provemos que para quaisquer $a,b \in I$ temos $a<b \Rightarrow f(a) < f(b)$. Sejam, então, $a,b \in I$ com $a<b$. Evidentemente, temos que $f$ e contínua em $[a,b]$ e derivável em $]a,b[$; daí, pelo Teorema do Valor Médio \eqref{teo:rolletvm}, existe $c \in \left] a,b \right[$ tal que 
        $
            f(b) - f(a) = f'(c) (b-a).
        $
    Como $f'(c) > 0$ e $b>a$, temos que $f(b) - f(a) > 0$, donde $f(a) < f(b)$. \itemproof
    
    \textbf{(b)} Segue analogamente.
\end{proof}

\begin{cor}
    Seja $f : D_f \subseteq \R \to \R$ uma função derivável até a 2ª ordem em $\left] a,b \right[ \subseteq D_f$. Se $f''(x) > 0$ para todo $x \in \left] a,b \right[$ e se existe $c \in \left] a,b \right[$ tal que $f'(c) = 0$, então $f$ é estritamente decrescente em $]a,c[$ e estritamente crescente em $]c,b[$.
\end{cor}

\begin{proof}
    Se $f''(x) > 0$ para todo $x \in \left] a,b \right[$, então $f'$ é estritamente crescente em $]a,b[$. Com isso, $f'(x) < f'(c) = 0$ para todo $x \in \left] a,c \right[$ e $f'(x) > f'(c) = 0$ para todo $x \in \left] c,b \right[$. Com isso, $f$ é estritamente decrescente em $]a,c[$ e estritamente crescente em $]c,b[$.
\end{proof}

\begin{defi}
    Seja $f : D_f \subseteq \R \to \R$ uma função derivável no intervalo aberto $I \subseteq D_f$.

    \textbf{(a)} Se 
        $
            f(x) > f(p) + f'(p) (x-p)
        $
    para todos $x,p \in I$, com $x \neq p$, diremos que $f$ tem a \textit{concavidade para cima} em $I$, ou que $f$ é \textit{convexa} em $I$.

    \textbf{(b)} Se
        $
            f(x) < f(p) + f'(p) (x-p)
        $
    para todos $x,p \in I$, com $x \neq p$, diremos que $f$ tem a \textit{concavidade para baixo} em $I$, ou que $f$ é \textit{côncava} em $I$.
\end{defi}

\begin{teo}
    Seja $f : D_f \subseteq \R \to \R$ uma função derivável até a 2ª ordem no intervalo aberto $I \subseteq D_f$.

    \textbf{(a)} Se $f''(x) > 0$ em $I$, então $f$ terá a concavidade para cima em $I$.

    \textbf{(b)} Se $f''(x)<0$ em $I$, então $f$ terá a concavidade para baixo em $I$.
\end{teo}

\begin{proof}
    \textbf{(a)} Sendo $p \in I$, provemos que para todo $x \in I$, com $x \neq p$, temos
        $
            f(x) > f(p) + f'(p)(x-p).
        $
    Definindo $g: I \to \R$ por $g(x) := f(x) - f(p) - f'(p)(x-p)$, basta provar que $g(x)>0$ para todo $x \in I$, com $x \neq p$. É fácil ver que $g'(x) = f'(x) - f'(p)$. Como $f''(x)>0$ em $I$, temos que $f'$ é estritamente crescente em $I$. Com isso, $g'(x) > 0$ para $x>p$ e $g'(x) < 0$ para $x < p$. Com isso, $g$ é estritamente decrescente em $\{ x \in I : x<p \}$ e estritamente crescente em $\{x \in I : x>p \}$. Com isso, sendo $g(p) = 0$, temos $g(x) > 0$ para todo $x \in I$, com $x \neq p$. \itemproof
    
    \textbf{(b)} Segue analogamente.
\end{proof}

\begin{defi}
    Seja $f : D_f \subseteq \R \to \R$ uma função contínua em $p \in D_f$. Diremos que $p$ é um \textit{ponto de inflexão} de $f$ se existirem $a,b \in \R$, com $p \in \left] a,b \right[ \subseteq D_f$, para os quais a concavidade de $f$ em $]a,p[$ é diferente da concavidade de $f$ em $]p,b[$.
\end{defi}


\begin{prop}
    \textbf{(a)} Seja $f: D_f \subseteq \R \to \R$ uma função derivável até a 3ª ordem no intervalo aberto $\left] a,b \right[ \subseteq D_f$. Se $f'''$ é contínua em $p \in \left] a,b \right[$, $f'''(p) \neq 0$ e $f''(p) = 0$, então $p$ é um ponto de inflexão de $f$.

    \textbf{(b)} Seja $f: D_f \subseteq \R \to \R$ uma função derivável até a 2ª ordem no intervalo aberto $I \subseteq D_f$. Se $f''$ é contínua em $p \in I$ e $p$ é um ponto de inflexão de $f$, então $f''(p)=0$.
\end{prop}

\begin{proof}
    \textbf{(a)} Suponha, sem perda de generalidade, que $f'''(p) > 0$. Como $f'''$ é contínua em $p$, pela conservação do sinal \eqref{consin2} existe $r_1 > 0$ tal que $f'''(x) > 0$ para todo $x \in \left] p-r_1, p+r_1 \right[$. Por outro lado, existe $r_2 > 0$ tal que $\left] p-r_2, p+r_2 \right[ \subseteq \left] a,b \right[$ (de fato, basta tomar $r_2 = \min{\{ b-p, p-a \}}$). Sendo, então, $r := \min{\{r_1,r_2\}}$, temos que $f'''(x) > 0$ para todo $x \in \left] p-r,p+r \right[ \subseteq \left] a,b \right[$. Com isso, $f''$ é estritamente crescente em $]p-r,p+r[$, e como $f''(p) = 0$, só pode ser $f''(x) < 0$ para todo $x \in \left] p-r,p+r \right[$ e $f''(x)>0$ para todo $x \in ]p,p+r[$. Logo, $p$ é um ponto de inflexão de $f$.

    \textbf{(b)} Se fosse $f''(p) \neq 0$, como $f''$ é contínua em $p$, pela conservação do sinal existiria $r>0$ tal que $f''(p)$ e $f''(x)$ teriam o mesmo sinal em $]p-r,p+r[$, donde $p$ não seria ponto de inflexão de $f$, absurdo. Logo, $f''(p) = 0$.
\end{proof}

\section{Regras de L'Hospital}

\begin{teo}
    (Regra de L'Hospital para indeterminações do tipo $0 / 0$)
    
    \textbf{(a)} Sejam $f:D_f \subseteq \R \to \R$ e $g: D_g \subseteq \R \to \R$ funções para as quais existe $r \in \R_{>0}$ tal que $f$ e $g$ são deriváveis e $g'(x) \neq 0$ em 
        \begin{itemize}
            \item $I := \left] p,p+r \right[ \subseteq D_f \cap D_g$ (caso $x \to p^+$); ou
            \item $I := \left]p-r, p \right[ \subseteq D_f \cap D_g$ (caso $x \to p^-$); ou
            \item $I := \left]p-r, p+r \right[ \setminus \{p\} \subseteq D_f \cap D_g$ (caso $x \to p$).
        \end{itemize}
    Se $\ds \lim{f(x)} = \lim{g(x)} = 0$ e $\ds \lim{\dfrac{f'(x)}{g'(x)}} \in \R \cup \{ \pm \infty \}$, então
        \[ \ds
            \lim{\dfrac{f(x)}{g(x)}} = \lim{\dfrac{f'(x)}{g'(x)}}. 
        \]
    \textbf{(b)} Sejam $f:D_f \subseteq \R \to \R$ e $g: D_g \subseteq \R \to \R$ funções para as quais existe $r \in \R$ tal que $f$ e $g$ são deriváveis e $g'(x) \neq 0$ em 
        \begin{itemize}
            \item $I := \left] r, + \infty \right[ \subseteq D_f \cap D_g$ (caso $x \to + \infty$); ou
            \item $I := \left]- \infty, r \right[ \subseteq D_f \cap D_g$ (caso $x \to - \infty$).
        \end{itemize}
    Se $\ds \lim{f(x)} = \lim{g(x)} = 0$ e $\ds \lim{\dfrac{f'(x)}{g'(x)}} \in \R \cup \{ \pm \infty \}$, então
        \[ \ds
            \lim{\dfrac{f(x)}{g(x)}} = \lim{\dfrac{f'(x)}{g'(x)}}. 
        \]
\end{teo}

\begin{proof}
    \textbf{(a)} Suponha $\ds \lim{\dfrac{f'(x)}{g'(x)}} = L \in \R$. Façamos o caso $x \to p^+$, isto é, 
        \[  \ds
            \lim_{x \to p^+}{f(x)} = \lim_{x \to p^+}{g(x)} = 0 \quad \text{e} \quad \lim_{x \to p^+}{\dfrac{f'(x)}{g'(x)}} = L.
        \]
    Defina $F,G : I \to \R$ por
        \[
            F(x) :=
                \begin{cases}
                    f(x) & x \in I \\
                    0 & x = p
                \end{cases}
            \qquad
            \text{e}
            \qquad 
            G(x) :=
                \begin{cases}
                    g(x) & x \in I \\
                    0 & x = p
                \end{cases}
        \]
    Afirmamos que $G'(x) \neq 0$ para todo $x \in I$ e $G(p) = 0$ resultam em $G(x) \neq 0$ para todo $x \in I$. De fato, se não fosse $G(x) \neq 0$ para todo $x \in I$, então existiria $a \in I$ com $G(a) = 0$; pelo Teorema do Valor Médio, existiria $b \in \left]p,a \right[$ tal que
        \[  
            \underbrace{G(a) - G(p)}_{=0} = G'(b) \underbrace{(a - p)}_{\neq 0},
        \]
    donde $G'(b) = 0$, contrariando a hipótese de ser $G'(x) \neq 0$ para todo $x \in I$. Agora, sendo $\ds \lim_{x \to p^+}{\dfrac{f'(x)}{g'(x)}} = L$, para todo $\epsilon \in \R_{>0}$ existe $\delta \in \R_{>0}$, com $\delta < r$, tal que
        \[ \ds
            p < x < p + \delta \Rightarrow \left| \dfrac{f'(x)}{g'(x)} - L \right| < \epsilon,
        \]
    isto é, $\ds \left| \dfrac{F'(x)}{G'(x)} - L \right| < \epsilon$. Por outro lado, o Teorema de Cauchy aplicado às funções $F$ e $G$ no intervalo $[p,x]$ nos diz que existe $q \in \left] p,x \right[$ tal que
        \[
            \dfrac{F(x) - F(p)}{G(x) - G(p)} = \dfrac{F'(q)}{G'(q)};
        \]
    daí,
        \[ \ds
            \left| \dfrac{F(x)}{G(x)} - L \right| =  \left| \dfrac{F(x) - F(p)}{G(x) - G(p)} - L \right| =   \left| \dfrac{F'(q)}{G'(q)} - L \right| < \epsilon,
        \]
    pois $p < q < x < p + \delta$. Com isso, $\ds \lim_{x \to p^+}{\dfrac{f(x)}{g(x)}} = L$, como queríamos provar.

    \textbf{(b)}
\end{proof}

\begin{teo}
    (Regra de L'Hospital para indeterminações do tipo $\infty /  \infty $)

    \textbf{(a)} Sejam $f:D_f \subseteq \R \to \R$ e $g: D_g \subseteq \R \to \R$ funções para as quais existe $r \in \R_{>0}$ tal que $f$ e $g$ são deriváveis e $g'(x) \neq 0$ em 
        \begin{itemize}
            \item $I := \left] p,p+r \right[ \subseteq D_f \cap D_g$ (caso $x \to p^+$); ou
            \item $I := \left]p-r, p \right[ \subseteq D_f \cap D_g$ (caso $x \to p^-$); ou
            \item $I := \left]p-r, p+r \right[ \setminus \{p\} \subseteq D_f \cap D_g$ (caso $x \to p$).
        \end{itemize}
    Se $\ds \lim{f(x)} = \lim{g(x)} = \pm \infty$ e $\ds \lim{\dfrac{f'(x)}{g'(x)}} \in \R \cup \{ \pm \infty \}$, então
        \[ \ds
            \lim{\dfrac{f(x)}{g(x)}} = \lim{\dfrac{f'(x)}{g'(x)}}. 
        \]
    \textbf{(b)} Sejam $f:D_f \subseteq \R \to \R$ e $g: D_g \subseteq \R \to \R$ funções para as quais existe $r \in \R$ tal que $f$ e $g$ são deriváveis e $g'(x) \neq 0$ em 
        \begin{itemize}
            \item $I := \left] r, + \infty \right[ \subseteq D_f \cap D_g$ (caso $x \to + \infty$); ou
            \item $I := \left]- \infty, r \right[ \subseteq D_f \cap D_g$ (caso $x \to - \infty$).
        \end{itemize}
    Se $\ds \lim{f(x)} = \lim{g(x)} = \pm \infty$ e $\ds \lim{\dfrac{f'(x)}{g'(x)}} \in \R \cup \{ \pm \infty \}$, então
        \[ \ds
            \lim{\dfrac{f(x)}{g(x)}} = \lim{\dfrac{f'(x)}{g'(x)}}. 
        \]
\end{teo}

\begin{proof}
\end{proof}

\section{Trigonometria, parte II}

\begin{teo} \label{teo:pisobre2}
    Existe um menor real $a>0$ tal que $\cos{a} = 0$ e $\sin{a} = 1$.
\end{teo}

\begin{proof}
\end{proof}

\begin{defi}
    Definimos $\pi := 2a$, em que $a$ é o menor real a que se refere o Teorema \eqref{teo:pisobre2}. Assim, $\ds \cos{\dfrac{\pi}{2}} = 0$ e $\ds \sin{\dfrac{\pi}{2}}=1$.
\end{defi}

\begin{teo}
    As funções $\sin$ e $\cos$ são periódicas com período $2 \pi$, isto é, $\sin{(x + 2\pi)} = \sin{x}$ e $\cos{(x+2\pi)} = \cos{x}$ para todo $x \in \R$.
\end{teo}

\section{Polinômio de Taylor}

\begin{defi} \label{defi:taylor}
    Seja $I \subset \R$ um intervalo aberto, $f: I \to \R$ uma função $n$ vezes diferenciável e $x_0 \in I$ um ponto de $I$. O polinômio
        \[
            P_n(x) := \sum_{k=0}^{n}{\dfrac{f^{(k)}(x_0)}{k!}(x-x_0)^{k}} 
        \]
    chama-se \textit{polinômio de Taylor} de \textit{ordem} $n$ de $f$ \textit{centrado}, ou \textit{em volta}, de $x_0$.
\end{defi}

\begin{prop}
    Nas condições da definição \eqref{defi:taylor}, $P^{(k)}_n (x_0) = f^{(k)}(x_0)$ para todo $k \leq n$.
\end{prop}

\begin{proof}
    
\end{proof}

\chapter{Integrais}

\section{A integral de Darboux}

\begin{defi}
    \leavevmode
        \begin{enumerate}[leftmargin=*, align=left, label=\textbf{(\alph*)}]
            \item Uma \textit{partição} de um intervalo $[a,b]$ é um conjunto finito $P := \{x_0, \ldots, x_n \}$ tal que $a = x_0 < x_1 < \cdots < x_n = b$. Isso é denotado por
                \[
                    P : a = x_0 < \cdots < x_n = b.
                \]
            O conjunto de todas as partições de $[a,b]$ é denotado por $\mathcal{P}{[a,b]}$.
            \item A \textit{amplitude} do $i$-ésimo intervalo $[x_{i-1}, x_i]$, para cada $i \in [n]$, é definida como $\Delta x_i := x_i - x_{i-1}$.
            \item Um \textit{refinamento} de uma partição $P \in \mathcal{P}{[a,b]}$ é uma partição $Q \in \mathcal{P}{[a,b]}$ tal que $P \subseteq Q$.
        \end{enumerate}
\end{defi}

\begin{defi} \label{defi:somassupinf}
    Sejam $f:[a,b] \to \R$ uma função limitada e $P : a = x_0 < \cdots < x_n = b$ uma partição de $[a,b]$.
        \begin{enumerate}[leftmargin=*, align=left, label=\textbf{(\alph*)}]
            \item A \textit{soma superior} de $f$ com relação à $P$ é definida como
                \[ \ds
                    U(f,P) := \sum_{i=1}^{n}{M_i \Delta{x_i}},
                \]
            onde
                $ \ds
                    M_i := \sup_{[x_{i-1}, x_i]}{f}
                $
            para cada $i \in [n]$.
            \item A \textit{soma inferior} de $f$ com relação à $P$ é definida como
                \[ \ds
                    L(f,P) := \sum_{i=1}^{n}{m_i \Delta{x_i}},
                \]
            onde
                $ \ds
                    m_i := \inf_{[x_{i-1}, x_i]}{f}
                $
            para cada $i \in [n]$.
        \end{enumerate}
\end{defi}

\begin{prop} \label{prop:somassupinf}
    Sejam $f:[a,b] \to \R$ uma função limitada e $P : a = x_0 < \cdots < x_n = b$ uma partição de $[a,b]$.
        \begin{enumerate}[leftmargin=*, align=left, label=\textbf{(\alph*)}]
            \item Tem-se $L(f,P) \leq U(f,P)$.
            \item Se $Q$ é um refinamento de $P$, então
                $
                    U(f,Q) \leq U(f,P)
                $
            e
                $
                    L(f,Q) \geq L(f,P)
                $.
            \item Se $Q$ é uma qualquer outra partição de $[a,b]$, então $L(f,P) \leq U(f,Q)$.
        \end{enumerate}
\end{prop}

\begin{proof}
    \leavevmode
        \begin{enumerate}[leftmargin=*, align=left, label=\textbf{(\alph*)}]
            \item Para todo $i \in [n]$ temos $m_i \leq M_i$ e $\Delta x_i > 0$. Logo $m_i \Delta x_i \leq M_i \Delta x_i$ para todo $i \in [n]$. Tomando a soma, temos que
                \[
                    \sum_{i=1}^{n} m_i \Delta x_i \leq \sum_{i=1}^{n} M_i \Delta x_i,
                \]
            de modo que $L(f,P) \leq U(f,P)$, como havíamos afirmado. \itemproof
            
            \item Façamos indução em $|Q \setminus P|$. Se $|Q \setminus P| = 1$, então $Q$ só tem um ponto a mais que $P$, isto é, existe $\bar{x} \in Q$ tal que $\bar{x} \notin P$. Em particular, existe um único índice $j \in [n]$ tal que $x_{j-1} < \bar{x} < x_j$. Com isso, tomando
                \[
                    M'_j := \sup_{[x_{j-1}, \bar{x}]}{f} \quad \text{e} \quad M''_j := \sup_{[\bar{x}, x_j]}{f},
                \]
            temos $M'_j, M''_j \leq M_j$, donde
                \begin{align*}
                    U(f,Q) &= \sum_{\substack{i = 1 \\ i \neq j  }}^{n} M_i \Delta x_i + M'_j(\bar{x} - x_{j-1}) + M''_j(x_j - \bar{x}) \\
                    &\leq \sum_{\substack{i = 1 \\ i \neq j  }}^{n} M_i \Delta x_i + M_j(\bar{x} - x_{j-1}) + M_j(x_j - \bar{x}) \\
                    &= \sum_{\substack{i = 1 \\ i \neq j  }}^{n} M_i \Delta x_i + M_j(x_j - x_{j-1}) \\
                    &= \sum_{i = 1}^{n} M_i \Delta x_i = U(f,P).
                \end{align*}
            Isso completa a base da indução. Agora, suponha que se $|Q \setminus P| = k > 1$, então $U(f,Q) \leq U(f,P)$. Tomando $Q'$ com só um ponto a mais que $Q$, temos que $U(f,Q') \leq U(f,Q)$, de modo que se $|Q' \setminus P| = k+1$ então $U(f,Q') \leq U(f,P)$. Isso completa o passo indutivo e, portanto, completa a prova. A prova de que $L(f,P') \geq L(f,P)$ segue de modo completamente análogo. \itemproof
    
            \item Para quaisquer partições $P$ e $Q$ de $[a,b]$, sempre existe um refinamento comum a ambas. De fato, basta tomar a união $P \cup Q$ e reindexar os índices conforme a definição. Assim, pelos itens anteriores,
                \[
                    L(f,P) \leq L(f, P \cup Q) \leq U( f, P \cup Q) \leq U(f,Q),
                \]
            como havíamos afirmado. \itemproof
        \end{enumerate}
\end{proof}

\begin{defi}
    Seja $f:[a,b] \to \R$ uma função limitada.
        \begin{enumerate}[leftmargin=*, align=left, label=\textbf{(\alph*)}]
            \item A \textit{integral inferior} de $f$ é definida como
                \[
                    \lowerint_{a}^{b} f(x) \, dx := \sup_{P \in \mathcal{P}{[a,b]}} L(f,P).
                \]
            \item A \textit{integral superior} de $f$ é definida como
                \[
                    \upperint_{a}^{b} f(x) \, dx := \inf_{P \in \mathcal{P}{[a,b]}} U(f,P).
                \]
        \end{enumerate}
\end{defi}

\begin{obs}
    A proposição \eqref{prop:somassupinf} garante que $\left\{ L(f,P) \in \R : P \in \mathcal{P}{[a,b]} \right\}$ é limitado superiormente; logo, pela propriedade do supremo, existe $\ds \sup_{P \in \mathcal{P}{[a,b]}} L(f,P)$. Analogamente, existe $\ds \inf_{P \in \mathcal{P}{[a,b]}} U(f,P)$. Isso garante que as definições de integral superior e inferior são consistentes (estão bem definidas).
\end{obs}

\begin{prop}
    Se uma função $f: [a,b] \to \R$ é limitada, então
        \[
             \lowerint_{a}^{b} f(x) \, dx \leq \upperint_{a}^{b} f(x) \, dx.
        \]
\end{prop}

\begin{proof}
    Vimos que $L(f, P) \leq U(f, Q)$ para quaisquer partições $P, Q \in \mathcal{P}{[a,b]}$. Fixando $Q$, temos que $U(f,Q)$ é uma cota superior de $L(f,P)$, para qualquer $P \in \mathcal{P}{[a,b]}$, de modo que
        \[ \ds
            \lowerint_{a}^{b} f(x) \, dx \leq U(f,Q).
        \]
    Com isso, $\ds \lowerint_{a}^{b} f(x) \, dx$ é uma cota inferior de $U(f,Q)$, para qualquer $Q \in \mathcal{P}{[a,b]}$, de modo que
        \[ \ds
            \lowerint_{a}^{b} f(x) \, dx \leq \upperint_{a}^{b} f(x) \, dx,
        \]
    como havíamos afirmado. \itemproof
\end{proof}

\begin{defi}
    Seja $f : [a,b] \to \R$ uma função limitada.
        \begin{enumerate}[leftmargin=*, align=left, label=\textbf{(\alph*)}]
            \item $f$ é \textit{integrável em $[a,b]$ segundo Darboux} se
                \[
                    \lowerint_{a}^{b} f(x) \, dx = \upperint_{a}^{b} f(x) \, dx.
                \]
            \item Seja $f$ \textit{Darboux-integrável} em $[a,b]$. A \textit{integral de Darboux de $f$ em $[a,b]$} é definida como
                \[
                    \int_{a}^{b} f(x) \, dx := \lowerint_{a}^{b} f(x) \, dx = \upperint_{a}^{b} f(x) \, dx.
                \]
            \item Se $f$ está definida em $c \in \R$, estendemos a definição dizendo que $f$ é integrável em $c$ colocando
                \[
                    \int_{c}^{c} f(x) \, dx = 0.
                \]
                
            \item Se $f$ é integrável, estendemos a definição colocando
                \[
                    \int_{b}^{a} f(x) \, dx := - \int_{a}^{b} f(x) \, dx.
                \]
        \end{enumerate}
\end{defi}

\begin{obs}
    Decorre das definições que, sendo $f: [a,b] \to \R$ integrável, vale
        \[
            L(f,P) \leq \lowerint_{a}^{b} f(x) \, dx = \int_{a}^{b} f(x) \, dx = \upperint_{a}^{b} f(x) \, dx \leq U(f,P),
        \]
    para qualquer $P \in \mathcal{P}[a,b]$.
\end{obs}

\begin{comment}
\begin{ex}
    \textbf{(a)} Se $f: [a,b] \to \R$ é dada por $f(x) = c \in \R$, então $f$ é integrável e
        $ \ds
            \int_{a}^{b} c \, dx = c(b-a).
        $
        
    \textbf{(b)} Se $f: [0,1] \to \R$ é dada por $f(x) = x$, então $f$ é integrável e
        $ \ds
            \int_{0}^{1} x \, dx = \frac{1}{2}.
        $
\end{ex}
\end{comment}

\begin{teo}[Critério de integrabilidade]
    Uma função limitada $f:[a,b] \to \R$ é integrável em $[a,b]$ se, e somente se, para todo $\epsilon \in \R_{>0}$ existe $P \in \mathcal{P}[a,b]$ tal que $U(f,P) - L(f,P) < \epsilon$.
\end{teo}

\begin{proof}
    \textbf{(a)} ($\Rightarrow$)\footnote{Esta prova depende de um resultado sobre supremos e ínfimos.} Sendo $f$ integrável, para todo $\epsilon \in \R_{>0}$ existem partições $P_1, P_2 \in \mathcal{P}$ tais que
        \[
            S(P_2,f) - \int_{a}^{b} f(x) \, dx < \dfrac{\epsilon}{2} \quad \text{e} \quad \int_{a}^{b} f(x) \, dx - s(P_1,f) < \dfrac{\epsilon}{2}.
        \]
    Com isso, sendo $P$ uma partição comum à $P_1$ e $P_2$, temos que
        \[
            S^+(f,P) \leq S(P_2,f) < \int_{a}^{b} f(x) \, dx + \dfrac{\epsilon}{2} < s(P_1,f) + \epsilon \leq S_-(f,P) + \epsilon,
        \]
    de modo que $S^+(f,P) - S_-(f,P) < \epsilon$.

    ($\Leftarrow$) Agora, suponha que para todo $\epsilon \in \R_{>0}$ existe uma partição $P$ de $[a,b]$ tal que $S^+(f,P) - S_-(f,P) < \epsilon$. Como
        \[
            S_-(f,P) \leq \lowerint_{a}^{b} f(x) \, dx \leq \upperint_{a}^{b} f(x) \, dx \leq S^+(f,P),
        \]
    temos que
        $ \ds
            0 \leq \upperint_{a}^{b} f(x) \, dx - \lowerint_{a}^{b} f(x) \, dx \leq S^+(f,P) - S_-(f,P) < \epsilon,
        $
    de modo que
        $ \ds
            \upperint_{a}^{b} f(x) \, dx - \lowerint_{a}^{b} f(x) \, dx < \epsilon
        $
    para todo $\epsilon \in \R_{>0}$, donde
        $ \ds
            \upperint_{a}^{b} f(x) \, dx = \lowerint_{a}^{b} f(x) \, dx.
        $
    Assim, $f$ é integrável.
\end{proof}

\subsection{Estendendo a definição}

\begin{prop}
    \textbf{(a)} Se $f : [a,b] \to \R$ é uma função tal que $f(x) = 0$ para todo $x \in [a,b] \setminus \{c\}$, onde $c \in [a,b]$ e $f(c) \neq 0$, então $f$ é integrável em $[a,b]$ e
        \[
            \int_{a}^{b} f(x) \, dx = 0.
        \]
% é nula, exceto em um ponto $c \in [a,b]$, então $f$ é integrável em $[a,b]$ e
    \textbf{(b)} Se $f : [a,b] \to \R$ é uma função tal que $f(x) = 0$ para todo $x \in [a,b] \setminus \{c_1, c_2, \ldots, c_n\}$, onde $c_i \in [a,b]$ e $f(c_i) \neq 0$ para todo $i \in [n]$, então $f$ é integrável em $[a,b]$ e
        \[
            \int_{a}^{b} f(x) \, dx = 0.
        \]
%Se a função $f : [a,b] \to \R$ é nula, exceto em um número finito de pontos $c_1, c_2, \ldots, c_n \in [a,b]$, então $f$ é integrável em $[a,b]$ e
    \textbf{(c)} Sejam $f,g : [a,b] \to \R$ funções tais que $f(x) = g(x)$ para todo $x \in [a,b] \setminus \{c_1, c_2, \ldots, c_n \}$, onde $c_i \in [a,b]$ e $f(c_i) \neq g(c_i)$ para todo $i \in [n]$. Se $f$ é integrável, então $g$ é integrável e
        \[
            \int_{a}^{b} g(x) \, dx  = \int_{a}^{b} f(x) \, dx.
        \]
\end{prop}

\begin{proof}
    Táboas, observação 4.1.16, página 171.
\end{proof}

\begin{defi}
    Seja $f:[a,b] \setminus \{c_1, c_2, \ldots, c_n \} \to \R$ uma função, $c_i \in [a,b]$ para todo $i \in [n]$. Diremos que $f$ é integrável em $[a,b]$ se qualquer extensão $g$ de $f$ a $[a,b]$ o for, pondo
        \[
            \int_{a}^{b} f(x) \, dx  := \int_{a}^{b} g(x) \, dx.
        \]
\end{defi}

\section{Resultados}

\begin{teo}
    \textbf{(a)} Toda função $f: [a,b] \to \R$ contínua é integrável.

    \textbf{(b)} Se a função $f:[a,b] \to \R$ é limitada e tem apenas um número finito de pontos de descontinuidade, então $f$ é integrável.
\end{teo}

\begin{proof}
    \textbf{(a)} Pelo teorema da limitação \eqref{teo.calc:limitacao}, $f$ é limitada. Pelo teorema \eqref{teo.calc:contunifocont}, $f$ é uniformemente contínua, de modo que para todo $\epsilon \in \R_{>0}$ existe $\delta \in \R_{>0}$ tal que
        \[
            |x-y| < \delta \Rightarrow |f(x) - f(y)| < \dfrac{\epsilon}{b-a}.
        \]
    Escolha uma partição $P : a = x_0 < \cdots < x_n = b$ tal que $\Delta x_i < \delta$ para todo $i \in [n]$. Uma tal partição existe: tomando $n \in \N$ tal que $\ds n > \frac{b-a}{\delta}$, é fácil ver que definindo $\ds x_i = a + i \cdot \frac{b-a}{n}$ para cada $i \in [n]_0$ temos $\Delta x_i < \delta$ para cada $i \in [n]$. Agora, $f$ é contínua em cada intervalo $[x_{i-1}, x_i]$, de modo que, pelo teorema de Weierstrass \eqref{teo.calc:weierstrass}, existem $a_i,b_i \in [x_{i-1}, x_i]$ tais que $f(a_i) = m_i$ e $f(b_i) = M_i$, onde
        \begin{align*}
            M_i &:= \sup_{x \in [x_{i-1}, x_i]}{f(x)}; \\
            m_i &:= \inf_{x \in [x_{i-1}, x_i]}{f(x)}.
        \end{align*}
    Como $|b_i - a_i| \leq \Delta x_i < \delta$ para todo $i \in [n]$, pela continuidade uniforme de $f$ temos $\ds M_i - m_i = |f(b_i) - f(a_i)| < \dfrac{\epsilon}{b-a}$ para todo $i \in [n]$, de modo que
        \[
            S^+(f,P) - S_-(f,P) = \sum_{i=1}^{n} (M_i - m_i) \Delta x_i < \dfrac{\epsilon}{b-a} \sum_{i=1}^{n} \Delta x_i = \epsilon.
        \]
    Assim, pelo critério de integrabilidade, $f$ é integrável. \itemproof

    \textbf{(b)} Como $f$ é limitada, existe $M \in \R_{>0}$ tal que $|f(x)| \leq M$ para todo $x \in [a,b]$. Se $f$ tem, digamos $p \in \N$ pontos de descontinuidade, sejam eles $x_j \in [a,b]$, para cada $j \in [p]$. Agora, seja $\epsilon \in \R_{>0}$ arbitrário. Para cada $j \in [p]$, tome $[c_j, d_j]$ centrado em $x_j$ tal que $[c_i, d_i] \cap [c_j, d_j] = \emptyset$ se $i \neq j$ e $\sum_{j=1}^{p} (d_j - c_j) < \epsilon$. Tomando $[a_j, b_j] = [c_j, d_j] \cap [a,b]$ para cada $j \in [p]$, sendo $A := [a,b] \setminus \bigcup_{j=1}^{p} \left]a_j,b_j \right[$ temos que $f$ é uniformemente contínua em $A$.
\end{proof}

\begin{teo}
    \textbf{(a)} Toda função $f:[a,b] \to \R$ monótona é integrável.

    \textbf{(b)} Se a função $f:[a,b] \to [m, M]$ é integrável e a função $g : [m, M] \to \R$ é contínua, então a função $g \circ f : [a,b] \to \R$ é integrável.
\end{teo}

\begin{proof}
    \textbf{(a)} Suponha, num primeiro caso, que $f$ seja crescente. Com isso, $f(a) \leq f(x) \leq f(b)$ para todo $x \in [a,b]$ e $f$ é limitada em $[a,b]$. Para todo $\epsilon \in \R_{>0}$ existe $n = n(\epsilon) \in \N$ suficientemente grande de modo que
        \[
            n > \dfrac{(b-a)[f(b) - f(a)]}{\epsilon}.
        \]
    Agora, a partição $P : a = x_0 < \cdots < x_n = b$ definida por $x_i = a +  i \cdot \frac{b-a}{n}$ para todo $i \in [n] \cup \{ 0\}$ é tal que $\Delta x_i = \frac{b-a}{n}$,  $M_i = f(x_i)$ e $m_i = f(x_{i-1})$ (pois $f$ é crescente), donde
        \begin{align*}
            S^+(f,P) - S_-(f,P) &= \sum_{i=1}^{n} \left[ M_i \dfrac{b-a}{n} \right] - \sum_{i=1}^{n} \left[ m_i \dfrac{b-a}{n} \right] \\
            &= \dfrac{b-a}{n} \cdot \sum_{i=1}^{n} [f(x_i) - f(x_{i-1})] \\
            &= \dfrac{b-a}{n} [f(b) - f(a)] \\
            &<\epsilon.
        \end{align*}
    Assim, pelo critério de integrabilidade, $f$ é integrável. No caso em que $f$ é decrescente, a demonstração é análoga. \itemproof
    
    \textbf{(b)}
\end{proof}


\begin{teo}
    Se $f,g : [a,b] \to \R$ são funções integráveis em $[a,b]$, então
        
        \textbf{(a)} $f + g : [a,b] \to \R$ é integrável em $[a,b]$ e
            \[
                \int_{a}^{b} [f(x) + g(x)] \, dx = \int_{a}^{b} f(x) \, dx + \int_{a}^{b} g(x) \, dx.
            \]
        \textbf{(b)} $c \cdot f : [a,b] \to \R$ (onde $c \in \R$ é uma constante) é integrável em $[a,b]$ e
            \[
                \int_{a}^{b} [c \cdot f(x)] \, dx = c \cdot \int_{a}^{b} f(x) \, dx.
            \]
        \textbf{(c)} 
            $ \ds
                \int_{a}^{b} f(x) \, dx  \leq \int_{a}^{b} g(x) \, dx
            $
        sempre que $f(x) \leq g(x)$ para todo $x \in [a,b]$.
        
        \textbf{(d)} $f \cdot g : [a,b] \to \R$ é integrável em $[a,b]$.

        \textbf{(e)} $|f| : [a,b] \to \R$ é integrável em $[a,b]$ e
            \[
                \left| \int_{a}^{b} f(x) \, dx  \right| \leq \int_{a}^{b} |f(x)| \, dx.
            \]
\end{teo}

\begin{proof}
    Táboas, página 173.
\end{proof}

\begin{prop}
    Se $I \subseteq \R$ é um intervalo fechado e $f : I \to \R$ é uma função integrável em $I$, então 
        \[
            \int_{a}^{b} f(x) \, dx = \int_{a}^{c} f(x) \, dx + \int_{c}^{b} f(x) \, dx
        \]
    para quaisquer $a,b,c \in I$.
\end{prop}

\begin{proof}
\end{proof}

\section{O Teorema Fundamental do Cálculo}

\begin{teo} \label{teo:TFC0}
    Sejam $I \subseteq \R$ um intervalo e $f , g: I \to \R$ funções contínuas.
        \begin{enumerate}[leftmargin=*, align=left, label=\textbf{(\alph*)}]
            \item Se $f'(x) = 0$ para todo $x \in I$ interior, então existe uma constante $C \in \R$ tal que $f(x) = C$ para todo $x \in I$.

            \item Se $f'(x) = g'(x)$ para todo $x \in I$ interior, então existe uma constante $C \in \R$ tal que $f(x) = g(x) + C$ para todo $x \in I$.
        \end{enumerate}
\end{teo}

\begin{proof}
\end{proof}

\begin{defi}
    Seja $I \subseteq \R$ um intervalo. Diremos que a função $F : I \to \R$ é uma \textit{primitiva} da função $f:I \to \R$ se $F'(x) = f(x)$ para todo $x \in I$.
\end{defi}

\begin{teo}(Fundamental do Cálculo, parte I) \label{ar1.teo:TFC1}
     Seja $f: [a,b] \to \R$ uma função integrável.
        \begin{enumerate}[leftmargin=*, align=left, label=\textbf{(\alph*)}]
            \item  A função $F: [a,b] \to \R$ definida por
                \[
                    F(x) := \int_{a}^{x} f(t) \, dt
                \]
            é uniformemente contínua em $[a,b]$.
            
            \item Se $f$ é contínua em $x_0 \in [a,b]$, então $F$ é derivável em $x_0$ e $F'(x_0) = f(x_0)$.
        \end{enumerate} 
\end{teo}

\begin{proof}
    \textbf{(a)} Precisamos provar que para todo $\epsilon \in \R_{>0}$ existe $\delta \in \R_{>0}$ tal que
        \[
            |x-y| < \delta \Rightarrow |F(x) - F(y)| < \epsilon
        \]
    para quaisquer $x,y \in [a,b]$. Como $f$ é integrável, $f$ é limitada, de modo que existe $M \in \R_{>0}$ tal que $|f(x)| \leq M$ para todo $x \in [a,b]$. Como, para quaisquer $x,y \in [a,b]$, temos
        \[
            |F(x) - F(y)| = \left| \int_{x}^{y} f(t) \, dt \right| \leq |M(y-x)| = M |x-y|,
        \]
    basta tomar $\delta \leq \dfrac{\epsilon}{M}$. De fato, se $\delta = \dfrac{\epsilon}{M}$ e $x,y \in [a,b]$ são tais que $|x-y| < \dfrac{\epsilon}{M}$, então $|F(x) - F(y)| \leq M |x-y| < M \cdot \dfrac{\epsilon}{M} = \epsilon$, como queríamos.

    \textbf{(b)} Para provar que $F$ é derivável em $x_0$ e $F'(x_0) = f(x_0)$, basta provar que
        \[
            \lim_{h \to 0} \left[ \dfrac{F(x_0+h) - F(x_0)}{h} - f(x_0) \right] = 0.
        \]
    Mais precisamente, basta provar que para todo $\epsilon \in \R_{>0}$ existe $\delta \in \R_{>0}$ tal que
        \[
            0 < |h| < \delta \Rightarrow \left| \dfrac{F(x_0+h) - F(x_0)}{h} - f(x_0) \right| < \epsilon
        \]
    para todo $h \in \R_{\neq 0}$ tal que $x_0 + h \in [a,b]$. Da continuidade de $f$ em $x_0$, para todo $\epsilon \in \R_{>0}$ existe $\delta \in \R_{>0}$ tal que
        \[
            |t-x_0| < \delta \Rightarrow |f(t) - f(x_0)| < \epsilon
        \]
    para todo $t \in [a,b]$. Veja que todo $h \in \R_{\neq 0}$ com $0 < |h| < \delta$ e $x_0 + h \in [a,b]$ é tal que $|t - x_0| \leq |h| < \delta$ para todo $t$ no intervalo definido por $x_0$ e $x_0 + h$ (especificamente, $t \in [x_0, x_0+h]$ se $h>0$ ou $t \in [x_0+h,x_0]$ se $h<0$), de modo que $|f(t) - f(x_0)| < \epsilon$ para todo $t$ no intervalo definido por $x_0$ e $x_0 + h$. Com isso,
        \begin{align*}
            \left |\dfrac{F(x_0+h) - F(x_0)}{h} - f(x_0) \right| &= \left| \dfrac{1}{h} \int_{x_0}^{x_0+h} [f(t) - f(x_0))] \, dt \right| \\
            &\leq \dfrac{1}{|h|}  \left| \int_{x_0}^{x_0+h} |f(t) - f(x_0)| \, dt  \right| \\
            &<  \dfrac{1}{|h|} \left| \int_{x_0}^{x_0+h} \epsilon \, dt \right| = \dfrac{1}{|h|} \epsilon |h| = \epsilon,
        \end{align*}
    o que prova que
        \[
            \lim_{h \to 0} \left[ \dfrac{F(x_0+h) - F(x_0)}{h} - f(x_0) \right] = 0,
        \]
    como havíamos afirmado. \itemproof
\end{proof}

\begin{cor}
    Seja $f : [a,b] \to \R$ uma função contínua em $[a,b]$.
        \begin{enumerate}[leftmargin=*, align=left, label=\textbf{(\alph*)}]
            \item A função $F: [a,b] \to \R$ definida por
                \[
                    F(x) := \int_{a}^{x} f(t) \, dt
                \]
            é uma primitiva de $f$ em $[a,b]$.
            
            \item Se $G: [a,b] \to \R$ é qualquer outra primitiva de $f$, então
                \[
                    G(x) = G(a) + \int_{a}^{x} f(t) \, dt
                \]
            para todo $x \in [a,b]$. Particularmente para $x = b$, temos
                \[
                    \int_{a}^{b} f(t) \, dt = G(b) - G(a).
                \]
        \end{enumerate}
\end{cor}

\begin{proof}
    
\end{proof}

\begin{teo}[Fundamental do Cálculo, parte II]
    Se $f : [a,b] \to \R $ é uma função integrável e $F : [a,b] \to \R$ é uma primitiva qualquer de $f$, então
        \[
            F(x) = F(a) + \int_{a}^{x} f(t) \, dt
        \]
    para todo $x \in [a,b] $. Particularmente para $x=b$, temos
        \[
            \int_{a}^{b} f(t) \, dt = F(b) - F(a).
        \]
\end{teo}

\begin{proof}
\end{proof}


\section{A Integral de Riemann}

\begin{defi}
    Seja $P : a < x_0 \cdots < x_n = b$ uma partição do intervalo $[a,b]$.
        \begin{enumerate}[leftmargin=*, align=left, label=\textbf{(\alph*)}]
            \item Definimos $\max \Delta x_i$ como a \textit{norma} de $P$, a qual denotaremos por $\| P \|$, isto é, $\| P \| := \max \Delta x_i$.

            \item (Partição marcada)

            \item (Soma de Riemann)
        \end{enumerate}
\end{defi}

\begin{defi}
    Seja $f : [a,b] \to \R$ uma função. Diremos que a soma de Riemann $S(f, P, \xi)$ tem limite $L \in \R$ quando $\| P \|$ tende a $0$, denotando isso por
        \[
            \lim_{\| P \| \to 0} S(f, P, \xi) = L,
        \]
    se para todo $\epsilon \in \R_{>0}$ existir $\delta = \delta (\epsilon) \in \R_{>0}$ tal que
        \[
            |S(f, P, \xi) - L| < \epsilon
        \]
    para toda partição marcada $(P, \xi)$ de $[a,b]$ com $\| P \| < \delta$. 
\end{defi}

\begin{prop}
    Seja $f : [a,b] \to \R$ uma função. O limite das somas de Riemann, quando existe, é único, isto é, se
        \[ \ds
            \lim_{\| P \| \to 0} S(f, P, \xi) = L_1 \quad \text{e} \quad \lim_{\| P \| \to 0} S(f, P, \xi) = L_2,
        \]
    então $L_1 = L_2$.
\end{prop}

\begin{proof}
\end{proof}

\begin{defi}
    Diremos que uma função $f : [a,b] \to \R$ é \textit{integrável em $[a,b]$ segundo Riemann} se
        $ \ds
            \lim_{\| P \| \to 0} S(f, P, \xi)
        $
    existir. Nesse caso, esse número real será chamado de \textit{integral de $f$ em $[a,b]$ segundo Riemann}, o qual será denotado por
        \[
            \int_{a}^{b} f(x) \, dx,
        \]
    isto é,
        \[
            \int_{a}^{b} f(x) \, dx := \lim_{\| P \| \to 0} S(f, P, \xi).
        \]
\end{defi}

\begin{prop}
    Se $f : [a,b] \to \R$ é uma função integrável segundo Riemann, então $f$ é limitada em $[a,b]$.
\end{prop}

\begin{proof}
\end{proof}

\begin{teo}
    Seja $f : [a,b] \to \R$ uma função.
        \begin{enumerate}[leftmargin=*, align=left, label=\textbf{(\alph*)}]
            \item $f$ é integrável segundo Riemann se, e somente se, é integrável segundo Darboux.

            \item Sendo $f$ integrável, as integrais de Riemann e Darboux coincidem.
        \end{enumerate}
\end{teo}

\begin{proof}
\end{proof}



\section{Integrais Impróprias}

\begin{defi}
    Seja 
\end{defi}






























\chapter{Demonstrações}

\begin{proof}
    \textbf{(a)} Consideremos o caso em que $x \to p$. Como $\ds \lim_{x \to p} f(x) = L_1$ e $\ds \lim_{x \to p} f(x) = L_2$, temos por definição que para todo $\epsilon > 0$ existem $\delta_1, \delta_2 > 0$ para os quais
    \begin{align*}  
            0 < | x - p | < \delta_1 \Rightarrow | f(x) - L_1 | < \dfrac{\epsilon}{2}; \\
            0 < | x - p | < \delta_2 \Rightarrow | f(x) - L_2 | < \dfrac{\epsilon}{2}.
        \end{align*}
    Tomando $\delta := \min \{\delta_1, \delta_2\}$, temos que para todo $\epsilon > 0$ existe $\delta > 0$ tal que
        \[
            0 < | x - p | < \delta \Rightarrow |f(x) - L_1| + |f(x) - L_2| < \epsilon.
        \]
    Com isso, temos que, para todo $\epsilon > 0$,
        \begin{align*}
            | L_1 - L_2 | &= | L_1 - f(x) + f(x) - L_2 | \\
            &\leq | L_1 - f(x) | + | f(x) - L_2 | \\
            &= |f(x) - L_1| + |f(x) - L_2| \\
            &< \epsilon.
        \end{align*}
    Daí, $L_1 = L_2$. \itemproof

    \textbf{(b)} \itemproof

    \textbf{(c)} \itemproof

    \textbf{(d)} (Verificar) Como, por hipótese, $\ds \lim_{x \to p} f(x) = L = \lim_{x \to p} h(x)$, temos
        \begin{align*}
            \forall \epsilon > 0, \exists \delta_1 > 0: 0 < \left| x - p \right| < \delta_1 \Rightarrow L - \epsilon < f(x) < L + \epsilon; \\
            \forall \epsilon > 0, \exists \delta_2 > 0: 0 < \left| x - p \right| < \delta_1 \Rightarrow L - \epsilon < h(x) < L + \epsilon.
        \end{align*}
    Pois tome $\delta = \min \left\{ \delta_1, \delta_2, r \right\}$; daí, vem
        \[
            \forall \epsilon > 0, \exists \delta > 0 : 0 < \left| x-p \right| < \delta \Rightarrow L - \epsilon < f(x) \leq g(x) \leq h(x) < L + \epsilon,
        \]
    e então
        \[
            \forall \epsilon > 0, \exists \delta > 0 : 0 < \left| x-p \right| < \delta \Rightarrow L - \epsilon < g(x) < L + \epsilon, 
        \]
    isto é, $\ds \lim_{x \to p} g(x) = L$.
\end{proof}

\begin{proof}
    \textbf{(a)} (Verificar) Consideremos o caso em que $x \to p$. Precisamos provar que
        \[
            \forall \epsilon > 0, \exists \delta > 0 : 0 < \left| x-p \right| < \delta \Rightarrow \lim_{u \to a} g(u) - \epsilon < g[f(x)] < \lim_{u \to a} g(u) + \epsilon.
        \]
     Como $\ds \lim_{u \to a} g(u) = g(a)$, temos que provar que
        \[
            \forall \epsilon > 0, \exists \delta > 0 : 0 < \left| x-p \right| < \delta \Rightarrow g(a) - \epsilon < g[f(x)] <g(a) + \epsilon. \tag{1} \label{eq:1}
        \]
    Por definição,
        \begin{multline*}
            \lim_{u \to a} g(u) = g(a) \Leftrightarrow \forall \epsilon > 0, \exists \delta_1 > 0: \\
            a - \delta_1 < u < a + \delta_1 \Rightarrow g(a) - \epsilon < g(u) < g(a) + \epsilon,
        \end{multline*}
    sendo esta última parte equivalente a 
        \[
            a - \delta_1 < f(x) < a + \delta_1 \Rightarrow g(a) - \epsilon < g[f(x)] < g(a) + \epsilon. \tag{2} \label{eq:2}
        \]
    Como, por hipótese, $\ds \lim_{x \to p} f(x) = a$, temos que
        \[
            \forall \epsilon > 0, \exists \delta > 0 : 0 < \left| x-p \right| < \delta \Rightarrow a - \epsilon < f(x) < a + \epsilon. \tag{3} \label{eq:3}
        \]
    Para $\epsilon = \delta_1$ em $\eqref{eq:3}$, existe um $\delta > 0$ tal que 
        \[
            0 < \left| x-p \right| < \delta \Rightarrow a - \delta_1 < f(x) < a + \delta_1. \tag{4} \label{eq:4}
        \]
    Daí, $\eqref{eq:4}$, com $\eqref{eq:2}$, resulta que
        \[
            \forall \epsilon > 0, \exists \delta > 0 : 0 < \left| x-p \right| < \delta \Rightarrow g(a) - \epsilon < g[f(x)] < g(a) + \epsilon,
        \]
    como queríamos provar. \itemproof
    
    \textbf{(b)} (Verificar) Precisamos provar que
        \[
            \forall \epsilon > 0, \exists \delta > 0 : 0 < \left| x-p \right| < \delta \Rightarrow \left| g[f(x)] - \lim_{u \to a} g(u) \right| < \epsilon,
        \]
    isto é,
        \[  
            0 < \left| x-p \right| < \delta \Rightarrow \left| g[f(x)] - L \right| < \epsilon.
        \]
    Bem, por definição,
        \[
            \lim_{u \to a} g(u) = g(a) \Leftrightarrow \forall \epsilon > 0, \exists \delta_1 > 0: 0 < \left| u-a \right| < \delta_1 \Rightarrow \left| g(u) - L \right| < \epsilon.
        \]
    Lembrando que $u := f(x)$, temos então
        \[
            0 < \left| f(x) - a \right| < \delta_1 \Rightarrow \left| g[f(x)] - L \right| < \epsilon.
        \]
    Por outro lado,
        \[
            \lim_{x \to p} f(x) = a \Leftrightarrow \forall \epsilon > 0, \exists \delta_2 > 0 : 0 < \left| x-p \right| < \delta_2 \Rightarrow \left| f(x) - a \right| < \epsilon,
        \]
    e então, tomando $\epsilon = \delta_1$, $0 < \left| x-p \right| < \delta_2 \Rightarrow \left| f(x) - a \right| < \delta_1$.

    Pois tome $\delta = \left\{ \delta_2, r \right\}$; teremos $0 < \left| x-p \right| < \delta \Rightarrow 0 < \left| f(x) - a \right| < \delta_1$.
\end{proof}


\begin{proof}
\end{proof}

\newpage

Um \textit{designador} é uma expressão que é um termo ou uma fórmula. Como se vê na definição de termo e fórmula, todo designador tem a forma \( \mathbf{u} \mathbf{v}_1 \ldots \mathbf{v}_n \), onde \( \mathbf{u} \) é um símbolo, \( \mathbf{v}_1, \ldots, \mathbf{v}_n \) são designadores, e \( n \) é um número natural determinado por \( \mathbf{u} \). Por exemplo, se \( \mathbf{u} \) é uma variável, então \( n = 0 \); se \( \mathbf{u} \) é um símbolo funcional \( k \)-ário, então \( n = k \); se \( \mathbf{u} \) é \( \exists \), então \( n = 2 \). Chamamos \( n \) de \textit{índice} de \( \mathbf{u} \). %ok

Dizemos que duas expressões são \textit{compatíveis} se uma delas pode ser obtida adicionando alguma expressão (possivelmente a expressão vazia) ao final da outra. 
    \begin{itemize}
        \item Se \( \mathbf{u} \mathbf{v} \) e \( \mathbf{u}'\mathbf{v}' \) são compatíveis, então \( \mathbf{u} \) e \( \mathbf{u}' \) são compatíveis; 
        \item se \( \mathbf{u} \mathbf{v} \) e \( \mathbf{u} \mathbf{v}' \) são compatíveis, então \( \mathbf{v} \) e \( \mathbf{v}' \) são compatíveis.
    \end{itemize} %ok

\textbf{Lema.} Seja $n$ um natural fixo. Se $\mathbf{u}_1, \ldots, \mathbf{u}_n$ e $\mathbf{u}'_1, \ldots, \mathbf{u}'_n$ são designadores, e $\mathbf{u}_1 \ldots \mathbf{u}_n$ e $\mathbf{u}'_1 \ldots \mathbf{u}'_n$ são compatíveis, então $\mathbf{u}'_i$ é $\mathbf{u}_i$, para $i = 1, \ldots, n$.

\begin{proof}
    Façamos indução no comprimento de $\mathbf{u}_1 \ldots \mathbf{u}_n$, isto é, no número de símbolos totais dessa expressão. Sendo $|\mathbf{u}_1 \ldots \mathbf{u}_n| = L$, provemos que se o resultado vale para toda sequência de designadores de comprimento menor que $L$, então vale para as sequências de designadores de comprimento $L$. Escrevendo $\mathbf{u}_1$ como $\mathbf{v} \mathbf{v}_1 \ldots \mathbf{v}_k$, onde $\mathbf{v}$ é um símbolo e $\mathbf{v}_1, \ldots, \mathbf{v}_k$ são designadores, vemos que $\mathbf{u}'_1$ é da forma $\mathbf{v} \mathbf{v}'_1 \ldots \mathbf{v}'_k$, onde $\mathbf{v}'_1, \ldots, \mathbf{v}'_k$ são designadores. Agora, como $\mathbf{u}_1$ é compatível com $\mathbf{u}'_1$ (isso segue do primeiro bullet point, fazendo $\mathbf{u}$ ser $\mathbf{u}_1$, $\mathbf{v}$ ser $\mathbf{u}_2 \ldots \mathbf{u}_n$, $\mathbf{u}'$ ser $\mathbf{u}'_1$ e $\mathbf{v}'$ ser $\mathbf{u}'_2 \ldots \mathbf{u}'_n$), temos que $\mathbf{v} \mathbf{v}_1 \ldots \mathbf{v}_k$ é compatível com $\mathbf{v} \mathbf{v}'_1 \ldots \mathbf{v}'_k$, de modo que  $\mathbf{v}_1 \ldots \mathbf{v}_k$ é compatível com $\mathbf{v}'_1 \ldots \mathbf{v}'_k$ (isso segue do segundo bullet point). Como, evidentemente, $|\mathbf{v}_1 \ldots \mathbf{v}_k| < L$, pela hipótese de indução concluímos que $\mathbf{v}_i$ é $\mathbf{v}'_i$ para cada $i = 1, \ldots, k$. Com isso, temos que $\mathbf{u}_1$ é $\mathbf{u}'_1$, de modo que $\mathbf{u}_2 \ldots \mathbf{u}_n$ é compatível com $\mathbf{u}'_2 \ldots \mathbf{u}'_n$ (pela segunda observação), e como $|\mathbf{u}_2 \ldots \mathbf{u}_n| < L$, pela hipótese de indução concluímos que $\mathbf{u}_i$ é $\mathbf{u}'_i$ para $i = 2, \ldots, n$. Isso nos dá a tese de indução, já que já sabemos que $\mathbf{u}_1$ é $\mathbf{u}'_1$.
\end{proof}

\begin{proof}
    Façamos indução forte no comprimento de $\mathbf{u}_1 \ldots \mathbf{u}_n$, isto é, no número de símbolos totais dessa expressão. Sendo $|\mathbf{u}_1 \ldots \mathbf{u}_n| = L$, provemos que se o resultado vale para toda sequência de designadores de comprimento menor que $L$, então vale para as sequências de designadores de comprimento $L$. Escrevendo $\mathbf{u}_1$ como $\mathbf{v} \mathbf{v}_1 \ldots \mathbf{v}_k$, onde $\mathbf{v}$ é um símbolo e $\mathbf{v}_1, \ldots, \mathbf{v}_k$ são designadores, vemos que $\mathbf{u}'_1$ é da forma $\mathbf{v} \mathbf{v}'_1 \ldots \mathbf{v}'_k$, onde $\mathbf{v}'_1, \ldots, \mathbf{v}'_k$ são designadores. Agora, como $\mathbf{u}_1$ é compatível com $\mathbf{u}'_1$ (isso segue da primeira observação ``se \( \mathbf{u} \mathbf{v} \) e \( \mathbf{u}'\mathbf{v}' \) são compatíveis, então \( \mathbf{u} \) e \( \mathbf{u}' \) são compatíveis'', fazendo $\mathbf{u}$ ser $\mathbf{u}_1$, $\mathbf{v}$ ser $\mathbf{u}_2 \ldots \mathbf{u}_n$, $\mathbf{u}'$ ser $\mathbf{u}'_1$ e $\mathbf{v}'$ ser $\mathbf{u}'_2 \ldots \mathbf{u}'_n$), temos que $\mathbf{v} \mathbf{v}_1 \ldots \mathbf{v}_k$ é compatível com $\mathbf{v} \mathbf{v}'_1 \ldots \mathbf{v}'_k$, de modo que  $\mathbf{v}_1 \ldots \mathbf{v}_k$ é compatível com $\mathbf{v}'_1 \ldots \mathbf{v}'_k$ (isso segue da segunda observação ``se \( \mathbf{u} \mathbf{v} \) e \( \mathbf{u} \mathbf{v}' \) são compatíveis, então \( \mathbf{v} \) e \( \mathbf{v}' \) são compatíveis''). Como, evidentemente, $|\mathbf{v}_1 \ldots \mathbf{v}_k| < L$, pela hipótese de indução concluímos que $\mathbf{v}_i$ é $\mathbf{v}'_i$ para cada $i = 1, \ldots, k$. Com isso, temos que $\mathbf{u}_1$ é $\mathbf{u}'_1$, de modo que $\mathbf{u}_2 \ldots \mathbf{u}_n$ é compatível com $\mathbf{u}'_2 \ldots \mathbf{u}'_n$ (pela segunda observação), e como $|\mathbf{u}_2 \ldots \mathbf{u}_n| < L$, pela hipótese de indução concluímos que $\mathbf{u}_i$ é $\mathbf{u}'_i$ para $i = 2, \ldots, n$. Isso nos dá a tese de indução, já que já sabemos que $\mathbf{u}_1$ é $\mathbf{u}'_1$.
   \end{proof}
 \newpage
Consideremos um conjunto enumerável correspondente aos símbolos do alfabeto e $X$ o conjunto de todas as sequências finitas de símbolos. Seja $Z$ o conjunto de todos os subconjuntos $Y$ de $X$ tais que
\begin{itemize}
    \item as variáveis proposicionais pertencem a $Y$;
    \item se $A$ pertence a $Y$, então $(\neg A)$ pertence a $Y$;
    \item se $A$ e $B$ pertencem a $Y$, então $(A \land B)$, $(A \lor B)$, $(A \rightarrow B)$ e $(A \leftrightarrow B)$ pertencem a $Y$.
\end{itemize}

\begin{teo}
Suponha que uma propriedade vale para toda fórmula atômica e que, se vale para as fórmulas $A$ e $B$, também vale para $(\neg A)$, $(A \land B)$, $(A \lor B)$, $(A \rightarrow B)$ e $(A \leftrightarrow B)$. Então essa propriedade vale para todas as fórmulas da lógica proposicional.
\end{teo}

\begin{proof}
    Tomando $F$ a interseção de $Z$, temos que $F$ satisfaz as condições acima (isto é, pertence à família $Z$) e é o menor conjunto (na ordem da inclusão) que pertence a $Z$. Isto é, se $Y \in Z$, então $F \subset Y$. Segue facilmente, daí, o teorema. Deixamos os detalhes ao leitor.
\end{proof}